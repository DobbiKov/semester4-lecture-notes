\chapter{Déterminants}
Ce chapitre est plutôt un cheatsheet des déterminants car je ne vais pas donner des preuves mais les propriétés utiles, les exemples et de l'intuition.

\begin{definition}
    Soit $A = [a_{i, j}] \in \mathcal{M}_{n}(\R)$ une matrice carée $n \times n$, alors:
    \[
        \operatorname{det}(A) = \sum_{\sigma \in S_n}^{} \operatorname{signe}(\sigma) \cdot \prod_{i=1}^{n} a_{i, \sigma(i)} 
    \] 
    où 
    \begin{itemize}
        \item $S_n$ est un groupe de toute permutation de  $\{1, \ldots, n\}$
        \item $\operatorname{signe}(\sigma)$ est une signe de pérmutation
    \end{itemize}
\end{definition}

Cette définition est très formelle, alors au bout de ce chapitre on va reformuler cette définition. D'abord, on va étudier les propriétés de déterminants:

\section{Propriétés les plus improtants}
\begin{prop} les propriétés de déterminant.
    Pour cette proposition, on note $\det(c_1, \ldots, c_n)$ un déterminant où $\forall i, \, r_i$ et $\forall i, \, y_i$ représentent une colonne (ou un vecteur colonne). Et $\forall i, \lambda_i \in \R$.
    \begin{enumerate}
        \item \textbf{Déterminant de la matrice identité est 1:}
            \[
            \det(I_n) = 1
            \] 
        \item \textbf{Déterminant de la matrice du rang 1 est son seul élément}:
            \[
                \det(\begin{bmatrix} a_{1,1} \end{bmatrix} ) = a_{1,1} \qquad \text{ où } a_{1,1} \in \R
            \] 
        \item \textbf{Linéarité 1}:
            \[
            \det(r_1, \ldots, r_i + y_i, \ldots, r_n) = \det(r_1, \ldots, r_i, \ldots, r_n) + \det(r_1, \ldots, y_i, \ldots, r_n)
            \] 
        \item \textbf{Linéarité 2}:
            \[
            \det(r_1, \ldots, \lambda_ir_i, \ldots, r_n) = \lambda_i\det(r_1, \ldots, r_i, \ldots, r_n) 
            \] 
            \begin{note}
               C'est pourquoi:
               \[
               \det(\lambda A) = \lambda^n\det(A)
               \] 
            \end{note}
        \item \textbf{Mêmes colonnes}: Supposons que $i \neq j$ et $c_i = c_j$ alors:
             \[
            \det(c_1, \ldots, c_i, \ldots, c_j, \ldots, c_n) = 0
            \] 
            S'il y a deux colonnes identiques, alors $\det$ est égale à 0.
        \item \textbf{Déplacements des colonnes}:
            
\[
    \det(c_1, \ldots, c_i, \ldots, c_j, \ldots, c_n) 
    = -\det(c_1, \ldots, 
    \underbrace{c_j , \ldots, 
    c_i}_{\text{permutation}}, \ldots, c_n)
\]
Autrement dire, une permutation des colonnes change la signe.

\item \textbf{Détérminant des matrices multipliées}: Soient $A, B \in \mathcal{M}_n(\R)$
    \[
        \det(AB) = \det(A)\det(B) 
    \] 

\item \textbf{Détérminant d'une matrice transposé}: Soit $A \in \mathcal{M}_n(\R)$
    \[
        \det(A^{T}) = \det(A)
    \] 


% Drawing the arc manually
\end{enumerate}
\end{prop}

\section{Développement par rapport à une ligne/colonne}
\begin{definition}\label{def:matrice-ligne-colonne-supprime}
    Soit $A = (a_{i, j}) \in \mathcal{M}_n(\R)$ une matrice carrée, i.e:
    \[
        A = 
        \begin{bmatrix} 
            a_{1, 1} & a_{1, 2} & \ldots & a_{1, i - 1} & a_{1, i} & a_{1, i+1} & \ldots & a_{1, n} \\
            a_{2, 1} & a_{2, 2} & \ldots & a_{2, i - 1} & a_{2, i} & a_{2, i+1} & \ldots & a_{2, n} \\
            \vdots   & \vdots   & \vdots & \vdots       & \vdots   & \vdots     & \vdots & \vdots \\
            a_{j-1, 1} & a_{j-1, 2} & \ldots & a_{j-1, i - 1} & a_{j-1, i} & a_{j-1, i+1} & \ldots & a_{j-1, n} \\
            a_{j, 1} & a_{j, 2} & \ldots & a_{j, i - 1} & a_{j, i} & a_{j, i+1} & \ldots & a_{j, n} \\
            a_{j+1, 1} & a_{j+1, 2} & \ldots & a_{j+1, i - 1} & a_{j+1, i} & a_{j+1, i+1} & \ldots & a_{j+1, n} \\
            \vdots   & \vdots   & \vdots & \vdots       & \vdots   & \vdots     & \vdots & \vdots \\
            a_{n, 1} & a_{n, 2} & \ldots & a_{n, i - 1} & a_{n, i} & a_{n, i+1} & \ldots & a_{n, n} 
        \end{bmatrix} 
    \] 

    Alors, $A_{j, i}$ est une matrice où la ligne $j$ et la colonne  $i$ sont supprimé, i.e:
    \[
        A_{j, i} = 
        \begin{bmatrix} 
            a_{1, 1} & a_{1, 2} & \ldots & a_{1, i - 1} & & a_{1, i+1} & \ldots & a_{1, n} \\
            a_{2, 1} & a_{2, 2} & \ldots & a_{2, i - 1} & & a_{2, i+1} & \ldots & a_{2, n} \\
            \vdots   & \vdots   & \vdots & \vdots       & & \vdots     & \vdots & \vdots \\
           a_{j-1, 1} & a_{j-1, 2} & \ldots & a_{j-1, i - 1} & & a_{j-1, i+1} & \ldots & a_{j-1, n} \\
             & & & & & & & \\
            a_{j+1, 1} & a_{j+1, 2} & \ldots & a_{j+1, i - 1} & & a_{j+1, i+1} & \ldots & a_{j+1, n} \\
            \vdots   & \vdots   & \vdots & \vdots       & \vdots   & & \vdots & \vdots \\
            a_{n, 1} & a_{n, 2} & \ldots & a_{n, i - 1} & & a_{n, i+1} & \ldots & a_{n, n} 
        \end{bmatrix} \in \mathcal{M}_{n-1}(\R)
    \] 
\end{definition}

Cela nous permet de développer le détérminant par rapport à une ligne ou une colonne:
\begin{prop}
    Soit $A = (a_{i, j}) \in \mathcal{M}_n(\R)$ une matrice carrée et soit $1 \le k \le n$
    \[
        \displaystyle \det(A) = \sum_{i=1}^{n} (-1)^{i + k} a_{k,i} \det(A_{k, i}) 
    \] 
    est le calcul de détérminant par rapport à $k^{\text{ième}}$ ligne.
\end{prop}
\begin{eg}
   Soit  
   \[
   A = 
   \begin{bmatrix} 
       1 & 4 & 5\\
       2 & 9 & 8\\
       3 & 7 & 6
   \end{bmatrix} \in \mathcal{M}_3(\R)
   \] 
\begin{figure}[H]
    \centering
    \incfig{mat-ligne-1-colonne-3}
    \caption{Développement par rapport à la deuxiemme ligne}
    \label{fig:mat-ligne-1-colonne-3}
\end{figure}
Donc:
\begin{align*}
    \det(A) &= \sum_{i=1}^{n} (-1)^{i + 2} a_{2, i} \det(A_{2, i}) \\
            &= (-1)^{1 + 2} \cdot a_{2, 1} \cdot \det(A_{2, 1}) + (-1)^{2 + 2} \cdot a_{2, 2} \cdot \det(A_{2,2})  + (-1)^{3 + 2} \cdot a_{2, 3} \cdot \det(A_{2, 3}) \\
            &= (-1)^{1 + 2} \cdot 2 \cdot \begin{vmatrix} 4 & 5 \\ 7 & 6 \end{vmatrix} + (-1)^{2 + 2} \cdot 9 \cdot \begin{vmatrix} 1 & 5 \\ 3 & 6 \end{vmatrix}  + (-1)^{3 + 2} \cdot 8 \cdot \begin{vmatrix} 1 & 4 \\ 3 & 7 \end{vmatrix} \\
            &= (-1) \cdot 2 \cdot (-11) + 1 \cdot 9 \cdot (-9) + (-1) \cdot 8 \cdot (-5)\\
            &= 22 - 81 + 40\\
            &= -19
\end{align*}
\end{eg}

\begin{prop}
    Soit $A = (a_{i, j}) \in \mathcal{M}_n(\R)$ une matrice carrée et soit $1 \le k \le n$
    \[
        \displaystyle \det(A) = \sum_{i=1}^{n} (-1)^{i + k} a_{i,k} \det(A_{i,k}) 
    \] 
    est le calcul de détérminant par rapport à $k^{\text{ième}}$ colonne.
\end{prop}

\begin{eg}
   Soit  
   \[
   A = 
   \begin{bmatrix} 
       1 & 4 & 5\\
       2 & 9 & 8\\
       3 & 7 & 6
   \end{bmatrix} \in \mathcal{M}_3(\R)
   \] 

\begin{figure}[H]
    \centering
    \incfig{mat-colonne-2}
    \caption{Développement par rapport à la deuxiemme colonne}
    \label{fig:mat-colonne-2}
\end{figure}
Donc:
\begin{align*}
    \det(A) &= \sum_{i=1}^{n} (-1)^{i + 2} a_{i, 2} \det(A_{i, 2}) \\
            &= (-1)^{1 + 2} \cdot a_{1, 2} \cdot \det(A_{1, 2}) + (-1)^{2 + 2} \cdot a_{2, 2} \cdot \det(A_{2,2})  + (-1)^{3 + 2} \cdot a_{3, 2} \cdot \det(A_{3, 2}) \\
            &= (-1)^{1 + 2} \cdot 4 \cdot \begin{vmatrix} 2 & 8 \\ 3 & 6 \end{vmatrix} + (-1)^{2 + 2} \cdot 9 \cdot \begin{vmatrix} 1 & 5 \\ 3 & 6 \end{vmatrix}  + (-1)^{3 + 2} \cdot 7 \cdot \begin{vmatrix} 1 & 5 \\ 2 & 8 \end{vmatrix} \\
            &= (-1) \cdot 4 \cdot (-12) + 1 \cdot 9 \cdot (-9) + (-1) \cdot 7 \cdot (-2)\\
            &= 48 - 81 + 14\\
            &= -19
\end{align*}
\end{eg}

\section{Déterminant d'une matrice triangulaire}
\begin{corollary}
   Le déterminant d'une matrice triangulaire est un produit de ces éléments diagonaux. I.e, soit une matrice triangulaire
   \[
   A = \begin{bmatrix} 
       a_{1, 1} & a_{1, 2} & \ldots & a_{1, n-1} & a_{1, n}\\
       0        & a_{2, 2} & \ldots & a_{2, n-1} & a_{2, n}\\
       \vdots   & \vdots   & \ddots & \vdots     & \vdots  \\
       0        & 0        & \ldots & 0          & a_{n, n}
       0        & 0        & \ldots & 0          & a_{n, n}
   \end{bmatrix} 
   \] 
   alors 
   \[
   \det(A) = a_{1,1} \cdot a_{2,2} \cdot \ldots \cdot a_{n,n}
   \] 
\end{corollary}

\begin{eg}
   Soit  
   \[
   A = 
   \begin{bmatrix} 
       1 & 4 & 5\\
       0 & 9 & 8\\
       0 & 0 & 6
   \end{bmatrix} \in \mathcal{M}_3(\R)
   \] 
Développons ce déterminant par rapport à la première colonne:
\begin{align*}
    \det(A) &= \sum_{i=1}^{n} (-1)^{i + 2} a_{i, 2} \det(A_{i, 2}) \\
            &= (-1)^{1 + 1} \cdot a_{1, 1} \cdot \det(A_{1, 1}) + (-1)^{2 + 1} \cdot a_{2, 1} \cdot \det(A_{2,1})  + (-1)^{3 + 1} \cdot a_{3, 1} \cdot \det(A_{3, 1}) \\
            &= (-1)^{2} \cdot 1 \cdot \begin{vmatrix} 9 & 8 \\ 0 & 6 \end{vmatrix} + \underbrace{(-1)^{3} \cdot 0 \cdot \begin{vmatrix} 4 & 5 \\ 0 & 6 \end{vmatrix}}_{= 0}  + \underbrace{(-1)^{4} \cdot 0 \cdot \begin{vmatrix} 4 & 5 \\ 9 & 8 \end{vmatrix}}_{= 0} \\
            &= \underbrace{1}_{= a_{1,1}} \cdot \begin{vmatrix} 9 & 8 \\ 0 & 6 \end{vmatrix}\\
            &= \det(\begin{bmatrix} 9 & 8 \\ 0 & 6 \end{bmatrix} =: B)\\
            &= (-1)^{1 + 1} \cdot b_{1, 1} \cdot \det(B_{1,1}) + (-1)^{2 + 1} \cdot b_{2, 1} \cdot \det(B_{2, 1}) \quad \substack{\text{ développement par rapport}\\\text{à la premiere colonne}}\\
            &= 1 \cdot \underbrace{9}_{a_{2,2}} \cdot \begin{vmatrix} 6 \end{vmatrix} + \underbrace{(-1) \cdot 0 \cdot \begin{vmatrix} 8 \end{vmatrix} }_{= 0}\\
            &= \underbrace{1}_{= a_{1,1}} \cdot \underbrace{9}_{= a_{2,2}} \cdot \underbrace{6}_{= a_{3,3}}
\end{align*}
\end{eg}

\section{Matrice adjointe}
D'abord, rappelons la définition de $A_{i,j}$. C'est une matrice carrée où $i^{\text{ième}}$ ligne et $j^{\text{ième}}$ colonne sont supprimé. (Voir la définition ~\ref{def:matrice-ligne-colonne-supprime}).
\begin{definition}
    Soit une matrice carrée $A = (a_{i, j}) \in \mathcal{M}_n(\R)$. On note
    \[
        b_{i, j} = (-1)^{i + j}\det(A_{i, j})
    \] 
    Ensuite, on note la matrice
    \[
        N = 
        \begin{bmatrix} 
            b_{1,1} & \ldots & b_{1, n}\\
            \vdots  & \ddots & \vdots  \\
            b_{n,1} & \ldots & b_{n, n}\\
        \end{bmatrix} 
    \] 
    Alors, la matrice adjointe de $A$ est définie comme:
    \[
        A^{*} = N^{T} = 
        \begin{bmatrix} 
            b_{1,1} & \ldots & b_{n, 1}\\
            \vdots  & \ddots & \vdots  \\
            b_{1,n} & \ldots & b_{n, n}\\
        \end{bmatrix} 
    \] 
\end{definition}

\begin{theorem}
    Soit $A \in \mathcal{M}_n{\R}$ une matrice carrée et $A^{*}$ sa matrice adjointe, alors on a:
     \[
         A^{*}A = A A^{*} = \det(A)I_n = 
         \begin{bmatrix}  
             \det(A) & 0        & 0      & \ldots & 0 & 0\\
             0       & \det(A)  & 0      & \ldots & 0 & 0\\
             \vdots  & \vdots   & \vdots & \ddots & \vdots & \vdots\\
             0       & 0        & 0      & \ldots & 0      & \det(A)
         \end{bmatrix}  
    \] 
\end{theorem}
Utilité de telle matrice?
\section{Matrice inverse}
\begin{theorem}
    Soit $A \in \mathcal{M}_n(\R)$ une matrice carrée telle que $\det(A) \neq 0$, alors:
    \[
        A^{-1} = \frac{1}{\det(A)}\cdot A^{*}
    \] 
    est la matrice inverse de $A$.
\end{theorem}
\begin{corollary}
   Si $A \in \mathcal{M}_n(\R)$ une matrice carrée inversible, alors:
   \[
       \det(A^{-1}) = \frac{1}{\det(A)}
   \] 
\end{corollary}
