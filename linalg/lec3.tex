\section{Isométries et Adjoints}
\subsection{Isométries}
\begin{definition}
    Une \textbf{isométrie} de $E$ (ou \textbf{transformation orthogonale}) est un endomorphisme  $f \in \mathcal{L}(E) := \mathcal{L}(E, E)$ préservant le produit vectoriel, i.e:
     \[
         \scalair{f(x), f(y)} = \scalair{x, y} \quad \forall x, y \in E
    \] 
\end{definition}
\begin{definition}
    Soient $x, y \in E$ deux vecteurs non nuls. On a, d'après l'inégalité de Cauchy-Schwarz (voir lemma \ref{lemma:inegalite-cauchy-schwarz}):
    \[
        \frac{| \scalair{x, y} |}{\|x\| \cdot \|y\|} \le 1
    \] 
    Alors, il existe un et un seul $\theta \in [0, \pi]$ tel que:
     \begin{equation}
        \cos \theta = \frac{ \scalair{x, y}}{\|x\| \cdot \|y\|} 
    \end{equation}
    $\theta$ est dit \textbf{angle} (non-orienté) entre les vecteurs  $x$ et  $y$.
\end{definition}
\begin{prop}\label{prop:isometrie-reserve-norme}
   Si  $f$ est une isométrie de  $E$, donc, on a:
   \[
   \|f(x)\| = \|x\| \quad \forall x \in E
   \] 
\end{prop}
\begin{preuve}
   Supposons que $f$ est une isométrie de  $E$. Soit  $x, y \in E$. Par définition:  $\scalair{f(x), f(y)} = \scalair{x, y}$, donc, posons $y := x$, alors, on a:
   \begin{align*}
       &\underbrace{\scalair{f(x), f(x)}}_{\|f(x)\|^2} = \underbrace{\scalair{x, x}}_{\|x\|^2}\\
       \iff &\|f(x)\|^2 = \|x\|^2\\
       \iff &\|f(x)\| = \|x\|
   \end{align*}
\end{preuve}
\begin{prop}
   Soit $f$ une isométrie dans  $E$, alors:
   \begin{enumerate}
       \item $f$ est bijective
       \item  $f$ présérve la distance euclidienne et les angles
   \end{enumerate}
\end{prop}
\begin{preuve}
   Soit $f$ une isométrie dans  $E$ et deux vecteurs  $u, v \in E$ 
   \begin{enumerate}
       \item  
           \begin{align*}
               \|f(u) - f(v)\| = \sqrt{\scalair{f(u), f(v)}} = \sqrt{\scalair{u, v}} = \|u - v\| 
           \end{align*}
       \item Soit $\theta_1$ angle entre  $f(u)$ et  $f(v)$ et $\theta_2$ angle entre  $u$ et  $v$, donc:
            \[
                \cos \theta_1 := \frac{\scalair{f(u), f(v)}}{\|f(u)\| \cdot \|f(v)\|}
           \] 
           \[
                \cos \theta_2 := \frac{\scalair{u, v}}{\|u\| \cdot \|v\|}
           \] 
           Par définition, $\scalair{f(u), f(v)} = \scalair{u, v}$, d'après proposition \ref{prop:isometrie-reserve-norme},  $\forall x, \|f(x)\| = \|x\|$, donc:
           \[
                \cos \theta_1 := \frac{\scalair{f(u), f(v)}}{\|f(u)\| \cdot \|f(v)\|} = \frac{\scalair{u, v}}{\|u\| \cdot \|v\|} = \cos \theta_2
           \] 
   \end{enumerate}
\end{preuve}
