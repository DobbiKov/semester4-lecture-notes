\section{Isométries et Adjoints}
\subsection{Isométries}
\begin{definition}
    Une \textbf{isométrie} de $E$ (ou \textbf{transformation orthogonale}) est un endomorphisme  $f \in \mathcal{L}(E) := \mathcal{L}(E, E)$ préservant le produit vectoriel, i.e:
     \[
         \scalair{f(x), f(y)} = \scalair{x, y} \quad \forall x, y \in E
    \] 
\end{definition}
\begin{definition}
    Soient $x, y \in E$ deux vecteurs non nuls. On a, d'après l'inégalité de Cauchy-Schwarz (voir lemma \ref{lemma:inegalite-cauchy-schwarz}):
    \[
        \frac{| \scalair{x, y} |}{\|x\| \cdot \|y\|} \le 1
    \] 
    Alors, il existe un et un seul $\theta \in [0, \pi]$ tel que:
     \begin{equation}
        \cos \theta = \frac{ \scalair{x, y}}{\|x\| \cdot \|y\|} 
    \end{equation}
    $\theta$ est dit \textbf{angle} (non-orienté) entre les vecteurs  $x$ et  $y$.
\end{definition}
\begin{prop}\label{prop:isometrie-reserve-norme}
   Si  $f$ est une isométrie de  $E$, donc, on a:
   \[
   \|f(x)\| = \|x\| \quad \forall x \in E
   \] 
\end{prop}
\begin{preuve}
   Supposons que $f$ est une isométrie de  $E$. Soit  $x, y \in E$. Par définition:  $\scalair{f(x), f(y)} = \scalair{x, y}$, donc, posons $y := x$, alors, on a:
   \begin{align*}
       &\underbrace{\scalair{f(x), f(x)}}_{\|f(x)\|^2} = \underbrace{\scalair{x, x}}_{\|x\|^2}\\
       \iff &\|f(x)\|^2 = \|x\|^2\\
       \iff &\|f(x)\| = \|x\|
   \end{align*}
\end{preuve}
\begin{prop}\label{prop:isometrie-bijective}
   Soit $f$ une isométrie dans  $E$, alors:
   \begin{enumerate}
       \item $f$ est bijective 
       \item  $f$ présérve la distance euclidienne et les angles
   \end{enumerate}
\end{prop}
\begin{preuve}
   Soit $f$ une isométrie dans  $E$ et deux vecteurs  $u, v \in E$ 
   \begin{enumerate}
       \item  
           \begin{align*}
               \|f(u) - f(v)\| = \sqrt{\scalair{f(u), f(v)}} = \sqrt{\scalair{u, v}} = \|u - v\| 
           \end{align*}
       \item Soit $\theta_1$ angle entre  $f(u)$ et  $f(v)$ et $\theta_2$ angle entre  $u$ et  $v$, donc:
            \[
                \cos \theta_1 := \frac{\scalair{f(u), f(v)}}{\|f(u)\| \cdot \|f(v)\|}
           \] 
           \[
                \cos \theta_2 := \frac{\scalair{u, v}}{\|u\| \cdot \|v\|}
           \] 
           Par définition, $\scalair{f(u), f(v)} = \scalair{u, v}$, d'après proposition \ref{prop:isometrie-reserve-norme},  $\forall x, \|f(x)\| = \|x\|$, donc:
           \[
                \cos \theta_1 := \frac{\scalair{f(u), f(v)}}{\|f(u)\| \cdot \|f(v)\|} = \frac{\scalair{u, v}}{\|u\| \cdot \|v\|} = \cos \theta_2
           \] 
   \end{enumerate}
\end{preuve}
\begin{definition}
    Soit $F$ un sous-espace vectoriel de  $E$, donc  $E = F \oplus F^{\perp}$ d'où  $\forall v \in E, \exists v_1 \in F, v_2 \in F^{\perp}$ tel que $v = v_1 + v_2$. On pose:
    \[
    s_F(v) = v_1 - v_2
    \] 
    et on appelle $s_F$ une symétrie orthogonale d'axe F.
\end{definition}
\begin{figure}[H]
    \centering
    \incfig{symetrie-orthogonale-axe-f}
    \caption{Symétrie orthogonale d'axe $F$}
    \label{fig:symetrie-orthogonale-axe-f}
\end{figure}
\begin{prop}
   La symétrie orthogonale est une isométrie. 
\end{prop}
\begin{proof}
   TODO ou pas besoin 
\end{proof}
\begin{prop}
   $f$ est une isométrie si et seulement si elle transforme toute base orthonormée en une base orthonormée. 
\end{prop}
\begin{preuve}
    Soit $f$ une isométrie, alors elle transforme toute base en une base car  $f$ est bijective par la prop. \ref{prop:isometrie-bijective}. 
    \begin{itemize}
        \item ($\implies$) Soit $\{e_i\}$ une base orthonormée, alors, on a:
             \[
                 \scalair{f(e_i), f(e_j)} = \scalair{e_i, e_j} = \delta_{i,j}
            \] 
            Donc, $\{f(e_i)\}$ est une base orthonormée.
        \item ($\impliedby$) Supposons, qu'il existe une base orthonormée $\{e_i\}$ telle que  $\{f(e_i)\}$ est aussi une base orthonormée. De plus, soit  $x = x_1e_1 + \ldots x_ne_n$ et $y = y_1e_1 + \ldots + y_ne_n$ avec $x_i, y_i \in \R$
            \par
            Comme $\{e_i\}$ est orthonormée, alors on a:
             \[
                 \scalair{x, y} = x_1y_1 + \ldots + x_ny_n = \sum_{i=1}^{n} x_iy_i
            \] 
            D'autre part:
            \begin{align*}
                \scalair{f(x), f(y)} &= \scalair{\sum_{i=1}^{n} x_if(e_i), \sum_{i=1}^{n} y_if(e_i)} = \sum_{i,j = 1}^{n} x_iy_j\scalair{f(e_i), f(e_j)}\\
                                     &= \sum_{i,j=1}^{n} x_iy_j\scalair{e_i, e_j} \underset{\text{car } f \text{ isométrie}}{=} = \sum_{i=1}^{n} x_iy_i \underset{\text{car } \{e_i\} \text{ orthonormée}} = \scalair{x, y}
            \end{align*}
            Donc $f$ est une isométrie.
    \end{itemize}
\end{preuve}
\begin{prop}
    Si $\{e_i\}$ est une base orthonormée, $f$ une isométrie et  $A = M(f)_{e_i}$, alors  $A^{T}A = I = AA^{T}$.
\end{prop}
\begin{preuve}
    Pour prouver cela, on va utiliser la proposition \ref{prop:prod-scal-par-matrice}. 
    \par
    Par définition de l'isométrie, on a:
    \begin{align*}
        &\scalair{f(x), f(y)} = \scalair{x, y} \quad \forall x, y \in E\\
        \iff &\underbrace{ (AX)^{T}(AY) }_{\scalair{f(x), f(y)}} = X^TA^TAY = \underbrace{X^TY}_{\scalair{x, y}}\\
        \iff &A^TA = I
    \end{align*}
\end{preuve}
\subsection{Endomorphisme adjoint}
\begin{prop}
   Soit $E$ un espace euclidien et  $f \in End(E)$. Il existe un et un seul endomorphisme  $f^* \in E$ tel que
   \[
       \scalair{f(x), y} = \scalair{x, f^*(y)}, \quad \forall x, y \in E
   \] 
   $f^*$ est dit  \textbf{adjoint} de $f$.
   \par
   Si  $\{e_i\}$ est une base orthonormée et  $A = M(f)_{e_i}$, alors la matrice $A^* = M(f^*)_{e_i}$ est la transposée de $A$, i.e  $A^* = A^T$
\end{prop}
\begin{preuve}
    Encore, pour la preuve, on va utiliser la proposition \ref{prop:prod-scal-par-matrice} qui est très utile, donc je vous conseil maîtriser ce concept.
    \par
    Soit $\{e_i\}$ une base orthonormée de $E$ et notons
     \[
    A = M(f)_{e_i} \quad A^* = M(f^*)_{e_i} \quad X = M(x)_{e_i} \quad Y = M(y)_{e_i}
    \] 
    Comme on est dans une base orthonormée, alors l'énoncé s'ecrit:
    \[
        \underbrace{(AX)^TY}_{\scalair{f(x),y}} = X^TA^TY = \underbrace{X^T(A^*Y)}_{\scalair{x, f^*(y)}} \quad \forall X, Y \in \mathcal{M}_{n, 1}(\R)
    \] 
    ce qui implique que $A^* = A$ et, de plus, démontre l'unicité de tel adjoint.
\end{preuve}
