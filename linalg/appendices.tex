\begin{appendices}
    \chapter{Rappels des concepts d'Algèbre Linéaire} 
    \section{Matrices}
    \subsection{Multiplication des matrices}
    \begin{definition}
        Soit $A \in \mathcal{M}_{p, n}(\R)$ et $B \in \mathcal{M}_{n, q}(\R)$ tels que $A = (a_{j, i})$ et $B = (b_{m, k})$, alors:
         \[
        AB = C = (c_{j, k} = \sum_{i=1}^{n} a_{j, i}b_{i, k})
        \] 
    \end{definition}
    \subsection{La trace}
    \begin{definition}
        La trace de la \( n \times n \) matrice carée \( A \), notée \( \text{tr}(A) \), est la somme des éléments diagonales

        \[
            \text{tr}(A) = a_{11} + a_{22} + \dots + a_{nn} = \sum_{i=1}^{n} a_{ii}
        \]

        où \( a_{ii} \) sont des éléments diagonales de la matrice \( A \). 
    \end{definition}

    \begin{property} de la trace.
       \begin{itemize}
           \item Linéarité:
               \[
                   \text{tr}(A + B) = \text{tr}(A) + \text{tr}(B)
               \]

               \[
                   \text{tr}(cA) = c \text{tr}(A), \quad c \in \mathbb{R} \text{ (ou } \mathbb{C} \text{)}
               \]
            \item  Transposé:
                \[
                    \text{tr}(A) = \text{tr}(A^T)
                \] 
            \item Multiplication des matrices:
                \[
                    \text{tr}(AB) = \text{tr}(BA), \quad \text{(si } A \text{ et } B \text{ sont de taille } n \times n)
                \]

                Cependant, la trace n'est pas distributive sur la multiplication :

                \[
                    \text{tr}(A B C) \neq \text{tr}(A) \text{tr}(B C)
                \]
            \item Valeurs propres:
                \[
                    \text{tr}(A) = \sum_{i=1}^{n} \lambda_i
                \]

                où \( \lambda_i \) sont les valeurs propres de \( A \). Cela fait de la trace un outil important en analyse spectrale.

            \item Trace de la Matrice Identité

                \[
                    \text{tr}(I_n) = n
                \]

                puisque tous les éléments diagonaux valent 1.
       \end{itemize} 
    \end{property}
    \begin{eg}
        Pour
        \[
            A = \begin{bmatrix} 3 & 2 & 1 \\ 4 & 5 & 6 \\ 7 & 8 & 9 \end{bmatrix}
        \]

        la trace est :

        \[
            \text{tr}(A) = 3 + 5 + 9 = 17
        \] 
    \end{eg}
    \begin{eg}
        Si

        \[
            B = \begin{bmatrix} 2 & 1 \\ 0 & 3 \end{bmatrix}, \quad C = \begin{bmatrix} 4 & 2 \\ 1 & 5 \end{bmatrix}
        \]

        alors

        \[
            \text{tr}(B + C) = \text{tr} \begin{bmatrix} 6 & 3 \\ 1 & 8 \end{bmatrix} = 6 + 8 = 14
        \]

        ce qui correspond bien à 

        \[
            \text{tr}(B) + \text{tr}(C) = (2+3) + (4+5) = 14
        \]

        confirmant ainsi la linéarité. 
    \end{eg}
\end{appendices}
