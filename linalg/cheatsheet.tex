\documentclass[a4paper]{article}

\usepackage[utf8]{inputenc}
\usepackage[T1]{fontenc}
\usepackage{textcomp}
\usepackage[english]{babel}
\usepackage{amsmath, amssymb, amsthm}


% figure support
\usepackage{import}
\usepackage{xifthen}
\pdfminorversion=7
\usepackage{pdfpages}
\usepackage{transparent}
\usepackage{hyperref}
\usepackage[margin=0.8in]{geometry}

\usepackage{setspace}
\setlength{\parindent}{0in}

\newcommand{\incfig}[1]{%
    \def\svgwidth{\columnwidth}
    \import{./figures/}{#1.pdf_tex}
}

\pdfsuppresswarningpagegroup=1

\newcommand{\N}{\mathbb{N}}
\newcommand{\R}{\mathbb{R}}
\newcommand{\Z}{\mathbb{Z}}
\newcommand{\Q}{\mathbb{Q}}
\newcommand{\scalair}[1]{\left\langle #1 \right\rangle}

\newtheorem{theorem}{Théorème}[section]
\newtheorem{definition}{Définition}[section]
\newtheorem{eg}{Exemple}[section]
\newtheorem{prop}{Proposition}[section]
\newtheorem{property}{Propriété(s)}[section]
\newtheorem*{notation}{Notation}
\newtheorem*{remark}{Remarque}

\author{Yehor KOROTENKO}
\title{CheatSheet pour l'Algèbre Linéaire}

\begin{document}
\maketitle
\section{Éspaces euclidiens}
   \begin{prop}
      Endomorphisme $f: E \to E$ un drapeau invariant (i.e $f(E_i) \subset E_i$) $\iff$ $Mat(f)$ triangulaire supérieure 
   \end{prop} 
\subsection{Produits scalairs et normes}
\begin{definition}
    Une forme bilinéaire sur $E$ (\textbf{produit scalair}) un espace euclidien est une application:
    \begin{align*}
        f: E \times E &\longrightarrow \R \\
        (u, v) &\longmapsto f(u, v) 
    \end{align*}
    qui vérifie ces propriétés:
    \begin{enumerate}
        \item \textbf{Bilinéarité}:
            \begin{enumerate}
                \item $f(u + \lambda v, w) = B(u, w) + \lambda B(v, w)$ avec $u, v, w \in E$ et  $\lambda \in \R$
                \item $f(u, v + \lambda w) = B(u, v) + \lambda B(v, w)$ avec $u, v, w \in E$ et $\lambda \in \R$
            \end{enumerate}
        \item \textbf{Symétrie}: $B(u, v) = B(v, u) \qquad \forall u, v \in E$ 
        \item \textbf{Définie positive}: $\forall u \in E, B(u, u) \ge 0$
        \item \textbf{Définie}: $B(u, u) = 0 \iff u = 0$
    \end{enumerate}
\end{definition}
\begin{remark}
    Le produit vectoriel est noté: $\scalair{., .}$ 
\end{remark}
\begin{definition}
    La norme $\forall X \in E$:
    \[
        \|X\| = \sqrt{\scalair{X, X}} 
    \] 
\end{definition}
\begin{prop}
   Les formules utiles: (pour $X, Y \in E$)
   \begin{enumerate}
       \item $|\scalair{X, Y}| \le \|X\| \cdot \|Y\|$
       \item $\|X + Y\|^2 = \|X\|^2 + 2\scalair{X, Y} + \|Y\|^2$ 
       \item $\|X + Y\|^2 + \|X - Y\|^2 = 2\left( \|X\|^2 + \|Y\|^2 \right) $
       \item $\scalair{X, Y} = \frac{1}{4}\left( \|X + Y\|^2 - \|X - Y\|^2 \right) $
   \end{enumerate}
\end{prop}
\subsection{Orthogonalité}
\begin{definition}
    $u, v \in E$ sont \textbf{orthogonaux} si $\scalair{u, v} = 0$ et on les notes  $u \perp v$
\end{definition}
\begin{definition}
    \textbf{Orthogonale de $A$}:
    \[
        A^{\perp} = \{ u \in E \mid \scalair{u, v} = 0 \quad \forall v \in A \}
    \] 
    aussi connu comme  \textbf{complement orthogonale}.
\end{definition}
\begin{prop}
   Si $E$ est un espace euclidien et  $A \subset E$ son sous-espace vectoriel, alors:
   \[
       E = A \oplus A^{\perp}
   \] 
   i.e tout vecteur $x \in E$ peut s'écrit comme  $x = e + e'$ où  $e \in A$ et  $e' \in A'$.
\end{prop}
\begin{prop}
   Si $f$ est une projection orthogonale sur  $F \subset E$, alors:
   \[
   f(f(x)) = f(x) \quad \forall x \in E
   \] 
\end{prop}
\begin{definition}
    La \textbf{projection orthogonale} sur un sous-espace $A \subset E$ est une application:
    \begin{align*}
        p_F: E &\longrightarrow F \\
        x &\longmapsto p_F(x = e + e') = e
    \end{align*}
\end{definition}
\begin{prop}
    \textbf{La distance} d'un vecteur $x$ à un sous-espace  $F$ est:
    \[
    \|x - p_F(x)\|
    \] 
\end{prop}
\begin{definition}
    Une \textbf{isométrie} de $E$ est un endomorphisme tel que:
     \[
         \forall x, y \in E, \quad \scalair{f(x), f(y)} = \scalair{x, y}
    \]
    de plus,
    \[
    \forall x \in E, \quad \|f(x)\| = \|x\|
    \] 
\end{definition}
\end{document}
