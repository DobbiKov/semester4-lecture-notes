\section{Bases orthonormales}
Soit $(E, <,>)$ un espace euclidien et  $F \subset E$ un sous-espace vectoriel ($dim(F) < \infty$) car $dim(E) < \infty$.
\begin{note}
    \[
        F^{\perp} := \{x \in E \mid <X, Z> = 0 \, \forall z \in F\} 
    \] 
    l'orthogonale de $F$.
\end{note}
\begin{theorem}
    On a $E = F \oplus F^{\perp}$.\\
    En particulier,  $dim(F^{\perp}) = dim(E) - dim(F)$ et  $F = (F^{\perp})^{\perp}$
\end{theorem}
\begin{preuve}
   On doit montrer que:
   \begin{enumerate}
       \item $F \cap F^{\perp} = \O$
       \item $E = F + F^{\perp}$ i.e  $\forall x \in E, \exists x' \in F, \, x'' \in F^{\perp}$ tq $x = x' + x''$ 
   \end{enumerate}
   \begin{enumerate}
       \item Soit $x \in F \cap F^{\perp}$\\
            $\implies$ $<X, Z> = 0 \, \forall Z \in F$ car $x \in F \implies <X, X> = 0 \implies x = 0 (<,> \text{ est définie})$ 
        \item Soit $x \in E$. Considérons  $f_x \in E^{*}$, i.e  $f_x: E \to \R, y \mapsto <x, y>$ et $f := f_{x|F}: F \to \R \implies f \in E^{*}$
            Lemme de Riesz $\implies$ $\exists! x' \in F$ tq $f = f_{x'}: F \to \R, z \mapsto <x', z>$\\
            $\implies f_{x}(z) = f_{x'}(z) = f(z)\, \forall z \in F$ (Attention: pas l'égalité pour tout $z$ dans  $E$)\\
            Posons $x'' := x - x'$, i.e  $x = x' + x'' \in F$. Montrons  $x'' \in  F^{\perp}$.\\
            Si $z \in F$,  $<x'', z> = <x - x', z> = <x, z> - <x', z> = 0$. Donc $x'' \in F^{\perp}$ et  $E = F \oplus F^{\perp}$ ($dim(E) = dim(F) + dim(F^{\perp})$) \\
            $F \subseteq (F^{\perp})^{\perp}$ car $<x, z> = 0 \, \forall x \in F \, \forall z \in F^{\perp}$
            \[
                dim(F) = dim(E) - dim(F^{\perp})
            \] 
            car $E = G \oplus G^{\perp}$, donc  $dim(G) = dim(E) - dim(G^{\perp})$ pour  $G = F^{\perp}, \, dim(F^{\perp}) = dim(G)$
   \end{enumerate}
\end{preuve}
\begin{definition}
    Soit $E$ un espace vectoriel muni d'un produit scalaire  $<,>$
     \begin{itemize}
         \item Une famille $(v_i)_{i \ge 0}$ de vecteurs de $E$ est dite \underline{orthogonale} si pour $i \neq j$ on a $<v_i, v_j> = 0$ i.e  $v_i \perp v_j$
         \item Une famille orthogonale de  $E$ est une famille orthogonale  $(v_i)_{i \ge  0}$ tq de plus $\|v_i\| = 1$ pour  $i \ge 0$
    \end{itemize}
\end{definition}
\begin{eg}
   \begin{enumerate}
       \item $E = \R^{n}$ muni du produit scalaire canonique. La base canonique $(e_1, \ldots, e_n)$ est orthogonale car 
           \[
           <e_i, e_j> = \begin{cases}
               1 \, i = j\\
               0 \, i \neq j
           \end{cases}
           \] 
       \item Dans $E = \mathcal{C}^{0}([-1, 1], \R)$ muni de $<f,g> = \int_{-1}^{1} f(t)g(t)\,d{t}$. La famille $(\cos(t), \sin(t))$ est orthogonale. La famille $(1, t^2)$ n'est pas orthogonale:
            \[
                <1, t^2> = \int_{-1}^{1} 1 t^2 \, d{t} = \frac{2}{3} \neq  0 
           \] 
   \end{enumerate} 
\end{eg}
\begin{prop}
    Une famille orthogonale constituée de vecteurs \underline{non-nuls} est libre. En particulier, une famille orthonormale est libre. 
\end{prop}
\begin{preuve}
    Suppososns $(v_1, \ldots, v_n)$ orthogonale avec $v_i \neq 0 \, \forall i = 1, \ldots, n$\\
    si $\sum_{j=1}^{n} \underset{\in \R}{\alpha_iv_i} = 0$, alors  
    \[
        \forall i \in \{1, \ldots, n\} 0 = <v_i, \sum_{j=1}^{n} \alpha_jv_j> = \sum_{j=1}^{n}\alpha_j <v_i, v_j> = \alpha_i \underset{\neq 0}{\|v_i\|^2}
    \] 
    Donc $\alpha_i = 0 \, \forall i = 1, \ldots, n$.\\
    Si $(v_1, \ldots, v_n)$ est orthonormale, alors $\|v_i\| = 1$. Donc  $v_i \neq 0, \, \forall i = 1, \ldots, n$.
\end{preuve}
\begin{definition}
    $(E, <,>)$ espace euclidien. Une famille  $B = (e_1, \ldots, e_n)$ est une base orthonormale (où BON) si elle est une base et famille orthonormale.
\end{definition}
\begin{theorem}
    $(E, <,>)$ espace euclidien. Alors, il admet une BON.
\end{theorem}
\begin{preuve}
   Soit $n := dim(E)$. Soit  $(e_1, \ldots, e_p)$ une famille orthogonale (du point de vue du cardinal $p$) tq  $e_i \neq 0 \, \forall i = 1, \ldots, p$.\\
   Supposons par l'absurde que $p < n$. Posons  $F = Vect(e_1, \ldots, e_p)$. Alors, $E = F \oplus F^{\perp}$ et  $dim(F) \le p < n$. Donc $F^{\perp} \neq  \{0\}$. Soit $x \in F^{\perp}, \, x \neq 0$. Alors, $(e_1, \ldots, e_p, x)$ est orthogonale de cardinale $> p$. Donc,  $p = n$ et  $(e_1, \ldots, e_n)$ est une base de $E$. Pour avoir une famille orthonormale  $(e_1', \ldots, e_n')$ il suffit de prendre $e_i' = \frac{1}{\|e_i\|}e_i \, \forall i = 1, \ldots, n$.
\end{preuve}
\begin{prop}
    Soit $(E, \scalar{}{})$ un espace euclidien et soit  $(e_1, \ldots, e_n)$ une BON de $E$. Si  $x \in E$, on a:
   \[
       x = \sum_{i=1}^{n} \scalar{x}{e_i}e_i
   \] 
Autrement dit, le réél $\scalar{x}{e_i}$ est la  $i^{\text{ème}}$ coordonnée de $x$ dans la base  $(e_1, \ldots, e_n)$.
\end{prop}
\begin{intuition}
    L'orthonormalité de la base nous simplifie la vie. Mais avant, petite introduction. Soit un e.v $E = \R^2$ et la base $(e_1, e_2) = (\begin{pmatrix} 1 \\ 0 \end{pmatrix}, \begin{pmatrix} 0\\ 1 \end{pmatrix})$. Soit un vecteur $\vec{v} = (2, 3)$ :
    \begin{center}
        \begin{tikzpicture}
            \begin{axis}[
                scale=1,
                axis lines=middle,        % Draw axes in the middle
                xmin=-2, xmax=4,          % X-axis range
                ymin=-2, ymax=4,          % Y-axis range
                xlabel={$x$},             % Label for X-axis
                ylabel={$y$},             % Label for Y-axis
                xtick={-2,-1,0,1,2,3,4},% X-axis ticks
                ytick={-2,-1,0,1,2,3,4},% Y-axis ticks
                ]
            \draw[color=red, ->, thick] (0, 0) -- node[below]{$e_1$}(1, 0);
            \draw[color=blue, ->, thick] (0, 0) -- node[left]{$e_2$}(0, 1);
            \draw[color=green, ->] (0, 0) --node[above]{$\vec{v}$} (2, 3);

            \draw[color=gray, ->, thick] (1, 0) -- node[below]{$e_1$}(2, 0);
            \draw[color=gray, ->, thick] (2, 0) -- node[left]{$e_2$}(2, 1);
            \draw[color=gray, ->, thick] (2, 1) -- node[left]{$e_2$}(2, 2);
            \draw[color=gray, ->, thick] (2, 2) -- node[left]{$e_2$}(2, 3);

            \node[right, above] (_) at (2, 3){$(2, 3)$};
        \end{axis} 
        \end{tikzpicture}
    \end{center}
    Donc, on peut écrire $\vec{v} = \vec{(2, 3)} = 2 \cdot \vec{e_1} + 3 \cdot \vec{e_2}$. Les $x$ et  $y$ (les coordonnées de $v$) nous donnes combien de parties de chaque vecteur de bases (le nombre peut être $\in \R$) et prendre leurs sommes, pour obtenir $\vec{v}$. (Le plus simple: combien on doit aller à gauche et en haut).
    \par
    Dans la base orthonormale $\scalair{v, e_i}$ nous donne combien on prend d'un vecteur $e_i$ pour faire le vecteur  $\vec{v}$ et  $\vec{e_i}$ donne la direction. D'où $\scalair{v, e_1}$ équivaut à $2$, et  $\scalair{v, e_2}$ à  $3$, puis: 
   \[
       \vec{v} = \underbrace{\scalair{v, e_1}}_{= 2} \cdot \vec{e_1} + \underbrace{\scalair{v, e_2}}_{= 3} \cdot \vec{e_2}
   \]  
\end{intuition}
\begin{preuve}
    Posons $y := \sum_{i=1}^{n} \scalar{x}{e_i}e_i$ . Alors, 
   \begin{align*}
       &\forall j = 1, \ldots, n,\\
       &\scalar{x - y}{e_j}\\ 
       = &\scalar{x}{e_j} - \scalar{y}{e_j}\\ 
       = &\scalar{x}{e_j} - \scalar{\sum_{i=1}^{n} \scalar{x}{e_i}e_i}{e_j}\\ 
       = &\scalar{x}{e_j} - \underbrace{ \sum_{i=1}^{n} \scalar{x}{e_i} }_{\substack{\text{moved out}\\ \text{like constant}}}\scalar{e_i}{e_j}\\ 
       = &\scalar{x}{e_j}\\ 
       -& \left(\scalar{x}{e_1}\underbrace{ \scalar{e_1}{e_j} }_{= 0} + \ldots + \scalar{x}{e_{j-1}}\underbrace{\scalar{e_{j-1}}{e_j}}_{= 0} + \scalar{x}{e_{j}}\underbrace{ \scalar{e_{j}}{e_j} }_{= 1} + \scalar{x}{e_{j+1}}\underbrace{ \scalar{e_{j+1}}{e_j} }_{= 0} + \ldots + \scalar{x}{e_{n}}\underbrace{ \scalar{e_{n}}{e_j} }_{= 0}\right)\\
        &\text{(} \forall i \neq j, \, \scalar{e_i}{e_j} = 0 \text{ car un produit scalaire des vecteur orthogonaux)}\\ 
        &\text{(} \forall j \, \scalar{e_j}{e_j} = 1 \text{ car un produit scalaire de même vecteur)}\\
       = &\scalar{x}{e_j} - \scalar{x}{e_j}\underset{= 1}{\scalar{e_j}{e_j}} = 0
   \end{align*}
   Donc, $x - y \in Vect(e_1, \ldots, e_n)^{\perp} = E^{\perp} = \{0\}$. Donc $x = y$
\end{preuve}
\begin{corollary}
    $\forall x \in E, \, \|x\|^2 = \sum_{i=1}^{n} \scalar{x}{e_i}^2$ 
\end{corollary}
\begin{preuve}
    Si $x = \sum_{i=1}^{n} \scalar{x}{e_i}e_i = \sum_{i=1}^{n} x_ie_i$ donc
    \[
        \|x\|^2 = \scalar{\sum_{i=1}^{n} x_ie_i}{\sum_{j=1}^{n} x_je_j} = \sum_{i,j=1}^{n} x_ix_j\scalar{e_i}{e_j} = \sum_{i=1}^{n} x_i^2
    \] 
\end{preuve}
\begin{prop} Soient $(E, <,>)$ un espace euclidien et $\epsilon = (e_1, \ldots, e_n)$ une BON. Soient $f \in \mathcal{L}(E, E)$ et $A = (a_{i,j})_{1 \le i,j \le n}$ la matrice représentative de $f$ dans  $\epsilon$, i.e,  $A = Mat_{\epsilon}(f)$ 
    \[
        a_{i,j} = <f(e_i), e_j> \, \forall i,j = 1, \ldots, n
    \] 
\end{prop}
\begin{preuve}
   $A$ est la matrice dont les colonnes sont les vecteurs  $f(e_j)$ écrits dans la base $\epsilon$:
    \[
        A = (f(e_1) | \ldots | f(e_n))\quad f(e_j) = \begin{pmatrix} a_{1,j}\\ \ldots\\ a_{n, j} \end{pmatrix} 
   \] 
   Car $\forall v \in E, \, v = c_1e_1 + \ldots c_ne_n$ donc $f(v) = c_1f(e_1) + \ldots c_nf(e_n)$ par la linéarité, donc il nous reste à étudier chaque $f(e_j)$
   \begin{align*}
       f(e_j) = a_{1, j}e_1 + \ldots a_{n, j}e_n \implies\\
       \langle f(e_j), e_i \rangle = \left\langle \sum_{k=1}^n a_{k,j} e_k, e_i \right\rangle = \sum_{k=1}^{n} a_{k,j}\scalar{e_k}{e_i} = a_{k, j}
   \end{align*}
   car $\scalar{e_k}{e_j} = \begin{cases}
       0 \text{ si } k \neq j\\
       1 \text{ si } k = j
   \end{cases}$
   Donc:
   \[
       a_{i, j} = <f(e_j), e_i>
   \] 
\end{preuve}

\section{Projections orthogonales}
Soit $(E, <,>)$ un espace euclidien,  $F \subseteq E$ un sous-espace vectoriel. Alors,  $E = F \oplus F^{\perp}$. Donc $\forall x \in E$ s'ecrit 
\[
x = \underset{\in F}{x_F} + \underset{\in F^{\perp}}{x_{F^{\perp}}}
\] 
\begin{definition}
    La \textbf{projection orthogonale} de $E$ dans  $F$ est la projection  $p_F$ de  $E$ sur  $F$ parallèlement  à $F^{\perp}$, i.e
    \begin{align*}
        p_F: E = F \oplus F^{\perp} &\longrightarrow F \\
        x = x_F + x_{F^{\perp}} &\longmapsto p_F(x = x_F + x_{F^{\perp}}) = x_F
    .\end{align*}
\end{definition}
\begin{remark}
   \begin{enumerate}
       \item $p_F$ est linéaire
       \item  $\forall x \in E \, p_{F}(x)$ est complétement caractérisé par la propriété suivante:\\
           Soit $y \in E$, alors
            \[
                y = p_F(x) \iff \left( \underset{\implies y = x_F}{y \in F \text{ et } x - y} \in F^{\perp} \right) 
           \] 
           En particulier $<p_F(x), x - P_F(x)> \,= 0$. Alors, si $(v_1, \ldots, v_R)$ est une BON de $F$, on a:
            \[
           \forall x \in E, \, p_F(x) = \sum_{i=1}^{k} <x, v_i>v_i
           \] 
           En effet, il suffit de vérfier que le vecteur $y = \sum_{i=1}^{k} <x, v_i>v_i$ vérfie:
           \[
               y \in F \text{ et } x - y \in F^{\perp}
           \] 
   \end{enumerate} 
\end{remark}
\begin{figure}[H]
   \centering 
\begin{tikzpicture}

% Draw the plane
\fill[gray!20] (-2,-1) -- (2,-1) -- (3,1) -- (-1,1) -- cycle;

% Draw the vectors
\draw[->, thick, black] (0,0) -- (2,2.2) node[anchor=south east] {\large $\mathbf{x}$};

\node[anchor=north, blue] (_) at ($(0,0)!0.5!(2,0)$) {\large $\text{proj}_\mathbf{F} \mathbf{x}$};
\node[anchor=west, blue] (_) at ($(2,0)!0.5!(2,2.2)$) {\large $\text{proj}_\mathbf{F^{\perp}} \mathbf{x}$};
% Add the labels for w and w perpendicular
\draw[->, thick, blue] (0,0) -- (2,0) ;
\draw[thick, black] (2,0) -- (2,3) node[anchor=west] {\large $\mathbf{F}^\perp$};
\draw[->, thick, blue] (2,0) -- (2,2.2);
\node[anchor=north west] (_) at (1.5, -0.5) {\large $\mathbf{F}$};
% Add the right angle symbol

\end{tikzpicture}
\caption{Projection}
\label{pic:projection}
\end{figure}
\begin{prop}
   Soit $x \in E$. Alors,
   \[
       \|x - p_F(x)\| = inf\{\|x - y\| \mid y \in F\}
   \] 
   i.e $\|x - p_F(x)\|$ est la distance de  $x$ à  $F$.\\
   Voir Figure~\ref{pic:projection}
\end{prop}
\begin{preuve}
   Comme $p_F(x) \in F$ il suffit de prouver que, si  $y \in F$, alors 
   \[
   \|x - p_F(x)\| \le \|x - y\|
   \] 
   Mais, $\underset{(x - p_F(x)) + (P_F(x) - y)}{\|x - y\|^2} = \|x - P_F(x)\|^2 + 2\overbrace{<\overset{\in F^{\perp}}{x - P_F(x)}, \overset{\in F}{P_F(x) - y}>}^{= 0} + \underbrace{\|P_F(x) - y\|^2}_{\ge 0} \ge \|x - P_F(x)\|^2$
\end{preuve}
\begin{theorem}Gram-Shmidt\\
    Soit $E$ un espace vectoriel muni d'un produit scalaire  $<,>$. Soit  $(v_1, \ldots, v_n)$ une famille libre d'élement $\in E$. Alors,  il existe une famille $(w_1, \ldots, w_n)$ orthogonale tq 
    \[
        \forall i = 1, \ldots, n \quad Vect(v_1, \ldots, v_i) = Vect(w_1, \ldots, w_i)
    \] 
\end{theorem}
\begin{preuve}
   Récurrence sur $i$\\ 
   \begin{itemize}
       \item $i = 1$:  $w_1 := v_1$ suffit
       \item $i \ge 1$: Supposons $(w_1, \ldots, w_i)$ construits. 
           Posons $F_i = Vect(w_1, \ldots, w_i) = Vect(v_1, \ldots, v_i)$. 
           Alors on prend $w_{i+1} := v_{i+1} - P_{F_i}(v_{i+1})$. 
           Donc,  $w_{i+1} \in F_i^{\perp}$ (par caractérisation de $P_{F_i}$) et $(w_1, \ldots, w_{i+1})$ est orthogonale. On note $P_{F_i}(v_{i+1}) \in F_i$, donc
            \[
                Vect(w_1, \ldots, w_i, w_{i+1}) \underset{w_{i+1} = v_{i+1} - P_{F_i}(v_{i+1})} = Vect(w_1, \ldots, w_i, v_{i+1}) = Vect(v_1, \ldots, v_i, v_{i+1})
           \] 
   \end{itemize}
\end{preuve}
\begin{remark}
   La preuve donne une récette concrète pour construir une BON.\\ 
   Soit $(E, <,>)$ un espace euclidien.  $(v_1, \ldots, v_n)$ base de $E$.\\
   Le but: construit une base  $(w_1', \ldots w_n')$ orthogonale de $E$ avec  $Vect(v_1, \ldots, v_i) = Vect(w_1, \ldots, w_i) \quad \forall i = 1, \ldots, n$\\
   Posons:
   \begin{enumerate}
       \item $w_1' := v_1$
       \item $w_{i+1}' = \sum_{j=1}^{i} \frac{<v_{i+1}, w_j'>}{<w_j', w_j'>}w_j'$. \\
           Alors, $(w_1, \ldots, w_n)$ avec $w_i = \frac{1}{\|w_i'\|}w_i'$ est une BON.
   \end{enumerate}
\end{remark}
