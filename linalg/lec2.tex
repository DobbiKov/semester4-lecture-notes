\section{Bases orthonormales}
Soit $(E, \scalair{,})$ un espace euclidien et  $F \subset E$ un sous-espace vectoriel ($dim(F) < \infty$) car $dim(E) < \infty$.
\begin{note}
    \[
        F^{\perp} := \{x \in E \mid \scalair{X, Z} = 0 \, \forall z \in F\} 
    \] 
    l'orthogonale de $F$.
\end{note}
\begin{theorem}
    On a $E = F \oplus F^{\perp}$.\\
    En particulier,  $dim(F^{\perp}) = dim(E) - dim(F)$ et  $F = (F^{\perp})^{\perp}$
\end{theorem}
\begin{preuve}
   On doit montrer que:
   \begin{enumerate}
       \item $F \cap F^{\perp} = \O$
       \item $E = F + F^{\perp}$ i.e  $\forall x \in E, \exists x' \in F, \, x'' \in F^{\perp}$ tq $x = x' + x''$ 
   \end{enumerate}
   \begin{enumerate}
       \item Soit $x \in F \cap F^{\perp}$\\
       $\implies$ $\scalair{X, Z} = 0 \, \forall Z \in F$ car $x \in F \implies \scalair{X, X} = 0 \implies x = 0 (\scalair{,} \text{ est définie})$ 
        \item Soit $x \in E$. Considérons  $f_x \in E^{*}$, i.e  $f_x: E \to \R, y \mapsto \scalair{x, y}$ et $f := f_{x|F}: F \to \R \implies f \in E^{*}$
            Lemme de Riesz $\implies$ $\exists! x' \in F$ tq $f = f_{x'}: F \to \R, z \mapsto \scalair{x', z}$\\
            $\implies f_{x}(z) = f_{x'}(z) = f(z)\, \forall z \in F$ (Attention: pas l'égalité pour tout $z$ dans  $E$)\\
            Posons $x'' := x - x'$, i.e  $x = x' + x'' \in F$. Montrons  $x'' \in  F^{\perp}$.\\
            Si $z \in F$,  $\scalair{x'', z} = \scalair{x - x', z} = \scalair{x, z} - \scalair{x', z} = 0$. Donc $x'' \in F^{\perp}$ et  $E = F \oplus F^{\perp}$ ($dim(E) = dim(F) + dim(F^{\perp})$) \\
            $F \subseteq (F^{\perp})^{\perp}$ car $\scalair{x, z} = 0 \, \forall x \in F \, \forall z \in F^{\perp}$
            \[
                dim(F) = dim(E) - dim(F^{\perp})
            \] 
            car $E = G \oplus G^{\perp}$, donc  $dim(G) = dim(E) - dim(G^{\perp})$ pour  $G = F^{\perp}, \, dim(F^{\perp}) = dim(G)$
   \end{enumerate}
\end{preuve}
\begin{definition}
    Soit $E$ un espace vectoriel muni d'un produit scalaire  $\scalair{,}$
     \begin{itemize}
         \item Une famille $(v_i)_{i \ge 0}$ de vecteurs de $E$ est dite \underline{orthogonale} si pour $i \neq j$ on a $\scalair{v_i, v_j} = 0$ i.e  $v_i \perp v_j$
         \item Une famille orthogonale de  $E$ est une famille orthogonale  $(v_i)_{i \ge  0}$ tq de plus $\|v_i\| = 1$ pour  $i \ge 0$
    \end{itemize}
\end{definition}
\begin{eg}
   \begin{enumerate}
       \item $E = \R^{n}$ muni du produit scalaire canonique. La base canonique $(e_1, \ldots, e_n)$ est orthogonale car 
           \[
           \scalair{e_i, e_j} = \begin{cases}
               1 \, i = j\\
               0 \, i \neq j
           \end{cases}
           \] 
       \item Dans $E = \mathcal{C}^{0}([-1, 1], \R)$ muni de $\scalair{f,g} = \int_{-1}^{1} f(t)g(t)\,d{t}$. La famille $(\cos(t), \sin(t))$ est orthogonale. La famille $(1, t^2)$ n'est pas orthogonale:
            \[
                \scalair{1, t^2} = \int_{-1}^{1} 1 t^2 \, d{t} = \frac{2}{3} \neq  0 
           \] 
   \end{enumerate} 
\end{eg}
\begin{prop}
    Une famille orthogonale constituée de vecteurs \underline{non-nuls} est libre. En particulier, une famille orthonormale est libre. 
\end{prop}
\begin{preuve}
    Suppososns $(v_1, \ldots, v_n)$ orthogonale avec $v_i \neq 0 \, \forall i = 1, \ldots, n$\\
    si $\sum_{j=1}^{n} \underset{\in \R}{\alpha_iv_i} = 0$, alors  
    \[
        \forall i \in \{1, \ldots, n\} 0 = \scalair{v_i, \sum_{j=1}^{n} \alpha_jv_j} = \sum_{j=1}^{n}\alpha_j \scalair{v_i, v_j} = \alpha_i \underset{\neq 0}{\|v_i\|^2}
    \] 
    Donc $\alpha_i = 0 \, \forall i = 1, \ldots, n$.\\
    Si $(v_1, \ldots, v_n)$ est orthonormale, alors $\|v_i\| = 1$. Donc  $v_i \neq 0, \, \forall i = 1, \ldots, n$.
\end{preuve}
\begin{intuition}
   Les vecteurs orthogonales (perpendiculaires) ne sont jamais dans l'un l'autre (i.e $e_i = \lambda e_j$ n'est pas possible) si les vecteurs sont liés, soit l'angle est $< 90º$ (donc les vecteurs ne sont pas orthogonales, absurd), (ils sont dans l'un l'autre, ils ne sont pas orthogonales, absurd). Donc ils sont bien libres.
\end{intuition}
\begin{definition}
    $(E, \scalair{,})$ espace euclidien. Une famille  $B = (e_1, \ldots, e_n)$ est une base orthonormale (où BON) si elle est une base et famille orthonormale.
\end{definition}
\begin{theorem}
    $(E, \scalair{,})$ espace euclidien. Alors, il admet une BON.
\end{theorem}
\begin{preuve}
   Soit $n := dim(E)$. Soit  $(e_1, \ldots, e_p)$ une famille orthogonale (du point de vue du cardinal $p$) tq  $e_i \neq 0 \, \forall i = 1, \ldots, p$.\\
Supposons par l'absurde que $p < n$. Posons  $F = Vect(e_1, \ldots, e_p)$. Alors, $E = F \oplus F^{\perp}$ et  $dim(F) \le p < n$. Donc $F^{\perp} \neq  \{0\}$. Soit $x \in F^{\perp}, \, x \neq 0$. Alors, $(e_1, \ldots, e_p, x)$ est orthogonale de cardinale $> p$. Donc,  $p = n$ et  $(e_1, \ldots, e_n)$ est une base de $E$. Pour avoir une famille orthonormale  $(e_1', \ldots, e_n')$ il suffit de prendre $e_i' = \frac{1}{\|e_i\|}e_i \, \forall i = \{1, \ldots, n\}$.
\end{preuve}
\begin{prop}
    Soit $(E, \scalar{}{})$ un espace euclidien et soit  $(e_1, \ldots, e_n)$ une BON de $E$. Si  $x \in E$, on a:
   \[
       x = \sum_{i=1}^{n} \scalar{x}{e_i}e_i
   \] 
Autrement dit, le réél $\scalar{x}{e_i}$ est la  $i^{\text{ème}}$ coordonnée de $x$ dans la base  $(e_1, \ldots, e_n)$.
\end{prop}
\begin{intuition}
    L'orthonormalité de la base nous simplifie la vie. Mais avant, petite introduction. Soit un e.v $E = \R^2$ et la base $(e_1, e_2) = (\begin{pmatrix} 1 \\ 0 \end{pmatrix}, \begin{pmatrix} 0\\ 1 \end{pmatrix})$. Soit un vecteur $\vec{v} = (2, 3)$ :
    \begin{center}
        \begin{tikzpicture}
            \begin{axis}[
                scale=1,
                axis lines=middle,        % Draw axes in the middle
                xmin=-2, xmax=4,          % X-axis range
                ymin=-2, ymax=4,          % Y-axis range
                xlabel={$x$},             % Label for X-axis
                ylabel={$y$},             % Label for Y-axis
                xtick={-2,-1,0,1,2,3,4},% X-axis ticks
                ytick={-2,-1,0,1,2,3,4},% Y-axis ticks
                ]
            \draw[color=red, ->, thick] (0, 0) -- node[below]{$e_1$}(1, 0);
            \draw[color=blue, ->, thick] (0, 0) -- node[left]{$e_2$}(0, 1);
            \draw[color=green, ->] (0, 0) --node[above]{$\vec{v}$} (2, 3);

            \draw[color=gray, ->, thick] (1, 0) -- node[below]{$e_1$}(2, 0);
            \draw[color=gray, ->, thick] (2, 0) -- node[left]{$e_2$}(2, 1);
            \draw[color=gray, ->, thick] (2, 1) -- node[left]{$e_2$}(2, 2);
            \draw[color=gray, ->, thick] (2, 2) -- node[left]{$e_2$}(2, 3);

            \node[right, above] (_) at (2, 3){$(2, 3)$};
        \end{axis} 
        \end{tikzpicture}
    \end{center}
    Donc, on peut écrire $\vec{v} = \vec{(2, 3)} = 2 \cdot \vec{e_1} + 3 \cdot \vec{e_2}$. Les $x$ et  $y$ (les coordonnées de $v$) nous donnes combien de parties de chaque vecteur de bases (le nombre peut être $\in \R$) et prendre leurs sommes, pour obtenir $\vec{v}$. (Le plus simple: combien on doit aller à gauche et en haut).
    \par
    Dans la base orthonormale $\scalair{v, e_i}$ nous donne combien on prend d'un vecteur $e_i$ pour faire le vecteur  $\vec{v}$ et  $\vec{e_i}$ donne la direction. D'où $\scalair{v, e_1}$ équivaut à $2$, et  $\scalair{v, e_2}$ à  $3$, puis: 
   \[
       \vec{v} = \underbrace{\scalair{v, e_1}}_{= 2} \cdot \vec{e_1} + \underbrace{\scalair{v, e_2}}_{= 3} \cdot \vec{e_2}
   \]  
   Habituelement, pour trouver les coordonnées dans une base, on devrait résoudre un système linéaire.
\end{intuition}
\begin{preuve}
    Posons $y := \sum_{i=1}^{n} \scalar{x}{e_i}e_i$ . Alors, 
   \begin{align*}
       &\forall j = 1, \ldots, n,\\
       &\scalar{x - y}{e_j}\\ 
       = &\scalar{x}{e_j} - \scalar{y}{e_j}\\ 
       = &\scalar{x}{e_j} - \scalar{\sum_{i=1}^{n} \scalar{x}{e_i}e_i}{e_j}\\ 
       = &\scalar{x}{e_j} - \underbrace{ \sum_{i=1}^{n} \scalar{x}{e_i} }_{\substack{\text{moved out}\\ \text{like constant}}}\scalar{e_i}{e_j}\\ 
       = &\scalar{x}{e_j}\\ 
       -& \left(\scalar{x}{e_1}\underbrace{ \scalar{e_1}{e_j} }_{= 0} + \ldots + \scalar{x}{e_{j-1}}\underbrace{\scalar{e_{j-1}}{e_j}}_{= 0} + \scalar{x}{e_{j}}\underbrace{ \scalar{e_{j}}{e_j} }_{= 1} + \scalar{x}{e_{j+1}}\underbrace{ \scalar{e_{j+1}}{e_j} }_{= 0} + \ldots + \scalar{x}{e_{n}}\underbrace{ \scalar{e_{n}}{e_j} }_{= 0}\right)\\
        &\text{(} \forall i \neq j, \, \scalar{e_i}{e_j} = 0 \text{ car un produit scalaire des vecteur orthogonaux)}\\ 
        &\text{(} \forall j \, \scalar{e_j}{e_j} = 1 \text{ car un produit scalaire de même vecteur)}\\
       = &\scalar{x}{e_j} - \scalar{x}{e_j}\underset{= 1}{\scalar{e_j}{e_j}} = 0
   \end{align*}
   Donc, $x - y \in Vect(e_1, \ldots, e_n)^{\perp} = E^{\perp} = \{0\}$. Donc $x = y$
\end{preuve}
\begin{corollary}
    $\forall x \in E, \, \|x\|^2 = \sum_{i=1}^{n} \scalar{x}{e_i}^2$ 
\end{corollary}
\begin{preuve}
    Si $x = \sum_{i=1}^{n} \scalar{x}{e_i}e_i = \sum_{i=1}^{n} x_ie_i$ donc
    \[
        \|x\|^2 = \scalar{\sum_{i=1}^{n} x_ie_i}{\sum_{j=1}^{n} x_je_j} = \sum_{i,j=1}^{n} x_ix_j\scalar{e_i}{e_j} = \sum_{i=1}^{n} x_i^2
    \] 
\end{preuve}
\section{Matrices et produits scalaires}
\begin{prop} Soient $(E, \scalair{,})$ un espace euclidien et $\epsilon = (e_1, \ldots, e_n)$ une BON. Soient $f \in \mathcal{L}(E, E)$ et $A = (a_{i,j})_{1 \le i,j \le n}$ la matrice représentative de $f$ dans  $\epsilon$, i.e,  $A = Mat_{\epsilon}(f)$ 
    \[
        a_{i,j} = \scalair{f(e_i), e_j} \, \forall i,j = 1, \ldots, n
    \] 
\end{prop}
\begin{preuve}
   $A$ est la matrice dont les colonnes sont les vecteurs  $f(e_j)$ écrits dans la base $\epsilon$:
    \[
        A = (f(e_1) | \ldots | f(e_n))\quad f(e_j) = \begin{pmatrix} a_{1,j}\\ \ldots\\ a_{n, j} \end{pmatrix} 
   \] 
   Car $\forall v \in E, \, v = c_1e_1 + \ldots c_ne_n$ donc $f(v) = c_1f(e_1) + \ldots c_nf(e_n)$ par la linéarité, donc il nous reste à étudier chaque $f(e_j)$
   \begin{align*}
       f(e_j) = a_{1, j}e_1 + \ldots a_{n, j}e_n \implies\\
       \langle f(e_j), e_i \rangle = \left\langle \sum_{k=1}^n a_{k,j} e_k, e_i \right\rangle = \sum_{k=1}^{n} a_{k,j}\scalar{e_k}{e_i} = a_{k, j}
   \end{align*}
   car $\scalar{e_k}{e_j} = \begin{cases}
       0 \text{ si } k \neq j\\
       1 \text{ si } k = j
   \end{cases}$
   Donc:
   \[
       a_{i, j} = \scalair{f(e_j), e_i}
   \] 
\end{preuve}


La matrice d'un produit vectoriel est très utile dans l'algèbre linéaire. Avant donner une definition:
\par
Soit $E$ un espace vectoriel de dimension finie  $n$, un espace  $K$ et une forme bilinéaire  $b: E \times E \longrightarrow K$. Si $\{e_1, \ldots, e_n\}$ est une base de $E$, alors:  $x = \sum_{i=1}^{n} x_ie_i$ et $y = \sum_{j=1}^{n} y_je_j$, alors on a:
\[
b(x, y) = \sum_{i,j = 1}^{n} x_iy_jb(e_i, e_j)
\] 
$b$ est donc détérminé par la conaissance des valeurs  $b(e_i, e_j)$ sur une base.
 \begin{definition}
     On appelle  \textbf{matrice de $b$} dans la base $\{e_i\}$ la matrice:
      \[
          M(b)_{e_i} = \begin{pmatrix} 
              b(e_1, e_1) & b(e_1, e_2) & \ldots & b(e_1, e_n)\\
              b(e_2, e_1) & b(e_2, e_2) & \ldots & b(e_2, e_n)\\
              \ldots & \ldots & \ldots & \ldots\\
              b(e_n, e_1) & \ldots & \ldots & b(e_n, e_n)
          \end{pmatrix} 
     \] 
     Ainsi l'élément de la $\text{i}^{\text{ème}}$ ligne et $\text{j}^{\text{ème}}$ colonne est le coefficient de $x_iy_j$.
\end{definition}
\begin{eg}
   La matrice du produit scalair canonique dans $\R^3$ est:
   \[
       \scalair{X, Y} = x_1y_1 + x_2y_2 + x_3y_3 
   \] 
   \[
       Mat(\scalair{,})_{e_i} = \begin{pmatrix} 
            1 & 0 & 0\\
            0 & 1 & 0\\
            0 & 0 & 1
       \end{pmatrix} 
   \] 
\end{eg}
\begin{prop}\label{prop:prod-scal-par-matrice} produit scalair représenté par une matrice.\par
   Notons:
   \begin{align*}
       \underbrace{A = M(b)_{e_i}}_{\text{matrice de produit scalair}} && \underbrace{X = M(x)_{e_i}}_{\substack{\text{coordonnées de $x$}\\ \text{dans la base  $e_i$}}} && \underbrace{Y = M(y)_{e_i}}_{\substack{\text{coordonnées de $y$}\\ \text{dans la base $e_i$}}} && (x, y \in E)
   \end{align*}
   Alors, on a:
   \[
       b(x, y) = X^{t}AY
   \] 
\end{prop}
\begin{eg}
    Repronnons l'exemple avec $b = \scalair{,}$ le produit scalair canonique dans  $\R^3$. Soit $X = \begin{pmatrix} 1 \\ 2 \\ -1 \end{pmatrix}$ et $Y = \begin{pmatrix} 2 \\ 3 \\ 1 \end{pmatrix} $ dans la base canonique de $\R^3$. Donc:
    \begin{align*}
        \scalair{x, y} = X^{t}AY &= \overbrace{(1, 2, -1)}^{X^{t}} \times \overbrace{\begin{pmatrix} 1 & 0 & 0\\ 0 & 1 & 0\\ 0 & 0 & 1 \end{pmatrix}}^{A} \times \overbrace{ \begin{pmatrix} 2 \\ 3 \\ 1 \end{pmatrix} }^{Y} \\
                                 &= \underbrace{(1, 2, -1)}_{X} \times \underbrace{ \begin{pmatrix} 2 \\ 3\\ 1 \end{pmatrix} }_{A \times Y} \\
                                 &= 1 \cdot 2 + 2 \cdot 3 + (-1) \cdot 1 = 2 + 6 - 1 = 7
    \end{align*}
\end{eg}
\begin{TODO}
   changement de base de la matrice d'une forme bilinéaire 
\end{TODO}

\section{Projections orthogonales}
Soit $(E, \scalair{,})$ un espace euclidien,  $F \subseteq E$ un sous-espace vectoriel. Alors,  $E = F \oplus F^{\perp}$. Donc $\forall x \in E$ s'ecrit 
\[
x = \underset{\in F}{x_F} + \underset{\in F^{\perp}}{x_{F^{\perp}}}
\] 
\begin{definition}
    La \textbf{projection orthogonale} de $E$ dans  $F$ est la projection  $p_F$ de  $E$ sur  $F$ parallèlement  à $F^{\perp}$, i.e
    \begin{align*}
        p_F: E = F \oplus F^{\perp} &\longrightarrow F \\
        x = x_F + x_{F^{\perp}} &\longmapsto p_F(x = x_F + x_{F^{\perp}}) = x_F
    .\end{align*}
\end{definition}
\begin{remark}
   \begin{enumerate}
       \item $p_F$ est linéaire
       \item  $\forall x \in E \, p_{F}(x)$ est complétement caractérisé par la propriété suivante:\\
           Soit $y \in E$, alors
            \[
                y = p_F(x) \iff \left( \underset{\implies y = x_F}{y \in F \text{ et } x - y} \in F^{\perp} \right) 
           \] 
       En particulier $\scalair{p_F(x), x - p_F(x)} \,= 0$. Alors, si $(v_1, \ldots, v_R)$ est une BON de $F$, on a:
            \[
                \forall x \in E, \, p_F(x) = \sum_{i=1}^{k} \scalair{x, v_i}v_i
           \] 
           En effet, il suffit de vérfier que le vecteur $y = \sum_{i=1}^{k} \scalair{x, v_i}v_i$ vérfie:
           \[
               y \in F \text{ et } x - y \in F^{\perp}
           \] 
   \end{enumerate} 
\end{remark}
\begin{figure}[H]
   \centering 
\begin{tikzpicture}

% Draw the plane
\fill[gray!20] (-2,-1) -- (2,-1) -- (3,1) -- (-1,1) -- cycle;

% Draw the vectors
\draw[->, thick, black] (0,0) -- (2,2.2) node[anchor=south east] {\large $\mathbf{x}$};

\node[anchor=north, blue] (_) at ($(0,0)!0.5!(2,0)$) {\large $\text{proj}_\mathbf{F} \mathbf{x}$};
\node[anchor=west, blue] (_) at ($(2,0)!0.5!(2,2.2)$) {\large $\text{proj}_\mathbf{F^{\perp}} \mathbf{x}$};
% Add the labels for w and w perpendicular
\draw[->, thick, blue] (0,0) -- (2,0) ;
\draw[thick, black] (2,0) -- (2,3) node[anchor=west] {\large $\mathbf{F}^\perp$};
\draw[->, thick, blue] (2,0) -- (2,2.2);
\node[anchor=north west] (_) at (1.5, -0.5) {\large $\mathbf{F}$};
% Add the right angle symbol

\end{tikzpicture}
\caption{Projection}
\label{pic:projection}
\end{figure}
\begin{figure}[ht]
    \centering
    \incfig{projection-with-bon}
    \caption{Projection avec BON}
    \label{fig:projection-with-bon}
\end{figure}
\begin{prop}
   Soit $x \in E$. Alors,
   \[
       \|x - p_F(x)\| = inf\{\|x - y\| \mid y \in F\}
   \] 
   i.e $\|x - p_F(x)\|$ est la distance de  $x$ à  $F$.\\
   Voir Figure~\ref{pic:projection}
\end{prop}
\begin{preuve}
   Comme $p_F(x) \in F$ il suffit de prouver que, si  $y \in F$, alors 
   \[
   \|x - p_F(x)\| \le \|x - y\|
   \] 
   Mais, $\underset{(x - p_F(x)) + (p_F(x) - y)}{\|x - y\|^2} = \|x - p_F(x)\|^2 + 2\overbrace{\scalair{\overset{\in F^{\perp}}{x - p_F(x)}, \overset{\in F}{p_F(x) - y}}}{= 0} + \underbrace{\|p_F(x) - y\|^2}_{\ge 0} \ge \|x - p_F(x)\|^2$
\end{preuve}
\begin{theorem}\label{thm:gram-schmidt}Gram-Shmidt\\
    Soit $E$ un espace vectoriel muni d'un produit scalaire  $\scalair{,}$. Soit  $(v_1, \ldots, v_n)$ une famille libre d'élement $\in E$. Alors,  il existe une famille $(w_1, \ldots, w_n)$ orthogonale tq 
    \[
        \forall i = 1, \ldots, n \quad Vect(v_1, \ldots, v_i) = Vect(w_1, \ldots, w_i)
    \] 
    De plus, ce théorème nous donne un procédé de construction d'une base orthonormée à partir d'une base quelconque.
\end{theorem}
\begin{preuve} du Théorème \ref{thm:gram-schmidt}
    Construisons la base orthogonale: $\{w_1, \ldots, w_p\}$. Posons d'abord:
    \[
    \begin{cases}
        w_1 = v_1\\
        w_2 = v_2 + \lambda w_1, \qquad \text{avec } \lambda \text{ tel que } w_1 \perp w_2
    \end{cases}
    \] 
    En imposant cette condition on trouve:
    \[
        0 = \scalair{v_2 + \lambda w_1, w_1} = \scalair{v_2, w_1} + \lambda \|w_1\|^2
    \] 
    Comme $w_1 \neq 0$, on obtient $\lambda = - \frac{\scalair{v_2, w_1}}{\|w_1\|^2}$. On remarque que:
    \[
    \begin{cases}
        v_1 = w_1\\
        v_2 = w_2 - \lambda w_1
    \end{cases}
    \] 
    donc $Vect\{v_1, v_2\} = Vect\{w_1, w_2\}$.
    \par
    Une fois construit $w_2$, on construit $w_3$ en posant:
    \begin{align*}
        &w_3 = v_3 + \mu w_1 + \nu w_2\\
        &\text{avec } \mu \text{ et } \nu \text{ tels que: } w_3 \perp w_1 \text{ et } w_3 \perp w_2
    \end{align*}
    On peut voir $w_3 = v_3 - \lambda' w_1 - \lambda'' w_2 $ comme $w_3 = v_3 - proj_{F_2}v_3$ où $F_i = Vect\{w_1, \ldots, w_i\}$
    \begin{figure}[H]
        \centering
        \incfig{projection-with-bon-thm}
        \caption{Vecteur par projection}
        \label{fig:projection-with-bon-thm}
    \end{figure}
    Ceci donne
    \begin{align*}
        0 &= \scalair{v_3 + \mu w_1 + \nu w_2, w_1} = \scalair{v_3, w_1} + \mu \underset{= \|w_1\|^2}{\scalair{w_1, w_1}} + \nu \underset{= 0}{\scalair{w_2, w_1}}\\
          &= \scalair{v_3, w_1} + \mu \|w_1\|^2 
    \end{align*}
    d'où $\mu = - \frac{\scalair{v_3, w_1}}{\|w_1\|^2}$. De même, en imposant que $w_3 \perp w_2$, on trouve $\nu = - \frac{\scalair{v_3, w_2}}{\|w_2\|^2}$. Comme
    \[
    \begin{cases}
        v_1 = w_1\\
        v_2 = w_2 - \lambda w_1\\
        v_3 = w_3 - \mu w_1 - \nu w_2
    \end{cases}
    \] 
    on voit bien que $Vect\{w_1, w_2, w_3\} = Vect\{v_1, v_2, v_3\}$. C'est-à-dire, $\{w_1, w_2, w_3\}$ est une base orthogonale de l'éspace engendre par $v_1, v_2, v_3$. On voit bien maintenant le procédé de récurrence.
    \par
    Supposons avoir construit $w_1, \ldots, w_{k-1}$ pour $k \le p$. On pose:
    \begin{align*}
        w_k &= v_k + \text{ combinaison linéaire des vecteurs déjà trouvés}\\
            &= v_k + \lambda_1w_1 + \ldots + \lambda_{k-1}w_{k-1}
    \end{align*}
    Les conditions $w_k \perp w_i$ (pour $i \in \{1, \ldots, k-1\}$) sont équivalentes à:
    \[
        \lambda_i = - \frac{\scalair{v_k, w_i}}{\|w_i\|^2}
    \] 
    comme on le vérifie immédiatement. Puisque $v_k = w_k - \lambda_1 - \ldots - \lambda_{k-1}w_{k-1}$, on voit par récurrence que $Vect\{w_1, \ldots, w_k\} = Vect\{v_1, \ldots, v_k\}$ $\iff$ $\{w_1, \ldots, w_k\}$ est une base orthogonale de $Vect\{v_1, \ldots, v_k\}$.
    \par
    Ce qu'il nous rester c'est à la normaliser, i.e  $\forall i \in \{1, \ldots, k\}$ $e_i = \frac{w_i}{\|w_i\|}$, d'où $\{e_1, \ldots, e_k\}$ est une base orthonormale de $F = Vect\{v_1, \ldots, v_k\}$.
\end{preuve}
\begin{prop} Pour comprendre cette proposition, je vous conseil de lire la section \ref{sec:isometrie-et-adjoints}
    \par
   Toute projection orthogonale est autoadjoint, i.e si $p$ est une projection orthogonale, donc:
   \[
   p^* = p
   \] 
   En notation matricielle: soit $A$ une matrice de la projection  $p$, donc:
    \[
   A^T = A
   \] 
\end{prop}
