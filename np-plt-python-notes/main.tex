\documentclass[a4paper]{article}

\usepackage[utf8]{inputenc}
\usepackage[T1]{fontenc}
\usepackage{textcomp}
\usepackage[english]{babel}
\usepackage{amsmath, amssymb, amsthm}


% figure support
\usepackage{import}
\usepackage{xifthen}
\pdfminorversion=7
\usepackage{pdfpages}
\usepackage{transparent}
\usepackage{hyperref}
\usepackage[margin=0.8in]{geometry}

\usepackage{setspace}
\setlength{\parindent}{0in}

\newcommand{\incfig}[1]{%
    \def\svgwidth{\columnwidth}
    \import{./figures/}{#1.pdf_tex}
}

\pdfsuppresswarningpagegroup=1

\newcommand{\N}{\mathbb{N}}
\newcommand{\R}{\mathbb{R}}
\newcommand{\Z}{\mathbb{Z}}
\newcommand{\Q}{\mathbb{Q}}

\newtheorem{theoreme}{Théorème}[section]
\newtheorem{definition}{Définition}[section]
\newtheorem{exemple}{Exemple}[section]
\newtheorem{proposition}{Proposition}[section]
\newtheorem{propriete}{Propriété(s)}[section]
\newtheorem*{notation}{Notation}
\newtheorem*{remarque}{Remarque}

\usepackage{minted}
\begin{document}
    \section{NumPy}
    \subsection{Meshgrid}
    In order to simplify ploting, there exists a function called Meshgrid. It takes $x$ and  $y$ values and returns all possible combinations.
     \begin{exemple}
         if there are two arrays:
         \begin{minted}[frame=lines, fontsize=\small, bgcolor=lightgray]{python}
x = [1, 2]
y = [3, 4]
        \end{minted} 
then meshgrid creates all possible combinations


         \begin{minted}[frame=lines, fontsize=\small, bgcolor=lightgray]{python}
xv, yv = np.meshgrid(x, y)
xv = [
[1, 2],
[1, 2]
]

yv = [
[ 3, 3 ],
[ 4, 4]
]
        \end{minted} 

Then, we have combinations of each x with each y.
    \end{exemple}

    \section{Matplotlib}
    \subsection{quiver}
    \begin{minted}{python}
plt.quiver(X, Y, U, V) 
    \end{minted}
    This function draws vectors, it takes vector origins ($X, Y$) and vector directions ($U, V$)
\end{document}
