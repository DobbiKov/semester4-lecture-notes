\documentclass[a4paper]{article}

\usepackage[utf8]{inputenc}
\usepackage[T1]{fontenc}
\usepackage{textcomp}
\usepackage[english]{babel}
\usepackage{amsmath, amssymb, amsthm}


% figure support
\usepackage{import}
\usepackage{xifthen}
\pdfminorversion=7
\usepackage{pdfpages}
\usepackage{transparent}
\usepackage{hyperref}
\usepackage[margin=0.8in]{geometry}

\usepackage{setspace}
\setlength{\parindent}{0in}

\newcommand{\incfig}[1]{%
    \def\svgwidth{\columnwidth}
    \import{./figures/}{#1.pdf_tex}
}

\pdfsuppresswarningpagegroup=1

\newcommand{\N}{\mathbb{N}}
\newcommand{\R}{\mathbb{R}}
\newcommand{\Z}{\mathbb{Z}}
\newcommand{\Q}{\mathbb{Q}}

\newtheorem{theoreme}{Théorème}[section]
\newtheorem{definition}{Définition}[section]
\newtheorem{exemple}{Exemple}[section]
\newtheorem{proposition}{Proposition}[section]
\newtheorem{propriete}{Propriété(s)}[section]
\newtheorem*{notation}{Notation}
\newtheorem*{remarque}{Remarque}
\title{Examen 2024}
\begin{document}
   \maketitle 
   \section*{Exercice 2}
   \[
   f(x, y) = \begin{cases}
       y^2x^{-1} \text{ si } x \neq 0\\
       0 \text{ si } x = 0
   \end{cases}
   \] 
   en polaires:
   \[
   f(x) = \begin{cases}
            \frac{r^2 \sin \theta}{\cos \theta} = \frac{r \sin \theta}{\cos \theta} \text{ si } \cos \theta \neq 0\\
            0 \text{ si } \cos \theta = 0
   \end{cases}
   \] 
   \subsection*{$f$ admet des dérivées dans toutes les directions en $(0, 0)$:}
   \par
   $\vec{u} = (a, b)$ et  $f(0, 0) = 0$
    \[
        f(t\vec{u}) = \begin{cases}
            \frac{t^2b^2}{ta} \text{ si } a \neq 0\\
            0 \text{ si } a = 0
        \end{cases}
   \] 
   \[
       \frac{d}{dt}f(t\vec{u})|_{t = 0} = \begin{cases}
           \frac{b^2}{a} \text{ si } a \neq 0\\
           0 \text{ si } a = 0
       \end{cases}
   \] 
   \par
   \subsection*{$f$ pas continue en  $(0, 0)$:}
   \par
   2 suites  $x_n, \hat{x}_n$ dans $\R^2$ tq $x_n \to (0, 0)$ et $\hat{x}_n \to (0, 0)$
   \[
   \lim_{n \to \infty} f(x_n) \neq \lim_{n \to \infty} f(\hat{x}_n)
   \] 
   \[
   x_n = (0, \frac{1}{n}) \quad f(x_n) = 0 \, \forall n
   \] 
   \[
   \hat{x}_n = (\frac{1}{n^2}, \frac{1}{n}) \quad f(\hat{x}_n)=1 \, \forall n
   \] 
   Lignes de niveau
   \[
   f(x, y) = c
   \] 
   2 cas:
   \begin{itemize}
       \item $c = 0$  $f(x, y) = 0 \iff x= 0 \text{ ou } y = 0$ 
       \item $c \neq 0$ $y^2 = cx$  $x \neq 0$
   \end{itemize}
\end{document}
