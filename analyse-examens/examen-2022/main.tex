\documentclass[a4paper]{article}

\usepackage[utf8]{inputenc}
\usepackage[T1]{fontenc}
\usepackage{textcomp}
\usepackage[english]{babel}
\usepackage{amsmath, amssymb, amsthm}


% figure support
\usepackage{import}
\usepackage{xifthen}
\pdfminorversion=7
\usepackage{pdfpages}
\usepackage{transparent}
\usepackage{hyperref}
\usepackage[margin=0.8in]{geometry}

\usepackage{setspace}
\setlength{\parindent}{0in}

\newcommand{\incfig}[1]{%
    \def\svgwidth{\columnwidth}
    \import{./figures/}{#1.pdf_tex}
}

\pdfsuppresswarningpagegroup=1

\newcommand{\N}{\mathbb{N}}
\newcommand{\R}{\mathbb{R}}
\newcommand{\Z}{\mathbb{Z}}
\newcommand{\Q}{\mathbb{Q}}

\newtheorem{theoreme}{Théorème}[section]
\newtheorem{definition}{Définition}[section]
\newtheorem{exemple}{Exemple}[section]
\newtheorem{proposition}{Proposition}[section]
\newtheorem{propriete}{Propriété(s)}[section]
\newtheorem*{notation}{Notation}
\newtheorem*{remarque}{Remarque}

\title{Examen 2022}
\begin{document}
\maketitle
\section*{Exercice 1} 
\[
    B(\R) = \{ u: \R \to \R \mid u \text{ bornée} \}
\] 
\[
\|u\|_{\infty} = \sup_{x \in \R} |u(x)|
\] 

$u: \R \to \R$ bornée: $\{ u(x): x \in \R \}$ bornée dans $\R$.
\begin{enumerate}
    \item si $u_n \to u$ pour $\|\cdot\|_{\infty}$, alors $u_n(x_0) \to u(x_0)$ $\forall x_0 \in \R$.
        \par
        On fixe $x_0 \in \R$,
        \[
        |u_n(x_0) - u(x_0)| \le \|u_n - u\|_{\infty}
        \] 
        donc si $\|u_n - u\|_{\infty} \to 0$ alors $|u_n(x_0) - u(x_0)| \xrightarrow[n \to \infty]{} 0$ 
        \begin{remarque}
        \[
            E = \mathcal{C}^0([a,b], \R)
        \] 
        \[
            \|u\|_1 = \int_{{a}}^{{b}} {|u(x)|} \: d{x}
        \] 
        $u_n \to u$ pour $\|\cdot \|_{1}$ n'entraine pas que $u_n(x_0) \to u(x_0) \forall x_0 \in [a, b]$ 
    \end{remarque}
    \item $A \subset B(\R) = $ ensemble des fonctions bornées et croissantes. $u_n \in A$ avec $u_n \to u$ pour $\|\cdot \|_{\infty}$, montrer que $u \in A$.
        \par
        On fixe $x_0, x_1 \in \R$ avec $x_0 \le x_1$. 
        \[
        u(x_1) - u(x_0) = \lim_{n \to \infty} u_n(x_1) - u_n(x_0)
        \] 
        $\forall n, u_n(x_1) - u_n(x_0) \ge 0$ car $u_n$ croissante, donc  $u(x_1) - u(x_0) \ge 0$, donc $u$ croissante aussi. Alors \underline{$A$ est fermé}.
    \item $C = \{ u \in B(\R) \text{ avec une limitie (finie) en } \pm\infty \}$ $C$ sous-espace vectoriel de $B(\R)$. 
        \begin{enumerate}
            \item 
                $A \subset C$, $C$ est fermé
            \item si  $u, v \in \mathbb{C}$ et $u^+ = \lim_{x \to \infty} u(x)$ et $v^+ = \lim_{x \to \infty} v(x)$, alors $|u^+ - v^+| \le \|u - v\|_{\infty}$
                \[
                |u^+| \le \|u\|_{\infty}
                \] 
                $|u(x)| \le \|u\|_{\infty} \quad \forall x \in \R$, donc 
                \[
                \lim_{x \to \infty} |u(x)| = u^+ \le \|u\|_{\infty}
                \] 
                (les inégalités larges passent à la limite)
            \item 
                $(u_n)$ suite dans  $C$ avec  $u_n \to u$ pour $\|\cdot\|_{\infty}$. $u_n^+ = \lim_{x \to \infty} u_n(x)$ et $(u_n^+)$ suite dans  $\R$. Montrer que $(u_n^+)$ converge dans  $\R$.
                \par
                $n, p \in \N$
                \[
                    (1) \quad |u_n^+ - u_p^+| \le \|u_n - u_p\|_{\infty}
                \] 
                $(u_n)$ converge vers  $u$ dans  $(B(\R), \|\cdot\|_{\infty})$ donc $(u_n)$ de  Cauchy (dans $B(\R), \|\cdot \|_{\infty}$)
                \[
                \forall \varepsilon > 0, \exists N \in \N \text{ tq si } n, p \ge N, \|u_n - u_p\|_{\infty} \le \varepsilon 
                \] 
                Par $(1)$  $(u_n^+)$ est de Cauchy dans  $\R$ donc  \underline{converge}.
            \item 
                $u_n \to u$ pour $\|\cdot\|_{\infty}$, on pose $l^+ = \lim_{n \to \infty} u_n^+$. \underline{But}: montrer que $\lim_{x \to \infty} u(x) = l^+$, donc $u \in C$, i.e montrer que 
                \[
                \lim_{n \to \infty} \lim_{x \to \infty} u_n(x) = \lim_{x \to \infty} \lim_{n \to \infty} u_n(x)
                \] 
                \par
                \[
                |l^+ - u(x)| \le |l^+ - u_n(x)| + |u_n(x) - u(x)| \le |l^+ - u_n^+| + |u_n^+ - u_n(x)| + |u_n(x) - u(x)|
                \] 
                $\forall \varepsilon > 0$, trouver $R \in \R$ tq si $x \ge R, |l^+ - u(x)| \le 3 \varepsilon$
                \[
                    |u_n(x) - u(x)| \le \|u_n - u\|_{\infty}
                \] 
                $u_n \to u$ pour $\|\cdot\|_{\infty}$ donc $\exists N \in \N$ si $n \ge N$
                \[
                |u_n(x) - u(x)| \le \varepsilon \forall x \in \R
                \] 
                $\lim_{n \to \infty} u_n^+ = l^+$: $\exists M \in \N$ tel que si $n \ge M$
                \[
                |l^+ - u_n^+| \le \varepsilon
                \] 
                Je fixe $n = \max(N, M)$ ou  $N + M$, pour ce  $n$ fixé: 
                \[
                u_n(x) \xrightarrow[x \to \infty]{} u_n^+
                \] 
                Donc $\exists R \ge 0$ tq si $x \ge R$
                \[
                |u_n(x) - u_n^+| \le \varepsilon
                \] 
                Si $x \ge R \quad |u(x) - l^+| \le \varepsilon$
        \end{enumerate}
\end{enumerate}
\end{document}
