% --- CHUNK_METADATA_START ---
% needs_review: True
% src_checksum: 253d291659ae7159a918d82c5fc9d307e05e3782cefee1a6f63f646fd9fec3f0
% --- CHUNK_METADATA_END ---
\begin{appendices}
    \chapter{Reminders of Linear Algebra concepts} 
    \section{Matrices}
    \subsection{Multiplication of matrices}
    \begin{definition}
        Let $A \in \mathcal{M}_{p, n}(\R)$ and $B \in \mathcal{M}_{n, q}(\R)$ such that $A = (a_{j, i})$ and $B = (b_{m, k})$, then:
         \[
        AB = C = (c_{j, k} = \sum_{i=1}^{n} a_{j, i}b_{i, k})
        \] 
    \end{definition}
    \subsection{The trace}
    \begin{definition}
        The trace of the \( n \times n \) square matrix \( A \), denoted \( \text{tr}(A) \), is the sum of the diagonal elements

        \[
            \text{tr}(A) = a_{11} + a_{22} + \dots + a_{nn} = \sum_{i=1}^{n} a_{ii}
        \]

        where \( a_{ii} \) are diagonal elements of the matrix \( A \). 
    \end{definition}

    \begin{property} of the trace.
       \begin{itemize}
           \item Linearity:
               \[
                   \text{tr}(A + B) = \text{tr}(A) + \text{tr}(B)
               \]

               \[
                   \text{tr}(cA) = c \text{tr}(A), \quad c \in \mathbb{R} \text{ (ou } \mathbb{C} \text{)}
               \]
            \item  Transposed:
                \[
                    \text{tr}(A) = \text{tr}(A^T)
                \] 
            \item Multiplication of matrices:
                \[
                    \text{tr}(AB) = \text{tr}(BA), \quad \text{(si } A \text{ et } B \text{ are of size } n \times n)
                \]

                However, the trace is not distributive over multiplication:

                \[
                    \text{tr}(A B C) \neq \text{tr}(A) \text{tr}(B C)
                \]
            \item Eigenvalues:
                \[
                    \text{tr}(A) = \sum_{i=1}^{n} \lambda_i
                \]

                where \( \lambda_i \) are the eigenvalues of \( A \). This makes the trace an important tool in spectral analysis.

            \item Trace of the Identity Matrix

                \[
                    \text{tr}(I_n) = n
                \]

                since all the diagonal elements are equal to 1.
       \end{itemize} 
    \end{property}
    \begin{eg}
        For
        \[
            A = \begin{bmatrix} 3 & 2 & 1 \\ 4 & 5 & 6 \\ 7 & 8 & 9 \end{bmatrix}
        \]

        the trace is:

        \[
            \text{tr}(A) = 3 + 5 + 9 = 17
        \] 
    \end{eg}
    \begin{eg}
        If

        \[
            B = \begin{bmatrix} 2 & 1 \\ 0 & 3 \end{bmatrix}, \quad C = \begin{bmatrix} 4 & 2 \\ 1 & 5 \end{bmatrix}
        \]

        then

        \[
            \text{tr}(B + C) = \text{tr} \begin{bmatrix} 6 & 3 \\ 1 & 8 \end{bmatrix} = 6 + 8 = 14
        \]

        which corresponds well to

        \[
            \text{tr}(B) + \text{tr}(C) = (2+3) + (4+5) = 14
        \]

        thus confirming linearity. 
    \end{eg}
\end{appendices}