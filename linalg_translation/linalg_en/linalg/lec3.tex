% --- CHUNK_METADATA_START ---
% needs_review: True
% src_checksum: bf29d983a077cade6e7b79974dcab3a974510ec0274808acf85fb66ff370949b
% --- CHUNK_METADATA_END ---
\section{Isometries and Adjoints}% --- CHUNK_METADATA_START ---
% needs_review: True
% src_checksum: 404e586606ba4465ff926abf10b88897e18e4ba1644bdf63730b802601b04887
% --- CHUNK_METADATA_END ---
\label{sec:isometrie-et-adjoints}% --- CHUNK_METADATA_START ---
% needs_review: True
% src_checksum: 73ae3e92ca1e7f1ec5cd7c60aa1e778221e2cc965369c246711a5e59f5873117
% --- CHUNK_METADATA_END ---
\subsection{Isometries}% --- CHUNK_METADATA_START ---
% needs_review: True
% src_checksum: f31879de7b3b511f25e54c3b0a768b70124f485082f987bdef45c36e98912a7c
% --- CHUNK_METADATA_END ---
\begin{definition}
    An \textbf{isometry} of $E$ (or \textbf{orthogonal transformation}) is an endomorphism $f \in \mathcal{L}(E) := \mathcal{L}(E, E)$ preserving the dot product, i.e.:
     \[
         \scalair{f(x), f(y)} = \scalair{x, y} \quad \forall x, y \in E
    \] 
\end{definition}% --- CHUNK_METADATA_START ---
% needs_review: True
% src_checksum: 226da0912aa6423aadeae262eeafcc3e84237ec5ceef3ec18cd9a1fd199371e6
% --- CHUNK_METADATA_END ---
\begin{definition}
    Let $x, y \in E$ be two non-zero vectors. We have, according to the Cauchy-Schwarz inequality (see lemma \ref{lemma:inegalite-cauchy-schwarz}):
    \[
        \frac{| \scalair{x, y} |}{\|x\| \cdot \|y\|} \le 1
    \] 
    Then, there exists one and only one $\theta \in [0, \pi]$ such that:
     \begin{equation}
        \cos \theta = \frac{ \scalair{x, y}}{\|x\| \cdot \|y\|} 
    \end{equation}
    $\theta$ is called the \textbf{angle} (non-oriented) between the vectors $x$ and $y$.
\end{definition}% --- CHUNK_METADATA_START ---
% needs_review: True
% src_checksum: 93326741099c1f67d38123ac1c58cd30fafdaf5a9fc1c9d3bfd189ebe3f67d49
% --- CHUNK_METADATA_END ---
\begin{prop}\label{prop:isometrie-reserve-norme}
   If $f$ is an isometry of $E$, then we have:
   \[
   \|f(x)\| = \|x\| \quad \forall x \in E
   \] 
\end{prop}% --- CHUNK_METADATA_START ---
% needs_review: True
% src_checksum: 8b6e25bd20ed73256f25ff5a38cadb126d0df5b8a1c53d50e83eb89da031c211
% --- CHUNK_METADATA_END ---
\begin{preuve}
   Suppose that $f$ is an isometry of $E$. Let $x, y \in E$. By definition: $\scalair{f(x), f(y)} = \scalair{x, y}$, therefore, let $y := x$, then, we have:
   \begin{align*}
       &\underbrace{\scalair{f(x), f(x)}}_{\|f(x)\|^2} = \underbrace{\scalair{x, x}}_{\|x\|^2}\\
       \iff &\|f(x)\|^2 = \|x\|^2\\
       \iff &\|f(x)\| = \|x\|
   \end{align*}
\end{preuve}% --- CHUNK_METADATA_START ---
% needs_review: True
% src_checksum: 02574ca751b6786c0767e44130459fb748e40551d55dbd7c9519b9234f153150
% --- CHUNK_METADATA_END ---
\begin{prop}\label{prop:isometrie-bijective}
   Let $f$ be an isometry in $E$, then:
   \begin{enumerate}
       \item $f$ is bijective
       \item  $f$ preserves the Euclidean distance and angles
   \end{enumerate}
\end{prop}% --- CHUNK_METADATA_START ---
% needs_review: True
% src_checksum: aa1693c4eb616c058387165d6c49b19bd90ddd4af5fb103c00d2ac8f14c38ffb
% --- CHUNK_METADATA_END ---
\begin{preuve}
   Let $f$ be an isometry in $E$ and two vectors $u, v \in E$ 
   \begin{enumerate}
       \item  
           \begin{align*}
               \|f(u) - f(v)\| = \sqrt{\scalair{f(u), f(v)}} = \sqrt{\scalair{u, v}} = \|u - v\| 
           \end{align*}
       \item Let $\theta_1$ be the angle between $f(u)$ and $f(v)$ and $\theta_2$ be the angle between $u$ and $v$, so:
            \[
                \cos \theta_1 := \frac{\scalair{f(u), f(v)}}{\|f(u)\| \cdot \|f(v)\|}
           \] 
           \[
                \cos \theta_2 := \frac{\scalair{u, v}}{\|u\| \cdot \|v\|}
           \] 
           By definition, $\scalair{f(u), f(v)} = \scalair{u, v}$, according to proposition \ref{prop:isometrie-reserve-norme}, $\forall x, \|f(x)\| = \|x\|$, so:
           \[
                \cos \theta_1 := \frac{\scalair{f(u), f(v)}}{\|f(u)\| \cdot \|f(v)\|} = \frac{\scalair{u, v}}{\|u\| \cdot \|v\|} = \cos \theta_2
           \] 
   \end{enumerate}
\end{preuve}% --- CHUNK_METADATA_START ---
% needs_review: True
% src_checksum: d385f4082b752467e8f8dbd40ae7f6e4d1b150aafef7f7a2d8753102cc334575
% --- CHUNK_METADATA_END ---
\begin{definition}
    Let $F$ be a vector subspace of $E$, therefore $E = F \oplus F^{\perp}$ where $\forall v \in E, \exists v_1 \in F, v_2 \in F^{\perp}$ such that $v = v_1 + v_2$. We set:
    \[
    s_F(v) = v_1 - v_2
    \] 
    and we call $s_F$ an orthogonal symmetry with axis F.
\end{definition}% --- CHUNK_METADATA_START ---
% needs_review: True
% src_checksum: 9774560f06771548c2bdd5a9eba119bad3c3fa44773dcfdd9dc44513d1f26142
% --- CHUNK_METADATA_END ---
\begin{figure}[H]
    \centering
    \incfig{symetrie-orthogonale-axe-f}
    \caption{Orthogonal symmetry with axis $F$}
    \label{fig:symetrie-orthogonale-axe-f}
\end{figure}% --- CHUNK_METADATA_START ---
% needs_review: True
% src_checksum: 414d6413494e32c606e09d080d0a277eca5d51f6763d4eb0e12e5c48f52250be
% --- CHUNK_METADATA_END ---
\begin{prop}
   Orthogonal symmetry is an isometry.
\end{prop}% --- CHUNK_METADATA_START ---
% needs_review: True
% src_checksum: 4836b9ba505742a648a3ea5786960cd719583b2ed9db26f6fb6e59a300bacb55
% --- CHUNK_METADATA_END ---
\begin{proof}
   TODO or not needed
\end{proof}% --- CHUNK_METADATA_START ---
% needs_review: True
% src_checksum: 163b9655d4c6ee6b5ed32a9e945de845daf3e519b3d872480793b91dad96ff6b
% --- CHUNK_METADATA_END ---
\begin{prop}\label{prop:isometrie-ssi-transforme-bon-en-bon}
   $f$ is an isometry if and only if it transforms every orthonormal basis into an orthonormal basis.
\end{prop}% --- CHUNK_METADATA_START ---
% needs_review: True
% src_checksum: 4ec95a52726e4981c90aca65978297d0557bf21f1532a54843088d610d510166
% --- CHUNK_METADATA_END ---
\begin{preuve}
    Let $f$ be an isometry, then it transforms any basis into a basis because $f$ is bijective by prop. \ref{prop:isometrie-bijective}.
    \begin{itemize}
        \item ($\implies$) Suppose that $f$ is an isometry. Let $\{e_i\}$ be an orthonormal basis, then we have:
             \[
                 \scalair{f(e_i), f(e_j)} = \scalair{e_i, e_j} = \delta_{i,j}
            \]
            Therefore, $\{f(e_i)\}$ is an orthonormal basis.
        \item ($\impliedby$) Suppose that there exists an orthonormal basis $\{e_i\}$ such that $\{f(e_i)\}$ is also an orthonormal basis. Moreover, let $x = x_1e_1 + \ldots x_ne_n$ and $y = y_1e_1 + \ldots + y_ne_n$ with $x_i, y_i \in \R$
            \par
            Since $\{e_i\}$ is orthonormal, then we have:
            \begin{equation}\label{eq:prod-scal-base-ortho}
                \scalair{x, y} = x_1y_1 + \ldots + x_ny_n = \sum_{i=1}^{n} x_iy_i
            \end{equation}
            On the other hand:
            \begin{align*}
                \scalair{f(x), f(y)} &= \scalair{\sum_{i=1}^{n} x_if(e_i), \sum_{i=1}^{n} y_if(e_i)} = \sum_{i,j = 1}^{n} x_iy_j\scalair{f(e_i), f(e_j)}\\
                                     &= \sum_{i,j=1}^{n} x_iy_j\scalair{e_i, e_j} \underset{\text{car } \{e_i\} \text{ orthonormée}}{=} = \sum_{i=1}^{n} x_iy_i \underset{\text{D'apres } \ref{eq:prod-scal-base-ortho}}{=} \scalair{x, y}
            \end{align*}
            Therefore $f$ is an isometry.
    \end{itemize}
\end{preuve}% --- CHUNK_METADATA_START ---
% needs_review: True
% src_checksum: 8d2950f635e9cffe3eefb6c7a9f8caceaf847f161b494f77e191990a45f5584a
% --- CHUNK_METADATA_END ---
\begin{prop}\label{prop:isometrie-ata-eg-i}
    If $\{e_i\}$ is an orthonormal basis, $f$ an isometry and $A = M(f)_{e_i}$, then $A^{T}A = I = AA^{T}$.
\end{prop}% --- CHUNK_METADATA_START ---
% needs_review: True
% src_checksum: 308e1996bec46376938c161a3389b6a69e2a68a372ac948cc0ddb54f07aa1f27
% --- CHUNK_METADATA_END ---
\begin{preuve}
    To prove this, we will use proposition \ref{prop:prod-scal-par-matrice}.
    \par
    By definition of isometry, we have:
    \begin{align*}
        &\scalair{f(x), f(y)} = \scalair{x, y} \quad \forall x, y \in E\\
        \iff &\underbrace{ (AX)^{T}(AY) }_{\scalair{f(x), f(y)}} = X^TA^TAY = \underbrace{X^TY}_{\scalair{x, y}}\\
        \iff &A^TA = I
    \end{align*}
\end{preuve}% --- CHUNK_METADATA_START ---
% needs_review: True
% src_checksum: 65746adebc98fdc96e5e56ec6d65df91030b03666332c9b11270d8eed9f46139
% --- CHUNK_METADATA_END ---
\begin{prop}
   If $A$ is a matrix of isometry in an orthonormal basis, then $det(A) = \pm 1$ 
\end{prop}% --- CHUNK_METADATA_START ---
% needs_review: True
% src_checksum: 23b9a2d19a5bed3852c398d91fb93c61fd03fadfdca45e33b28839618a75a043
% --- CHUNK_METADATA_END ---
\begin{preuve}
    By proposition \ref{prop:isometrie-ata-eg-i}, we have: $A^TA = I$, hence:
     \begin{align*}
         det(A^TA) = det(I) = 1 \implies& det(A)^2 = 1 \quad \text{ (because }  det(A^T) = det(A) \text{)}\\
                                \implies& det(A) = \pm 1
    \end{align*}
\end{preuve}% --- CHUNK_METADATA_START ---
% needs_review: True
% src_checksum: 7c8be0f83e3ecdc7e2399c8f0a68e8f7d4306d5dbcf3cf9c373f63d87616a7fd
% --- CHUNK_METADATA_END ---
\begin{intuition}
   An isometry performs a rotation or a reflection; it preserves distances, and therefore the area (or volume) of a figure constructed by the base of this transformation is equal to $1$. 
\end{intuition}% --- CHUNK_METADATA_START ---
% needs_review: True
% src_checksum: 09c3327a6fca672cd2ff7bbc61249f10cf92ae3695cbf2ec363dbf5510fec4c2
% --- CHUNK_METADATA_END ---
\subsection{Adjoint endomorphism}% --- CHUNK_METADATA_START ---
% needs_review: True
% src_checksum: f5a9abf46d1ad9c63bf6534aac796f8754fed2637a6ff0f5c45c6831dc66dfcb
% --- CHUNK_METADATA_END ---
\begin{prop}
   Let $E$ be a Euclidean space and $f \in End(E)$. There exists one and only one endomorphism $f^* \in E$ such that
   \[
       \scalair{f(x), y} = \scalair{x, f^*(y)}, \quad \forall x, y \in E
   \] 
   $f^*$ is called the \textbf{adjoint} of $f$.
   \par
   If $\{e_i\}$ is an orthonormal basis and $A = M(f)_{e_i}$, then the matrix $A^* = M(f^*)_{e_i}$ is the transpose of $A$, i.e. $A^* = A^T$
\end{prop}% --- CHUNK_METADATA_START ---
% needs_review: True
% src_checksum: 5e40b99de63f8219b8e35fcf11ba61e30bfb1741076e801edd00e6e8b82c3a16
% --- CHUNK_METADATA_END ---
\begin{preuve}
    Again, for the proof, we will use proposition \ref{prop:prod-scal-par-matrice} which is very useful, so I advise you to master this concept.
    \par
    Let $\{e_i\}$ be an orthonormal basis of $E$ and let us denote
     \[
    A = M(f)_{e_i} \quad A^* = M(f^*)_{e_i} \quad X = M(x)_{e_i} \quad Y = M(y)_{e_i}
    \] 
    Since we are in an orthonormal basis, the statement is written:
    \[
        \underbrace{(AX)^TY}_{\scalair{f(x),y}} = X^TA^TY = \underbrace{X^T(A^*Y)}_{\scalair{x, f^*(y)}} \quad \forall X, Y \in \mathcal{M}_{n, 1}(\R)
    \] 
    which implies that $A^* = A$ and, furthermore, demonstrates the uniqueness of such adjoint.
\end{preuve}% --- CHUNK_METADATA_START ---
% needs_review: True
% src_checksum: 69e2e232a1cf77e5370bdc547dbbc03cf5a33e341123571f6f13ffa4a2a97849
% --- CHUNK_METADATA_END ---
\section{Orthogonal Groups}% --- CHUNK_METADATA_START ---
% needs_review: True
% src_checksum: 30e9713b279ada9e4dc1ba4cf518680437c3f81a6bf4160c2fc7ddaa21f5608d
% --- CHUNK_METADATA_END ---
Reminder:% --- CHUNK_METADATA_START ---
% needs_review: True
% src_checksum: 7f5170f3039994d2267944bfacea33bab1b2b27457ae3c7c865342196546ad13
% --- CHUNK_METADATA_END ---
\begin{definition}\label{def:general-linear-group}
    A general linear group:
    \[
        GL(n, \R) = \{A \in \mathcal{M}_{n}(\R) \mid det(A) \neq 0\}
    \] 
    is a group of all linear transformations (square matrices) that are invertible (because $det(A) \neq 0$).
\end{definition}% --- CHUNK_METADATA_START ---
% needs_review: True
% src_checksum: 69a067bb55ef45d25eea5e96ca06f9e04e1254a999ec2722830a3aab6f48df5b
% --- CHUNK_METADATA_END ---
\begin{definition} \textbf{Orthogonal Group}:
    The set:
    \[
        O(n, \R) := \{A \in \mathcal{M}_{n}(\R) \mid A^TA = I\} = \{A \in \mathcal{M}_{n}(\R) \mid AA^T = I\}
    \] 
    satisfies the following properties:
    \begin{enumerate}
        \item if $A, B \in O(n, \R)$, then $AB \in O(n, \R)$
        \item $I \in O(n, \R)$
        \item if $A \in O(n, \R)$ then $A^{-1} \in O(n, \R)$
    \end{enumerate}
    In particular, $O(n, \R)$ is a subgroup of $GL(n, \R)$ (group of invertible matrices) (see definition \ref{def:general-linear-group}).
\end{definition}% --- CHUNK_METADATA_START ---
% needs_review: True
% src_checksum: 8ca4e70a0d8b16cb958e9eee5fa78577f55c9906dfd354069234ff3982a705a6
% --- CHUNK_METADATA_END ---
\begin{intuition}
    The meaning of orthogonal matrices is clear: they represent the matrices of orthogonal transformations (isometries) in \textbf{an orthonormal basis} (see defn \ref{def:orthogonal}).
\end{intuition}% --- CHUNK_METADATA_START ---
% needs_review: True
% src_checksum: 884c53f8606bf123e9b18d510e42e1f06bef30774481997531dad56cbbfcc4e0
% --- CHUNK_METADATA_END ---
We can notice that if $det(A) = 1$, this isometry represents a rotation; furthermore, we have the following definition:% --- CHUNK_METADATA_START ---
% needs_review: True
% src_checksum: 7d8c7b2b1e0c5e074414c9ccb9842dc4f78ec36f618d7cd8ef79e023ceeb9160
% --- CHUNK_METADATA_END ---
\begin{definition}
    The set of direct orthogonal matrices (i.e. such that $det(A) = 1$)
    \[
    SO(n, \R) = \{A \in O(n, \R) \mid det(A) = 1 \}
    \] 
    is a group, called the \textbf{special orthogonal group}.
\end{definition}% --- CHUNK_METADATA_START ---
% needs_review: True
% src_checksum: c5dbacd3e34cbf5a2300990c8929db644508fb545359d81af28c2434e33b6972
% --- CHUNK_METADATA_END ---
\begin{eg}
   The matrix
   \[
       A = \frac{1}{3} \begin{pmatrix} 2 & -1 & 2\\ 2 & 2 & -1\\ -1 & 2 & 2 \end{pmatrix} 
   \] 
   is orthogonal. We can verify that $A^TA = I$, or, it is sufficient to show that $c_1, c_2, c_3$ is an orthonormal family, i.e.:
   \[
       \|c_i\|^2 = 1 \quad \text{ and } \quad \scalair{c_i, c_j} = 0 \quad \text{ if } i \neq j
   \] 
   We can interpret $A$ as the matrix of a transformation $f$ in the canonical basis $\{e_i\}$, so we have: $c_i = f(e_i)$, according to proposition \ref{prop:isometrie-ssi-transforme-bon-en-bon} $f$ is orthogonal. Moreover, we see that $det(A) = +1$. Consequently, $f$ is a direct orthogonal transformation.
\end{eg}% --- CHUNK_METADATA_START ---
% needs_review: True
% src_checksum: 135d781de4dc50b2a9c113785c40f142345ea9689665ffb66c938b3a10834f9e
% --- CHUNK_METADATA_END ---
\begin{prop}
   The change-of-basis matrix from an orthonormal basis to an orthonormal basis is an orthogonal matrix. 
\end{prop}% --- CHUNK_METADATA_START ---
% needs_review: True
% src_checksum: 0bec23217c28ab51a0c6e64cc22eb8834a44ebb9f3d2160385f659491a79a8b1
% --- CHUNK_METADATA_END ---
\begin{preuve}
   I'm providing intuition. A transition matrix transforms one basis into another; it transforms the vectors of the basis, so it transforms the basis of the O.N.B. into vectors of the basis of the O.N.B. Therefore, according to proposition \ref{prop:isometrie-ssi-transforme-bon-en-bon}, this matrix is orthogonal.
\end{preuve}