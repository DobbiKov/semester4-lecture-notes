% --- CHUNK_METADATA_START ---
% needs_review: True
% src_checksum: d2593aec0de109b6fd1ab51dca106b64e4599f7b7be08482718767d419c69efe
% --- CHUNK_METADATA_END ---
\usepackage[utf8]{inputenc}
\usepackage[T1]{fontenc}
\usepackage{textcomp}

\usepackage{url}

%  \usepackage{hyperref}
%  \hypersetup{
%      colorlinks,
%      linkcolor={black},
%      citecolor={black},
%      urlcolor={blue!80!black}
%  }


\usepackage{graphicx}
\usepackage{float}
\usepackage[usenames,dvipsnames]{xcolor}

%  \usepackage{cmbright}


\usepackage{amsmath, amsfonts, mathtools, amsthm, amssymb}
\usepackage{mathrsfs}
\usepackage{cancel}

\newcommand\N{\ensuremath{\mathbb{N}}}
\newcommand\R{\ensuremath{\mathbb{R}}}
\newcommand\Z{\ensuremath{\mathbb{Z}}}
\renewcommand\O{\ensuremath{\emptyset}}% --- CHUNK_METADATA_START ---
% needs_review: True
% src_checksum: 487383c8026aaeee672f5dd4ad5d92bc3544de70cbfb64fa110fcba510232d42
% --- CHUNK_METADATA_END ---
\newcommand\Q{\ensuremath{\mathbb{Q}}}
\newcommand\C{\ensuremath{\mathbb{C}}}
\let\implies\Rightarrow
\let\impliedby\Leftarrow
\let\iff\Leftrightarrow
\let\epsilon\varepsilon

%  горизонтальна лінія
\newcommand\hr{
    \noindent\rule[0.5ex]{\linewidth}{0.5pt}
}

\usepackage{tikz}
\usepackage{pgfplots}
\usepackage{tikz-cd}

%  теореми
\usepackage{thmtools}
\usepackage{thm-restate}
\usepackage[framemethod=TikZ]{mdframed}
\mdfsetup{skipabove=1em,skipbelow=0em, innertopmargin=12pt, innerbottommargin=8pt}

\theoremstyle{definition}

\makeatletter

\declaretheoremstyle[
    headfont=\bfseries\sffamily\color{ForestGreen!70!black}% --- CHUNK_METADATA_START ---
% needs_review: True
% src_checksum: 06cdb61085c5b02952e27dc4272d540b497997afe34f4b08190bef7ffff9a4df
% --- CHUNK_METADATA_END ---
, bodyfont=\normalfont,
    mdframed={
        linewidth=2pt,
        rightline=false, topline=false, bottomline=false,
        linecolor=ForestGreen, backgroundcolor=ForestGreen!5,
    }
]{thmgreenbox}

\declaretheoremstyle[
    headfont=\bfseries\sffamily\color{NavyBlue!70!black}, bodyfont=\normalfont,
    mdframed={
        linewidth=2pt,
        rightline=false, topline=false, bottomline=false,
        linecolor=NavyBlue, backgroundcolor=NavyBlue!5,
    }
]{thmbluebox}

\declaretheoremstyle[
    headfont=\bfseries\sffamily\color{NavyBlue!70!black}, bodyfont=\normalfont,
    mdframed={
        linewidth=2pt,
        rightline=false, topline=false, bottomline=false,
        linecolor=NavyBlue
    }
]% --- CHUNK_METADATA_START ---
% needs_review: True
% src_checksum: 70847fcbadc8d9f3ca939892eaeaedd8409aebbc3491a605a62e34baa8a19d09
% --- CHUNK_METADATA_END ---
]{thmblueline}

\declaretheoremstyle[
    headfont=\bfseries\sffamily\color{RawSienna!70!black}, bodyfont=\normalfont,
    mdframed={
        linewidth=2pt,
        rightline=false, topline=false, bottomline=false,
        linecolor=RawSienna, backgroundcolor=RawSienna!5,
    }
]{thmredbox}

\declaretheoremstyle[
    headfont=\bfseries\sffamily\color{RawSienna!70!black}, bodyfont=\normalfont,
    numbered=no,
    mdframed={
        linewidth=2pt,
        rightline=false, topline=false, bottomline=false,
        linecolor=RawSienna, backgroundcolor=RawSienna!1,
    },
    qed=\qedsymbol
]{thmproofbox}
% --- CHUNK_METADATA_START ---
% needs_review: True
% src_checksum: 38ba5fb484d02353ca64e46dd7526b6ba04e4c8adc3a0dcfac7dc732b93b2417
% --- CHUNK_METADATA_END ---
\declaretheoremstyle[
    headfont=\bfseries\sffamily\color{NavyBlue!70!black}, bodyfont=\normalfont,
    numbered=no,
    mdframed={
        linewidth=2pt,
        rightline=false, topline=false, bottomline=false,
        linecolor=NavyBlue, backgroundcolor=NavyBlue!1,
    },
]{thmexplanationbox}

\declaretheorem[numberwithin=chapter, style=thmgreenbox, name=Definition]{definition}
\declaretheorem[sibling=definition, style=thmredbox, name=Corollary]{corollary}
\declaretheorem[sibling=definition, style=thmredbox, name=Proposition]{prop}
\declaretheorem[sibling=definition, style=thmredbox, name=Theorem]% --- CHUNK_METADATA_START ---
% needs_review: True
% src_checksum: 59c04866b292162300742f226dbea63d743816196156331f85ffdb7324acce4e
% --- CHUNK_METADATA_END ---
{theorem}
\declaretheorem[sibling=definition, style=thmredbox, name=Lemma]{lemma}
\declaretheorem[sibling=definition, style=thmbluebox,  name=Example]{eg}
\declaretheorem[sibling=definition, style=thmbluebox,  name=Nonexample]{noneg}
\declaretheorem[sibling=definition, style=thmblueline, name=Remark]{remark}




\declaretheorem[numbered=no, style=thmexplanationbox, name=Proof]{explanation}
\declaretheorem[numbered=no, style=thmproofbox, name=Proof]{replacementproof}
\declaretheorem[style=thmbluebox,  numbered=no, name=Exercise]{ex}
\declaretheorem[style=thmblueline, numbered=no, name=Note]{note}% --- CHUNK_METADATA_START ---
% needs_review: True
% src_checksum: 4531c05616fbe9675ef9c77e7846bb411b46e270bb68e8d9991fc66ad2afdb9d
% --- CHUNK_METADATA_END ---
%  \renewenvironment{proof}[1][\proofname]{\begin{replacementproof}}{\end{replacementproof}}


%  \AtEndEnvironment{eg}{\null\hfill$\diamond$}%


\newtheorem*{uovt}{UOVT}
\newtheorem*{notation}{Notation}
\newtheorem*{previouslyseen}{As previously seen}
\newtheorem*{problem}{Problem}
\newtheorem*{observe}{Observe}
\newtheorem*{property}{Property}
\newtheorem*{intuition}{Intuition}


\declaretheoremstyle[
    headfont=\bfseries\sffamily\color{RawSienna!70!black}, bodyfont=\normalfont,
    mdframed={
        linewidth=2pt,
        rightline=false, topline=false, bottomline=false,
        linecolor=RawSienna, backgroundcolor=RawSienna!5,
    }]
% --- CHUNK_METADATA_START ---
% needs_review: True
% src_checksum: e7a3c52ddebf0d765edf9885b4792e3ebac45159186b5c5d65d2ca076319d1c2
% --- CHUNK_METADATA_END ---
]{todo}
\declaretheorem[numbered=no, style=todo, name=TODO]{TODO}


\usepackage{etoolbox}

\AtEndEnvironment{vb}{\null\hfill$\diamond$}% 
\AtEndEnvironment{intermezzo}{\null\hfill$\diamond$}% 





%  http://tex.stackexchange.com/questions/22119/how-can-i-change-the-spacing-before-theorems-with-amsthm
%  \def\thm@space@setup{%
%    \thm@preskip=\parskip \thm@postskip=0pt
%  }


\usepackage{xifthen}

\def\testdateparts#1{\dateparts#1\relax}
\def\dateparts#1 #2 #3 #4 #5\relax{
    \marginpar{\small\textsf{\mbox{#1 #2 #3 #5}}}
}

\def\@lesson{}% 
\newcommand{\lesson}[3]{
    \ifthenelse{\isempty{#3}}{%
        \def\@lesson{Lecture #1}%
    }{%
        \def\@lesson{Lecture #1: #3}%
    }%
    \subsection*{\@lesson}
    \testdateparts{#2}
}
% --- CHUNK_METADATA_START ---
% needs_review: True
% src_checksum: 0c755ffa0ba0b6888a347f399f51c2323e130155e6b80a6ed99a3197969d1dc0
% --- CHUNK_METADATA_END ---
%  химерні заголовки
\usepackage{fancyhdr}
\pagestyle{fancy}

%  \fancyhead[LE,RO]{Gilles Castel}
\fancyhead[RO,LE]{\@lesson}
\fancyhead[RE,LO]{}
\fancyfoot[LE,RO]{\thepage}
\fancyfoot[C]{\leftmark}
\renewcommand{\headrulewidth}{0pt}

\makeatother

%  підтримка рисунків (https://castel.dev/post/lecture-notes-2)
\usepackage{import}
\usepackage{xifthen}
\pdfminorversion=7
\usepackage{pdfpages}
\usepackage{transparent}
\usepackage[margin=0.8in]{geometry}
\newcommand{\incfig}[1]{%
    \def\svgwidth{\columnwidth}
    \import{./figures/}{#1.pdf_tex}
}

%  %http://tex.stackexchange.com/questions/76273/multiple-pdfs-with-page-group-included-in-a-single-page-warning
% --- CHUNK_METADATA_START ---
% needs_review: True
% src_checksum: edd0e2d7b64684eac5a088619f35404ade24f868b66ad2b8b8cdbcc60819e5d8
% --- CHUNK_METADATA_END ---
\pdfsuppresswarningpagegroup=1

\usetikzlibrary{calc, arrows.meta, positioning}

\author{Єгор Коротенко}