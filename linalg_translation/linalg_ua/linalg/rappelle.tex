% --- CHUNK_METADATA_START ---
% needs_review: True
% src_checksum: 796974b4f5f9bede27e32c40566cca28b82298e100893bb1063766a54fd6f218
% --- CHUNK_METADATA_END ---
\section{Корисні поняття}% --- CHUNK_METADATA_START ---
% needs_review: True
% src_checksum: ba138866b5fd960e146e3aa2c010fc19c8212ef8a44a169d97c38b326551b6dc
% --- CHUNK_METADATA_END ---
\begin{theorem}
   du rang: Нехай дано застосування $f: E \to E'$ з $E$ та $E'$ векторними просторами, тоді: 
   \[
        dim(E) = dim(Ker(f)) + dim(Im(f)) 
   \] 
   це також означає, що $Ker(f) \oplus  Im(f) = E$ (образ і ядро знаходяться в прямій сумі). 
\end{theorem}% --- CHUNK_METADATA_START ---
% needs_review: True
% src_checksum: 041d698309c46f9b7a9ccb58ffcf376509a0452a8da537e1d1de9eb2cd7ad4be
% --- CHUNK_METADATA_END ---
\begin{intuition}
   Будь-яке лінійне відображення може бути представлене як матриця
\end{intuition}