% --- CHUNK_METADATA_START ---
% needs_review: True
% src_checksum: 73ccb62af63a937a6d3cefeabad6c5a7d3cb68369db6989cb6a2d3dc188473b7
% --- CHUNK_METADATA_END ---
\section{Ортонормовані базиси}% --- CHUNK_METADATA_START ---
% needs_review: True
% src_checksum: 48af4b80a9de6ca29bdbace6d0944a33fd8c9c15b77cb1e7025dd95b7fca5e7c
% --- CHUNK_METADATA_END ---
Нехай $(E, \scalair{,})$ буде евклідовим простором і $F \subset E$ векторним підпростором ($dim(F) < \infty$), оскільки $dim(E) < \infty$.% --- CHUNK_METADATA_START ---
% needs_review: True
% src_checksum: 1cf6ab4654a737c18a935159f01dc60a50331e445ab0fbb27d17129b05c48e8c
% --- CHUNK_METADATA_END ---
\begin{note}
    \[
        F^{\perp} := \{x \in E \mid \scalair{X, Z} = 0 \, \forall z \in F\} 
    \] 
    ортогонал до $F$.
\end{note}% --- CHUNK_METADATA_START ---
% needs_review: True
% src_checksum: c3a10a7268a935140dd832d42c2e16a56f122abbeb4749bea75affc54b3c9bde
% --- CHUNK_METADATA_END ---
\begin{theorem}
    Маємо $E = F \oplus F^{\perp}$.\\
    Зокрема, $dim(F^{\perp}) = dim(E) - dim(F)$ і $F = (F^{\perp})^{\perp}$
\end{theorem}% --- CHUNK_METADATA_START ---
% needs_review: True
% src_checksum: e86a1a6bea99ae019ad5cbd5c5d91c61a4dff3c6b217fb2f6aa9c3b347b40c36
% --- CHUNK_METADATA_END ---
\begin{proof}
   Ми повинні показати, що:
   \begin{enumerate}
       \item $F \cap F^{\perp} = \O$
       \item $E = F + F^{\perp}$ тобто  $\forall x \in E, \exists x' \in F, \, x'' \in F^{\perp}$ такий що $x = x' + x''$
   \end{enumerate}
   \begin{enumerate}
       \item Нехай $x \in F \cap F^{\perp}$\\
       $\implies$ $\scalair{X, Z} = 0 \, \forall Z \in F$ оскільки $x \in F \implies \scalair{X, X} = 0 \implies x = 0 (\scalair{,} \text{ визначено})$
        \item Нехай $x \in E$. Розглянемо  $f_x \in E^{*}$, тобто  $f_x: E \to \R, y \mapsto \scalair{x, y}$ і $f := f_{x|F}: F \to \R \implies f \in E^{*}$
            Лема Ріса $\implies$ $\exists! x' \in F$ такий що $f = f_{x'}: F \to \R, z \mapsto \scalair{x', z}\\$
            $\implies f_{x}(z) = f_{x'}(z) = f(z)\, \forall z \in F$ (Увага: не рівність для всіх $z$ у  $E$)\\
            Покладемо $x'' := x - x'$, тобто  $x = x' + x'' \in F$. Доведемо  $x'' \in  F^{\perp}$.\\
            Якщо $z \in F$,  $\scalair{x'', z} = \scalair{x - x', z} = \scalair{x, z} - \scalair{x', z} = 0$. Отже $x'' \in F^{\perp}$ і  $E = F \oplus F^{\perp}$ ($dim(E) = dim(F) + dim(F^{\perp})$) \\
            $F \subseteq (F^{\perp})^{\perp}$ оскільки $\scalair{x, z} = 0 \, \forall x \in F \, \forall z \in F^{\perp}$
            \[
                dim(F) = dim(E) - dim(F^{\perp})
            \]
            оскільки $E = G \oplus G^{\perp}$, отже  $dim(G) = dim(E) - dim(G^{\perp})$ для  $G = F^{\perp}, \, dim(F^{\perp}) = dim(G)$
   \end{enumerate}
\end{proof}% --- CHUNK_METADATA_START ---
% needs_review: True
% src_checksum: 27851f8b7925d124fd18b2554233f86afbf8232d6432f79cac746f2904377647
% --- CHUNK_METADATA_END ---
\begin{definition}
    Нехай $E$ — векторний простір, оснащений скалярним добутком $\scalair{,}$
     \begin{itemize}
         \item Сім'я $(v_i)_{i \ge 0}$ векторів з $E$ називається \underline{ортогональною}, якщо для $i \neq j$ ми маємо $\scalair{v_i, v_j} = 0$, тобто $v_i \perp v_j$
         \item Ортонормальна сім'я з $E$ — це ортогональна сім'я $(v_i)_{i \ge  0}$, така що до того ж $\|v_i\| = 1$ для $i \ge 0$
    \end{itemize}
\end{definition}% --- CHUNK_METADATA_START ---
% needs_review: True
% src_checksum: 0d2505dfb8639405790b38b3457ed812bf61a68b5c0e75296f1fa02e779ce15c
% --- CHUNK_METADATA_END ---
\begin{eg}
   \begin{enumerate}
       \item $E = \R^{n}$ оснащене стандартним скалярним добутком. Канонічний базис $(e_1, \ldots, e_n)$ є ортогональним, тому що
           \[
           \scalair{e_i, e_j} = \begin{cases}
               1 \text{ якщо } i = j\\
               0 \text{ якщо } i \neq j
           \end{cases}
           \]
       \item У $E = \mathcal{C}^{0}([-1, 1], \R)$ оснащене $\scalair{f,g} = \int_{-1}^{1} f(t)g(t)\,d{t}$. Сімейство $(\cos(t), \sin(t))$ є ортогональним. Сімейство $(1, t^2)$ не є ортогональним:
            \[
                \scalair{1, t^2} = \int_{-1}^{1} 1 t^2 \, d{t} = \frac{2}{3} \neq  0
           \]
   \end{enumerate}
\end{eg}% --- CHUNK_METADATA_START ---
% needs_review: True
% src_checksum: af5bd0f71063b14b9ef145dbe3af0af69158d553f33bee3c0be990eee7d22802
% --- CHUNK_METADATA_END ---
\begin{prop}
    Ортогональна сім'я, що складається з \underline{ненульових} векторів, є лінійно незалежною. Зокрема, ортонормована сім'я є лінійно незалежною. 
\end{prop}% --- CHUNK_METADATA_START ---
% needs_review: True
% src_checksum: 956a44f4ea8ed6a9bcd2cb11b5ea44ada4fdbb810014826686bfa90c5a387053
% --- CHUNK_METADATA_END ---
\begin{preuve}
    Припустимо, $(v_1, \ldots, v_n)$ ортогональні з $v_i \neq 0 \, \forall i = 1, \ldots, n$\\
    якщо $\sum_{j=1}^{n} \underset{\in \R}{\alpha_iv_i} = 0$, тоді  
    \[
        \forall i \in \{1, \ldots, n\} 0 = \scalair{v_i, \sum_{j=1}^{n} \alpha_jv_j} = \sum_{j=1}^{n}\alpha_j \scalair{v_i, v_j} = \alpha_i \underset{\neq 0}{\|v_i\|^2}
    \] 
    Отже, $\alpha_i = 0 \, \forall i = 1, \ldots, n$.\\
    Якщо $(v_1, \ldots, v_n)$ є ортонормальною, тоді $\|v_i\| = 1$. Отже,  $v_i \neq 0, \, \forall i = 1, \ldots, n$.
\end{preuve}% --- CHUNK_METADATA_START ---
% needs_review: True
% src_checksum: 87c48cf174cbeb60915f4f20a06e6f50a5f4ce984ed03c19114b2a5a1c37a705
% --- CHUNK_METADATA_END ---
\begin{intuition}
   Ортогональні (перпендикулярні) вектори ніколи не знаходяться один в одному (тобто $e_i = \lambda e_j$ неможливо), якщо вектори лінійно залежні, або кут $< 90º$ (отже, вектори не є ортогональними, абсурд), (вони знаходяться один в одному, вони не є ортогональними, абсурд). Отже, вони справді лінійно незалежні.
\end{intuition}% --- CHUNK_METADATA_START ---
% needs_review: True
% src_checksum: 295b6bd389a92be0c98563e3c43ed9be1871c81e35b1882caeccef39b2a1245c
% --- CHUNK_METADATA_END ---
\begin{definition}
    $(E, \scalair{,})$ евклідів простір. Сім'я $B = (e_1, \ldots, e_n)$ є ортонормальним базисом (де БОН), якщо вона є базисом і ортонормальною сім'єю.
\end{definition}% --- CHUNK_METADATA_START ---
% needs_review: True
% src_checksum: 5a8279067605c9c96f64e94f3a7c9f05a2800934a6a8a80be35d1aaaf2a73cd5
% --- CHUNK_METADATA_END ---
\begin{theorem}
    $(E, \scalair{,})$ евклідів простір. Тоді він допускає БОН.
\end{theorem}% --- CHUNK_METADATA_START ---
% needs_review: True
% src_checksum: e318537d26b99537acadb714f5cb06e6ea5ba9583d68b020578679227f59bf2c
% --- CHUNK_METADATA_END ---
\begin{preuve}
   Нехай $n := dim(E)$. Нехай $(e_1, \ldots, e_p)$ ортогональна сім'я (з точки зору потужності $p$) така що $e_i \neq 0 \, \forall i = 1, \ldots, p$.\\
Припустимо суперечливо, що $p < n$. Покладемо $F = Vect(e_1, \ldots, e_p)$. Тоді, $E = F \oplus F^{\perp}$ і $dim(F) \le p < n$. Отже $F^{\perp} \neq \{0\}$. Нехай $x \in F^{\perp}, \, x \neq 0$. Тоді, $(e_1, \ldots, e_p, x)$ є ортогональною потужності $> p$. Отже, $p = n$ і $(e_1, \ldots, e_n)$ є базисом $E$. Щоб отримати ортонормальну сім'ю $(e_1', \ldots, e_n')$ достатньо взяти $e_i' = \frac{1}{\|e_i\|}e_i \, \forall i = \{1, \ldots, n\}$.
\end{preuve}% --- CHUNK_METADATA_START ---
% needs_review: True
% src_checksum: c253feaaaf8ea9284c21cbf47c98c9c66d47171dba831f1fdad3d80d992cb660
% --- CHUNK_METADATA_END ---
\begin{prop}
    Нехай $(E, \scalar{}{})$ евклідів простір, і нехай  $(e_1, \ldots, e_n)$ ортонормальний базис $E$. Якщо  $x \in E$, маємо:
   \[
       x = \sum_{i=1}^{n} \scalar{x}{e_i}e_i
   \] 
Іншими словами, дійсне число $\scalar{x}{e_i}$ є $i^{\text{-та}}$ координата $x$ у базисі  $(e_1, \ldots, e_n)$.
\end{prop}% --- CHUNK_METADATA_START ---
% needs_review: True
% src_checksum: fc668c016669972f30b88cd5e9b2991340c3210e8aa83e0c514c9afe3127ded3
% --- CHUNK_METADATA_END ---
\begin{intuition}
    Ортогональність базису спрощує нам життя. Але спочатку невеликий вступ. Нехай векторний простір $E = \R^2$ і базис $(e_1, e_2) = (\begin{pmatrix} 1 \\ 0 \end{pmatrix}, \begin{pmatrix} 0\\ 1 \end{pmatrix})$. Нехай вектор $\vec{v} = (2, 3)$ :
    \begin{center}
        \begin{tikzpicture}
            \begin{axis}[
                scale=1,
                axis lines=middle,        % Draw axes in the middle
                xmin=-2, xmax=4,          % X-axis range
                ymin=-2, ymax=4,          % Y-axis range
                xlabel={$x$},             % Label for X-axis
                ylabel={$y$},             % Label for Y-axis
                xtick={-2,-1,0,1,2,3,4},% X-axis ticks
                ytick={-2,-1,0,1,2,3,4},% Y-axis ticks
                ]
            \draw[color=red, ->, thick] (0, 0) -- node[below]{$e_1$}(1, 0);
            \draw[color=blue, ->, thick] (0, 0) -- node[left]{$e_2$}(0, 1);
            \draw[color=green, ->] (0, 0) --node[above]{$\vec{v}$} (2, 3);

            \draw[color=gray, ->, thick] (1, 0) -- node[below]{$e_1$}(2, 0);
            \draw[color=gray, ->, thick] (2, 0) -- node[left]{$e_2$}(2, 1);
            \draw[color=gray, ->, thick] (2, 1) -- node[left]{$e_2$}(2, 2);
            \draw[color=gray, ->, thick] (2, 2) -- node[left]{$e_2$}(2, 3);

            \node[right, above] (_) at (2, 3){$(2, 3)$};
        \end{axis} 
        \end{tikzpicture}
    \end{center}
    Отже, ми можемо записати $\vec{v} = \vec{(2, 3)} = 2 \cdot \vec{e_1} + 3 \cdot \vec{e_2}$. Значення $x$ та $y$ (координати $v$) показують, скільки частин кожного базисного вектора (число може бути $\in \R$) потрібно взяти і просумувати, щоб отримати $\vec{v}$. (Простіше кажучи: наскільки далеко ми повинні піти вліво і вгору).
    \par
    У ортонормальному базисі $\scalair{v, e_i}$ вказує, скільки потрібно взяти вектора $e_i$, щоб утворити вектор  $\vec{v}$, а  $\vec{e_i}$ задає напрямок. Звідси $\scalair{v, e_1}$ еквівалентно $2$, і  $\scalair{v, e_2}$ до  $3$, потім: 
   \[
       \vec{v} = \underbrace{\scalair{v, e_1}}_{= 2} \cdot \vec{e_1} + \underbrace{\scalair{v, e_2}}_{= 3} \cdot \vec{e_2}
   \]  
   Зазвичай, щоб знайти координати в базисі, слід розв'язувати лінійну систему, тоді як ортонормальний базис дозволяє отримати їх шляхом обчислення скалярного добутку з кожним вектором базису, що значно простіше.
\end{intuition}% --- CHUNK_METADATA_START ---
% needs_review: True
% src_checksum: aea3e28cd57e20ec69577d68fb9b0a1782d005d58409320b6f31ec98254394c3
% --- CHUNK_METADATA_END ---
\begin{preuve}
    Покладемо $y := \sum_{i=1}^{n} \scalar{x}{e_i}e_i$ . Тоді, 
   \begin{align*}
       &\forall j = 1, \ldots, n,\\
       &\scalar{x - y}{e_j}\\ 
       = &\scalar{x}{e_j} - \scalar{y}{e_j}\\ 
       = &\scalar{x}{e_j} - \scalar{\sum_{i=1}^{n} \scalar{x}{e_i}e_i}{e_j}\\ 
       = &\scalar{x}{e_j} - \underbrace{ \sum_{i=1}^{n} \scalar{x}{e_i} }_{\substack{\text{винесено}\\ \text{як константу}}}\scalar{e_i}{e_j}\\ 
       = &\scalar{x}{e_j}\\ 
       -& \left(\scalar{x}{e_1}\underbrace{ \scalar{e_1}{e_j} }_{= 0} + \ldots + \scalar{x}{e_{j-1}}\underbrace{\scalar{e_{j-1}}{e_j}}_{= 0} + \scalar{x}{e_{j}}\underbrace{ \scalar{e_{j}}{e_j} }_{= 1} + \scalar{x}{e_{j+1}}\underbrace{ \scalar{e_{j+1}}{e_j} }_{= 0} + \ldots + \scalar{x}{e_{n}}\underbrace{ \scalar{e_{n}}{e_j} }_{= 0}\right)\\
        &\text{(} \forall i \neq j, \, \scalar{e_i}{e_j} = 0 \text{ оскільки це скалярний добуток ортогональних векторів)}\\ 
        &\text{(} \forall j \, \scalar{e_j}{e_j} = 1 \text{ оскільки це скалярний добуток того ж вектора)}\\
       = &\scalar{x}{e_j} - \scalar{x}{e_j}\underset{= 1}{\scalar{e_j}{e_j}} = 0
   \end{align*}
   Отже, $x - y \in Vect(e_1, \ldots, e_n)^{\perp} = E^{\perp} = \{0\}$. Отже $x = y$
\end{preuve}% --- CHUNK_METADATA_START ---
% needs_review: True
% src_checksum: 62f3efa5a94162d7abae77eaecb8119b2fe1c6b32a0943293c1aae43e7e31e74
% --- CHUNK_METADATA_END ---
\begin{corollary}
    $\forall x \in E, \, \|x\|^2 = \sum_{i=1}^{n} \scalar{x}{e_i}^2$ 
\end{corollary}% --- CHUNK_METADATA_START ---
% needs_review: True
% src_checksum: 9bfc133359e84b6700396f72b13466a8e04d0343cb89d0a7ed122d2a24ff1511
% --- CHUNK_METADATA_END ---
\begin{preuve}
    Якщо $x = \sum_{i=1}^{n} \scalar{x}{e_i}e_i = \sum_{i=1}^{n} x_ie_i$ тому
    \[
        \|x\|^2 = \scalar{\sum_{i=1}^{n} x_ie_i}{\sum_{j=1}^{n} x_je_j} = \sum_{i,j=1}^{n} x_ix_j\scalar{e_i}{e_j} = \sum_{i=1}^{n} x_i^2
    \] 
\end{preuve}% --- CHUNK_METADATA_START ---
% needs_review: True
% src_checksum: a6214a94d4e043fe5be223adf1ed2f11683d8677a04d2b0fc4052e2e34dd25c6
% --- CHUNK_METADATA_END ---
\section{Матриці та скалярні добутки}% --- CHUNK_METADATA_START ---
% needs_review: True
% src_checksum: 093eb312b5bef7229d54b74b2ee759bfaca5736436a22ee873986782bc80f654
% --- CHUNK_METADATA_END ---
\begin{prop} Нехай $(E, \scalair{,})$ евклідовий простір та $\epsilon = (e_1, \ldots, e_n)$ ортонормований базис. Нехай $f \in \mathcal{L}(E, E)$ та $A = (a_{i,j})_{1 \le i,j \le n}$ матриця, що представляє $f$ у $\epsilon$, тобто, $A = Mat_{\epsilon}(f)$
    \[
        a_{i,j} = \scalair{f(e_i), e_j} \, \forall i,j = 1, \ldots, n
    \] 
\end{prop}% --- CHUNK_METADATA_START ---
% needs_review: True
% src_checksum: dd44d42745a798aeb93703353ab8f513f389f43e7fe08fd4c6af0b3ed3b29b9c
% --- CHUNK_METADATA_END ---
\begin{preuve}
   $A$ є матрицею, стовпцями якої є вектори $f(e_j)$, записані в базисі $\epsilon$:
    \[
        A = (f(e_1) | \ldots | f(e_n))\quad f(e_j) = \begin{pmatrix} a_{1,j}\\ \ldots\\ a_{n, j} \end{pmatrix} 
   \] 
   Оскільки $\forall v \in E, \, v = c_1e_1 + \ldots c_ne_n$ тому $f(v) = c_1f(e_1) + \ldots c_nf(e_n)$ за лінійністю, отже нам залишається дослідити кожен $f(e_j)$
   \begin{align*}
       f(e_j) = a_{1, j}e_1 + \ldots a_{n, j}e_n \implies\\
       \langle f(e_j), e_i \rangle = \left\langle \sum_{k=1}^n a_{k,j} e_k, e_i \right\rangle = \sum_{k=1}^{n} a_{k,j}\scalar{e_k}{e_i} = a_{k, j}
   \end{align*}
   car $\scalar{e_k}{e_j} = \begin{cases}
       0 \text{ якщо } k \neq j\\
       1 \text{ якщо } k = j
   \end{cases}$
   Отже:
   \[
       a_{i, j} = \scalair{f(e_j), e_i}
   \] 
\end{preuve}% --- CHUNK_METADATA_START ---
% needs_review: True
% src_checksum: 3d6e562394bd16d44eacf1fdbf398a06c0ebb4899ed706e2fd0453018ad5cf75
% --- CHUNK_METADATA_END ---
Матриця векторного добутку дуже корисна в лінійній алгебрі. Перш ніж дати визначення:
\par
Нехай $E$ векторний простір скінченної розмірності $n$, простір $K$ і білінійна форма $b: E \times E \longrightarrow K$. Якщо $\{e_1, \ldots, e_n\}$ є базисом $E$, то: $x = \sum_{i=1}^{n} x_ie_i$ і $y = \sum_{j=1}^{n} y_je_j$, тоді маємо:
\[
b(x, y) = \sum_{i,j = 1}^{n} x_iy_jb(e_i, e_j)
\] 
$b$ отже, визначається знанням значень $b(e_i, e_j)$ на базі.% --- CHUNK_METADATA_START ---
% needs_review: True
% src_checksum: 48094a454bf8ba48c06e4394c13f6784cf74f6f5ae372abcf1d103d24a3d40fa
% --- CHUNK_METADATA_END ---
\begin{definition}
     Називається  \textbf{матрицею $b$} у базисі $\{e_i\}$ матриця:
      \[
          M(b)_{e_i} = \begin{pmatrix} 
              b(e_1, e_1) & b(e_1, e_2) & \ldots & b(e_1, e_n)\\
              b(e_2, e_1) & b(e_2, e_2) & \ldots & b(e_2, e_n)\\
              \ldots & \ldots & \ldots & \ldots\\
              b(e_n, e_1) & \ldots & \ldots & b(e_n, e_n)
          \end{pmatrix} 
     \] 
     Таким чином, елемент $\text{i-того}$ рядка та $\text{j-того}$ стовпця є коефіцієнтом $x_iy_j$.
\end{definition}% --- CHUNK_METADATA_START ---
% needs_review: True
% src_checksum: 62c294d50e70f069edb118954d1062019befc9fb23a60d3d2fae64b860280b3b
% --- CHUNK_METADATA_END ---
\begin{eg}
   Матриця канонічного скалярного добутку в $\R^3$ дорівнює:
   \[
       \scalair{X, Y} = x_1y_1 + x_2y_2 + x_3y_3 
   \] 
   \[
       Mat(\scalair{,})_{e_i} = \begin{pmatrix} 
            1 & 0 & 0\\
            0 & 1 & 0\\
            0 & 0 & 1
       \end{pmatrix} 
   \] 
\end{eg}% --- CHUNK_METADATA_START ---
% needs_review: True
% src_checksum: b0c83a2f49748ae4e903bde7a5deb981b49ddbea840f17eddb5300625781a59c
% --- CHUNK_METADATA_END ---
\begin{prop}\label{prop:prod-scal-par-matrice} скалярний добуток, представлений матрицею.\par
   Зазначимо:
   \begin{align*}
       \underbrace{A = M(b)_{e_i}}_{\text{матриця скалярного добутку}} && \underbrace{X = M(x)_{e_i}}_{\substack{\text{координати $x$}\\ \text{у базисі $e_i$}}} && \underbrace{Y = M(y)_{e_i}}_{\substack{\text{координати $y$}\\ \text{у базисі $e_i$}}} && (x, y \in E)
   \end{align*}
   Тоді маємо:
   \[
       b(x, y) = X^{t}AY
   \] 
\end{prop}% --- CHUNK_METADATA_START ---
% needs_review: True
% src_checksum: 9d0901364708fc729bba61d8e31be5464c242130c9e13ca350ba448dda5a614a
% --- CHUNK_METADATA_END ---
\begin{eg}
    Знову розглянемо приклад з $b = \scalair{,}$ канонічний скалярний добуток в $\R^3$. Нехай $X = \begin{pmatrix} 1 \\ 2 \\ -1 \end{pmatrix}$ та $Y = \begin{pmatrix} 2 \\ 3 \\ 1 \end{pmatrix} $ в канонічному базисі $\R^3$. Отже:
    \begin{align*}
        \scalair{x, y} = X^{t}AY &= \overbrace{(1, 2, -1)}^{X^{t}} \times \overbrace{\begin{pmatrix} 1 & 0 & 0\\ 0 & 1 & 0\\ 0 & 0 & 1 \end{pmatrix}}^{A} \times \overbrace{ \begin{pmatrix} 2 \\ 3 \\ 1 \end{pmatrix} }^{Y} \\
                                 &= \underbrace{(1, 2, -1)}_{X} \times \underbrace{ \begin{pmatrix} 2 \\ 3\\ 1 \end{pmatrix} }_{A \times Y} \\
                                 &= 1 \cdot 2 + 2 \cdot 3 + (-1) \cdot 1 = 2 + 6 - 1 = 7
    \end{align*}
\end{eg}% --- CHUNK_METADATA_START ---
% needs_review: True
% src_checksum: 54a4cd7b54b5866d578d31539dfc2487344f6e00176ef1f50e52b853fd2b6e6a
% --- CHUNK_METADATA_END ---
\begin{TODO}
   зміна базису матриці білінійної форми 
\end{TODO}% --- CHUNK_METADATA_START ---
% needs_review: True
% src_checksum: f02c7be5d857b5eed9458f065ffbe0b92fabe4a86739c6e6b8f844815fbed1b9
% --- CHUNK_METADATA_END ---
\section{Ортогональні проєкції}% --- CHUNK_METADATA_START ---
% needs_review: True
% src_checksum: 3cf8ef01da181c9899c2c7ef308c301ce20ec3f321d9dc0adeb31fc8b92fac39
% --- CHUNK_METADATA_END ---
Нехай $(E, \scalair{,})$ - евклідів простір, $F \subseteq E$ - векторний підпростір. Тоді, $E = F \oplus F^{\perp}$. Отже, $\forall x \in E$ записується як
\[
x = \underset{\in F}{x_F} + \underset{\in F^{\perp}}{x_{F^{\perp}}}
\]% --- CHUNK_METADATA_START ---
% needs_review: True
% src_checksum: 7ee17c23574b5c31f16245c762feb1689038b4764a68cdc48f11abebfaf21d0c
% --- CHUNK_METADATA_END ---
\begin{definition}
    \textbf{Ортогональна проєкція} з $E$ в $F$ — це проєкція $p_F$ з $E$ на $F$ паралельно до $F^{\perp}$, тобто
    \begin{align*}
        p_F: E = F \oplus F^{\perp} &\longrightarrow F \\
        x = x_F + x_{F^{\perp}} &\longmapsto p_F(x = x_F + x_{F^{\perp}}) = x_F
    .\end{align*}
\end{definition}% --- CHUNK_METADATA_START ---
% needs_review: True
% src_checksum: bdcdf0f2c368d93b2f418fae1ea96b243f406c45c822c89d9f5056e59d4f5329
% --- CHUNK_METADATA_END ---
\begin{remark}
   \begin{enumerate}
       \item $p_F$ є лінійним
       \item  $\forall x \in E \, p_{F}(x)$ повністю характеризується наступною властивістю:\\
           Нехай $y \in E$, тоді
            \[
                y = p_F(x) \iff \left( \underset{\implies y = x_F}{y \in F \text{ та } x - y} \in F^{\perp} \right) 
           \] 
       Зокрема $\scalair{p_F(x), x - p_F(x)} \,= 0$. Тоді, якщо $(v_1, \ldots, v_R)$ є ортонормованим базисом $F$, маємо:
            \[
                \forall x \in E, \, p_F(x) = \sum_{i=1}^{k} \scalair{x, v_i}v_i
           \] 
           Дійсно, достатньо перевірити, що вектор $y = \sum_{i=1}^{k} \scalair{x, v_i}v_i$ задовольняє:
           \[
               y \in F \text{ та } x - y \in F^{\perp}
           \] 
   \end{enumerate} 
\end{remark}% --- CHUNK_METADATA_START ---
% needs_review: True
% src_checksum: 3aa6469e224aa9391b7275942ad2ee83ef12453991db092e8687d27715531a0e
% --- CHUNK_METADATA_END ---
\begin{figure}[H]
   \centering 
\begin{tikzpicture}

% Draw the plane
\fill[gray!20] (-2,-1) -- (2,-1) -- (3,1) -- (-1,1) -- cycle;

% Draw the vectors
\draw[->, thick, black] (0,0) -- (2,2.2) node[anchor=south east] {\large $\mathbf{x}$};

\node[anchor=north, blue] (_) at ($(0,0)!0.5!(2,0)$) {\large $\text{proj}_\mathbf{F} \mathbf{x}$};
\node[anchor=west, blue] (_) at ($(2,0)!0.5!(2,2.2)$) {\large $\text{proj}_\mathbf{F^{\perp}} \mathbf{x}$};
% Add the labels for w and w perpendicular
\draw[->, thick, blue] (0,0) -- (2,0) ;
\draw[thick, black] (2,0) -- (2,3) node[anchor=west] {\large $\mathbf{F}^\perp$};
\draw[->, thick, blue] (2,0) -- (2,2.2);
\node[anchor=north west] (_) at (1.5, -0.5) {\large $\mathbf{F}$};
% Add the right angle symbol

\end{tikzpicture}
\caption{Проекція}
\label{pic:projection}
\end{figure}% --- CHUNK_METADATA_START ---
% needs_review: True
% src_checksum: fef2202683aa574defaf64635f8d48a8974d78d79ccc9dc70debe12602180d44
% --- CHUNK_METADATA_END ---
\begin{figure}[ht]
    \centering
    \incfig{projection-with-bon}
    \caption{Проєкція з ОНБ}
    \label{fig:projection-with-bon}
\end{figure}% --- CHUNK_METADATA_START ---
% needs_review: True
% src_checksum: 39fd6b59580833a0b7891f464b5fe02810e005f144922030608790b87c8ffebe
% --- CHUNK_METADATA_END ---
\begin{prop}
   Нехай $x \in E$. Тоді,
   \[
       \|x - p_F(x)\| = inf\{\|x - y\| \mid y \in F\}
   \] 
   тобто $\|x - p_F(x)\|$ є відстань від  $x$ до  $F$.\\
   Див. Figure~\ref{pic:projection}
\end{prop}% --- CHUNK_METADATA_START ---
% needs_review: True
% src_checksum: eaf28919b4717902ddaef3d4356df8d73f7d2e7f52cbe116e05394d2c107ed56
% --- CHUNK_METADATA_END ---
\begin{preuve}
   Оскільки $p_F(x) \in F$ достатньо довести, що, якщо  $y \in F$, тоді 
   \[
   \|x - p_F(x)\| \le \|x - y\|
   \] 
   Але, $\underset{(x - p_F(x)) + (p_F(x) - y)}{\|x - y\|^2} = \|x - p_F(x)\|^2 + 2\overbrace{\scalair{\overset{\in F^{\perp}}{x - p_F(x)}, \overset{\in F}{p_F(x) - y}}}{= 0} + \underbrace{\|p_F(x) - y\|^2}_{\ge 0} \ge \|x - p_F(x)\|^2$
\end{preuve}% --- CHUNK_METADATA_START ---
% needs_review: True
% src_checksum: 58448e126f08a50a09fc0715ad54230783e07f1dcf3d3cf5d1e964c99ddb6752
% --- CHUNK_METADATA_END ---
\begin{theorem}\label{thm:gram-schmidt}Грам-Шмідт\\
    Нехай $E$ — векторний простір, оснащений скалярним добутком $\scalair{,}$. Нехай $(v_1, \ldots, v_n)$ — лінійно незалежна сім'я елементів $\in E$. Тоді, існує сім'я $(w_1, \ldots, w_n)$ ортогональна така що
    \[
        \forall i = 1, \ldots, n \quad Vect(v_1, \ldots, v_i) = Vect(w_1, \ldots, w_i)
    \]
    Крім того, ця теорема дає нам метод побудови ортонормованого базису з довільного базису.
\end{theorem}% --- CHUNK_METADATA_START ---
% needs_review: True
% src_checksum: 330b4b7f1bdcbebbdd214c06eebd87c11d27badb990601b0996dd8b25cbb96d7
% --- CHUNK_METADATA_END ---
\begin{preuve} Теореми \ref{thm:gram-schmidt}
    Побудуємо ортогональний базис: $\{w_1, \ldots, w_p\}$. Спершу покладемо:
    \[
    \begin{cases}
        w_1 = v_1\\
        w_2 = v_2 + \lambda w_1, \qquad \text{де } \lambda \text{ такий, що } w_1 \perp w_2
    \end{cases}
    \] 
    Накладаючи цю умову, знаходимо:
    \[
        0 = \scalair{v_2 + \lambda w_1, w_1} = \scalair{v_2, w_1} + \lambda \|w_1\|^2
    \] 
    Оскільки $w_1 \neq 0$, отримуємо $\lambda = - \frac{\scalair{v_2, w_1}}{\|w_1\|^2}$. Зауважимо, що:
    \[
    \begin{cases}
        v_1 = w_1\\
        v_2 = w_2 - \lambda w_1
    \end{cases}
    \] 
    отже $Vect\{v_1, v_2\} = Vect\{w_1, w_2\}$.
    \par
    Після побудови $w_2$, будуємо $w_3$, поклавши:
    \begin{align*}
        &w_3 = v_3 + \mu w_1 + \nu w_2\\
        &\text{де } \mu \text{ та } \nu \text{ такі, що: } w_3 \perp w_1 \text{ та } w_3 \perp w_2
    \end{align*}
    Можна розглядати $w_3 = v_3 - \lambda' w_1 - \lambda'' w_2 $ як $w_3 = v_3 - proj_{F_2}v_3$ де $F_i = Vect\{w_1, \ldots, w_i\}$
    \begin{figure}[H]
        \centering
        \incfig{projection-with-bon-thm}
        \caption{Вектор за допомогою проекції}
        \label{fig:projection-with-bon-thm}
    \end{figure}
    Це дає
    \begin{align*}
        0 &= \scalair{v_3 + \mu w_1 + \nu w_2, w_1} = \scalair{v_3, w_1} + \mu \underset{= \|w_1\|^2}{\scalair{w_1, w_1}} + \nu \underset{= 0}{\scalair{w_2, w_1}}\\
          &= \scalair{v_3, w_1} + \mu \|w_1\|^2 
    \end{align*}
    звідки $\mu = - \frac{\scalair{v_3, w_1}}{\|w_1\|^2}$. Аналогічно, накладаючи умову, що $w_3 \perp w_2$, знаходимо $\nu = - \frac{\scalair{v_3, w_2}}{\|w_2\|^2}$. Оскільки
    \[
    \begin{cases}
        v_1 = w_1\\
        v_2 = w_2 - \lambda w_1\\
        v_3 = w_3 - \mu w_1 - \nu w_2
    \end{cases}
    \] 
    добре видно, що $Vect\{w_1, w_2, w_3\} = Vect\{v_1, v_2, v_3\}$. Тобто, $\{w_1, w_2, w_3\}$ є ортогональним базисом простору, породженого $v_1, v_2, v_3$. Тепер добре видно процес рекурсії.
    \par
    Припустимо, що ми побудували $w_1, \ldots, w_{k-1}$ для $k \le p$. Покладемо:
    \begin{align*}
        w_k &= v_k + \text{ лінійна комбінація вже знайдених векторів}\\
            &= v_k + \lambda_1w_1 + \ldots + \lambda_{k-1}w_{k-1}
    \end{align*}
    Умови $w_k \perp w_i$ (для $i \in \{1, \ldots, k-1\}$) еквівалентні:
    \[
        \lambda_i = - \frac{\scalair{v_k, w_i}}{\|w_i\|^2}
    \] 
    як це негайно перевіряється. Оскільки $v_k = w_k - \lambda_1 - \ldots - \lambda_{k-1}w_{k-1}$, за індукцією бачимо, що $Vect\{w_1, \ldots, w_k\} = Vect\{v_1, \ldots, v_k\}$ $\iff$ $\{w_1, \ldots, w_k\}$ є ортогональним базисом $Vect\{v_1, \ldots, v_k\}$.
    \par
    Нам залишається лише нормувати її, тобто  $\forall i \in \{1, \ldots, k\}$ $e_i = \frac{w_i}{\|w_i\|}$, звідки $\{e_1, \ldots, e_k\}$ є ортонормальним базисом $F = Vect\{v_1, \ldots, v_k\}$.
\end{preuve}% --- CHUNK_METADATA_START ---
% needs_review: True
% src_checksum: 751a5f3f1ef8a135bcd0fe7e3da6b01ade6279ba3c8727b2968a2107f53aa410
% --- CHUNK_METADATA_END ---
\begin{prop} Щоб зрозуміти цю пропозицію, раджу прочитати розділ \ref{sec:isometrie-et-adjoints}
    \par
   Будь-яка ортогональна проєкція є самоспряженою, тобто якщо $p$ є ортогональною проєкцією, тоді:
   \[
   p^* = p
   \] 
   У матричному записі: нехай $A$ матриця проєкції $p$, тоді:
    \[
   A^T = A
   \] 
\end{prop}
