% --- CHUNK_METADATA_START ---
% needs_review: True
% src_checksum: 50343702a42fcde4b60393fcf79ec2720a6bbf66c41089ef10a063bd7e9e2528
% --- CHUNK_METADATA_END ---
\documentclass[a4paper]{report}
\usepackage[backend=bibtex]{biblatex}
\addbibresource{refs.bib}
% --- CHUNK_METADATA_START ---
% needs_review: True
% src_checksum: c989bfd3ebf41acd1af67154558c73f359661474e55820d188f711ad6f74328a
% --- CHUNK_METADATA_END ---
\usepackage[utf8]{inputenc}
\usepackage[T2A]{fontenc}

\usepackage{lmodern}
\usepackage{textcomp}

\usepackage[ukrainian]{babel}
\usepackage[toc,page]{appendix}
% \AtBeginDocument{%
%   \addto\captionsukrainian{%
%     \renewcommand{\appendixname}{Appendix}%
%   }
% }


\usepackage{url}

\usepackage{hyperref}
\hypersetup{
    colorlinks,
    linkcolor={black},
    citecolor={black},
    urlcolor={blue!80!black}
}

\usepackage{graphicx}
\usepackage{wrapfig}
\usepackage{adjustbox}
\usepackage{float}
\usepackage[usenames,dvipsnames]{xcolor}

\usepackage{listings}

\lstset{
    language=Python,
    basicstyle=\ttfamily\footnotesize,
    keywordstyle=\color{blue},
    stringstyle=\color{red},
    commentstyle=\color{gray},
    showstringspaces=false,
    frame=single,
    numbers=left,
    numberstyle=\tiny,
    breaklines=true,
    tabsize=4
}% --- CHUNK_METADATA_START ---
% needs_review: True
% src_checksum: 35e274708155b50e1296725bef11ce565afd60f0203d9dfa979f19057f19cd55
% --- CHUNK_METADATA_END ---
%  \usepackage{cmbright}


\usepackage{amsmath, amsfonts, mathtools, amsthm, amssymb}
\usepackage{mathrsfs}
\usepackage{cancel}

\newcommand\N{\ensuremath{\mathbb{N}}}
\newcommand\R{\ensuremath{\mathbb{R}}}
\newcommand\Z{\ensuremath{\mathbb{Z}}}
\renewcommand\O{\ensuremath{\emptyset}}
\newcommand\Q{\ensuremath{\mathbb{Q}}}
\renewcommand\C{\ensuremath{\mathbb{C}}}
\let\implies\Rightarrow
\let\impliedby\Leftarrow
\let\iff\Leftrightarrow
\let\epsilon\varepsilon

%  horizontal rule
\newcommand\hr{
    \noindent\rule[0.5ex]{\linewidth}{0.5pt}
}

\usepackage{tikz}
%  \usepackage{tikzmark}
\usepackage{pgfplots}% --- CHUNK_METADATA_START ---
% needs_review: True
% src_checksum: 6f630fe01a2fd793720b662f606ec5b9269304740b72d6ae4c92bd94dbc4cf37
% --- CHUNK_METADATA_END ---
\usepackage{tikz-cd}

\usetikzlibrary{calc, arrows.meta, positioning, angles, quotes, patterns}

%  theorems
\usepackage{thmtools}
\usepackage{thm-restate}
\usepackage[framemethod=TikZ]{mdframed}
\mdfsetup{skipabove=1em,skipbelow=0em, innertopmargin=12pt, innerbottommargin=8pt}

\theoremstyle{definition}

\makeatletter

\declaretheoremstyle[
    headfont=\bfseries\sffamily\color{ForestGreen!70!black}, bodyfont=\normalfont,
    mdframed={
        linewidth=2pt,
        rightline=false, topline=false, bottomline=false,
        linecolor=ForestGreen, backgroundcolor=ForestGreen!5,
    }
]{thmgreenbox}
% --- CHUNK_METADATA_START ---
% needs_review: True
% src_checksum: 95ac1cbed3de7c4059d6e7c798f0c79ccb5f329d1ced37cb03fd86807b2b50ac
% --- CHUNK_METADATA_END ---
\declaretheoremstyle[
    headfont=\bfseries\sffamily\color{NavyBlue!70!black}, bodyfont=\normalfont,
    mdframed={
        linewidth=2pt,
        rightline=false, topline=false, bottomline=false,
        linecolor=NavyBlue, backgroundcolor=NavyBlue!5,
    }
]{thmbluebox}

\declaretheoremstyle[
    headfont=\bfseries\sffamily\color{NavyBlue!70!black}, bodyfont=\normalfont,
    mdframed={
        linewidth=2pt,
        rightline=false, topline=false, bottomline=false,
        linecolor=NavyBlue
    }
]{thmblueline}

\declaretheoremstyle[
    headfont=\bfseries\sffamily\color{RawSienna!70!black}% --- CHUNK_METADATA_START ---
% needs_review: True
% src_checksum: 21247ef5cc586e85a1629d9070830990e59d62fa8adf694527922c776d6a4406
% --- CHUNK_METADATA_END ---
, bodyfont=\normalfont,
    mdframed={
        linewidth=2pt,
        rightline=false, topline=false, bottomline=false,
        linecolor=RawSienna, backgroundcolor=RawSienna!5,
    }
]{thmredbox}

\declaretheoremstyle[
    headfont=\bfseries\sffamily\color{RawSienna!70!black}, bodyfont=\normalfont,
    numbered=no,
    mdframed={
        linewidth=2pt,
        rightline=false, topline=false, bottomline=false,
        linecolor=RawSienna, backgroundcolor=RawSienna!1,
    },
    qed=\qedsymbol
]{thmproofbox}

\declaretheoremstyle[
    headfont=\bfseries\sffamily\color{NavyBlue!70!black}, bodyfont=% --- CHUNK_METADATA_START ---
% needs_review: True
% src_checksum: cb48ecae9ffd2aea7668e1c099876d7ec238ed1c52f772495f3069585721f181
% --- CHUNK_METADATA_END ---
\normalfont,
    numbered=no,
    mdframed={
        linewidth=2pt,
        rightline=false, topline=false, bottomline=false,
        linecolor=NavyBlue, backgroundcolor=NavyBlue!1,
    },
]{thmexplanationbox}

\declaretheorem[numberwithin=chapter, style=thmgreenbox, name=Definition]{definition}
\declaretheorem[sibling=definition, style=thmredbox, name=Corollary]{corollary}
\declaretheorem[sibling=definition, style=thmredbox, name=Proposition]{prop}
\declaretheorem[sibling=definition, style=thmredbox, name=Theorem]{theorem}
\declaretheorem[sibling=definition, style=thmredbox, name=Lemma]{lemma}
% --- CHUNK_METADATA_START ---
% needs_review: True
% src_checksum: 133ec0a8c08cd33dfdd103df18f548a0bc889ed3e3af9f5aeaa294e7705e2cd8
% --- CHUNK_METADATA_END ---
\declaretheorem[sibling=definition, style=thmbluebox,  name=Example]{eg}
\declaretheorem[sibling=definition, style=thmbluebox,  name=Nonexample]{noneg}
\declaretheorem[sibling=definition, style=thmblueline, name=Remark]{remark}




\declaretheorem[numbered=no, style=thmexplanationbox, name=Proof]{explanation}
\declaretheorem[numbered=no, style=thmproofbox, name=Proof]{preuve}
\declaretheorem[style=thmbluebox,  numbered=no, name=Exercise]{ex}
\declaretheorem[style=thmblueline, numbered=no, name=Note]{note}

%  \renewenvironment{proof}[1][\proofname]{\begin{replacementproof}}{\end{replacementproof}}
% --- CHUNK_METADATA_START ---
% needs_review: True
% src_checksum: ee40156a4f68d7cba674746ec43032913865317fcc31679bbbd0032787911a63
% --- CHUNK_METADATA_END ---
%  \AtEndEnvironment{eg}{\null\hfill$\diamond$}%


\newtheorem*{uovt}{UOVT}
\newtheorem*{notation}{Notation}
\newtheorem*{previouslyseen}{As previously seen}
\newtheorem*{problem}{Problem}
\newtheorem*{observe}{Observe}
\newtheorem*{property}{Property}
\newtheorem*{intuition}{Intuition}


\declaretheoremstyle[
    headfont=\bfseries\sffamily\color{RawSienna!70!black}, bodyfont=\normalfont,
    mdframed={
        linewidth=2pt,
        rightline=false, topline=false, bottomline=false,
        linecolor=RawSienna, backgroundcolor=RawSienna!5,
    }
]{todo}
\declaretheorem[numbered=no, style=todo, name=TODO]
% --- CHUNK_METADATA_START ---
% needs_review: True
% src_checksum: 99fe31c94cf82c099d90ca98f2fe23181e39cd0b0d125335c7eccda8a49edf0f
% --- CHUNK_METADATA_END ---
{TODO}


\usepackage{etoolbox}

\AtEndEnvironment{vb}{\null\hfill$\diamond$}% 
\AtEndEnvironment{intermezzo}{\null\hfill$\diamond$}% 





%  http://tex.stackexchange.com/questions/22119/how-can-i-change-the-spacing-before-theorems-with-amsthm
%  \def\thm@space@setup{%
%    \thm@preskip=\parskip \thm@postskip=0pt
%  }


\usepackage{xifthen}

\def\testdateparts#1{\dateparts#1\relax}
\def\dateparts#1 #2 #3 #4 #5\relax{
    \marginpar{\small\textsf{\mbox{#1 #2 #3 #5}}}
}

\def\@lesson{}% 
\newcommand{\lesson}[3]{
    \ifthenelse{\isempty{#3}}{%
        \def\@lesson{Lecture #1}%
    }{%
        \def\@lesson{Lecture #1: #3}%
    }%
    \subsection*{\@lesson}
    \testdateparts{#2}
}% --- CHUNK_METADATA_START ---
% needs_review: True
% src_checksum: 0c755ffa0ba0b6888a347f399f51c2323e130155e6b80a6ed99a3197969d1dc0
% --- CHUNK_METADATA_END ---
%  химерні заголовки
\usepackage{fancyhdr}
\pagestyle{fancy}

%  \fancyhead[LE,RO]{Gilles Castel}
\fancyhead[RO,LE]{\@lesson}
\fancyhead[RE,LO]{}
\fancyfoot[LE,RO]{\thepage}
\fancyfoot[C]{\leftmark}
\renewcommand{\headrulewidth}{0pt}

\makeatother

%  підтримка рисунків (https://castel.dev/post/lecture-notes-2)
\usepackage{import}
\usepackage{xifthen}
\pdfminorversion=7
\usepackage{pdfpages}
\usepackage{transparent}
\usepackage[margin=0.8in]{geometry}
\newcommand{\incfig}[1]{%
    \def\svgwidth{\columnwidth}
    \import{./figures/}{#1.pdf_tex}
}

%  %http://tex.stackexchange.com/questions/76273/multiple-pdfs-with-page-group-included-in-a-single-page-warning
% --- CHUNK_METADATA_START ---
% needs_review: True
% src_checksum: 446f4d1f4015c7ba76adb15825ad69e5cfffc38fc5d825d0013371f876db7739
% --- CHUNK_METADATA_END ---
\pdfsuppresswarningpagegroup=1
\pgfplotsset{compat=1.11}
\usepackage{subcaption}

\author{Єгор Коротенко}

\newcommand{\scalar}[2]{\langle #1, #2 \rangle}
\newcommand{\scalair}[1]{\left\langle #1 \right\rangle}

%  fancy chapters
\usepackage{lipsum}
\usepackage[Lenny]{fncychap}
\ChNameUpperCase
\ChNumVar{\fontencoding{T2A}\fontsize{40}{42}\fontfamily{ptm}\selectfont}
\ChTitleVar{\Large\sc}

\title{Нотатки до курсу Лінійної Алгебри 2}% --- CHUNK_METADATA_START ---
% needs_review: True
% src_checksum: 49d34b0366eb0a68eb35c75d6d8aeb1d60be3c49779501a7e1cdf4941acdab46
% --- CHUNK_METADATA_END ---
\begin{document}% --- CHUNK_METADATA_START ---
% needs_review: True
% src_checksum: 66cd782567cea37bd98f32fc53f26630d55f24107cbb50d091ef757c8931466d
% --- CHUNK_METADATA_END ---
\maketitle% --- CHUNK_METADATA_START ---
% needs_review: True
% src_checksum: 2e77d5221beddb246cad8b2818266702eac135e1d92a2af3eaecaef2eb99f01a
% --- CHUNK_METADATA_END ---
\begin{abstract}
    Це мої нотатки, зроблені для курсу лінійної алгебри 2 в Університеті Париж-Сакле, який викладає професор Йоганнес Аншютц.
    Основна частина цих нотаток посилається на книгу "Лінійна алгебра", написану Жозефом Гріфоном \cite{grifone}.

   Мої нотатки з інших предметів доступні на моєму сайті: \href{https://dobbikov.com/lecture_notes}{dobbikov.com}

  Ці нотатки перекладені українською та англійською мовами за допомогою інструменту \texttt{sci-trans-git} \cite{korotenko-sci-trans-git}
\end{abstract}% --- CHUNK_METADATA_START ---
% needs_review: True
% src_checksum: d5651f07d4547738bb40415742f67c52ad35f597c6eb6a04c2fd094194cc8aa7
% --- CHUNK_METADATA_END ---
\tableofcontents
% --- CHUNK_METADATA_START ---
% needs_review: True
% src_checksum: 5da0735687c2aea3697e5e896ffcb2de79d53828c4ce70778e3f5615aecdb14a
% --- CHUNK_METADATA_END ---
\chapter{Euclidean Spaces}% --- CHUNK_METADATA_START ---
% needs_review: True
% src_checksum: 09c6effe383b397c2e287b2302b5c82f4782172b70cdf3be336a68a99ec35441
% --- CHUNK_METADATA_END ---
“Classical” linear algebra deals with vector spaces, where we only talk about linear combinations, subspaces, bases, matrices, etc. At some point, this is no longer sufficient. To be able to explore stronger, more complex, and useful notions, we will need to calculate the length of a vector, the angles between two vectors, the relative positioning between vectors, etc. To be able to study these concepts, we introduce the notion of a dot product (bilinear form) and then the vector spaces equipped with this product.

This chapter is devoted to the study of the two main notions:% --- CHUNK_METADATA_START ---
% needs_review: True
% src_checksum: 70055e76d35e68266dcc2b09983fa30f0cc425313d489a398c66762a9624032b
% --- CHUNK_METADATA_END ---
\begin{itemize}
    \item scalar products
    \item Euclidean spaces
\end{itemize}% --- CHUNK_METADATA_START ---
% needs_review: True
% src_checksum: 479ce8267a8dc66449e53b838492edeab11f9f116eed87da087547c1cd388ae6
% --- CHUNK_METADATA_END ---
\section{Introduction}% --- CHUNK_METADATA_START ---
% needs_review: True
% src_checksum: 0ac44e2012e6f37806b8ec7c72ad867302be795f6c99ca78cd8ba87ccba55dd5
% --- CHUNK_METADATA_END ---
The vector spaces considered in this chapter are real. We assume that $E$ is an $\R$-vector space.

Scalar product:% --- CHUNK_METADATA_START ---
% needs_review: True
% src_checksum: aac3ae1bec48ecd43cdb867de27dbe9d4e32a291690935f3e07ad94ea30b3963
% --- CHUNK_METADATA_END ---
\begin{definition}
    A bilinear form on $E$ is a map
    \begin{align*}
        B: E \times E &\longrightarrow \R \\
        (u, v) &\longmapsto B((u, v))
    \end{align*}
    that satisfies the following conditions $\forall u, v, w \in E$ $\forall \lambda \in \R$:
    \begin{enumerate}
        \item $B(u + \lambda v, w) = B(u, w) + \lambda B(v, w)$
        \item  $B(u, v + \lambda w) = B(u, v) + \lambda B(v, w)$
    \end{enumerate}
    B is said to be
    \begin{enumerate}
        \item symmetric if $B(u, v) = B(v, u)$  $\forall u, v \in E$
        \item positive if $B(., u) \ge 0 \, \forall u \in E$
        \item defined if $B(u, u) = 0 \iff u = 0$
    \end{enumerate}
\end{definition}% --- CHUNK_METADATA_START ---
% needs_review: True
% src_checksum: b3e3014d86bf527aed2249534faff461531a00e3c7469faa8feaef31f7788090
% --- CHUNK_METADATA_END ---
\begin{notation}
   Scalar product is denoted: $<u, v>$ 
\end{notation}% --- CHUNK_METADATA_START ---
% needs_review: True
% src_checksum: 8c68c516bc5901f21ff7541f291db7303a684176baa8fee049e9397452534c86
% --- CHUNK_METADATA_END ---
\begin{eg}.
   \begin{enumerate}
       \item $E = \R^n$,  $X = (x_1, \ldots, x_n), Y = (y_1, \ldots, y_n) \in E$\\
           \[
               <X, Y> := \sum_{n=1}^{n} x_iy_i
           \] 
           It is called the "canonical (or usual) scalar product".
        \item $E = \R^2$ and  $<X, Y> = 2x_1y_1 + x_2y_2$
        \item $E = \mathcal{C}^0([-1, 1], \R) \ni f, g$ (a space of continuous functions)
            \[
                <f, g> := \int_{-1}^{1} f(t) \cdot g(t) \: d{t} 
            \] 
        \item $E = \mathcal{M}_n(\R) \ni A, B$
             \[
            <A, B> := Tr(A^tB)
            \] 
   \end{enumerate} 
\end{eg}% --- CHUNK_METADATA_START ---
% needs_review: True
% src_checksum: 3754243cd92b07a4e8a3690f08a016f59688f2b965a017a38bca50239762e19f
% --- CHUNK_METADATA_END ---
\begin{prop}
    A non-zero vector space has an infinite number of different scalar products.
\end{prop}% --- CHUNK_METADATA_START ---
% needs_review: True
% src_checksum: 73752ce65fbe2fdff7cab331cd5cc247e172ed8b55063a0cfebe8d6a4a21be40
% --- CHUNK_METADATA_END ---
\begin{definition}
    A Euclidean space is a pair $(E, < . >)$ where $E$ is a $\R$-vector space \underline{of finite dimension} and $< . >$ is an inner product on $E$.
\end{definition}% --- CHUNK_METADATA_START ---
% needs_review: True
% src_checksum: de6e7256acedf77ba6b59dd8c8638e3b17598fcb2783a559a6ba2e747ecc603d
% --- CHUNK_METADATA_END ---
\begin{property} Let $(E, < . >)$ be a Euclidean space. We define:
    \[
    \|X\| := \sqrt{<X,X>} \qquad X \in E 
    \] 
    the norm (or length) of $X$. (It is well defined because $<., .>$ is always positive)
\end{property}% --- CHUNK_METADATA_START ---
% needs_review: True
% src_checksum: a0c34c74b776c4b2712e0bc68ea2baf5a169429392c9041fa9951ee767e094fd
% --- CHUNK_METADATA_END ---
\begin{property}
   Let $X, Y \in E$ be, then:
   \[
       \|X + Y\|^2 = \|X\|^2 + 2\scalair{X, Y} + \|Y\|^2
   \] 
\end{property}% --- CHUNK_METADATA_START ---
% needs_review: True
% src_checksum: 8432f24eca144a9f41a6cbb5161d6ea135f840aad65be511c92557c145409cf5
% --- CHUNK_METADATA_END ---
\begin{proof}
   \begin{align*}
       \|X + Y\|^2 = \sqrt{\scalair{X + Y, X + Y}}^2 &= \scalair{X + Y, X + Y} \\ 
                                                     &= \scalair{X, X + Y} + \scalair{Y, X + Y}  \\
                                                     &= \scalair{X, X} + \scalair{X, Y} + \scalair{Y, X} + \scalair{Y, Y}\\
                                                     &= \|X\|^2 + 2\scalair{X, Y} + \|Y\|
   \end{align*} 
\end{proof}% --- CHUNK_METADATA_START ---
% needs_review: True
% src_checksum: 01d462ad6bb3b1ce2908c760049b50ee548204ecba31f845da8c1db0f8bf13f6
% --- CHUNK_METADATA_END ---
\begin{lemma}\label{lemma:inegalite-cauchy-schwarz} Cauchy-Schwarz inequality
   We have
   \[
   |<u, v>| \le \|u\| \cdot \|v\| \qquad \forall u, v \in E
   \] 
   with equality if and only if $u$ and  $v$ are collinear, i.e  $\exists \, t \in R$ such that $u = tv$ or  $v = tu$
\end{lemma}% --- CHUNK_METADATA_START ---
% needs_review: True
% src_checksum: dd7937806464a5e6da7e6b458df5a7ae1a082e9a65aec713595521a1e6323273
% --- CHUNK_METADATA_END ---
\begin{explanation}
   If $v = 0$, clear\\
   If $v \neq 0$ we consider $\forall t \in \R$
   \begin{align*}
       \|u + tv\|^2 &= <u + tv, u + tv>\\ 
                    &= <u, u + tv> + t<v, u + tv>\\
                    &= <u, u> + t<u, v> + t<v, u> + t^2<v, v>\\
                    &= \|u\|^2 + 2t<u, v> + t^2 \|v\|^2 = f(t)
   \end{align*}
   \begin{center}
       \begin{tikzpicture}
           \begin{axis}[
               axis lines = center,
               xlabel = $x$,
               ylabel = $y$,
               samples = 100,
               domain = -3:6,
               xmin = -3, xmax = 6,
               ymin = -3, ymax = 6,
               width=5cm,
               height=5cm
               ]
               \addplot[red, thick] {(x - 3)^2 + 2};
               \addplot[blue, thick] {(x + 1)^2};
           \end{axis}
       \end{tikzpicture}
   \end{center}
Case 1: $f(t)$ has no distinct roots
\begin{align*}
    &\Delta = 4<u, v>^2 = 4\|u\|^2\|v\|^2 \le 0\\
    \implies & <u, v>^2 \le \|u\|^2 \cdot \|v\|^2\\
    \implies & |<u, v>| \le \|u\|\|v\|
\end{align*}
Case 2: $f(t)$ has only one root:\\
\begin{align*}
    &\Delta = 0\\
    \implies & \exists t \in \R \text{ tq } \|u + tv\|^2 = 0\\
    \implies &u + tv = 0 \implies u = -tv
\end{align*}
\end{explanation}% --- CHUNK_METADATA_START ---
% needs_review: True
% src_checksum: c1714456eb2dec6573390652b4931da225ab71012c26330b844da6910872d60d
% --- CHUNK_METADATA_END ---
The following definition will be studied in the analysis course:% --- CHUNK_METADATA_START ---
% needs_review: True
% src_checksum: 7f4d91e24d8e87eb9d751db268d06f36046760b562f9679cbc4ec4ec849c4e91
% --- CHUNK_METADATA_END ---
\begin{definition}
    We say that $N: E \to \R_+$ is a norm if:
    \begin{enumerate}
        \item $N(\lambda u) = |\lambda| \cdot N(u)$ \quad  $\forall \lambda \in \R, \forall u \in E$
        \item $N(u) = 0 \implies u = 0$
        \item $N(u + v) \le N(u) + N(v)$ \quad $\forall u, v \in E$
    \end{enumerate}
\end{definition}% --- CHUNK_METADATA_START ---
% needs_review: True
% src_checksum: 29fe003cc6ce4d1aea6b0410774a5d08729125a2118436a2ce7784a58c116681
% --- CHUNK_METADATA_END ---
\begin{lemma}
   The application
   \[
   \sqrt{<.,.>} = \| . \|: E \to \R_+ 
   \] 
   is called a Euclidean norm.
\end{lemma}% --- CHUNK_METADATA_START ---
% needs_review: True
% src_checksum: 7c736f0253c57b9007eb5f8c29d4e29b3a3e6059a266b244847745b6ff0c0277
% --- CHUNK_METADATA_END ---
\begin{explanation}
    1) and 2) are done\\
    \begin{itemize}
        \item $\| u + v \|^2 = \|u\|^2 + 2<u,v> + \|v\|^2 \le \|u\|^2 + 2\|u\|\|v\| + \|v\|^2 = (\|u\| + \|v\|)^2$
            \[
            \implies \|u + v\|^2 \le \|u\|^2 + \|v\|^2
            \] 
    \end{itemize}
\end{explanation}% --- CHUNK_METADATA_START ---
% needs_review: True
% src_checksum: 6fc2f21b40ab21daed7de5d63fc64583726c71d583db47bd8bf31d39bc96c70e
% --- CHUNK_METADATA_END ---
\begin{prop}
   We have the following identities $\forall u, v \in E$ 
   \begin{enumerate}
       \item Parallelogram identity:
           \[
           \|u + v\|^2 + \|u - v\|^2 = 2(\|u^2\| + \|v\|^2)
           \] 
       \item Polarization identity:
           \[
               \scalair{u, v} = \frac{1}{4}(\|u + v\|^2 - \|u - v\|^2)
           \] 
   \end{enumerate}
\end{prop}% --- CHUNK_METADATA_START ---
% needs_review: True
% src_checksum: 8ea48ceb96f53479a13cb0809bb9b4c21c511cb1436fe021b40a9523876d5a28
% --- CHUNK_METADATA_END ---
\begin{explanation}.
   \begin{enumerate}
       \item 
           \begin{align*}
               \|u + v\|^2 &= \scalair{u + v, u + v}\\
                           &= \|u\|^2 + 2\scalair{u,v} + \|v\|^2
           \end{align*}
       \item $\|u - v\|^2 = \|u\|^2 - 2\scalair{u, v} + \|v\|^2$
   \end{enumerate} 
   For a:
   \begin{itemize}
       \item 
           $(1) + (2)$:  $\|u + v\|^2 + \|u - v\|^2 = 2 (\|u\|^2 + \|v\|^2)$
       \item $(1) - (2)$:  $\|u + v\|^2 - \|u - v\|^2 = 4\scalair{u, v}$ 
   \end{itemize}
\end{explanation}% --- CHUNK_METADATA_START ---
% needs_review: True
% src_checksum: c705ccdbbe83531645858a091730563d185c1d4357934fea51fc437d8c8d25c7
% --- CHUNK_METADATA_END ---
\section{Orthogonality}% --- CHUNK_METADATA_START ---
% needs_review: True
% src_checksum: 3b7793f1f64934c9b08010a74f388a75b5aebd58d5c0eabfab6bb9d63a13877c
% --- CHUNK_METADATA_END ---
Let $E$ be an $\R$-vector space and $\scalair{ , }$ an inner product on $E$.% --- CHUNK_METADATA_START ---
% needs_review: True
% src_checksum: a50d257209420676b32029c180974a5ba8ce99821cf6bb087f70ea743ed18003
% --- CHUNK_METADATA_END ---
\begin{definition}\label{def:orthogonal}
     $u, v \in E$ are said to be \underline{orthogonal} if $<u, v> = 0$. We denote $u \perp v$
      \begin{itemize}
         \item Two subsets $A, B$ of $E$ are orthogonal if:
              \[
             \forall u \in A, \forall v \in B, \quad <u, v> = 0
             \] 
         \item If $A \subseteq E$ we call the \textbf{orthogonal of $A$}, denoted $A^{\perp}$, the set
              \[
                  A^{\perp} = \{ u \in E \mid <u, v> = 0 \quad \forall v \in A \}
             \]
             Also known as \textbf{orthogonal complement of $A$}
         \item A family $(v_1, \ldots, v_n)$ of vectors in $E$ is said to be orthogonal if $\forall i \neq j, v_i \perp v_j$. It is said to be orthonormal if it is orthogonal and additionally $\|v_i\| = 1 \quad \forall i \in \{ 1, \ldots, n \}$
     \end{itemize}
\end{definition}% --- CHUNK_METADATA_START ---
% needs_review: True
% src_checksum: 0fea0c9eedad08e574166dbfecd45aa77a49a34b6c1afc0c8c3ca35b679e2395
% --- CHUNK_METADATA_END ---
\begin{eg}
   $E = \R^n$, $< , >$ canonical scalar product
   \[
       v_i = (0, \ldots, 0, \underbrace{1}_{i}, 0, \ldots, 0)
   \] 
   \[
   <v_i, v_j> = \begin{cases}
       1 \text{ si } i = j\\  
       0 \text{ si } i \neq  j
   \end{cases}
   \] 
   $(v_1, \ldots, v_n)$ is a canonical basis
\end{eg}% --- CHUNK_METADATA_START ---
% needs_review: True
% src_checksum: 6fe38b98e0c58272e2277c935867cc0f2da465ae23e70ab50ffbf4a07fbe2003
% --- CHUNK_METADATA_END ---
\begin{prop}
    \begin{enumerate}
        \item 
            If $A \subseteq E$ then $A^{\perp}$ is a vector subspace of $E$ 
        \item If $A \subseteq B$ then $B^{\perp} \subseteq A^{\perp}$
        \item $A^{\perp} = Vect(A)^{\perp}$
        \item $A \subset (A^{\perp})^{\perp}$ 
    \end{enumerate}
\end{prop}% --- CHUNK_METADATA_START ---
% needs_review: True
% src_checksum: 19b7087b6a48384d722d6b841661a817b54245a8f5f18649f66ec1bddb8ac4ac
% --- CHUNK_METADATA_END ---
\begin{explanation}
   Exercise
\end{explanation}% --- CHUNK_METADATA_START ---
% needs_review: True
% src_checksum: 0c811ad2e1c9baccb837657ac7fe9c9e79bf8a8869817013fbeebbc1fee459ad
% --- CHUNK_METADATA_END ---
\begin{eg}
   \begin{enumerate}
       \item $E = \mathcal{C}^0([-1, 1], \R)$
            \[
                <f, g> := \int_{-1}^{1} f(t) \cdot g(t) \: d{t} 
            \] 
            \begin{center}
       \begin{tikzpicture}
           \begin{axis}[
               axis lines = center,
               xlabel = $x$,
               ylabel = $y$,
               samples = 100,
               domain = -4:4,
               xmin = -4, xmax = 4,
               ymin = -2, ymax = 2,
               width=10cm,
               height=5cm
               ]
               \addplot[red, thick] {sin(deg(x))};
           \end{axis}
       \end{tikzpicture}
   \end{center}
   Then, $f(t) = \cos(t)$, $g(t) = \sin(t)$ are orthogonal: $2\cos(t)\sin(t) = \sin(2t)$
   \[
       \int_{-1}^{1} \cos(t)\sin(t)\:d{t} = \frac{1}{2}\int_{-1}^{1} \sin(2t) \: d{t} = 0  
   \] 
   \end{enumerate} 
\end{eg}% --- CHUNK_METADATA_START ---
% needs_review: True
% src_checksum: 34a1b6efd8afd5a191fedcc4063c7162e4ff251b3b80e839ad5d5aa5c887835f
% --- CHUNK_METADATA_END ---
\begin{definition}
    If $E$ is a Euclidean space, the set
    \[
        L(E, \R) = \{ f: E \to \R \mid f \text{ is linear}\}
    \] is called the "dual of $E$".
    It is denoted $E^*$. An element $f \in E^*$ is called a linear form.
\end{definition}% --- CHUNK_METADATA_START ---
% needs_review: True
% src_checksum: 3b0fe845bcfce365ebe09bee7ac9bde0d0486f12b24754d48e80e6331fd867ae
% --- CHUNK_METADATA_END ---
Recall:% --- CHUNK_METADATA_START ---
% needs_review: True
% src_checksum: c62046bf44d7a01ed59a1a4b7bea4075b61a00b15df8ed0bc5edfae3be41e91d
% --- CHUNK_METADATA_END ---
\begin{prop}
    If $F, F'$ are two finite-dimensional vector spaces, then $dim(L(F, F')) = dim(F)\cdot dim(F')$\\
    In particular, $dim(F^*) = dim(F)$. Indeed, if $n = (e_1, \ldots, e_p)$ is a basis of $F$ and $n' = (e'_1, \ldots, e'_q)$ is a basis of $F'$, then the mapping
    \begin{align*}
        : L(F, F') &\longrightarrow Mat_{f\times p}(\R) \\
        f &\longmapsto (f) = Mat_{n,n'}(f)
    .\end{align*}
    is an isomorphism. Therefore $dim(F, F) = qp$
\end{prop}% --- CHUNK_METADATA_START ---
% needs_review: True
% src_checksum: c9aa8b287720d3caad43357c6f50e57667cb4596ca051151224ebd08f7781253
% --- CHUNK_METADATA_END ---
\begin{theorem}
    Rank Theorem: If  $F$ is a finite-dimensional vector space and  $f: F \to F'$ is linear, then $dim(F) = dim(Ker(f)) + dim(Im(f))$
\end{theorem}% --- CHUNK_METADATA_START ---
% needs_review: True
% src_checksum: d2a5584e9f9f2893dffad3bfa59128e0eff437ca2b8cc8fdfdca03e8312c5f4a
% --- CHUNK_METADATA_END ---
\begin{prop}
    If $F, F'$ are two finite-dimensional vector spaces \underline{such that} $dim(F) = dim(F')$ and $f: F \to F'$ is linear, then $f$ is an isomorphism $\iff Ker(f) = {0}$
\end{prop}% --- CHUNK_METADATA_START ---
% needs_review: True
% src_checksum: 6c10e113159dc42fee26f775e32f5b658f01d15a5a580ec4fb86fb5c44d8a610
% --- CHUNK_METADATA_END ---
\begin{explanation}
   Recall that if $G, G'$ are finite-dimensional subspaces in the same vector space, then:
   \[
   G = G' \iff G \subseteq G' \text{ and } dim(G) = dim(G')
   \] 
   $\implies$) $f$ is injective  $\implies$ $Ker(f) = {0}$\\
   $\impliedby$) Let $Ker(f) = {0}$.\\
   Then, necessarily  $dim(Ker(f)) = 0$ and by the rank theorem, we have  $dim(F) = dim(Im(f))$, so  $Im(f) = F'$
\end{explanation}% --- CHUNK_METADATA_START ---
% needs_review: True
% src_checksum: 96310fb223d39512a621685fe14d4231b8b736cf346ef901b9446c1f0f46329a
% --- CHUNK_METADATA_END ---
\begin{lemma} Riesz's Lemma:\\
    Let $(E, \scalair{.,.})$ be a finite-dimensional Euclidean space and $f \in E^*$. Then, $\exists! u \in E$ such that $f(x) = \scalair{u, x} \quad \forall x \in E$. The linear form $f$ is given by an inner product with a vector.
\end{lemma}% --- CHUNK_METADATA_START ---
% needs_review: True
% src_checksum: db83365bab1f1a066c9d4aceecc2aaa09b915c739ca3c1af5b5b251e538fbf9e
% --- CHUNK_METADATA_END ---
\begin{notation}
   For any $v \in E$, we denote by $f_v$ the mapping:
   \begin{align*}
       f_v: E &\longrightarrow \R \\
       x &\longmapsto f_v(x) = <v, x>
   .\end{align*}
   $f_v$ is linear $\forall v \in E$, i.e. $E^*$
\end{notation}% --- CHUNK_METADATA_START ---
% needs_review: True
% src_checksum: b785e304a80823e3776a8546fe9e9abdab13358fac4e47eabd1971e31b5ed134
% --- CHUNK_METADATA_END ---
\begin{explanation} Riesz Lemma\\
   Consider the mapping
   \begin{align*}
       \phi: E &\longrightarrow E^* \\
       v &\longmapsto \phi(v) = f_v
   .\end{align*}
   $\phi$ is linear (exercise). $\phi$ is injective:
   \[
   v \in Ker(\phi) \iff f_v(x) = 0 \quad \forall x \in E
   \] 
   in particular for x = v, we have:
   \[
   0 = f_v(v) = <v,v> \implies v = 0
   \] 
   \begin{align*}
       dim(E) = dim(E^*) &\implies \phi \text{ is an isomorphism}\\
                         &\implies \phi \text{ bijective}
   \end{align*}
   \[
   \forall f \in E^*, \exists! n \in E \text{ such that } \phi(n) = f, \text{ i.e } f(x) = <n, x> \, \forall x \in E
   \] 
   In this case $E = \R^n$, the Riesz Lemma is very simple to understand:\\
   Let $f: \R^n \to \R$ be a linear form. If we denote $(e_1, \ldots, e_n)$ the canonical basis of $\R^n$, any $x \in \R^n$ can be written as
    \begin{align*}
        x = \sum_{n=1}^{n} \alpha_ie_i \qquad \alpha_i \in \R, \forall i \in \{1, \ldots, n\}\\
        \implies f(x) = \sum_{n=1}^{n} \alpha_if(e_i) = <(\alpha_1, \ldots, \alpha_n), (a_1, \ldots, a_n)> = <(a_1, \ldots, a_n), (\alpha_1, \ldots, \alpha_n)>
   \end{align*}
\end{explanation}
% --- CHUNK_METADATA_START ---
% needs_review: True
% src_checksum: 73ccb62af63a937a6d3cefeabad6c5a7d3cb68369db6989cb6a2d3dc188473b7
% --- CHUNK_METADATA_END ---
\section{Ортонормовані базиси}% --- CHUNK_METADATA_START ---
% needs_review: True
% src_checksum: 48af4b80a9de6ca29bdbace6d0944a33fd8c9c15b77cb1e7025dd95b7fca5e7c
% --- CHUNK_METADATA_END ---
Нехай $(E, \scalair{,})$ буде евклідовим простором і $F \subset E$ векторним підпростором ($dim(F) < \infty$), оскільки $dim(E) < \infty$.% --- CHUNK_METADATA_START ---
% needs_review: True
% src_checksum: 1cf6ab4654a737c18a935159f01dc60a50331e445ab0fbb27d17129b05c48e8c
% --- CHUNK_METADATA_END ---
\begin{note}
    \[
        F^{\perp} := \{x \in E \mid \scalair{X, Z} = 0 \, \forall z \in F\} 
    \] 
    ортогонал до $F$.
\end{note}% --- CHUNK_METADATA_START ---
% needs_review: True
% src_checksum: c3a10a7268a935140dd832d42c2e16a56f122abbeb4749bea75affc54b3c9bde
% --- CHUNK_METADATA_END ---
\begin{theorem}
    Маємо $E = F \oplus F^{\perp}$.\\
    Зокрема, $dim(F^{\perp}) = dim(E) - dim(F)$ і $F = (F^{\perp})^{\perp}$
\end{theorem}% --- CHUNK_METADATA_START ---
% needs_review: True
% src_checksum: e86a1a6bea99ae019ad5cbd5c5d91c61a4dff3c6b217fb2f6aa9c3b347b40c36
% --- CHUNK_METADATA_END ---
\begin{proof}
   Ми повинні показати, що:
   \begin{enumerate}
       \item $F \cap F^{\perp} = \O$
       \item $E = F + F^{\perp}$ тобто  $\forall x \in E, \exists x' \in F, \, x'' \in F^{\perp}$ такий що $x = x' + x''$
   \end{enumerate}
   \begin{enumerate}
       \item Нехай $x \in F \cap F^{\perp}$\\
       $\implies$ $\scalair{X, Z} = 0 \, \forall Z \in F$ оскільки $x \in F \implies \scalair{X, X} = 0 \implies x = 0 (\scalair{,} \text{ визначено})$
        \item Нехай $x \in E$. Розглянемо  $f_x \in E^{*}$, тобто  $f_x: E \to \R, y \mapsto \scalair{x, y}$ і $f := f_{x|F}: F \to \R \implies f \in E^{*}$
            Лема Ріса $\implies$ $\exists! x' \in F$ такий що $f = f_{x'}: F \to \R, z \mapsto \scalair{x', z}\\$
            $\implies f_{x}(z) = f_{x'}(z) = f(z)\, \forall z \in F$ (Увага: не рівність для всіх $z$ у  $E$)\\
            Покладемо $x'' := x - x'$, тобто  $x = x' + x'' \in F$. Доведемо  $x'' \in  F^{\perp}$.\\
            Якщо $z \in F$,  $\scalair{x'', z} = \scalair{x - x', z} = \scalair{x, z} - \scalair{x', z} = 0$. Отже $x'' \in F^{\perp}$ і  $E = F \oplus F^{\perp}$ ($dim(E) = dim(F) + dim(F^{\perp})$) \\
            $F \subseteq (F^{\perp})^{\perp}$ оскільки $\scalair{x, z} = 0 \, \forall x \in F \, \forall z \in F^{\perp}$
            \[
                dim(F) = dim(E) - dim(F^{\perp})
            \]
            оскільки $E = G \oplus G^{\perp}$, отже  $dim(G) = dim(E) - dim(G^{\perp})$ для  $G = F^{\perp}, \, dim(F^{\perp}) = dim(G)$
   \end{enumerate}
\end{proof}% --- CHUNK_METADATA_START ---
% needs_review: True
% src_checksum: 27851f8b7925d124fd18b2554233f86afbf8232d6432f79cac746f2904377647
% --- CHUNK_METADATA_END ---
\begin{definition}
    Нехай $E$ — векторний простір, оснащений скалярним добутком $\scalair{,}$
     \begin{itemize}
         \item Сім'я $(v_i)_{i \ge 0}$ векторів з $E$ називається \underline{ортогональною}, якщо для $i \neq j$ ми маємо $\scalair{v_i, v_j} = 0$, тобто $v_i \perp v_j$
         \item Ортонормальна сім'я з $E$ — це ортогональна сім'я $(v_i)_{i \ge  0}$, така що до того ж $\|v_i\| = 1$ для $i \ge 0$
    \end{itemize}
\end{definition}% --- CHUNK_METADATA_START ---
% needs_review: True
% src_checksum: 0d2505dfb8639405790b38b3457ed812bf61a68b5c0e75296f1fa02e779ce15c
% --- CHUNK_METADATA_END ---
\begin{eg}
   \begin{enumerate}
       \item $E = \R^{n}$ оснащене стандартним скалярним добутком. Канонічний базис $(e_1, \ldots, e_n)$ є ортогональним, тому що
           \[
           \scalair{e_i, e_j} = \begin{cases}
               1 \text{ якщо } i = j\\
               0 \text{ якщо } i \neq j
           \end{cases}
           \]
       \item У $E = \mathcal{C}^{0}([-1, 1], \R)$ оснащене $\scalair{f,g} = \int_{-1}^{1} f(t)g(t)\,d{t}$. Сімейство $(\cos(t), \sin(t))$ є ортогональним. Сімейство $(1, t^2)$ не є ортогональним:
            \[
                \scalair{1, t^2} = \int_{-1}^{1} 1 t^2 \, d{t} = \frac{2}{3} \neq  0
           \]
   \end{enumerate}
\end{eg}% --- CHUNK_METADATA_START ---
% needs_review: True
% src_checksum: af5bd0f71063b14b9ef145dbe3af0af69158d553f33bee3c0be990eee7d22802
% --- CHUNK_METADATA_END ---
\begin{prop}
    Ортогональна сім'я, що складається з \underline{ненульових} векторів, є лінійно незалежною. Зокрема, ортонормована сім'я є лінійно незалежною. 
\end{prop}% --- CHUNK_METADATA_START ---
% needs_review: True
% src_checksum: 956a44f4ea8ed6a9bcd2cb11b5ea44ada4fdbb810014826686bfa90c5a387053
% --- CHUNK_METADATA_END ---
\begin{preuve}
    Припустимо, $(v_1, \ldots, v_n)$ ортогональні з $v_i \neq 0 \, \forall i = 1, \ldots, n$\\
    якщо $\sum_{j=1}^{n} \underset{\in \R}{\alpha_iv_i} = 0$, тоді  
    \[
        \forall i \in \{1, \ldots, n\} 0 = \scalair{v_i, \sum_{j=1}^{n} \alpha_jv_j} = \sum_{j=1}^{n}\alpha_j \scalair{v_i, v_j} = \alpha_i \underset{\neq 0}{\|v_i\|^2}
    \] 
    Отже, $\alpha_i = 0 \, \forall i = 1, \ldots, n$.\\
    Якщо $(v_1, \ldots, v_n)$ є ортонормальною, тоді $\|v_i\| = 1$. Отже,  $v_i \neq 0, \, \forall i = 1, \ldots, n$.
\end{preuve}% --- CHUNK_METADATA_START ---
% needs_review: True
% src_checksum: 87c48cf174cbeb60915f4f20a06e6f50a5f4ce984ed03c19114b2a5a1c37a705
% --- CHUNK_METADATA_END ---
\begin{intuition}
   Ортогональні (перпендикулярні) вектори ніколи не знаходяться один в одному (тобто $e_i = \lambda e_j$ неможливо), якщо вектори лінійно залежні, або кут $< 90º$ (отже, вектори не є ортогональними, абсурд), (вони знаходяться один в одному, вони не є ортогональними, абсурд). Отже, вони справді лінійно незалежні.
\end{intuition}% --- CHUNK_METADATA_START ---
% needs_review: True
% src_checksum: 295b6bd389a92be0c98563e3c43ed9be1871c81e35b1882caeccef39b2a1245c
% --- CHUNK_METADATA_END ---
\begin{definition}
    $(E, \scalair{,})$ евклідів простір. Сім'я $B = (e_1, \ldots, e_n)$ є ортонормальним базисом (де БОН), якщо вона є базисом і ортонормальною сім'єю.
\end{definition}% --- CHUNK_METADATA_START ---
% needs_review: True
% src_checksum: 5a8279067605c9c96f64e94f3a7c9f05a2800934a6a8a80be35d1aaaf2a73cd5
% --- CHUNK_METADATA_END ---
\begin{theorem}
    $(E, \scalair{,})$ евклідів простір. Тоді він допускає БОН.
\end{theorem}% --- CHUNK_METADATA_START ---
% needs_review: True
% src_checksum: e318537d26b99537acadb714f5cb06e6ea5ba9583d68b020578679227f59bf2c
% --- CHUNK_METADATA_END ---
\begin{preuve}
   Нехай $n := dim(E)$. Нехай $(e_1, \ldots, e_p)$ ортогональна сім'я (з точки зору потужності $p$) така що $e_i \neq 0 \, \forall i = 1, \ldots, p$.\\
Припустимо суперечливо, що $p < n$. Покладемо $F = Vect(e_1, \ldots, e_p)$. Тоді, $E = F \oplus F^{\perp}$ і $dim(F) \le p < n$. Отже $F^{\perp} \neq \{0\}$. Нехай $x \in F^{\perp}, \, x \neq 0$. Тоді, $(e_1, \ldots, e_p, x)$ є ортогональною потужності $> p$. Отже, $p = n$ і $(e_1, \ldots, e_n)$ є базисом $E$. Щоб отримати ортонормальну сім'ю $(e_1', \ldots, e_n')$ достатньо взяти $e_i' = \frac{1}{\|e_i\|}e_i \, \forall i = \{1, \ldots, n\}$.
\end{preuve}% --- CHUNK_METADATA_START ---
% needs_review: True
% src_checksum: c253feaaaf8ea9284c21cbf47c98c9c66d47171dba831f1fdad3d80d992cb660
% --- CHUNK_METADATA_END ---
\begin{prop}
    Нехай $(E, \scalar{}{})$ евклідів простір, і нехай  $(e_1, \ldots, e_n)$ ортонормальний базис $E$. Якщо  $x \in E$, маємо:
   \[
       x = \sum_{i=1}^{n} \scalar{x}{e_i}e_i
   \] 
Іншими словами, дійсне число $\scalar{x}{e_i}$ є $i^{\text{-та}}$ координата $x$ у базисі  $(e_1, \ldots, e_n)$.
\end{prop}% --- CHUNK_METADATA_START ---
% needs_review: True
% src_checksum: fc668c016669972f30b88cd5e9b2991340c3210e8aa83e0c514c9afe3127ded3
% --- CHUNK_METADATA_END ---
\begin{intuition}
    Ортогональність базису спрощує нам життя. Але спочатку невеликий вступ. Нехай векторний простір $E = \R^2$ і базис $(e_1, e_2) = (\begin{pmatrix} 1 \\ 0 \end{pmatrix}, \begin{pmatrix} 0\\ 1 \end{pmatrix})$. Нехай вектор $\vec{v} = (2, 3)$ :
    \begin{center}
        \begin{tikzpicture}
            \begin{axis}[
                scale=1,
                axis lines=middle,        % Draw axes in the middle
                xmin=-2, xmax=4,          % X-axis range
                ymin=-2, ymax=4,          % Y-axis range
                xlabel={$x$},             % Label for X-axis
                ylabel={$y$},             % Label for Y-axis
                xtick={-2,-1,0,1,2,3,4},% X-axis ticks
                ytick={-2,-1,0,1,2,3,4},% Y-axis ticks
                ]
            \draw[color=red, ->, thick] (0, 0) -- node[below]{$e_1$}(1, 0);
            \draw[color=blue, ->, thick] (0, 0) -- node[left]{$e_2$}(0, 1);
            \draw[color=green, ->] (0, 0) --node[above]{$\vec{v}$} (2, 3);

            \draw[color=gray, ->, thick] (1, 0) -- node[below]{$e_1$}(2, 0);
            \draw[color=gray, ->, thick] (2, 0) -- node[left]{$e_2$}(2, 1);
            \draw[color=gray, ->, thick] (2, 1) -- node[left]{$e_2$}(2, 2);
            \draw[color=gray, ->, thick] (2, 2) -- node[left]{$e_2$}(2, 3);

            \node[right, above] (_) at (2, 3){$(2, 3)$};
        \end{axis} 
        \end{tikzpicture}
    \end{center}
    Отже, ми можемо записати $\vec{v} = \vec{(2, 3)} = 2 \cdot \vec{e_1} + 3 \cdot \vec{e_2}$. Значення $x$ та $y$ (координати $v$) показують, скільки частин кожного базисного вектора (число може бути $\in \R$) потрібно взяти і просумувати, щоб отримати $\vec{v}$. (Простіше кажучи: наскільки далеко ми повинні піти вліво і вгору).
    \par
    У ортонормальному базисі $\scalair{v, e_i}$ вказує, скільки потрібно взяти вектора $e_i$, щоб утворити вектор  $\vec{v}$, а  $\vec{e_i}$ задає напрямок. Звідси $\scalair{v, e_1}$ еквівалентно $2$, і  $\scalair{v, e_2}$ до  $3$, потім: 
   \[
       \vec{v} = \underbrace{\scalair{v, e_1}}_{= 2} \cdot \vec{e_1} + \underbrace{\scalair{v, e_2}}_{= 3} \cdot \vec{e_2}
   \]  
   Зазвичай, щоб знайти координати в базисі, слід розв'язувати лінійну систему, тоді як ортонормальний базис дозволяє отримати їх шляхом обчислення скалярного добутку з кожним вектором базису, що значно простіше.
\end{intuition}% --- CHUNK_METADATA_START ---
% needs_review: True
% src_checksum: aea3e28cd57e20ec69577d68fb9b0a1782d005d58409320b6f31ec98254394c3
% --- CHUNK_METADATA_END ---
\begin{preuve}
    Покладемо $y := \sum_{i=1}^{n} \scalar{x}{e_i}e_i$ . Тоді, 
   \begin{align*}
       &\forall j = 1, \ldots, n,\\
       &\scalar{x - y}{e_j}\\ 
       = &\scalar{x}{e_j} - \scalar{y}{e_j}\\ 
       = &\scalar{x}{e_j} - \scalar{\sum_{i=1}^{n} \scalar{x}{e_i}e_i}{e_j}\\ 
       = &\scalar{x}{e_j} - \underbrace{ \sum_{i=1}^{n} \scalar{x}{e_i} }_{\substack{\text{винесено}\\ \text{як константу}}}\scalar{e_i}{e_j}\\ 
       = &\scalar{x}{e_j}\\ 
       -& \left(\scalar{x}{e_1}\underbrace{ \scalar{e_1}{e_j} }_{= 0} + \ldots + \scalar{x}{e_{j-1}}\underbrace{\scalar{e_{j-1}}{e_j}}_{= 0} + \scalar{x}{e_{j}}\underbrace{ \scalar{e_{j}}{e_j} }_{= 1} + \scalar{x}{e_{j+1}}\underbrace{ \scalar{e_{j+1}}{e_j} }_{= 0} + \ldots + \scalar{x}{e_{n}}\underbrace{ \scalar{e_{n}}{e_j} }_{= 0}\right)\\
        &\text{(} \forall i \neq j, \, \scalar{e_i}{e_j} = 0 \text{ оскільки це скалярний добуток ортогональних векторів)}\\ 
        &\text{(} \forall j \, \scalar{e_j}{e_j} = 1 \text{ оскільки це скалярний добуток того ж вектора)}\\
       = &\scalar{x}{e_j} - \scalar{x}{e_j}\underset{= 1}{\scalar{e_j}{e_j}} = 0
   \end{align*}
   Отже, $x - y \in Vect(e_1, \ldots, e_n)^{\perp} = E^{\perp} = \{0\}$. Отже $x = y$
\end{preuve}% --- CHUNK_METADATA_START ---
% needs_review: True
% src_checksum: 62f3efa5a94162d7abae77eaecb8119b2fe1c6b32a0943293c1aae43e7e31e74
% --- CHUNK_METADATA_END ---
\begin{corollary}
    $\forall x \in E, \, \|x\|^2 = \sum_{i=1}^{n} \scalar{x}{e_i}^2$ 
\end{corollary}% --- CHUNK_METADATA_START ---
% needs_review: True
% src_checksum: 9bfc133359e84b6700396f72b13466a8e04d0343cb89d0a7ed122d2a24ff1511
% --- CHUNK_METADATA_END ---
\begin{preuve}
    Якщо $x = \sum_{i=1}^{n} \scalar{x}{e_i}e_i = \sum_{i=1}^{n} x_ie_i$ тому
    \[
        \|x\|^2 = \scalar{\sum_{i=1}^{n} x_ie_i}{\sum_{j=1}^{n} x_je_j} = \sum_{i,j=1}^{n} x_ix_j\scalar{e_i}{e_j} = \sum_{i=1}^{n} x_i^2
    \] 
\end{preuve}% --- CHUNK_METADATA_START ---
% needs_review: True
% src_checksum: a6214a94d4e043fe5be223adf1ed2f11683d8677a04d2b0fc4052e2e34dd25c6
% --- CHUNK_METADATA_END ---
\section{Матриці та скалярні добутки}% --- CHUNK_METADATA_START ---
% needs_review: True
% src_checksum: 093eb312b5bef7229d54b74b2ee759bfaca5736436a22ee873986782bc80f654
% --- CHUNK_METADATA_END ---
\begin{prop} Нехай $(E, \scalair{,})$ евклідовий простір та $\epsilon = (e_1, \ldots, e_n)$ ортонормований базис. Нехай $f \in \mathcal{L}(E, E)$ та $A = (a_{i,j})_{1 \le i,j \le n}$ матриця, що представляє $f$ у $\epsilon$, тобто, $A = Mat_{\epsilon}(f)$
    \[
        a_{i,j} = \scalair{f(e_i), e_j} \, \forall i,j = 1, \ldots, n
    \] 
\end{prop}% --- CHUNK_METADATA_START ---
% needs_review: True
% src_checksum: dd44d42745a798aeb93703353ab8f513f389f43e7fe08fd4c6af0b3ed3b29b9c
% --- CHUNK_METADATA_END ---
\begin{preuve}
   $A$ є матрицею, стовпцями якої є вектори $f(e_j)$, записані в базисі $\epsilon$:
    \[
        A = (f(e_1) | \ldots | f(e_n))\quad f(e_j) = \begin{pmatrix} a_{1,j}\\ \ldots\\ a_{n, j} \end{pmatrix} 
   \] 
   Оскільки $\forall v \in E, \, v = c_1e_1 + \ldots c_ne_n$ тому $f(v) = c_1f(e_1) + \ldots c_nf(e_n)$ за лінійністю, отже нам залишається дослідити кожен $f(e_j)$
   \begin{align*}
       f(e_j) = a_{1, j}e_1 + \ldots a_{n, j}e_n \implies\\
       \langle f(e_j), e_i \rangle = \left\langle \sum_{k=1}^n a_{k,j} e_k, e_i \right\rangle = \sum_{k=1}^{n} a_{k,j}\scalar{e_k}{e_i} = a_{k, j}
   \end{align*}
   car $\scalar{e_k}{e_j} = \begin{cases}
       0 \text{ якщо } k \neq j\\
       1 \text{ якщо } k = j
   \end{cases}$
   Отже:
   \[
       a_{i, j} = \scalair{f(e_j), e_i}
   \] 
\end{preuve}% --- CHUNK_METADATA_START ---
% needs_review: True
% src_checksum: 3d6e562394bd16d44eacf1fdbf398a06c0ebb4899ed706e2fd0453018ad5cf75
% --- CHUNK_METADATA_END ---
Матриця векторного добутку дуже корисна в лінійній алгебрі. Перш ніж дати визначення:
\par
Нехай $E$ векторний простір скінченної розмірності $n$, простір $K$ і білінійна форма $b: E \times E \longrightarrow K$. Якщо $\{e_1, \ldots, e_n\}$ є базисом $E$, то: $x = \sum_{i=1}^{n} x_ie_i$ і $y = \sum_{j=1}^{n} y_je_j$, тоді маємо:
\[
b(x, y) = \sum_{i,j = 1}^{n} x_iy_jb(e_i, e_j)
\] 
$b$ отже, визначається знанням значень $b(e_i, e_j)$ на базі.% --- CHUNK_METADATA_START ---
% needs_review: True
% src_checksum: 48094a454bf8ba48c06e4394c13f6784cf74f6f5ae372abcf1d103d24a3d40fa
% --- CHUNK_METADATA_END ---
\begin{definition}
     Називається  \textbf{матрицею $b$} у базисі $\{e_i\}$ матриця:
      \[
          M(b)_{e_i} = \begin{pmatrix} 
              b(e_1, e_1) & b(e_1, e_2) & \ldots & b(e_1, e_n)\\
              b(e_2, e_1) & b(e_2, e_2) & \ldots & b(e_2, e_n)\\
              \ldots & \ldots & \ldots & \ldots\\
              b(e_n, e_1) & \ldots & \ldots & b(e_n, e_n)
          \end{pmatrix} 
     \] 
     Таким чином, елемент $\text{i-того}$ рядка та $\text{j-того}$ стовпця є коефіцієнтом $x_iy_j$.
\end{definition}% --- CHUNK_METADATA_START ---
% needs_review: True
% src_checksum: 62c294d50e70f069edb118954d1062019befc9fb23a60d3d2fae64b860280b3b
% --- CHUNK_METADATA_END ---
\begin{eg}
   Матриця канонічного скалярного добутку в $\R^3$ дорівнює:
   \[
       \scalair{X, Y} = x_1y_1 + x_2y_2 + x_3y_3 
   \] 
   \[
       Mat(\scalair{,})_{e_i} = \begin{pmatrix} 
            1 & 0 & 0\\
            0 & 1 & 0\\
            0 & 0 & 1
       \end{pmatrix} 
   \] 
\end{eg}% --- CHUNK_METADATA_START ---
% needs_review: True
% src_checksum: b0c83a2f49748ae4e903bde7a5deb981b49ddbea840f17eddb5300625781a59c
% --- CHUNK_METADATA_END ---
\begin{prop}\label{prop:prod-scal-par-matrice} скалярний добуток, представлений матрицею.\par
   Зазначимо:
   \begin{align*}
       \underbrace{A = M(b)_{e_i}}_{\text{матриця скалярного добутку}} && \underbrace{X = M(x)_{e_i}}_{\substack{\text{координати $x$}\\ \text{у базисі $e_i$}}} && \underbrace{Y = M(y)_{e_i}}_{\substack{\text{координати $y$}\\ \text{у базисі $e_i$}}} && (x, y \in E)
   \end{align*}
   Тоді маємо:
   \[
       b(x, y) = X^{t}AY
   \] 
\end{prop}% --- CHUNK_METADATA_START ---
% needs_review: True
% src_checksum: 9d0901364708fc729bba61d8e31be5464c242130c9e13ca350ba448dda5a614a
% --- CHUNK_METADATA_END ---
\begin{eg}
    Знову розглянемо приклад з $b = \scalair{,}$ канонічний скалярний добуток в $\R^3$. Нехай $X = \begin{pmatrix} 1 \\ 2 \\ -1 \end{pmatrix}$ та $Y = \begin{pmatrix} 2 \\ 3 \\ 1 \end{pmatrix} $ в канонічному базисі $\R^3$. Отже:
    \begin{align*}
        \scalair{x, y} = X^{t}AY &= \overbrace{(1, 2, -1)}^{X^{t}} \times \overbrace{\begin{pmatrix} 1 & 0 & 0\\ 0 & 1 & 0\\ 0 & 0 & 1 \end{pmatrix}}^{A} \times \overbrace{ \begin{pmatrix} 2 \\ 3 \\ 1 \end{pmatrix} }^{Y} \\
                                 &= \underbrace{(1, 2, -1)}_{X} \times \underbrace{ \begin{pmatrix} 2 \\ 3\\ 1 \end{pmatrix} }_{A \times Y} \\
                                 &= 1 \cdot 2 + 2 \cdot 3 + (-1) \cdot 1 = 2 + 6 - 1 = 7
    \end{align*}
\end{eg}% --- CHUNK_METADATA_START ---
% needs_review: True
% src_checksum: 54a4cd7b54b5866d578d31539dfc2487344f6e00176ef1f50e52b853fd2b6e6a
% --- CHUNK_METADATA_END ---
\begin{TODO}
   зміна базису матриці білінійної форми 
\end{TODO}% --- CHUNK_METADATA_START ---
% needs_review: True
% src_checksum: f02c7be5d857b5eed9458f065ffbe0b92fabe4a86739c6e6b8f844815fbed1b9
% --- CHUNK_METADATA_END ---
\section{Ортогональні проєкції}% --- CHUNK_METADATA_START ---
% needs_review: True
% src_checksum: 3cf8ef01da181c9899c2c7ef308c301ce20ec3f321d9dc0adeb31fc8b92fac39
% --- CHUNK_METADATA_END ---
Нехай $(E, \scalair{,})$ - евклідів простір, $F \subseteq E$ - векторний підпростір. Тоді, $E = F \oplus F^{\perp}$. Отже, $\forall x \in E$ записується як
\[
x = \underset{\in F}{x_F} + \underset{\in F^{\perp}}{x_{F^{\perp}}}
\]% --- CHUNK_METADATA_START ---
% needs_review: True
% src_checksum: 7ee17c23574b5c31f16245c762feb1689038b4764a68cdc48f11abebfaf21d0c
% --- CHUNK_METADATA_END ---
\begin{definition}
    \textbf{Ортогональна проєкція} з $E$ в $F$ — це проєкція $p_F$ з $E$ на $F$ паралельно до $F^{\perp}$, тобто
    \begin{align*}
        p_F: E = F \oplus F^{\perp} &\longrightarrow F \\
        x = x_F + x_{F^{\perp}} &\longmapsto p_F(x = x_F + x_{F^{\perp}}) = x_F
    .\end{align*}
\end{definition}% --- CHUNK_METADATA_START ---
% needs_review: True
% src_checksum: bdcdf0f2c368d93b2f418fae1ea96b243f406c45c822c89d9f5056e59d4f5329
% --- CHUNK_METADATA_END ---
\begin{remark}
   \begin{enumerate}
       \item $p_F$ є лінійним
       \item  $\forall x \in E \, p_{F}(x)$ повністю характеризується наступною властивістю:\\
           Нехай $y \in E$, тоді
            \[
                y = p_F(x) \iff \left( \underset{\implies y = x_F}{y \in F \text{ та } x - y} \in F^{\perp} \right) 
           \] 
       Зокрема $\scalair{p_F(x), x - p_F(x)} \,= 0$. Тоді, якщо $(v_1, \ldots, v_R)$ є ортонормованим базисом $F$, маємо:
            \[
                \forall x \in E, \, p_F(x) = \sum_{i=1}^{k} \scalair{x, v_i}v_i
           \] 
           Дійсно, достатньо перевірити, що вектор $y = \sum_{i=1}^{k} \scalair{x, v_i}v_i$ задовольняє:
           \[
               y \in F \text{ та } x - y \in F^{\perp}
           \] 
   \end{enumerate} 
\end{remark}% --- CHUNK_METADATA_START ---
% needs_review: True
% src_checksum: 3aa6469e224aa9391b7275942ad2ee83ef12453991db092e8687d27715531a0e
% --- CHUNK_METADATA_END ---
\begin{figure}[H]
   \centering 
\begin{tikzpicture}

% Draw the plane
\fill[gray!20] (-2,-1) -- (2,-1) -- (3,1) -- (-1,1) -- cycle;

% Draw the vectors
\draw[->, thick, black] (0,0) -- (2,2.2) node[anchor=south east] {\large $\mathbf{x}$};

\node[anchor=north, blue] (_) at ($(0,0)!0.5!(2,0)$) {\large $\text{proj}_\mathbf{F} \mathbf{x}$};
\node[anchor=west, blue] (_) at ($(2,0)!0.5!(2,2.2)$) {\large $\text{proj}_\mathbf{F^{\perp}} \mathbf{x}$};
% Add the labels for w and w perpendicular
\draw[->, thick, blue] (0,0) -- (2,0) ;
\draw[thick, black] (2,0) -- (2,3) node[anchor=west] {\large $\mathbf{F}^\perp$};
\draw[->, thick, blue] (2,0) -- (2,2.2);
\node[anchor=north west] (_) at (1.5, -0.5) {\large $\mathbf{F}$};
% Add the right angle symbol

\end{tikzpicture}
\caption{Проекція}
\label{pic:projection}
\end{figure}% --- CHUNK_METADATA_START ---
% needs_review: True
% src_checksum: fef2202683aa574defaf64635f8d48a8974d78d79ccc9dc70debe12602180d44
% --- CHUNK_METADATA_END ---
\begin{figure}[ht]
    \centering
    \incfig{projection-with-bon}
    \caption{Проєкція з ОНБ}
    \label{fig:projection-with-bon}
\end{figure}% --- CHUNK_METADATA_START ---
% needs_review: True
% src_checksum: 39fd6b59580833a0b7891f464b5fe02810e005f144922030608790b87c8ffebe
% --- CHUNK_METADATA_END ---
\begin{prop}
   Нехай $x \in E$. Тоді,
   \[
       \|x - p_F(x)\| = inf\{\|x - y\| \mid y \in F\}
   \] 
   тобто $\|x - p_F(x)\|$ є відстань від  $x$ до  $F$.\\
   Див. Figure~\ref{pic:projection}
\end{prop}% --- CHUNK_METADATA_START ---
% needs_review: True
% src_checksum: eaf28919b4717902ddaef3d4356df8d73f7d2e7f52cbe116e05394d2c107ed56
% --- CHUNK_METADATA_END ---
\begin{preuve}
   Оскільки $p_F(x) \in F$ достатньо довести, що, якщо  $y \in F$, тоді 
   \[
   \|x - p_F(x)\| \le \|x - y\|
   \] 
   Але, $\underset{(x - p_F(x)) + (p_F(x) - y)}{\|x - y\|^2} = \|x - p_F(x)\|^2 + 2\overbrace{\scalair{\overset{\in F^{\perp}}{x - p_F(x)}, \overset{\in F}{p_F(x) - y}}}{= 0} + \underbrace{\|p_F(x) - y\|^2}_{\ge 0} \ge \|x - p_F(x)\|^2$
\end{preuve}% --- CHUNK_METADATA_START ---
% needs_review: True
% src_checksum: 58448e126f08a50a09fc0715ad54230783e07f1dcf3d3cf5d1e964c99ddb6752
% --- CHUNK_METADATA_END ---
\begin{theorem}\label{thm:gram-schmidt}Грам-Шмідт\\
    Нехай $E$ — векторний простір, оснащений скалярним добутком $\scalair{,}$. Нехай $(v_1, \ldots, v_n)$ — лінійно незалежна сім'я елементів $\in E$. Тоді, існує сім'я $(w_1, \ldots, w_n)$ ортогональна така що
    \[
        \forall i = 1, \ldots, n \quad Vect(v_1, \ldots, v_i) = Vect(w_1, \ldots, w_i)
    \]
    Крім того, ця теорема дає нам метод побудови ортонормованого базису з довільного базису.
\end{theorem}% --- CHUNK_METADATA_START ---
% needs_review: True
% src_checksum: 330b4b7f1bdcbebbdd214c06eebd87c11d27badb990601b0996dd8b25cbb96d7
% --- CHUNK_METADATA_END ---
\begin{preuve} Теореми \ref{thm:gram-schmidt}
    Побудуємо ортогональний базис: $\{w_1, \ldots, w_p\}$. Спершу покладемо:
    \[
    \begin{cases}
        w_1 = v_1\\
        w_2 = v_2 + \lambda w_1, \qquad \text{де } \lambda \text{ такий, що } w_1 \perp w_2
    \end{cases}
    \] 
    Накладаючи цю умову, знаходимо:
    \[
        0 = \scalair{v_2 + \lambda w_1, w_1} = \scalair{v_2, w_1} + \lambda \|w_1\|^2
    \] 
    Оскільки $w_1 \neq 0$, отримуємо $\lambda = - \frac{\scalair{v_2, w_1}}{\|w_1\|^2}$. Зауважимо, що:
    \[
    \begin{cases}
        v_1 = w_1\\
        v_2 = w_2 - \lambda w_1
    \end{cases}
    \] 
    отже $Vect\{v_1, v_2\} = Vect\{w_1, w_2\}$.
    \par
    Після побудови $w_2$, будуємо $w_3$, поклавши:
    \begin{align*}
        &w_3 = v_3 + \mu w_1 + \nu w_2\\
        &\text{де } \mu \text{ та } \nu \text{ такі, що: } w_3 \perp w_1 \text{ та } w_3 \perp w_2
    \end{align*}
    Можна розглядати $w_3 = v_3 - \lambda' w_1 - \lambda'' w_2 $ як $w_3 = v_3 - proj_{F_2}v_3$ де $F_i = Vect\{w_1, \ldots, w_i\}$
    \begin{figure}[H]
        \centering
        \incfig{projection-with-bon-thm}
        \caption{Вектор за допомогою проекції}
        \label{fig:projection-with-bon-thm}
    \end{figure}
    Це дає
    \begin{align*}
        0 &= \scalair{v_3 + \mu w_1 + \nu w_2, w_1} = \scalair{v_3, w_1} + \mu \underset{= \|w_1\|^2}{\scalair{w_1, w_1}} + \nu \underset{= 0}{\scalair{w_2, w_1}}\\
          &= \scalair{v_3, w_1} + \mu \|w_1\|^2 
    \end{align*}
    звідки $\mu = - \frac{\scalair{v_3, w_1}}{\|w_1\|^2}$. Аналогічно, накладаючи умову, що $w_3 \perp w_2$, знаходимо $\nu = - \frac{\scalair{v_3, w_2}}{\|w_2\|^2}$. Оскільки
    \[
    \begin{cases}
        v_1 = w_1\\
        v_2 = w_2 - \lambda w_1\\
        v_3 = w_3 - \mu w_1 - \nu w_2
    \end{cases}
    \] 
    добре видно, що $Vect\{w_1, w_2, w_3\} = Vect\{v_1, v_2, v_3\}$. Тобто, $\{w_1, w_2, w_3\}$ є ортогональним базисом простору, породженого $v_1, v_2, v_3$. Тепер добре видно процес рекурсії.
    \par
    Припустимо, що ми побудували $w_1, \ldots, w_{k-1}$ для $k \le p$. Покладемо:
    \begin{align*}
        w_k &= v_k + \text{ лінійна комбінація вже знайдених векторів}\\
            &= v_k + \lambda_1w_1 + \ldots + \lambda_{k-1}w_{k-1}
    \end{align*}
    Умови $w_k \perp w_i$ (для $i \in \{1, \ldots, k-1\}$) еквівалентні:
    \[
        \lambda_i = - \frac{\scalair{v_k, w_i}}{\|w_i\|^2}
    \] 
    як це негайно перевіряється. Оскільки $v_k = w_k - \lambda_1 - \ldots - \lambda_{k-1}w_{k-1}$, за індукцією бачимо, що $Vect\{w_1, \ldots, w_k\} = Vect\{v_1, \ldots, v_k\}$ $\iff$ $\{w_1, \ldots, w_k\}$ є ортогональним базисом $Vect\{v_1, \ldots, v_k\}$.
    \par
    Нам залишається лише нормувати її, тобто  $\forall i \in \{1, \ldots, k\}$ $e_i = \frac{w_i}{\|w_i\|}$, звідки $\{e_1, \ldots, e_k\}$ є ортонормальним базисом $F = Vect\{v_1, \ldots, v_k\}$.
\end{preuve}% --- CHUNK_METADATA_START ---
% needs_review: True
% src_checksum: 751a5f3f1ef8a135bcd0fe7e3da6b01ade6279ba3c8727b2968a2107f53aa410
% --- CHUNK_METADATA_END ---
\begin{prop} Щоб зрозуміти цю пропозицію, раджу прочитати розділ \ref{sec:isometrie-et-adjoints}
    \par
   Будь-яка ортогональна проєкція є самоспряженою, тобто якщо $p$ є ортогональною проєкцією, тоді:
   \[
   p^* = p
   \] 
   У матричному записі: нехай $A$ матриця проєкції $p$, тоді:
    \[
   A^T = A
   \] 
\end{prop}

% --- CHUNK_METADATA_START ---
% needs_review: True
% src_checksum: bf29d983a077cade6e7b79974dcab3a974510ec0274808acf85fb66ff370949b
% --- CHUNK_METADATA_END ---
\section{Isometries and Adjoints}% --- CHUNK_METADATA_START ---
% needs_review: True
% src_checksum: 404e586606ba4465ff926abf10b88897e18e4ba1644bdf63730b802601b04887
% --- CHUNK_METADATA_END ---
\label{sec:isometrie-et-adjoints}% --- CHUNK_METADATA_START ---
% needs_review: True
% src_checksum: 73ae3e92ca1e7f1ec5cd7c60aa1e778221e2cc965369c246711a5e59f5873117
% --- CHUNK_METADATA_END ---
\subsection{Isometries}% --- CHUNK_METADATA_START ---
% needs_review: True
% src_checksum: f31879de7b3b511f25e54c3b0a768b70124f485082f987bdef45c36e98912a7c
% --- CHUNK_METADATA_END ---
\begin{definition}
    An \textbf{isometry} of $E$ (or \textbf{orthogonal transformation}) is an endomorphism $f \in \mathcal{L}(E) := \mathcal{L}(E, E)$ preserving the dot product, i.e.:
     \[
         \scalair{f(x), f(y)} = \scalair{x, y} \quad \forall x, y \in E
    \] 
\end{definition}% --- CHUNK_METADATA_START ---
% needs_review: True
% src_checksum: 226da0912aa6423aadeae262eeafcc3e84237ec5ceef3ec18cd9a1fd199371e6
% --- CHUNK_METADATA_END ---
\begin{definition}
    Let $x, y \in E$ be two non-zero vectors. We have, according to the Cauchy-Schwarz inequality (see lemma \ref{lemma:inegalite-cauchy-schwarz}):
    \[
        \frac{| \scalair{x, y} |}{\|x\| \cdot \|y\|} \le 1
    \] 
    Then, there exists one and only one $\theta \in [0, \pi]$ such that:
     \begin{equation}
        \cos \theta = \frac{ \scalair{x, y}}{\|x\| \cdot \|y\|} 
    \end{equation}
    $\theta$ is called the \textbf{angle} (non-oriented) between the vectors $x$ and $y$.
\end{definition}% --- CHUNK_METADATA_START ---
% needs_review: True
% src_checksum: 93326741099c1f67d38123ac1c58cd30fafdaf5a9fc1c9d3bfd189ebe3f67d49
% --- CHUNK_METADATA_END ---
\begin{prop}\label{prop:isometrie-reserve-norme}
   If $f$ is an isometry of $E$, then we have:
   \[
   \|f(x)\| = \|x\| \quad \forall x \in E
   \] 
\end{prop}% --- CHUNK_METADATA_START ---
% needs_review: True
% src_checksum: 8b6e25bd20ed73256f25ff5a38cadb126d0df5b8a1c53d50e83eb89da031c211
% --- CHUNK_METADATA_END ---
\begin{preuve}
   Suppose that $f$ is an isometry of $E$. Let $x, y \in E$. By definition: $\scalair{f(x), f(y)} = \scalair{x, y}$, therefore, let $y := x$, then, we have:
   \begin{align*}
       &\underbrace{\scalair{f(x), f(x)}}_{\|f(x)\|^2} = \underbrace{\scalair{x, x}}_{\|x\|^2}\\
       \iff &\|f(x)\|^2 = \|x\|^2\\
       \iff &\|f(x)\| = \|x\|
   \end{align*}
\end{preuve}% --- CHUNK_METADATA_START ---
% needs_review: True
% src_checksum: 02574ca751b6786c0767e44130459fb748e40551d55dbd7c9519b9234f153150
% --- CHUNK_METADATA_END ---
\begin{prop}\label{prop:isometrie-bijective}
   Let $f$ be an isometry in $E$, then:
   \begin{enumerate}
       \item $f$ is bijective
       \item  $f$ preserves the Euclidean distance and angles
   \end{enumerate}
\end{prop}% --- CHUNK_METADATA_START ---
% needs_review: True
% src_checksum: aa1693c4eb616c058387165d6c49b19bd90ddd4af5fb103c00d2ac8f14c38ffb
% --- CHUNK_METADATA_END ---
\begin{preuve}
   Let $f$ be an isometry in $E$ and two vectors $u, v \in E$ 
   \begin{enumerate}
       \item  
           \begin{align*}
               \|f(u) - f(v)\| = \sqrt{\scalair{f(u), f(v)}} = \sqrt{\scalair{u, v}} = \|u - v\| 
           \end{align*}
       \item Let $\theta_1$ be the angle between $f(u)$ and $f(v)$ and $\theta_2$ be the angle between $u$ and $v$, so:
            \[
                \cos \theta_1 := \frac{\scalair{f(u), f(v)}}{\|f(u)\| \cdot \|f(v)\|}
           \] 
           \[
                \cos \theta_2 := \frac{\scalair{u, v}}{\|u\| \cdot \|v\|}
           \] 
           By definition, $\scalair{f(u), f(v)} = \scalair{u, v}$, according to proposition \ref{prop:isometrie-reserve-norme}, $\forall x, \|f(x)\| = \|x\|$, so:
           \[
                \cos \theta_1 := \frac{\scalair{f(u), f(v)}}{\|f(u)\| \cdot \|f(v)\|} = \frac{\scalair{u, v}}{\|u\| \cdot \|v\|} = \cos \theta_2
           \] 
   \end{enumerate}
\end{preuve}% --- CHUNK_METADATA_START ---
% needs_review: True
% src_checksum: d385f4082b752467e8f8dbd40ae7f6e4d1b150aafef7f7a2d8753102cc334575
% --- CHUNK_METADATA_END ---
\begin{definition}
    Let $F$ be a vector subspace of $E$, therefore $E = F \oplus F^{\perp}$ where $\forall v \in E, \exists v_1 \in F, v_2 \in F^{\perp}$ such that $v = v_1 + v_2$. We set:
    \[
    s_F(v) = v_1 - v_2
    \] 
    and we call $s_F$ an orthogonal symmetry with axis F.
\end{definition}% --- CHUNK_METADATA_START ---
% needs_review: True
% src_checksum: 9774560f06771548c2bdd5a9eba119bad3c3fa44773dcfdd9dc44513d1f26142
% --- CHUNK_METADATA_END ---
\begin{figure}[H]
    \centering
    \incfig{symetrie-orthogonale-axe-f}
    \caption{Orthogonal symmetry with axis $F$}
    \label{fig:symetrie-orthogonale-axe-f}
\end{figure}% --- CHUNK_METADATA_START ---
% needs_review: True
% src_checksum: 414d6413494e32c606e09d080d0a277eca5d51f6763d4eb0e12e5c48f52250be
% --- CHUNK_METADATA_END ---
\begin{prop}
   Orthogonal symmetry is an isometry.
\end{prop}% --- CHUNK_METADATA_START ---
% needs_review: True
% src_checksum: 4836b9ba505742a648a3ea5786960cd719583b2ed9db26f6fb6e59a300bacb55
% --- CHUNK_METADATA_END ---
\begin{proof}
   TODO or not needed
\end{proof}% --- CHUNK_METADATA_START ---
% needs_review: True
% src_checksum: 163b9655d4c6ee6b5ed32a9e945de845daf3e519b3d872480793b91dad96ff6b
% --- CHUNK_METADATA_END ---
\begin{prop}\label{prop:isometrie-ssi-transforme-bon-en-bon}
   $f$ is an isometry if and only if it transforms every orthonormal basis into an orthonormal basis.
\end{prop}% --- CHUNK_METADATA_START ---
% needs_review: True
% src_checksum: 4ec95a52726e4981c90aca65978297d0557bf21f1532a54843088d610d510166
% --- CHUNK_METADATA_END ---
\begin{preuve}
    Let $f$ be an isometry, then it transforms any basis into a basis because $f$ is bijective by prop. \ref{prop:isometrie-bijective}.
    \begin{itemize}
        \item ($\implies$) Suppose that $f$ is an isometry. Let $\{e_i\}$ be an orthonormal basis, then we have:
             \[
                 \scalair{f(e_i), f(e_j)} = \scalair{e_i, e_j} = \delta_{i,j}
            \]
            Therefore, $\{f(e_i)\}$ is an orthonormal basis.
        \item ($\impliedby$) Suppose that there exists an orthonormal basis $\{e_i\}$ such that $\{f(e_i)\}$ is also an orthonormal basis. Moreover, let $x = x_1e_1 + \ldots x_ne_n$ and $y = y_1e_1 + \ldots + y_ne_n$ with $x_i, y_i \in \R$
            \par
            Since $\{e_i\}$ is orthonormal, then we have:
            \begin{equation}\label{eq:prod-scal-base-ortho}
                \scalair{x, y} = x_1y_1 + \ldots + x_ny_n = \sum_{i=1}^{n} x_iy_i
            \end{equation}
            On the other hand:
            \begin{align*}
                \scalair{f(x), f(y)} &= \scalair{\sum_{i=1}^{n} x_if(e_i), \sum_{i=1}^{n} y_if(e_i)} = \sum_{i,j = 1}^{n} x_iy_j\scalair{f(e_i), f(e_j)}\\
                                     &= \sum_{i,j=1}^{n} x_iy_j\scalair{e_i, e_j} \underset{\text{car } \{e_i\} \text{ orthonormée}}{=} = \sum_{i=1}^{n} x_iy_i \underset{\text{D'apres } \ref{eq:prod-scal-base-ortho}}{=} \scalair{x, y}
            \end{align*}
            Therefore $f$ is an isometry.
    \end{itemize}
\end{preuve}% --- CHUNK_METADATA_START ---
% needs_review: True
% src_checksum: 8d2950f635e9cffe3eefb6c7a9f8caceaf847f161b494f77e191990a45f5584a
% --- CHUNK_METADATA_END ---
\begin{prop}\label{prop:isometrie-ata-eg-i}
    If $\{e_i\}$ is an orthonormal basis, $f$ an isometry and $A = M(f)_{e_i}$, then $A^{T}A = I = AA^{T}$.
\end{prop}% --- CHUNK_METADATA_START ---
% needs_review: True
% src_checksum: 308e1996bec46376938c161a3389b6a69e2a68a372ac948cc0ddb54f07aa1f27
% --- CHUNK_METADATA_END ---
\begin{preuve}
    To prove this, we will use proposition \ref{prop:prod-scal-par-matrice}.
    \par
    By definition of isometry, we have:
    \begin{align*}
        &\scalair{f(x), f(y)} = \scalair{x, y} \quad \forall x, y \in E\\
        \iff &\underbrace{ (AX)^{T}(AY) }_{\scalair{f(x), f(y)}} = X^TA^TAY = \underbrace{X^TY}_{\scalair{x, y}}\\
        \iff &A^TA = I
    \end{align*}
\end{preuve}% --- CHUNK_METADATA_START ---
% needs_review: True
% src_checksum: 65746adebc98fdc96e5e56ec6d65df91030b03666332c9b11270d8eed9f46139
% --- CHUNK_METADATA_END ---
\begin{prop}
   If $A$ is a matrix of isometry in an orthonormal basis, then $det(A) = \pm 1$ 
\end{prop}% --- CHUNK_METADATA_START ---
% needs_review: True
% src_checksum: 23b9a2d19a5bed3852c398d91fb93c61fd03fadfdca45e33b28839618a75a043
% --- CHUNK_METADATA_END ---
\begin{preuve}
    By proposition \ref{prop:isometrie-ata-eg-i}, we have: $A^TA = I$, hence:
     \begin{align*}
         det(A^TA) = det(I) = 1 \implies& det(A)^2 = 1 \quad \text{ (because }  det(A^T) = det(A) \text{)}\\
                                \implies& det(A) = \pm 1
    \end{align*}
\end{preuve}% --- CHUNK_METADATA_START ---
% needs_review: True
% src_checksum: 7c8be0f83e3ecdc7e2399c8f0a68e8f7d4306d5dbcf3cf9c373f63d87616a7fd
% --- CHUNK_METADATA_END ---
\begin{intuition}
   An isometry performs a rotation or a reflection; it preserves distances, and therefore the area (or volume) of a figure constructed by the base of this transformation is equal to $1$. 
\end{intuition}% --- CHUNK_METADATA_START ---
% needs_review: True
% src_checksum: 09c3327a6fca672cd2ff7bbc61249f10cf92ae3695cbf2ec363dbf5510fec4c2
% --- CHUNK_METADATA_END ---
\subsection{Adjoint endomorphism}% --- CHUNK_METADATA_START ---
% needs_review: True
% src_checksum: f5a9abf46d1ad9c63bf6534aac796f8754fed2637a6ff0f5c45c6831dc66dfcb
% --- CHUNK_METADATA_END ---
\begin{prop}
   Let $E$ be a Euclidean space and $f \in End(E)$. There exists one and only one endomorphism $f^* \in E$ such that
   \[
       \scalair{f(x), y} = \scalair{x, f^*(y)}, \quad \forall x, y \in E
   \] 
   $f^*$ is called the \textbf{adjoint} of $f$.
   \par
   If $\{e_i\}$ is an orthonormal basis and $A = M(f)_{e_i}$, then the matrix $A^* = M(f^*)_{e_i}$ is the transpose of $A$, i.e. $A^* = A^T$
\end{prop}% --- CHUNK_METADATA_START ---
% needs_review: True
% src_checksum: 5e40b99de63f8219b8e35fcf11ba61e30bfb1741076e801edd00e6e8b82c3a16
% --- CHUNK_METADATA_END ---
\begin{preuve}
    Again, for the proof, we will use proposition \ref{prop:prod-scal-par-matrice} which is very useful, so I advise you to master this concept.
    \par
    Let $\{e_i\}$ be an orthonormal basis of $E$ and let us denote
     \[
    A = M(f)_{e_i} \quad A^* = M(f^*)_{e_i} \quad X = M(x)_{e_i} \quad Y = M(y)_{e_i}
    \] 
    Since we are in an orthonormal basis, the statement is written:
    \[
        \underbrace{(AX)^TY}_{\scalair{f(x),y}} = X^TA^TY = \underbrace{X^T(A^*Y)}_{\scalair{x, f^*(y)}} \quad \forall X, Y \in \mathcal{M}_{n, 1}(\R)
    \] 
    which implies that $A^* = A$ and, furthermore, demonstrates the uniqueness of such adjoint.
\end{preuve}% --- CHUNK_METADATA_START ---
% needs_review: True
% src_checksum: 69e2e232a1cf77e5370bdc547dbbc03cf5a33e341123571f6f13ffa4a2a97849
% --- CHUNK_METADATA_END ---
\section{Orthogonal Groups}% --- CHUNK_METADATA_START ---
% needs_review: True
% src_checksum: 30e9713b279ada9e4dc1ba4cf518680437c3f81a6bf4160c2fc7ddaa21f5608d
% --- CHUNK_METADATA_END ---
Reminder:% --- CHUNK_METADATA_START ---
% needs_review: True
% src_checksum: 7f5170f3039994d2267944bfacea33bab1b2b27457ae3c7c865342196546ad13
% --- CHUNK_METADATA_END ---
\begin{definition}\label{def:general-linear-group}
    A general linear group:
    \[
        GL(n, \R) = \{A \in \mathcal{M}_{n}(\R) \mid det(A) \neq 0\}
    \] 
    is a group of all linear transformations (square matrices) that are invertible (because $det(A) \neq 0$).
\end{definition}% --- CHUNK_METADATA_START ---
% needs_review: True
% src_checksum: 69a067bb55ef45d25eea5e96ca06f9e04e1254a999ec2722830a3aab6f48df5b
% --- CHUNK_METADATA_END ---
\begin{definition} \textbf{Orthogonal Group}:
    The set:
    \[
        O(n, \R) := \{A \in \mathcal{M}_{n}(\R) \mid A^TA = I\} = \{A \in \mathcal{M}_{n}(\R) \mid AA^T = I\}
    \] 
    satisfies the following properties:
    \begin{enumerate}
        \item if $A, B \in O(n, \R)$, then $AB \in O(n, \R)$
        \item $I \in O(n, \R)$
        \item if $A \in O(n, \R)$ then $A^{-1} \in O(n, \R)$
    \end{enumerate}
    In particular, $O(n, \R)$ is a subgroup of $GL(n, \R)$ (group of invertible matrices) (see definition \ref{def:general-linear-group}).
\end{definition}% --- CHUNK_METADATA_START ---
% needs_review: True
% src_checksum: 8ca4e70a0d8b16cb958e9eee5fa78577f55c9906dfd354069234ff3982a705a6
% --- CHUNK_METADATA_END ---
\begin{intuition}
    The meaning of orthogonal matrices is clear: they represent the matrices of orthogonal transformations (isometries) in \textbf{an orthonormal basis} (see defn \ref{def:orthogonal}).
\end{intuition}% --- CHUNK_METADATA_START ---
% needs_review: True
% src_checksum: 884c53f8606bf123e9b18d510e42e1f06bef30774481997531dad56cbbfcc4e0
% --- CHUNK_METADATA_END ---
We can notice that if $det(A) = 1$, this isometry represents a rotation; furthermore, we have the following definition:% --- CHUNK_METADATA_START ---
% needs_review: True
% src_checksum: 7d8c7b2b1e0c5e074414c9ccb9842dc4f78ec36f618d7cd8ef79e023ceeb9160
% --- CHUNK_METADATA_END ---
\begin{definition}
    The set of direct orthogonal matrices (i.e. such that $det(A) = 1$)
    \[
    SO(n, \R) = \{A \in O(n, \R) \mid det(A) = 1 \}
    \] 
    is a group, called the \textbf{special orthogonal group}.
\end{definition}% --- CHUNK_METADATA_START ---
% needs_review: True
% src_checksum: c5dbacd3e34cbf5a2300990c8929db644508fb545359d81af28c2434e33b6972
% --- CHUNK_METADATA_END ---
\begin{eg}
   The matrix
   \[
       A = \frac{1}{3} \begin{pmatrix} 2 & -1 & 2\\ 2 & 2 & -1\\ -1 & 2 & 2 \end{pmatrix} 
   \] 
   is orthogonal. We can verify that $A^TA = I$, or, it is sufficient to show that $c_1, c_2, c_3$ is an orthonormal family, i.e.:
   \[
       \|c_i\|^2 = 1 \quad \text{ and } \quad \scalair{c_i, c_j} = 0 \quad \text{ if } i \neq j
   \] 
   We can interpret $A$ as the matrix of a transformation $f$ in the canonical basis $\{e_i\}$, so we have: $c_i = f(e_i)$, according to proposition \ref{prop:isometrie-ssi-transforme-bon-en-bon} $f$ is orthogonal. Moreover, we see that $det(A) = +1$. Consequently, $f$ is a direct orthogonal transformation.
\end{eg}% --- CHUNK_METADATA_START ---
% needs_review: True
% src_checksum: 135d781de4dc50b2a9c113785c40f142345ea9689665ffb66c938b3a10834f9e
% --- CHUNK_METADATA_END ---
\begin{prop}
   The change-of-basis matrix from an orthonormal basis to an orthonormal basis is an orthogonal matrix. 
\end{prop}% --- CHUNK_METADATA_START ---
% needs_review: True
% src_checksum: 0bec23217c28ab51a0c6e64cc22eb8834a44ebb9f3d2160385f659491a79a8b1
% --- CHUNK_METADATA_END ---
\begin{preuve}
   I'm providing intuition. A transition matrix transforms one basis into another; it transforms the vectors of the basis, so it transforms the basis of the O.N.B. into vectors of the basis of the O.N.B. Therefore, according to proposition \ref{prop:isometrie-ssi-transforme-bon-en-bon}, this matrix is orthogonal.
\end{preuve}
\section{Suites de Cauchy}
\begin{definition}
    $(x_n)_{n \in \N}$ suite dans $E$ est de \underline{Cauchy} si:
     \[
    \forall \epsilon > 0 \, \exists N(\epsilon) \in \N \text{ tel que } \forall n, p \ge N(\epsilon), d(x_n, x_p) \le \epsilon
    \] 
\end{definition}
\begin{intuition}
   Une suite de Cauchy c'est comme on mesure un point et on le localise, i.e:
   \begin{enumerate}
       \item On dit qu'il est entre $0$ et  $1$.
       \item Ensuite, on precise plus et on dit qu'il est entre  $0.5$ et  $0.6$.
       \item Puis, entre  $0.55$ et  $0.56$
   \end{enumerate}
   On peut infiniment augmenter le niveau de précision. C'est ça l'idée d'une suite de Cauchy.
\end{intuition}
\begin{prop}
   \begin{enumerate}
       \item Toute suite de Cauchy est bornée.
        \item Toute suite convergente est de Cauchy
   \end{enumerate} 
\end{prop}
\begin{preuve}
   \begin{enumerate}
       \item voir poly
       \item Soit $(x_n)$ une suite avec  $\lim_{n \to \infty} x_n = x$ avec $x \in E$.
           \begin{itemize}
               \item 
                   \underline{Hyp:}  $\frac{\epsilon}{2} > 0 \, \exists N(\frac{\epsilon}{2}) \in \N \text{ tel que } \forall n \ge N(\frac{\epsilon}{2}), d(x_n, x) \le \epsilon/2$
                \item 
                    \underline{À montrer:} $\epsilon > 0 \, \exists M(\epsilon) \in \N \text{ tel que } \forall n, p \ge M(\epsilon), d(x_n, x_p) \le \epsilon$
           \end{itemize}
           \[
               d(x_n, x_p) < d(x_n, x) + d(x, x_p) \text{ si } n, p \ge  N(\frac{\epsilon}{2}) \, d(x_n, x_p) \le 2 \frac{\epsilon}{2} = \epsilon
           \] 
   \end{enumerate} 
\end{preuve}
\begin{definition}
    $(E, d)$ est \underline{complet} si toute suite de cauchy dans  $E$ est convergente.
\end{definition}
\begin{definition}
    Un éspace métrique $(E, d)$ est \textbf{complet} si toute suite  $(x_n)_{n \in \N}$ d'éléments de  $E$ converge vers une limite  $x$ qui appartient aussi à  $E$.
\end{definition}
\begin{eg}
    Un éspace métrique $(]0, 1], d)$ avec $d$ une distance euclidienne n'est pas complet, car  soit une suite: $x_n = \frac{1}{n}$ dont la limite est $0$. Par contre,  $0 \not\in ]0, 1]$. Donc cet éspace n'est pas complet. 
\end{eg}
\begin{figure}[h]
   \centering 
   \begin{tikzpicture}
       \draw[->] (-1, 0) -- (2, 0); 
       \node[below] (_) at (2,0){$x$};

       \node (_) at (0,0){]};
       \node[below] (_) at (0,-0.3){$0$};
       \node (_) at (1,0){]};
       \node[below] (_) at (1,-0.3){$1$};
       \draw[color=red] (0,0)--(1,0);
   \end{tikzpicture}
   \caption{$(]0, 1], d)$ n'est pas complet}
\end{figure}
\begin{eg}
   Un éspace $(\Q, d)$ n'est pas complet. Car on peut prendre une suite  $x_n$ tendant vers  $\sqrt{2} \not\in \Q$.
\end{eg}

\begin{figure}[H]
    \centering
    \incfig{q_not_complete}
    \caption{$\Q$ pas complet}
    \label{fig:q_not_complete}
\end{figure}
\begin{prop}
   $\R^d$ muni de la distance usuelle est complet. 
\end{prop}
\begin{preuve}
   \[
   X_n = (x_{1,n}, \ldots, x_{d,n})
   \]  
   \[
   |x_i - y_i| \le d(X, Y) = \|X - Y\|_2 \quad \forall 1 \le i \le d
   \] 
   les suites réelles $(x_{i,n})_{n \in \N}$ sont de Cauchy si $(X_n)$ est de Cauchy.
\end{preuve}
\begin{property}
   $\R$ est complet 
\end{property}
\begin{preuve}
    (Suit de la propriété de la borne supérieure) 
    \par
    Il existe $x_i \in \R$ avec $1 \le i \le d$ tels que $|x_{i,n} - x_{i}| \xrightarrow[n \to \infty]{} 0$
    \[
        d(X, Y) \le \sqrt{d} \underset{1 \le i \le d}{max} |x_i - y_i| 
    \] 
    donc $X_n \xrightarrow[n \to \infty]{} X$, $X = (x_1, \ldots, x_d)$
\end{preuve}
\section{Sous-suites}
\begin{definition}
    Soit $(x_n)_{n \in \N}$ une suite dans $E$. Une suite  
    \[
        (y_n)_{n \in \N} \text{ avec } y_n = x_{\phi(n)}
    \] 
    où $\phi: \N \to \N$ est \underline{strictement croissante} est appelée \textbf{sous-suite} de la suite $(x_n)$.
\end{definition}
\begin{eg}
    Soit une application $\phi: \N \to \N$ telle que $\phi(n) = 2n$. Donc  $(x_n)_{\phi(n)}$ est une sous-suite de  $(x_n)_{n \in \N}$ et:
    \[
        (x_n)_{\phi(n)} = \{x_0, x_2, x_4, \ldots\}
    \] 
\end{eg}
\begin{prop}
   \begin{enumerate}
       \item Toute sous-suite d'une suite convergente converge vers la limite de cette suite.
           \par
           Cela signifie que, $\forall (x_n)_{n \in \N}$ tq $\exists x \in E, \lim_{n \to \infty} x_n = x$
           \[
               \forall \phi: \N \to \N \text{ strictement croissante}, \lim_{n \to \infty} x_{\phi(n)} = x
           \] 
        \item Si $(x_n)$ est de Cauchy et admet une sous-suite qui converge vers  $X$, alors  $(x_n)$ converge vers  $x$.
   \end{enumerate} 
\end{prop}
\begin{preuve}
   \begin{enumerate}
       \item Soit $(x_n)$ avec  $\lim x_n = x$
            \[
                \forall \epsilon > 0 \, \exists M(\epsilon) \text{ tq si } n \ge N(\epsilon), d(x_n ,x) \le \epsilon 
           \] 
           Soit $y_n = x_{\phi(n)}$ une sous-suite.
           \begin{itemize}
               \item \underline{But:} Soit $\epsilon > 0$, trouver $N(\epsilon)$ tq si  $n \ge  N(\epsilon), \, d(\underbrace{y_n}_{:= x_{\phi(n)}}, x) \le \epsilon$
           \end{itemize}
           Je choisis $N(\epsilon)$ tel que si  $n \ge N(\epsilon)$ alors $\phi(n) \ge M(\epsilon)$, donc $d(y_n, x) d(x_{\phi(n)}, x) \le  \epsilon$. C'est possible car $\phi(n) \xrightarrow[n \to \infty]{} \infty$, $N(\epsilon) = M(\epsilon)$ 
        \item 
            \begin{itemize}
                \item  \underline{Hyp1:} $\forall \epsilon > 0 \, \exists M(\epsilon)$ tq si $n, p \ge M(\epsilon)$ $d(x_n, x_p) \le \epsilon$
                \item  \underline{Hyp2:} $\forall \epsilon > 0 \, \exists P(\epsilon)$ tq si $p \ge P(\epsilon), d(y_p, x) \le \epsilon$, $d(y_p, x) = d(x_{\phi(p)}, x)$
            \end{itemize}
            \begin{align*}
                d(x_n, x) \le d(x_n, x_{\phi(p)}) + d(x_{\phi(p)}, x) \quad \text{par l'inégalité triangulaire}
            \end{align*}
            \begin{align*}
                d(x_n, x_{\phi(p)}) \le \epsilon \text{ si } n \ge M(\epsilon) \text{ et } \phi(p) \ge M(\epsilon)
            \end{align*}
            \begin{align*}
                d(x_{\phi(p)}, x) \le \epsilon \text{ si } p \ge P(\epsilon)
            \end{align*}
            Si $n \ge M(\epsilon)$, je choisis $p$ tel que  $\phi(p) \ge  M(\epsilon)$ et $p \ge P(\epsilon)$. Je fixe ce $p$!
             \[
            \text{si } n \ge M(\epsilon) \text{ alors } d(x_n, x) \le 2\epsilon
            \] 
   \end{enumerate} 
\end{preuve}

\section{Procédé de construction de l'intérieur et l'adhérence}
J'ai $A \subset \R$ ou $\R^2$ (ou $\R^3$). Je dois trouver $Int(A)$ et  $Adh(A)$
 \begin{enumerate}
    \item Je dessine $A$ sur une feuille
    \item Je pense que  $Int(A) = C$ ($C$ dit être inclu dans  $A$!)
         \begin{enumerate}
             \item Je montre que \underline{$C$ est ouvert} (facile), donc
                 \[
                 C \subset Int(A)
                 \] 
                 car $Int(A)$ est le plus grand ouvert inclu dans  $A$.
            \item Je montre que $Int(A) \subset C$, i.e je montre que les points dans $A$ mais pas dans  $C$ ne sont pas dans  $Int(A)$: je prends $X \in A, X \not\in C$, je montre que $X \not\in Int(A)$
                Je construit une suite $(X_n)$ avec  $X_n \not\in A$ mais  $X_n \to X$.
         \end{enumerate}
    \item Je pense que $Adh(A) = B$ (il faut que $A \subset B$)
        \begin{enumerate}
            \item Je montre que $B$ est fermé (facile)
                \[
                \text{donc } Adh(A) \subset B
                \] 
            \item  On montre que $B \subset Adh(A)$: On fixe $X \in B$, on cherche une suite  $(X_n)$ avec  $X_n \in A$ et  $X_n \xrightarrow[n \to \infty]{} X$. 
                \underline{On regarde seulement les} $X \in B, X \not\in A$
        \end{enumerate}
\end{enumerate}
\begin{eg}
   \[
       A = \{(x, y) \in \R^2 \mid 2x + 3y \le 4, x \neq y\}
   \]  
\begin{figure}[H]
    \centering
    \incfig{example-interieur}
    \caption{Exemple de l'intérieur}
    \label{fig:example-interieur}
\end{figure}
.
\begin{itemize}
    \item 
        Je dévine que $Int(A) = C = \{(x, y) \mid 2x + 3y < 4, x \neq y\}$
    \item
        Convect: $\{(x, y) \mid 2x + 3y < 4, x < y\} \cup \{(x, y) \mid 2x + 3y < 4, x > y\}$
\end{itemize}
Je construit une suite $(X_n)$ avec  $X_n \not\in A$ mais $X_n \to X$. Soit $X \in A, X \not\in C$, $X = (x, y)$ donc:  $2x + 3y = 4 \quad x \neq y$
\[
X_n = (x, y + \frac{1}{n})
\] 
\[
2x_n + 3y_n = 2x + 3y + \frac{3}{n} = 4 + \frac{3}{n} > 4
\] 
\[
X_n \not\in A \text{ mais } X_n \to X
\] 
\end{eg}
\begin{eg}
   \[
       A = \{(x, y) \in \R^2 \mid x > 0, y = x^{-1}\}
   \]  
   $Int(A) = \O$? $C = \O$
\begin{figure}[H]
    \centering
    \incfig{example-interieur-hyperbola}
    \caption{Exemple de l'intérieur de l'hyperbole}
    \label{fig:example-interieur-hyperbola}
\end{figure}
$\O$ ouvert, donc $C \subset Int(A)$
\par
Soit $X \in A \quad X \not\in C$, donc $X \in A$.
 \[
X_n := (x, y + \frac{1}{n}) \quad X_n \not\in A
\] 
\[
x_ny_n = xy + \frac{x}{n} = 1 + \frac{x}{n} \neq 1
\] 
\[
X_n \xrightarrow[n \to \infty]{} X \text{ donc } X \not\in Int(A)
\] 
\[
Int(A) = \O
\] 
\end{eg}
\begin{eg}
   \[
       A = \{(x, y) \in \R^2 \mid x > 0, y = x^{-1}\}
   \]  
   $Adh(A) = ?$
   \par
   Je pense que  $Adh(A) = A$ ($B = A$). Il suffit de montrer que $A$ \underline{est fermé}.
    \[
   x > 0 \quad y \le \frac{1}{x} \quad y \ge \frac{1}{x}
   \] 
   Si $X_n = (x_n, y_n)$  \quad $X_n \in A$ et  $X_n \to X$, alors $X \in A$
    \[
        X = (x, y) \quad \substack{x_n \to x \\ y_n \to y} \quad \substack{x_n \to x \\ \frac{1}{x_n} \to y} \quad (x_n > 0)
   \] 
   donc $x > 0$ et  $y = \frac{1}{x}$ donc $X \in A$
    \[
   A \text{ est fermé}
   \] 
\end{eg}
\begin{eg}
   \[
       A = \{(x, y) \in \R^2 \mid 2x + 3y \le 4, x \neq y\}
   \]  
\begin{figure}[H]
    \centering
    \incfig{example-adherence}
    \caption{example-adherence}
    \label{fig:example-adherence}
\end{figure}
.
\begin{enumerate}
    \item 
        $B$ est fermé (facile), donc  $Adh(A) \subset B$
    \item Soit $X \in B$. On montre que  $X \in Adh(A)$ (on cherche $X_n \in A$ avec  $X_n \to X$) 
        \par
        Je regarde juste $X \in B, X \not\in A$
        \[
        X_n = (x_n, y_n) \in A \quad x_n \to x \text{ et } y_n \to y
        \] 
        \[
        x_n = x + \frac{1}{n}, y_n = y = x
        \] 
        \[
        X_n \to X \text{ et } 2x_n + 3y_n = 2x + 3y - \frac{2}{n} \le 4 et x_n \neq y_n 
        \] 
        donc $X_n \in A$
\end{enumerate}
\end{eg}
\begin{eg}
   \[
       A = \{(x, y) \mid |x| \le 1, |y| < 1\}
   \]  
   \[
       Int(A) = \{(x, y) \mid |x| < 1, |y| < 1 \}
   \] 
   \[
       Adh(A) = \{(x, y) \mid |x| \le 1, |y| \le 1\}
   \] 
\end{eg}
\begin{eg}
   \[
       A = \{(x,y) \mid x > 0, y = \sin(\frac{1}{n}) \}
   \]  
   $Adh(A) = A \cup \{(0, y) \mid -1 \le y \le 1\}$
   $Int(A) = $
\end{eg}

\section{Compacité}
\begin{definition}
   Soit $F \subset E$. Un recouvrement ouvert de $F$ est une collection  $(U_i)_{i \in I}$ où $U_i$ sont des ouverts et $F \subset \cup_{i \in I} U_i$ ("les $U_i$ recouvrent  $F$")
\end{definition}
\begin{eg}
\begin{figure}[H]
    \centering
    \incfig{recouvrement-ouvert}
    \caption{recouvrement-ouvert}
    \label{fig:recouvrement-ouvert}
\end{figure}
\begin{itemize}
    \item $U_x = B(x, \frac{1}{2})$
    \item $\bigcup_{x \in F} U_x$ contient $F$
    \item  $(U_x)_{x \in F}$ recouvrement ouvert de $F$
\end{itemize}
\end{eg}

\begin{definition}
    $K \subset E$ est compact si de tout recouvrement ouvert $(U_i)_{i \in I}$ de $F$ on peut extraire un sous-recouvrement fini: je peux choisir  $i_1, \ldots, i_n \in I$ tels que 
    \[
    F \subset U_{i_1} \cup U_{i_2} \cup \ldots \cup U_{i_n}
    \] 
\end{definition}
\begin{property}
   Un ensemble fini est compact. 
   \[
       F = \{a_1, \ldots, a_p\} \quad a_j \in E
   \] 
   $(U_i)_{i \in I}$ recouvre $F$.
   Je choisit  $a_j$ (point de $F$), il existe un $i \in I$ noté  $i(j)$ tel que 
    \[
   a_j \in U_{i(j)} \quad F \subset U_{i(1)} \cup \ldots \cup U_{i(p)}
   \] 
\end{property}
\begin{theorem}
    Caractérisation à l'aide de suites. \par
    $K \subset E$ est compact ssi toute suite d'éléments de $K$ admet une sous-suite qui converge vers un élément de  $K$.
\end{theorem}
\begin{eg}
    \begin{figure}[H]
        \centering
        \incfig{compactness-with-sequences}
        \caption{Compacité avec les suites}
        \label{fig:compactness-with-sequences}
    \end{figure}
    \begin{itemize}
        \item $E = \R^2$
        \item $F = B(x_0, r)$ pas compact
        \item $x_n \in F$,  $x_n \to x$, $x \not\in F$
        \item si $y_n = x_{\phi(n)}$, $y_n \to x$ mais $x \not\in F$
    \end{itemize}
\end{eg}
\begin{eg}
    
\begin{figure}[H]
    \centering
    \incfig{suite-sans-sous-suite-convergente}
    \caption{suite-sans-sous-suite-convergente}
    \label{fig:suite-sans-sous-suite-convergente}
\end{figure}
\[
    F = \{(x, y): x \ge 0, -\frac{1}{x} \le y \le \frac{1}{x} \}
\] 
$u_n = (n, 0)$  $(u_n)$ suite dans  $F$ sans sous-suite convergente.
\end{eg}

\begin{prop}
    \begin{enumerate}
        \item $K$ compact $\implies$ $K$ fermé et borné. (réciproque est fausse en général!)
        \item Si $K$ compact et $F$ fermé, alors  $K \cap F$ est compact.
        \item Si $K$ compact, toute suite de Cauchy dans  $K$ converge dans  $K$
    \end{enumerate}
\end{prop}

\begin{preuve}
   \begin{enumerate}
       \item Soit $K$ compact.  $K$ fermé si  $(u_n)$ suite dans  $K$ qui converge vers  $u$, alors  $u \in K$.
           \par
           \underline{clair:}  $(u_n)$ a une suite-suite  $v_n = u_{\phi(n)}$ avec $v_n \to v \in K$, $u_n \to u$, donc $v_n \to u$ $\implies$ $u = v$  $\implies$ $u \in K$
           \par
           $K$ \underline{est borné}:
           \par
              Soit $U_x = \bigcup_{x \in K} B(x, 1)$ un recouvrement ouvert de $K$. Or  $K$ est compact, donc il existent  $x_1, \ldots, x_n \in K$, tels que $K \subset \bigcup_{i = 1, \ldots, n} B(x_i, 1) $, donc $K$ est borné.
        \item $K$ compact et $F$ fermé. $(u_n)$ une suite dans $K \cap F$. $u_n \in K$. $ \exists$ sous-suite $v_n = u_{\phi(n)}$ avec $v_n \to x \in K$. $v_n \in F, v_n \to x$, $F$ fermé donc $x \in F$, $x \in K \cap F$.
        \item Soit $(u_n)$ suite de Cauchy dans $K$. $(u_n)$ a une sous-suite $v_n = u_ { \phi(n)}$ qui converge vers $x \in K$. $u_n \to x \in K$
   \end{enumerate} 
\end{preuve}

\subsection{Compacité dans $\R^n$ avec la distance usuelle}
\begin{theorem}
    (Borel-Lebesgue)
    \par
    dans $\R^n$ avec la distance usuelle $K$ est compact ssi  $K$ est fermé et borné
\end{theorem}

\begin{prop}\label{prop:boules-fermes-sont-compactes}
   Les boules fermées $B_f(x_0, r)$ sont compactes dans $\R^n$. 
\end{prop}
\begin{itemize}
    \item Implique le théorème: Soit $K$ fermé et borné.  $K$ borné, donc  $K \subset B_f(0, r)$ avec $r$ grand, donc  $K = K \cap B_f(0, r)$. Donc  $K$ compact.
\end{itemize}
\begin{preuve} de la prop. \ref{prop:boules-fermes-sont-compactes}
   \begin{enumerate}
       \item $n = 1$.  À montrer: $[a, b]$ est compact.
           \par
           Soit  $(U_i)_{i \in I}$ un recouvrement ouvert de $[a, b]$. Soit  $F$: les  $x \in [a, b]$ tels que  $[a, x]$ est récouvert par un nombre fini de  $U_i$.
           \par
           \underline{But:} montrer que  $b \in F$! (si $x \in F$, et  $x' \le x$ $x' \in F$)
            \begin{enumerate}
                \item $F \neq \O$: $a \in F$  $[a, a] = \{ a \}$
                \item  $c = sup(F)$. \underline{On montre que $c = b$} \par
                    Supposons que $c < b$.
                     \begin{itemize}
                        \item $c$ appartient à un des  $U_i$ noté  $U_{i_0}$
                        \item $U_{i_0}$ est ouvert, $c \in U_{i_0}$ donc $\exists \delta_0 > 0$ tel que $]c - \delta_0, c + \delta_0[ \subset U_{i_0}$
                        \item $c = sup(F)$:  $\forall \delta > 0, \, \exists x_{\delta} \in F$ avec $c - \delta < x_{\delta} \le c$
                            \[
                            \delta = \delta_{0,2} \quad \exists x_{\delta_0} \in F, c - \delta_{0,2} < x_{\delta_0}
                            \] 
                            $[a, x_{\delta_0}]$ reouvert par $U_{i_1} \cup \ldots \cup U_{i_n}$ et $]c - \delta_0, c + \delta_0[ \subset U_{i_0}$ donc $[a, c + \delta_{0,2}]$ est reouvert par $U_{i_0} \cup U_{i_1} \cup \ldots \cup U_{i_n}$, donc $c + \delta_{0, 2} \in F$ contredit que $c = sup(F)$. Donc  $c = b$.
                            \par
                            $F$ c'est  $[a, b[$ ou $[a, b]$.  $b \in F$  $\exists U_{i_1}, \ldots, U_{i_n}$ tq $[a, b] \subset U_{i_1} \cup \ldots \cup U_{i_n}$, $[a, b]$ compact.
                    \end{itemize}
            \end{enumerate}
   \end{enumerate}
\end{preuve}

\section{Limites et continuité}
\subsection{Limites}
Je prends $(E_1, d_1), (E_2, d_2)$ deux espaces métriques et $F: E_1 \to E_2$. $x_0 \in E_1, l \in E_2$.
\begin{definition}.
    \begin{enumerate}
        \item Limite:
            \[
            \lim_{x \to x_0} F(x) = l
            \] 
            si $\forall \epsilon > 0, \exists \delta > 0$ tq si $d_1(x_0, x) < \delta$ alors $d_2(l, F(x)) < \epsilon$
        \item $F$ continue en  $x_0$ si $\lim_{x \to x_0} F(x) = F(x_0)$
        \item $F$ est continue (sur $E$) si elle est continue en tout $x_0$ de $E$
\end{enumerate}
\end{definition}
\begin{prop}\label{prop:continuité-de-fonctions}
   Les propriétés suivantes sont équivalentes: 
   \begin{enumerate}
       \item $F: (E_1, d_1) \to  (E_2, d_2)$ est continue.
       \item $\forall U_2 \subset  E_2$ ouvert, $F^{-1}(U_2)$ est ouvert dans $E_1$.
       \item $\forall F_2 \subset E_2$ fermé, $F^{-1}(F_2) \subset E_1$ est fermé.
        \item $\forall (x_n)$ suite dans $E_1$ avec $\lim_{n \to \infty} x_n = x$ on a:
            \[
            \lim_{n \to \infty} F(x_n) = F(x)
            \] 
   \end{enumerate}
\end{prop}
\begin{figure}[H]
    \centering
    \incfig{continuite-topologique}
    \caption{continuite-topologique}
    \label{fig:continuite-topologique}
\end{figure}
\begin{eg}
   \[
       U = \{(x, y) \in \R^2: x \sin(y) - e^x > 1\}
   \]  
   \begin{align*}
       F: \R^2 &\longrightarrow \R \\
       (x, y) &\longmapsto F((x, y)) = x \sin(y) - e^x
   \end{align*}
   évidemment continue.
   \[
       U = F^{-1}(\underbrace{]1, +\infty[}_{\text{ouvert de } \R})
   \] 
\end{eg}
\begin{preuve}
   $1 \implies 2 \implies 3 \implies 4 \implies 1$ 
   \begin{itemize}
       \item[$1 \implies 2$:] Hyp: $F$ continue et  $U_2 \subset E_2$ est ouvert.
           \par
           Conclusion: $U_1 = F^{-1}(U_2)$ est ouvert?
           \par
           Je fixe $x_0 \in U_1$ ($F(x_0) \in U_2$).
           \begin{enumerate}
               \item $U_2$ ouvert $\implies$ $\exists \epsilon_0 > 0$ tq $B_2(F(x_0), \epsilon_0) \subset U_2$
               \item $F$ continue en  $x_0$: 
                   \[
                   \forall \epsilon > 0, \, \exists \delta > 0 \text{ tq } d_1(x_0, x) < \delta \implies d_2(F(x_0), F(x)) < \epsilon
                   \] 
                   \[
                   x \in B_1(x_0, \delta) \implies F(x) \in B_2(F(x_0), \epsilon)
                   \] 
                   $\delta_0 = $ le  $\delta$ qui marche pour  $\epsilon_0$
                    \[
                   x \in B_1(x_0, \delta_0) \implies F(x) \in B_2(F(x_0), \epsilon_0)
                   \] 
                   Donc $B_1(x_0, \delta_0) \subset F^{-1}(U_2)$. Donc $F^{-1}(U_2)$ ouvert.
           \end{enumerate}
       \item[$2 \implies 3$:]: $F^{-1}(U_2)^{c} = F^{-1}(U_2^c)$
   \end{itemize}
\end{preuve}

\begin{eg} résultat de cette proposition.
    Prenons la fonction: $f(x) = x^2$.  $f^{-1}(]4, 9[) = \{x \in \R \mid 4 < x^2 < 9\} = ]-3, -2[ \cup ]2, 3[$. Autrement dire, la continuité de $f$ (évident) donne que  $U = ]4, 9[$ ouvert, alors $f^{-1}(U)$ aussi ouvert.
    \begin{figure}[H]
        \centering
        \begin{tikzpicture}
            \begin{axis}[
                axis lines = middle,
                xlabel = $x$,
                ylabel = {$f(x) = x^2$},
                xmin=-5, xmax=5,
                ymin=0, ymax=10,
                samples=100,
                domain=-5:5,
                xtick={-4,-3,-2,-1,0,1,2,3,4,5},  % Custom x-axis numbers
                ytick={0,4,9},   % Custom y-axis numbers
                thick
                ]
                \addplot[blue, thick] {x^2};
                \coordinate (a) at (2, 0);
                \node (_) at (a){$]$};
                \coordinate (b) at (-2, 0);
                \node (_) at (b){$[$};
                \coordinate (c) at (3, 0);
                \node (_) at (c){$[$};
                \coordinate (d) at (-3, 0);
                \node (_) at (d){$]$};

                \coordinate (fa) at (2, 4);
                \coordinate (fb) at (-2, 4);
                \coordinate (fc) at (3, 9);
                \coordinate (fd) at (-3, 9);
                \draw[color=red] (a) -- (c);
                \draw[color=red] (b) -- (d);

                \draw[dashed] (a) -- (fa) -- (fb) -- (b);
                \draw[dashed] (c) -- (fc) -- (fd) -- (d);
            \end{axis}
        \end{tikzpicture} 
        \caption{Exemple en $f(x) = x^2$}
    \end{figure}
\end{eg}

% --- CHUNK_METADATA_START ---
% needs_review: True
% src_checksum: 74d2bbf6d9ead8152362c6422ea0e12b99956aeff8fb5d19ced1481ede0d1fd6
% --- CHUNK_METADATA_END ---
\section{Анулюючі многочлени}% --- CHUNK_METADATA_START ---
% needs_review: True
% src_checksum: 388160a58d43eb5002bc29e27e78e2b0a5de5c8f76dc2dcdba11ae574192a3c9
% --- CHUNK_METADATA_END ---
У попередніх розділах ми дізналися, що для того, щоб з'ясувати, чи є матриця діагоналізованою, необхідно вивчити власні простори, що не завжди дуже легко і не є найшвидшим способом. Отже, в цьому розділі ми розглянемо один з інших методів вивчення діагоналізовності, одним з цих методів є вивчення анулюючих многочленів.% --- CHUNK_METADATA_START ---
% needs_review: True
% src_checksum: 4f22b0eb38fef4a7f6b68aa62328910c678d684e80e4446e2fa7d4f580e723ba
% --- CHUNK_METADATA_END ---
\begin{remark}
   У цьому розділі я не пишу більшість доказів, а скоріше інтуїцію, чому це правда і чому це працює. 
\end{remark}% --- CHUNK_METADATA_START ---
% needs_review: True
% src_checksum: 7c070585845657f62897af4e2ffbadbc8a5167ad9d1b1527b19e195b640bb76d
% --- CHUNK_METADATA_END ---
\begin{definition}\label{def:polynome-annulateur}
    Нехай $f \in \mathbb{K}^n$ ендоморфізм. Поліном $Q(X) \in K[X]$ є \textbf{анулюючим поліномом} для $f$ якщо $Q(f) = 0$.
\end{definition}% --- CHUNK_METADATA_START ---
% needs_review: True
% src_checksum: 48dcdcf5adba02023072764f9fcd70860b6c2ad9d7180ac921fe7a3f5bf3ad0f
% --- CHUNK_METADATA_END ---
\begin{eg}
   Нехай $f$ проєкція, тоді, ми знаємо, що $f^2 = f$, звідки $f^2 - f = 0$, тому $Q(X) = X^2 - X = X(X - 1)$ є анулюючим поліномом для $f$.
\end{eg}% --- CHUNK_METADATA_START ---
% needs_review: True
% src_checksum: 03256e33031d32a446c95907fa9769fe691de26f127fe4c077e485c26abea50a
% --- CHUNK_METADATA_END ---
Важливо те, що анулюючі многочлени тісно пов'язані з власними значеннями:% --- CHUNK_METADATA_START ---
% needs_review: True
% src_checksum: 9a77faa938f34b4f2e8edcc8dcd6e5b77140255c74309638229856c2cc93f3dc
% --- CHUNK_METADATA_END ---
\begin{prop}
   Нехай $Q(X)$ є анулюючим многочленом для $f$, тоді власні значення $f$ знаходяться серед коренів $Q$, тобто:
   \[
       \operatorname{Sp}(f) \subset \operatorname{Rac}(Q)
   \] 
\end{prop}% --- CHUNK_METADATA_START ---
% needs_review: True
% src_checksum: 833385c5bcc9e301f6931f1b5e1a7892bb0fcdf984f5f0ae5e735357fcfd7f26
% --- CHUNK_METADATA_END ---
\begin{proof}
    Нехай $Q(X) = a_n X^n + a_{n-1} X^{n-1} + \ldots + a_0$ — анулюючий многочлен для $f$ і $\lambda$ — власне значення для $f$. Отже, $\exists v \neq 0 \in E$ така що $f(v) = \lambda v$, більше того:
    \[
        Q(f) = a_n f^n + a_{n-1} f^{n-1} + \ldots + a_0 \operatorname{Id} = 0
    \] 
    Але $f(v) = \lambda v$, тому $f^2(v) = f(\lambda v) = \lambda^2 v$, звідки $f^k(v) = \lambda^k v$ $\forall k \in \N$, тоді:
    \[
        Q(f(v)) = 0 = (a_n f^n + a_{n-1} f^{n-1} + \ldots + a_0 \operatorname{Id})v = (a_n \lambda^n + a_{n-1} \lambda^{n-1} + \ldots + a_0 \operatorname{Id})v = 0
     \] 
     Але $v \neq 0$, тому $a_n \lambda^n + a_{n-1} \lambda^{n-1} + \ldots + a_0 \operatorname{Id} = 0$, звідки $\lambda$ є коренем $Q$.
\end{proof}% --- CHUNK_METADATA_START ---
% needs_review: True
% src_checksum: 12483be17214331a02678b687db0b149599401ac99fa148c4b90cab081705332
% --- CHUNK_METADATA_END ---
\begin{note}
    Однак, рівність не є загалом вірною, наприклад $\operatorname{Id}^2 = \operatorname{Id}$, отже $Q(X) = X^2 - X = X(X - 1)$ обнуляє $\operatorname{Id}$ з коренями $0$ та $1$, але $0$ не є власним значенням для $\operatorname{Id}$.
\end{note}% --- CHUNK_METADATA_START ---
% needs_review: True
% src_checksum: 0554600a2792f8f908c854bf1ca22807161de363016b32ab96ea1974f3eb92ae
% --- CHUNK_METADATA_END ---
\begin{theorem}\label{thm:cayley-hamilton} \textbf{Кайлі-Гамільтона}. Нехай $f \in K^n$ ендоморфізм та $P_f(X)$ його характеристичний поліном, тоді
     \[
    P_f(f) = 0
    \] 
    Іншими словами, характеристичний поліном ендоморфізму є його анулюючим поліномом.
\end{theorem}% --- CHUNK_METADATA_START ---
% needs_review: True
% src_checksum: 19651c78378f539b9b35c4882f09c6e6970cbccd4bcf852752b98b7987e62c09
% --- CHUNK_METADATA_END ---
%  TODO: довести <15-04-25, dobbikov> %
% --- CHUNK_METADATA_START ---
% needs_review: True
% src_checksum: 26efee0238d78e545da285e09fcdde858c12ec8f95b7eeca9504301396c9f37d
% --- CHUNK_METADATA_END ---
\begin{intuition}
   Характеристичний поліном описує нам структуру $f$, тобто які операції потрібно виконати, щоб втратити принаймні один вимір, якщо ми отримуємо множники вигляду $(X - \lambda)^n$, отже, потрібно застосувати $f(v) - \lambda v) = v_r$, а потім до результату $v_r$ знову, тобто $f(v_r) - \lambda v_r$, і повторюємо $n$ разів (це відбувається у випадках тригоналізованих матриць)

   Теорема залишається вірною навіть у випадках, коли ендоморфізм не є тригоналізовним, оскільки ми можемо вибрати замикання $K'$ поля $K$, в якому знаходиться наш ендоморфізм, і він стає тригоналізовним (наприклад, $\mathbb{C}$ для $\mathbb{R}$).

   Крім того, характеристичний поліном дає нам $\ker(P_f(X)) = E$, тобто вектори, які стають нульовими під дією $P_f(f)$, цікавий факт полягає в тому, що всі вектори з $E$ належать до цього ядра, і тому $\forall v \in E$, $p_f(f)v = 0$, звідки $p_f(f) = 0$.
\end{intuition}% --- CHUNK_METADATA_START ---
% needs_review: True
% src_checksum: d88ea0e42078a13cb42b5707ac14746b75251ad291fdb416378807d11b627dd6
% --- CHUNK_METADATA_END ---
\begin{definition}
    Нехай $Q$ — розкладений поліном:
     \[
         Q(X) = (X - a_1)^{\alpha_1} \cdots (X - a_r)^{\alpha_r}
    \] 
    Поліном 
    \[
    Q_1 = (X - a_1) \cdots (X - a_r)
    \] 
    називається \textbf{радикалом} $Q$ (тобто розкладений поліном (той самий поліном, але без степенів біля дужок). 
    \par Більше того, $Q_1 \mid Q$, тобто радикал полінома ділить сам поліном.
\end{definition}% --- CHUNK_METADATA_START ---
% needs_review: True
% src_checksum: 3eda218f4e180ff0801bdb823ca755587104598cbc2e16cf834a81850705086a
% --- CHUNK_METADATA_END ---
\begin{prop}
    Нехай $f$ є ендоморфізмом і
     \[
         P_f(X) = (-1)^n(X - \lambda_1)^{\alpha_1} \cdots (X - \lambda_p)^{\alpha_p}
    \]
    є його характеристичним многочленом. Тоді, якщо $f$ є діагоналізовним, радикал $Q_1$ анулює $f$ також, тобто
     \[
    Q_1(f) = (f - \lambda_1) \cdots (f - \lambda_r) = 0
    \]
\end{prop}% --- CHUNK_METADATA_START ---
% needs_review: True
% src_checksum: 4b4b0dbdc69c2b8816e01712982bae8a99770c9cbb0864f6a33505968b722423
% --- CHUNK_METADATA_END ---
\begin{intuition}
   Я даю інтуїцію доведення. Якщо $f$ є діагоналізованою з характеристичним поліномом
     \[
         P_f(X) = (-1)^n(X - \lambda_1)^{\alpha_1} \cdots (X - \lambda_p)^{\alpha_p}
    \] 
    з $r := \alpha_i > 1$ це \underline{не означає}, що потрібно застосовувати $(f - \lambda_i \operatorname{Id})$ $r$ разів для зменшення розмірності як у випадку тригоналізованих матриць, але це означає, що $E_{\lambda_i}$ власний простір власного значення $\lambda_i$ має розмірність $\alpha_i = r$ і тому $\forall v \in E_{\lambda_i}, f(v) = \lambda_i v$. 

    Оскільки $E = E_{\lambda_1} \oplus \ldots \oplus E_{\lambda_p}$, якщо $v \in E$, тоді $\exists i \in \{1, \ldots, p\}$ така що $v \in E_{\lambda_i}$ і тому $f(v) - \lambda_i v = 0$ тобто $(f - \lambda_i \operatorname{Id})(v) = 0$. Звідси радикал $P_f$ анулює $f$.
\end{intuition}% --- CHUNK_METADATA_START ---
% needs_review: True
% src_checksum: b6670673bc3faad631505556abf072a29b10672fb6393278e6641d68428f127b
% --- CHUNK_METADATA_END ---
\section{Лема про ядра}% --- CHUNK_METADATA_START ---
% needs_review: True
% src_checksum: 20068f9667592f6e79b67c87d69fe214fed4ca81af52f565e9c8c839d07cb20f
% --- CHUNK_METADATA_END ---
\begin{lemma}\label{lemma:lemme-des-noyaux} \textbf{про ядра}
   Нехай $f \in K^n$ ендоморфізм і 
   \[
   Q(X) = Q_1(X) \cdots Q_p(X)
   \] 
   многочлен, розкладений у добуток попарно взаємно простих многочленів. Якщо $Q(f) = 0$, то:
    \[
        E = \operatorname{Ker} Q_1(f) \oplus \ldots \oplus \operatorname{Ker} Q_p(f)
   \] 
\end{lemma}% --- CHUNK_METADATA_START ---
% needs_review: True
% src_checksum: fe3c2c4f501dfbcfda20991bd7579ae44cc78b5fed9961c2873de026fa6891d6
% --- CHUNK_METADATA_END ---
\begin{intuition}
    Оскільки $Q(f) = 0$, тому $\forall v \in E, Q(f)(v) = 0$ отже
    $\operatorname{Ker}(Q(f)) = E$. $\exists v_1, \ldots, v_p$ такі що $v = v_1 +
    \ldots + v_p$. Але усі поліноми попарно взаємно прості, тоді лише один з них анулює $v_i$ тому $v_i \in \operatorname{Ker}Q_i(f)$ і
    це залишається правдою для всіх $v_1, \ldots, v_p$. І оскільки поліноми
    взаємно прості, тож якщо $k \neq j$ і $Q_k(v_i) = 0$, тоді $Q_j(v_i) \neq
    0$ бо $Q_j$ і $Q_k$ відрізняються. Тоді, $\forall i, j \,
    \operatorname{Ker}Q_i \cap \operatorname{Ker}Q_j = \{0\}$.
\end{intuition}% --- CHUNK_METADATA_START ---
% needs_review: True
% src_checksum: 776749d74cb1979ac8d0d6380de9b0d40e671d9773a72e3a38896192822cd28f
% --- CHUNK_METADATA_END ---
\begin{remark}
   Повернімося до прикладу $f$, яка є проєкцією, отже $f^2 - f = 0$ і $Q(X) = X^2 - X = X(X-1)$ анулює $f$. Проте $X$ і $X-1$ є взаємно простими, тоді 
    \[
        E = \operatorname{Ker}f \oplus \operatorname{Ker}(f - \operatorname{Id})
   \] 
    Щоб бути більш загальною, нехай $f$ є ендоморфізмом, і $Q(X) = (X - \lambda_1) \cdots (X - \lambda_p)$ така що $Q(f) = 0$, маємо:
     \[
         E = \underbrace{\operatorname{Ker}(f - \lambda_1 \operatorname{Id})}_{E_{\lambda_1}} \oplus \ldots \oplus \underbrace{\operatorname{Ker}(f - \lambda_p \operatorname{Id})}_{E_{\lambda_p}}
    \] 
    Звісно, $\lambda_i \neq \lambda_j$. І тоді $f$ є діагоналізованим, оскільки прямою сумою цих власних підпросторів.
\end{remark}% --- CHUNK_METADATA_START ---
% needs_review: True
% src_checksum: 7d9db5623101af13df78a593fd54a7570398e179ea58cfb4076059a2c0479cc3
% --- CHUNK_METADATA_END ---
\begin{corollary}
    Ендоморфізм $f$ є діагоналізованим тоді і тільки тоді, якщо існує анулюючий поліном $Q$ для $f$, який є розкладним і має лише прості корені \footnote{розкладний: $(X - \lambda_i)^{\alpha_i}$ - $X$ в степені $1$! прості корені: якщо $\alpha_i = 1$ також, тобто множники $(X - \lambda)$ в степені 1!} 
\end{corollary}% --- CHUNK_METADATA_START ---
% needs_review: True
% src_checksum: dcd9523065c4f0b772b8a2061f9f39ad63175ce452061c83e205f2b9c6f0953c
% --- CHUNK_METADATA_END ---
\section{Пошук анулюючих многочленів. Мінімальний многочлен}% --- CHUNK_METADATA_START ---
% needs_review: True
% src_checksum: a4bcf6aa64911c5b2fc23192f1480c1669b97c09b90a90b7f1cac551142db332
% --- CHUNK_METADATA_END ---
\begin{definition}
    Називається \textbf{мінімальний многочлен} для $f$, позначений $m_f(X)$ - нормалізований многочлен \footnote{тобто з коефіцієнтом $1$ при члені найвищого степеня, тобто: $1*X^n + a_{n-1}X^{n-1} + \ldots + a_0$} який анулює $f$ найменшого степеня.
\end{definition}% --- CHUNK_METADATA_START ---
% needs_review: True
% src_checksum: 3f3410373979f6d254b617ac467e93c977d12edbd837b37b0e1846f8a256dd43
% --- CHUNK_METADATA_END ---
\begin{prop}
   Анулюючі многочлени $f$ мають вигляд:
   \[
       Q(X) = A(X)m_f(X) \quad \text{ де } \quad A(X) \in K[X]
   \] 
   \text{ тобто } $m_f(X)$ \text{ ділить } $Q(X)$. 
\end{prop}% --- CHUNK_METADATA_START ---
% needs_review: True
% src_checksum: dfd8008cd4cd354832b2c84d1a95ec6dd91c323dd4aabf1ca63406a19f4b7427
% --- CHUNK_METADATA_END ---
\begin{prop}
   Корені мінімального полінома $m_f(X)$ є точно коренями характеристичного полінома $P_f(X)$, тобто власні значення.
\end{prop}% --- CHUNK_METADATA_START ---
% needs_review: True
% src_checksum: 9ae173893a86059f9cc9135a475e272f1f4917c32eaf8a7526a47e4a27bfe4dd
% --- CHUNK_METADATA_END ---
\begin{preuve}
   Ми знаємо, що $P_f(X) = A(X)m_f(X)$ тому якщо $\lambda$ є коренем $m_f(X)$, тоді вона є коренем $P_f(X)$ також. Навпаки, якщо $\lambda$ є коренем $P_f(X)$ тоді вона є власним значенням, а $m_f(X)$ анулює $f$, отже $\lambda$ також є коренем $m_f(X)$.
\end{preuve}% --- CHUNK_METADATA_START ---
% needs_review: True
% src_checksum: ee5dbc1d494bec59b3ab46c90aa27756840120a6431779b000cedda40c958836
% --- CHUNK_METADATA_END ---
\begin{theorem}
    Ендоморфізм $f$ є діагоналізовним тоді і тільки тоді, коли його мінімальний многочлен є розкладним і всі його корені прості.
\end{theorem}% --- CHUNK_METADATA_START ---
% needs_review: True
% src_checksum: a8de942faaeca497b14477980a0a02c0c1723833b56c4e22a9454495fe23aec2
% --- CHUNK_METADATA_END ---
\begin{eg}
   \begin{enumerate}
       \item $A = \begin{pmatrix} 
            -1 & 1 & 1\\
            1  & -1 & 1\\
            1  & 1  & -1
           \end{pmatrix} $. $P_A(X) = -(X - 1)(X + 2)^2$, отже, маємо дві можливості:
           \begin{itemize}
               \item $m_A(X) = (X - 1)(X + 2)$ - отже  $A$ діагоналізована
               \item  $m_A(X) = (X - 1)(X + 2)^2$ - отже $A$ не діагоналізована
           \end{itemize}
           Обчислимо:
           \[
            (A - I)(A + 2I) = \begin{pmatrix} 
                -2 & 1 & 1\\
                1 & -2 & 1\\
                1 & 1  & -2
            \end{pmatrix}\begin{pmatrix} 
                1 & 1 & 1\\
                1 & 1 & 1\\
                1 & 1 & 1
            \end{pmatrix} = \begin{pmatrix} 
                0 & 0 & 0\\
                0 & 0 & 0\\
                0 & 0 & 0
            \end{pmatrix}   
           \] 
           Отже, $m_f(X) = (X - 1)(X + 2)$ і тому $A$ є діагоналізованою.
        \item  $A = \begin{pmatrix} 
                3 & -1 & 1\\
                2 & 0  & 1\\
                1 & -1 & 2
            \end{pmatrix} $. Маємо: $P_A(X) = -(X - 1)(X - 2)^2$, отже:
            \[
            m_A(X) = \begin{cases}
                (X-1)(X-2) \quad \text{ тобто $A$ діагоналізована}\\
                (X-1)(X-2)^2 \quad \text{ тобто $A$ не діагоналізована}
            \end{cases}
            \] 
            Обчислимо:
            \[
                (A - I)(A - 2I) = \begin{pmatrix} 
                    2 & -1 & 1\\
                    2 & -1 & 1\\
                    1 & -1 & 1\\
                \end{pmatrix} 
                \begin{pmatrix} 
                    1 & -1 & 1\\
                    2 & -2 & 1\\
                    1 & -1 & 0
                \end{pmatrix} 
                =
            \begin{pmatrix}
            1 & -2 & 1 \\
            1 & -2 & 1 \\
            0 & -2 & 2
            \end{pmatrix} \neq \begin{pmatrix} 0 & 0 & 0\\  0 & 0 & 0\\ 0 & 0 & 0 \end{pmatrix} 
            \] 
            Звідси $m_A(X) \neq (X-1)(X-2)$ і тому $A$ не є діагоналізованою.
   \end{enumerate} 
\end{eg}


% --- CHUNK_METADATA_START ---
% needs_review: True
% src_checksum: 253d291659ae7159a918d82c5fc9d307e05e3782cefee1a6f63f646fd9fec3f0
% --- CHUNK_METADATA_END ---
\begin{appendices}
    \chapter{Reminders of Linear Algebra concepts} 
    \section{Matrices}
    \subsection{Multiplication of matrices}
    \begin{definition}
        Let $A \in \mathcal{M}_{p, n}(\R)$ and $B \in \mathcal{M}_{n, q}(\R)$ such that $A = (a_{j, i})$ and $B = (b_{m, k})$, then:
         \[
        AB = C = (c_{j, k} = \sum_{i=1}^{n} a_{j, i}b_{i, k})
        \] 
    \end{definition}
    \subsection{The trace}
    \begin{definition}
        The trace of the \( n \times n \) square matrix \( A \), denoted \( \text{tr}(A) \), is the sum of the diagonal elements

        \[
            \text{tr}(A) = a_{11} + a_{22} + \dots + a_{nn} = \sum_{i=1}^{n} a_{ii}
        \]

        where \( a_{ii} \) are diagonal elements of the matrix \( A \). 
    \end{definition}

    \begin{property} of the trace.
       \begin{itemize}
           \item Linearity:
               \[
                   \text{tr}(A + B) = \text{tr}(A) + \text{tr}(B)
               \]

               \[
                   \text{tr}(cA) = c \text{tr}(A), \quad c \in \mathbb{R} \text{ (ou } \mathbb{C} \text{)}
               \]
            \item  Transposed:
                \[
                    \text{tr}(A) = \text{tr}(A^T)
                \] 
            \item Multiplication of matrices:
                \[
                    \text{tr}(AB) = \text{tr}(BA), \quad \text{(si } A \text{ et } B \text{ are of size } n \times n)
                \]

                However, the trace is not distributive over multiplication:

                \[
                    \text{tr}(A B C) \neq \text{tr}(A) \text{tr}(B C)
                \]
            \item Eigenvalues:
                \[
                    \text{tr}(A) = \sum_{i=1}^{n} \lambda_i
                \]

                where \( \lambda_i \) are the eigenvalues of \( A \). This makes the trace an important tool in spectral analysis.

            \item Trace of the Identity Matrix

                \[
                    \text{tr}(I_n) = n
                \]

                since all the diagonal elements are equal to 1.
       \end{itemize} 
    \end{property}
    \begin{eg}
        For
        \[
            A = \begin{bmatrix} 3 & 2 & 1 \\ 4 & 5 & 6 \\ 7 & 8 & 9 \end{bmatrix}
        \]

        the trace is:

        \[
            \text{tr}(A) = 3 + 5 + 9 = 17
        \] 
    \end{eg}
    \begin{eg}
        If

        \[
            B = \begin{bmatrix} 2 & 1 \\ 0 & 3 \end{bmatrix}, \quad C = \begin{bmatrix} 4 & 2 \\ 1 & 5 \end{bmatrix}
        \]

        then

        \[
            \text{tr}(B + C) = \text{tr} \begin{bmatrix} 6 & 3 \\ 1 & 8 \end{bmatrix} = 6 + 8 = 14
        \]

        which corresponds well to

        \[
            \text{tr}(B) + \text{tr}(C) = (2+3) + (4+5) = 14
        \]

        thus confirming linearity. 
    \end{eg}
\end{appendices}



\nocite{*}
\printbibliography% --- CHUNK_METADATA_START ---
% needs_review: True
% src_checksum: 2dc670e7ffe1f12aa0326631b39a0b6d72da153425c7f5f9aed627a71c1487d6
% --- CHUNK_METADATA_END ---
\end{document}
