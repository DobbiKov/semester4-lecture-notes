\documentclass[a4paper]{article}

\usepackage[utf8]{inputenc}
\usepackage[T1]{fontenc}
\usepackage{textcomp}
\usepackage[english]{babel}
\usepackage{amsmath, amssymb, amsthm}


% figure support
\usepackage{import}
\usepackage{xifthen}
\pdfminorversion=7
\usepackage{pdfpages}
\usepackage{transparent}
\usepackage{hyperref}
\usepackage[margin=0.8in]{geometry}

\usepackage{setspace}
\setlength{\parindent}{0in}

\newcommand{\incfig}[1]{%
    \def\svgwidth{\columnwidth}
    \import{./figures/}{#1.pdf_tex}
}

\pdfsuppresswarningpagegroup=1

\newcommand{\N}{\mathbb{N}}
\newcommand{\R}{\mathbb{R}}
\newcommand{\Z}{\mathbb{Z}}
\newcommand{\Q}{\mathbb{Q}}

\newtheorem{theoreme}{Théorème}[section]
\newtheorem{definition}{Définition}[section]
\newtheorem{exemple}{Exemple}[section]
\newtheorem{prop}{Proposition}[section]
\newtheorem{propriete}{Propriété(s)}[section]
\newtheorem*{notation}{Notation}
\newtheorem*{remarque}{Remarque}

\newcommand{\scalair}[1]{\left\langle #1 \right\rangle}

\title{Oraux des Mathématiques: Séance 2}
\author{Yehor KOROTENKO}
\date{\today}
\begin{document}
\section{Algèbre}
\begin{prop}\label{prop:isometrie-ssi-transforme-bon-en-bon}
   $f$ est une isométrie si et seulement si elle transforme toute base orthonormée en une base orthonormée. 
\end{prop}
\begin{proof}
    Soit $f$ une isométrie, alors elle transforme toute base en une base car  $f$ est bijective d'apres le cours.
    \begin{itemize}
        \item ($\implies$) Supposons que $f$ est une isométrie. Soit $\{e_i\}$ une base orthonormée, alors, on a:
             \[
                 \scalair{f(e_i), f(e_j)} = \scalair{e_i, e_j} = \delta_{i,j}
            \] 
            Donc, $\{f(e_i)\}$ est une base orthonormée.
        \item ($\impliedby$) Supposons, qu'il existe une base orthonormée $\{e_i\}$ telle que  $\{f(e_i)\}$ est aussi une base orthonormée. De plus, soit  $x = x_1e_1 + \ldots x_ne_n$ et $y = y_1e_1 + \ldots + y_ne_n$ avec $x_i, y_i \in \R$
            \par
            Comme $\{e_i\}$ est orthonormée, alors on a:
            \begin{equation}
                \scalair{x, y} = x_1y_1 + \ldots + x_ny_n = \sum_{i=1}^{n} x_iy_i
            \end{equation}
            D'autre part:
            \begin{align*}
                \scalair{f(x), f(y)} &= \scalair{\sum_{i=1}^{n} x_if(e_i), \sum_{i=1}^{n} y_if(e_i)} = \sum_{i,j = 1}^{n} x_iy_j\scalair{f(e_i), f(e_j)}\\
                                     &= \sum_{i=1}^{n} x_iy_i = \scalair{x, y}
            \end{align*}
            Donc $f$ est une isométrie.
    \end{itemize}
\end{proof}

\section{Analyse}
\begin{prop}
    \begin{enumerate}
        \item $K$ compact $\implies$ $K$ fermé et borné. (réciproque est fausse en général!)
        \item Si $K$ compact et $F$ fermé, alors  $K \cap F$ est compact.
        \item Si $K$ compact, toute suite de Cauchy dans  $K$ converge dans  $K$
    \end{enumerate}
\end{prop}

\begin{proof}
   \begin{enumerate}
       \item Soit $K$ compact.  $K$ fermé si  $(u_n)$ suite dans  $K$ qui converge vers  $u$, alors  $u \in K$.
           \par
           \underline{clair:}  $(u_n)$ a une sous-suite  $v_n = u_{\phi(n)}$ avec $v_n \to v \in K$, $u_n \to u$, donc $v_n \to u$ $\implies$ $u = v$  $\implies$ $u \in K$
           Donc $K$ est fermé.
           \par
           $K$ \underline{est borné}:
           Soit $U_x = \bigcup_{x \in K} B(x, 1)$ un recouvrement ouvert de $K$. Or  $K$ est compact, donc il existent  $x_1, \ldots, x_n \in K$, tels que $K \subset \bigcup_{i = 1, \ldots, n} B(x_i, 1) $, donc $K$ est borné.
        \item $K$ compact et $F$ fermé. $(u_n)$ une suite dans $K \cap F$. $u_n \in K$. $ \exists$ sous-suite $v_n = u_{\phi(n)}$ avec $v_n \to x \in K$. $v_n \in F, v_n \to x$, $F$ fermé donc $x \in F$, $x \in K \cap F$. Donc $K \cap F$ est séquentiellement compact, donc compact.
        \item Soit $K$ un espace métrique compact et soit $(x_n)$ une suite de Cauchy dans $K$. Nous allons montrer que $(x_n)$ converge dans $K$.
            \par 
            Supposons que $(u_n)$ est de Cauchy, i.e  $\forall \epsilon > 0, \exists N \in \N, \forall n, m \ge N, d(u_n, u_m) < \epsilon$. Soit $\epsilon > 0$. Or  $K$ est compact, alors il existe  $v_n = u_{\phi(n)}$ tq  $v_n \xrightarrow[n \to +\infty]{} x \in K$. Donc $\exists M \in \N, \forall n \ge M, d(v_n, x) < \frac{\epsilon}{2}$. Or $u_n$ est de Cauchy, donc  $\forall n \ge N, \phi(n) \ge N, d(u_n, u_{\phi(n)}) < \frac{\epsilon}{2}$. L'inégalité triangulaire donne:
            \[
            d(u_n, x) \le d(u_n, u_{\phi(n)}) + d(u_{\phi(n)}, x) = \frac{\epsilon}{2} + \frac{\epsilon}{2} = \epsilon
            \] 
            Donc $u_n$ converge vers  $x$.
   \end{enumerate} 
\end{proof}

\end{document}
