\documentclass[a4paper]{article}

\usepackage[utf8]{inputenc}
\usepackage[T1]{fontenc}
\usepackage{textcomp}
\usepackage[english]{babel}
\usepackage{amsmath, amssymb, amsthm}


% figure support
\usepackage{import}
\usepackage{xifthen}
\pdfminorversion=7
\usepackage{pdfpages}
\usepackage{transparent}
\usepackage{hyperref}
\usepackage[margin=0.8in]{geometry}

\usepackage{setspace}
\setlength{\parindent}{0in}

\newcommand{\incfig}[1]{%
    \def\svgwidth{\columnwidth}
    \import{./figures/}{#1.pdf_tex}
}

\pdfsuppresswarningpagegroup=1

\newcommand{\N}{\mathbb{N}}
\newcommand{\R}{\mathbb{R}}
\newcommand{\Z}{\mathbb{Z}}
\newcommand{\Q}{\mathbb{Q}}

\newtheorem{theoreme}{Théorème}[section]
\newtheorem{definition}{Définition}[section]
\newtheorem{exemple}{Exemple}[section]
\newtheorem{proposition}{Proposition}[section]
\newtheorem{propriete}{Propriété(s)}[section]
\newtheorem*{notation}{Notation}
\newtheorem*{remarque}{Remarque}

\begin{document}
\section{Algèbre}
\begin{theoreme}
    Soit $A \in \mathcal{M}_n(\mathbb{K})$. On a:
    \[
        \operatorname{Com}(A)^{T}A = A\operatorname{Com}(A)^{T} = \det(A)I_n
    \] 
\end{theoreme}
\begin{proof}
    On pose $B = (b_{i, j})_{\substack{1 \le i \le n \\ 1 \le j \le n}}$ où
    \begin{align*}
        b_{i, j} &= \sum_{k=1}^{n} \operatorname{cof}_{k,i}(A)a_{k,j} \text{ par la multiplication des matrices}\\
                 &= \sum_{k=1}^{n} (-1)^{k+i}\det(A_{k,i})a_{k,j} \text{ par la définition de la cofacteur}
    \end{align*}
    On remarque que si $j = i$ (l'élément diagonal), on a:
     \[
    b_{i, j} = \sum_{k=1}^{n} (-1)^{k+i}\det(A_{k,i})a_{k,i} 
    \] 
    ce qui est le développement par rapport à la i-ème colonne.
    \par
    Alors, pour le cas où $j \neq i$ on a:

    \[
        b_{i, j} = \sum_{k=1}^{n} (-1)^{k+i}\det(A_{k,i})a_{k,j} 
    \] 
    ce qui est aussi un développement sauf que si on aurait changé la j-ième colonne par la i-ème:
    \[
        b_{i, j} = 
        \begin{vmatrix} 
            a_{1, 1} & \ldots & a_{1, i} & \ldots & a_{1, i} & \ldots & a_{1, n}\\
            \vdots   &        &          &        &          &        & \vdots \\
            a_{n, 1} & \ldots & a_{n, i} & \ldots & a_{n, i} & \ldots & a_{n, n}\\
        \end{vmatrix} = 0 \text{ car deux colonnes identiques}
    \] 
    Alors, $b_{i, j} = \begin{cases}
        \det(A) \text{ si } i = j\\
        0 \text{ si } i \neq  j
    \end{cases}$, d'où
    \[
    B = \begin{bmatrix} 
        \det(A) & 0 & \ldots & 0\\
        0       & \det(A) & \ldots & 0\\
        \vdots  & \ddots  & \ddots & \vdots\\
        0       & 0       & \ldots & \det(A) 
    \end{bmatrix}  = \det(A)I_n
    \] 
\end{proof}

\section{Analyse}
\begin{theoreme}
    Soit $f: \R^n \to \R$ une fonction continue et $K \subset \R^n$ un compact. Alors $f(K)$ est bornée et atteint ses bornes.
\end{theoreme}
\begin{proof}
   D'après le cours, $f(K)$ est compact dans  $\R$ car $K$ compact et  $f$ continue, donc bornée. Alors, $\exists a, b$ tq $a < b$ et  $\forall X \in K, a \le f(X) \le b$. Montrons que $\exists X_a, X_b \in K$ tq $a = f(X_a)$ et  $b = f(X_b)$.
   \par
   Soit  $a = \inf_{X \in K}(f)$, or  $f$ continue, donc  il existe une suite $X_n$ à valeur dans $K$, telle que  $f(X_n) \xrightarrow[n \to \infty]{} a$. Or $K$ compact, donc  il existe une sous-suite $X_{\phi(n)}$ qui converge vers une limite $X \in K$. Par l'unicité de limite, 
   \[
       a = \lim_{n \to \infty} f(X_{\phi(n)}) = f(X)
   \] 
   Soit  $b = \sup_{X \in K}(f)$, or  $f$ continue, donc  il existe une suite $X_n$ à valeur dans $K$, telle que  $f(X_n) \xrightarrow[n \to \infty]{} b$. Or $K$ compact, donc  il existe une sous-suite $X_{\phi(n)}$ qui converge vers une limite $X \in K$. Par l'unicité de limite, 
   \[
       b = \lim_{n \to \infty} f(X_{\phi(n)}) = f(X)
   \] 
   Donc les deux bornes de $f(K)$  $a$ et  $b$ sont atteintes.
\end{proof}

\section{Si jamais}
\begin{theoreme}
    Soit $f: \R^n \to \R^p$ une application continue et $K \subset \R^n$ un compact. Alors $f(K) \subset \R^p$ est compact.    
\end{theoreme}
\begin{proof}
    Soit $(Y_n) = (f(X_n))$ une suite dans  $f(K)$. La suite  $(X_n)$ est à valeur dans le compact  $K$, donc il existe une sous-suite  $(X_{\phi(n)})$ qui converge vers une limite $X$. Or  $f$ continue, donc  $\lim_{n \to \infty} Y_{\phi(n)} = \lim_{n \to \infty} f(X_{\phi(n)}) = f(X)$, donc de toute suite à valeurs dans $f(K)$ on peut extraire une sous-suite qui converge vers une limite aussi dans $f(K)$, alors  $f(K)$ est compact.
\end{proof}
\end{document}
