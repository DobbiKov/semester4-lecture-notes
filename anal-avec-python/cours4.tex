\section{Interpolation: définition-motivation-exemples}
\subsection{Définition}
\begin{definition}
    Soient $(x_i, y_i)_{i = \{1, \ldots, N\}}$ un nuage de points(exemple un ensemble discret de point du graphe d'une fonction). Interpoler ce nuage de points correspond à chercher un polynôme de degré $N-1$, qui passe par chacun de ces points.
\end{definition}
\begin{figure}[H]
    \centering
    \incfig{definition-example}
    \caption{L'exemple visuel de la définition}
    \label{fig:definition-example}
\end{figure}
Questions:
\begin{enumerate}
    \item Comment le construire?
    \item $P_{N-1} \in \R_{N-1}[X]$
    \item $P_{N-1}(x_i) = y_i$
\end{enumerate}

\subsection{Motivations}
\begin{itemize}
    \item La solution d'un problème est fournie par une formule représentative: Noyau de la chaleur (i.e convolution) est un cherche la solution en un nombre de points.
    \begin{itemize}
        \item On approche alors la fonction par un polynôme: i.e chercher le polynôme de degré "bas" proche de la fonction
    \end{itemize}
    \item La solution d'un problème n'est connue qu'à table des valeurs en un nombre fini de points et on souhaite l'évaluer partout.
        \begin{itemize}
            \item l'intérpolation
        \end{itemize}
    \item On peut utiliser l'intérpolation dans 
        \begin{itemize}
            \item l'intégration numérique
            \item la résolution numérique des EDO
            \item la visualisation scientifique
        \end{itemize}
\end{itemize}

\begin{definition}
    Un tel polynôme est appelé \textbf{polynôme intérpolateur de lagrange} de degré $N-1$ de ces points.
\end{definition}
\subsection{exemples d'intérpolation}
\begin{theorem}
    Polynôme intérpolateur de degré 1.
    \par
    Soient $(x_1, y_1), \, (x_2, y_2)$ 2 points distincts de $\R^2$
    \begin{itemize}
        \item Il existe une unique droite passant par les 2 points.
            \[
                (x, y) \in \mathcal{D} \iff (x-x_1)(y_2 - y_1) - (y - y_1)(x_2 - x_1) = 0
            \] 
        \item Si de plus, $x_1 \neq  x_2$, il existe un unique polynôme de degré $1$ (i.e $P_1 \in \R_1[X]$) tq:
            \[
                (x, y) \in \mathcal{D} \iff y = P(x) \text{ avec } P_1 = \frac{(x - x_1)y_1 - (x - x_2)y_2}{x_2 - x_1}
            \] 
    \end{itemize}
\end{theorem}
\begin{preuve}
    \begin{itemize}
        \item 
    \begin{align*}
        M \begin{pmatrix} x \\ y \end{pmatrix} \in \mathcal{D} &\iff \vec{MM_1} / \vec{M_1M_2}\\
                                                               &\iff det(\vec{MM_1}, \vec{M_1M_2}) = 0\\
                                                               &\iff \begin{vmatrix} x - x_1 & x_2 - x_1 \\ y - y_1 & y_2 - y_1 \end{vmatrix} = 0\\
                                                               &\iff (x - x_1)(y_2 - y_1) - (y - y_1)(x_2 - x_1) = 0
    \end{align*}
\item Si $x_1 \neq x_2$
    \begin{align*}
        M \in \mathcal{D} &\iff y - y_1 = \frac{(x - x_1)(y_2 - y_1)}{x_2 - x_1}\\
                          &\iff y = P_1(X)
    \end{align*}
\end{itemize}
\end{preuve}
\begin{remark}
    On a l'écriture équivalente de $P_1$:
   \begin{itemize}
       \item 
           \[
               P_1 \frac{x_0y_1 - x_1y_2}{x_2 - x_1} + X \frac{y_2 - y_1}{x_2 - x_1} \equiv a_0 + a_1X
           \] 
           C'est l'écriture dans la base $(1, X)$ de  $\R_1[X]$
       \item 
           \[
               P_1 = \underbrace{ \frac{x - x_2}{x_1 - x_2} }_{l_1}y_1 + \underbrace{ \frac{x - x_1}{x_2 - x_1} }_{l_2}y_2
           \] 
           C'est l'écriture dans la base $(l_1, l_2)$ de $\R_1[X]$
           \par
           RQ:
           \[
           l_1(x_1) = 1 \quad l_1(x_2) = 0 \quad l_2(x_1) = 0 \quad l_2(x_2) = 1
           \] 
           (base de lagrange)
        \item 
            \[
            P_1 = y_1 + \frac{y_2 - y_1}{x_2 - x_1}(x - x_1)
            \] 
            C'est l'écriture dans la base $(1, x - x_1)$ de $\R_1[X]$ (base de newton)
   \end{itemize} 
\end{remark}
\begin{eg}
   Méthode de calcul employle 
   \par
   Chercher le polynôme interpolateur de lagrange aux points $(x_1, y_1), (x_2, y_2), (x_3, y_3)$
   \begin{itemize}
       \item \underline{Méthode 1}: $x_1 \neq x_2 \neq x_3$ \par
           $P_2$ est un polynôme de degré 2
           \[
           P_2 = a_0 + a_1x + a_2x^2
           \] 
           \begin{lemma}
              \[
              P_2(x_1) = y_1, \quad P_2(x_2) = y_2 \qquad \text{i.e } a_0 + a_1x_1 + a_2x_1^2 = y_1
              \]  
           \end{lemma}
           \[
               \underbrace{
                   \begin{bmatrix} 
                       1 & x_1 & x_1^2\\
                       1 & x_2 & x_2^2\\
                       1 & x_3 & x_3^2
                   \end{bmatrix} 
               }_{M}\underbrace{\begin{bmatrix} a_0 \\ a_1 \\ a_2 \end{bmatrix} }_{} = \underbrace{\begin{bmatrix} y_1 \\ y_2 \\ y_3 \end{bmatrix} }_{} \implies \begin{bmatrix} a_0 \\ a_1 \\ a_2 \end{bmatrix} = \underbrace{M^{-1}}_{???}\begin{bmatrix} y_1 \\ y_2 \\ y_3 \end{bmatrix} 
           \] 
           \begin{remark}
              Par 2 points 
              \[
                  M = \begin{bmatrix} 1 & x_1 \\ 1 & x_2 \end{bmatrix} \text{ et } M^{-1} = \frac{1}{x_2 - x_1}\begin{bmatrix} x_2 & -x_1\\ -1 & 1 \end{bmatrix} 
              \] 
           \end{remark}
        \item \underline{Méthode 2}:\par
            \[
            P_2 = a_0 + a_1(x - x_1) + a_2(x - x_1)(x - x_2)
            \] 
            \begin{align*}
                \underset{i = 1,2,3}{P_2(x_i) = y_i} \implies &\begin{cases}
                    a_0 &=y_1\\
                    a_0 + a_1(x_2 - x_1) &= y_2\\
                    a_0 + a_1(x_3 - x_1) + a_2(x_3 - x_1)(x_3 - x_2) &= y_3
                \end{cases}\\
                                                        \implies &\begin{cases}
                                                                  a_0 &= y_1\\
                                                                  a_1 &= \frac{y_2 - y_1}{x_2 - x_1}\\
                                                                  a_2 &= \frac{y_3 - y_1 - \frac{y_2 - y_1}{x_2 - x_1}(x_3 - x_1)}{(x_3 - x_1)(x_3 - x_2)}
                                                              \end{cases}
            \end{align*}
            \begin{remark}
               On a:
               \[
               a_2 = \frac{\frac{y_2 - y_1}{x_2 - x_1} - \frac{y_3 - y_2}{x_3 - x_2}}{x_3 - x_1}
               \] 
               càd 
               \[
               P_3 = y_1 + \frac{y_2 - y_1}{x_2 - x_1}(x - x_1) + \frac{\frac{y_2 - y_1}{x_2 - x_1} - \frac{y_3 - y_2}{x_3 - x_2}}{x_3 - x_2}(x-x_1)(x-x_2)
               \] 
               \begin{center}
                   \begin{tabular}{| c | c | c | c |}
                       \hline
                   $x_1$ & $y_1 =: a_3$ &  &  \\
                       \hline
                   $x_2$ & $y_2$ & $\frac{y_2 - y_1}{x_2 - x_1} =: a_1$ & \\
                   \hline
                   $x_3$ & $y_3$ & $\frac{y_3 - y_2}{x_3 - x_2}$ & $\frac{\frac{y_2 - y_1}{x_2 - x_1} - \frac{y_3 - y_2}{x_3 - x_2}}{x_3 - x_1} =: a_2$\\
                   \hline
                   \end{tabular}
               \end{center}
            \end{remark}
        \item \underline{Méthode 3}:
            \par
            \[
            P_3 = \frac{(x - x_2)(x - x_3)}{(x_1 - x_2)(x_1 - x_3)}y_1 + \frac{(x - x_1)(x - x_3)}{(x_2 - x_1)(x_2 - x_3)}y_2 + \frac{(x - x_1)(x - x_2)}{(x_3 - x_1)(x_3 - x_2)}y_3 = \sum_{i=1}^{3} \underbrace{\left( \prod_{j=1}^{3} \frac{x - x_j}{x_i - x_j}  \right)}_{l_i(x)} y_i
            \] 
            \begin{remark}
               $(x_1, y_1), (x_2, y_2)$ 
               \[
               y = \frac{x - x_2}{x_1 - x_2}y_1 + \frac{x - x_1}{x_2 - x_1}y_2
               \] 
            \end{remark}
   \end{itemize}
\end{eg}
\section{Polynôme interpolateur de lagrange}
\subsection{Définition et propriétés}
\begin{theorem}
    (existence et utilité) \par
    Soit $x_1, \ldots, x_n$ des réels 2 à 2 distincts et $y_1, \ldots, y_n$ des rééls quelconques: Il existe \underline{un unique} polynôme $P \in \R_{n-1}[X]$ (i.e de degré $n-1$) tel que $p(x_i) = y_i, \, i = 1, \ldots, n$
    \par
    On dit que $P$ est le polynôme interpolateur de lagrange aux points  $(x_1, y_1), \, \ldots, \, (x_n, y_n)$
\end{theorem}
\begin{preuve}
   Soit \begin{align*}
       \Phi: \R_{n-1}[X] &\longrightarrow \R^{n-1} \\
       P &\longmapsto \Phi(P) = (P(x_1), \ldots, P(x_n))
   .\end{align*} 
   on a:
   \begin{itemize}
       \item $\Phi$ linéaire
       \item  $\Phi$ injective
   \end{itemize}
   En effet, $\Phi(P) = 0 \iff P(x_i) = 0 \iff P \equiv 0$ car $deg(P) \le n-1$.
   D'où $\Phi$ isomorphisme d'espace vectoriel et la surjection assure le résultat.
\end{preuve}
\begin{definition}
    Si $f$ est continue sur  $[a, b] \to \R$, $x_1, \ldots, x_n \in [a, b]$ 2 à 2 distincts, alors, l'unique $P \in \R_{n-1}[X]$ tq $P(x_i) = f(x_i) \, i = 1, \ldots, n$ est appelé \underline{polynôme d'interpolation de lagrange} de $f$ aux points  $x_1, \ldots, x_n$
\end{definition}
\subsection{Estiomation d'erreur}
\begin{theorem}
    l'erreur  \par
    Soient
    \begin{itemize}
        \item $a < b$  $f: [a, b] \to \R$ continue
        \item $x_1, \ldots, x_n$ 2 à 2 distincts de $[a, b]$
        \item  $P$ polyôme d'interpolation de lagrange de  $f$ aux points  $x_i$
    \end{itemize}
    Si $f$ est $n$ fois dérivable sur  $]a, b[$, alors, pour tout  $a \in [a, b]$, il existe  $t \in ]a, b[$ tq 
     \[
         f(x) - P(x) = \omega_n(x)\frac{f^{(n)}(t)}{n!}
    \] 
    où $\omega_n(x) = (x - x_1)\ldots(x - x_n)$
\end{theorem}
\begin{corollary}
    Si $f^{(n)}$ est bornée par  $M$ sur  $]a, b[$, alors  $\forall x \in [a, b]$ 
    \[
    |f(x) - P(x)| \le \frac{M}{n!}|\omega_n(x)| \le \frac{M}{n!}(b - a)^n
    \] 
\end{corollary}
\begin{preuve}
    à faire
\end{preuve}

\subsection{Implémentation avec python}
\begin{lstlisting}
   from scipy.interpolite import lagrange
   x = np.array([x_1, x_2, x_3])
   y = np.array([y_1, y_2, y_3])
   p = lagrange(x, y)
   print(p) # affiche le polynome
   print(p(3)) # collable
\end{lstlisting}

\section{Construction des polynôme d'intérpolation de lagrange}
$x_0, \ldots, x_{n-1}$ 2 à 2 distincts
\subsection{Interpolation dans la base canonique (Vandermonde)}
\subsubsection{Construction}
\[
P = \sum_{i=1}^{n-1} a_ix^i
\] 
\[
P(x_i) = y_i, \, i = 0, \ldots, n-1
\] 
\[
    \underbrace{
        \begin{bmatrix} 
            1 & x_0 & \ldots & x_0^{n-1}\\
            1 & x_1 & \ldots & x_1^{n-1}\\
            \ldots & \ldots & \ldots & \ldots\\
            1 & x_{n-1} & \ldots & x_{n-1}^{n-1}
        \end{bmatrix} 
    }_{V(x_0, \ldots, x_{n-1})} \underbrace{\begin{bmatrix} a_0 \\ \ldots \\ a_{n-1} \end{bmatrix} }_{a} 
    =
    \underbrace{\begin{bmatrix} y_1 \\ \ldots \\ y_{n-1} \end{bmatrix} }_{b}
\] 
Matrice de Vandermonde
\begin{itemize}
    \item elle pleine
    \item malconditionnée
\end{itemize}
\begin{lstlisting}
def VDM_Mat(x):
    x_n = np.reshape(x, (x.size, 1))
    return x_n ** np.arange(x.size)
\end{lstlisting}
\begin{lstlisting}
def VDM_Poly(x, y):
    M = VDM_Mat(x)
    a = np.linalg.solve(M, y)
    return a
\end{lstlisting}
\subsubsection{Evaluation efficace de $P$ algorithme de Horner}
\begin{prop}
   Si $X$ est un réel et  $Q$ est le polynôme défini par 
   \[
       Q(X) = a_0X^n + a_1X^{n-1} + \ldots + a_{n-1}X + a_n
   \] 
   alors la suite
   \[
   \begin{cases}
       q_0 = a_0\\
       q_k = q_{k-1}x + a_k, \, k=1,\ldots,n
   \end{cases}
   \] 
   vérfiie $q_n = Q(x)$
\end{prop}
\begin{preuve}
    (laissé exo)
    \[
    Q(X) = X^2 + 2X + 1 \equiv (X+2)X + 1
    \] 
\end{preuve}
\begin{lstlisting}
def Horner(P, xx): 
   y = 0
   for a in P:
        y = y * xx + a
   return y
\end{lstlisting}
\begin{lstlisting}
def IntuP_VDM(x, y, xx):
    a = VDM_Poly(x, y)
    yy = Horner(a[::-1], xx)
    return yy
\end{lstlisting}
\subsection{Interpolation dans la base duale: Formule de lagrange et points barycentrique}
\subsubsection{Construction}
Idée prendre pour base de $\R_{n-1}[X]$ l'image réciproque de la base canonique de $\R^{n-1}$ pour $\Phi$ 
\[
L_i(x_j) = \begin{cases}
    1 \text{ si } i \neq j\\
    0 \text{ sinon}
\end{cases}
\] 
\[
    L_i(x) = \frac{\prod_{j\neq i \, j = 0}^{n-1}(x - x_j) }{\prod_{j = 0 \, j \neq i}^{n-1} (x_i - x_j) } = \prod_{j = 1 \, j \neq i}^{n-1} \frac{x - x_j}{x_i - x_j} 
\] 
\[
P(X) = \sum_{i=0}^{n-1} y_iL_i(X)
\] 
