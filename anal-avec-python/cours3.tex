\chapter{cours3:\\Interpolation polynomiale}
On va essayer de construire des polynôms qui passent par un ensemble (nuages) de points donnés.

Si ces points sont les valeurs d'une fonction, on amerait:
\begin{itemize}
    \item savoir si le polynôme construit est d'autant plus proche de la fonction que le nombre de point est grand. C'est-à-dre, est-ce que nute des "erreurs" tend vers zero lorsque le nombre de points tend vers l'infini.
    \item Si oui, comment quantifier cette convergence? C'est-à-dire, quelle est la vitesse (ordre) de cette convergence.
        \begin{figure}[H]
            \centering
            \incfig{evolution-de-population-en-annee}
            \caption{evolution-de-population-en-annee}
            \label{fig:evolution-de-population-en-annee}
        \end{figure}
\end{itemize}
\begin{enumerate}
    \item Approche 1: approximation linéaire. 
        \begin{itemize}
            \item Polynôme de degré 1
        \end{itemize}
    \item Approche 2: 
        \begin{itemize}
            \item polynôme de degré 1
            \item approximation quadratique
        \end{itemize}
    \item Approche 3: prise en compre d'Historique
\end{enumerate}

\section*{Rappels sur les nuts numériques\\
          Vitesse (ordre) de convergence\\
          valeur ajoutée par itérations
      }
\begin{definition}
    Soit $(x_n)_n \subset \R^n$ une suite qui converge vers $x^* \in \R^n$, pour une norme $\| \, \|$ de  $\R^n$
    \begin{itemize}
        \item Si $k_1 = \lim_{x \to \infty} \frac{\|x_{n+1} - x^*\|}{\|x_n - x^*\|}$ existe et $k_1 \in ]-1, 1[ \setminus \{0\}$. On dit que la suite convere \underline{linéairement} vers $x^*$ ou que la convergence est d'ordre  $1$.
        \item Si  $k_1 = 0$, $k_2 = \lim_{n \to \infty} \frac{\|x_{n+1} - x^*\|}{\|x_n - x^*\|^2}$ existe et non nul. On dit que la suite coverge \underline{quadratiquement} vers $x^*$, ou que la convergence est \underline{d'ordre $2$}.  
        \item Si $k_q = \lim_{n \to \infty} \frac{\|x_{n+1} - x^*\|}{\|x_n - x^*\|^q}$ existe et $\neq 0$ la convergence est \underline{ d'ordre $q$ }. La constante $K_q$ est appelée \underline{constante asymptotique d'erreur}.
    \end{itemize}
\end{definition}
\begin{eg}
    \begin{enumerate}
        \item 
   $x_n = (0.2)^n$ 
   \begin{itemize}
       \item On a $\lim_{n \to \infty} x_n = 0$. La convergence vers $x^* = 0$.
       \item  $\lim_{n \to \infty} \frac{|x_{n+1} - x^*|}{|x_n - x^*|} = \lim_{n \to \infty} \frac{(0.2)^{n+1}}{(0.2)^n} = 0.2 \in ]-1, 1[ \setminus \{0\}$
   \end{itemize}
   D'où
   \begin{itemize}
       \item $x_n$ converge à \underline{l'ordre $1$} 
       \item Sa constante asymptotique est \underline{$k_1 = 0.2$}
   \end{itemize}
   \item
       $I_n = (0.2)^{2^n}$. On a  $\lim_{n \to \infty} I_n = 0$\\
       On a:
       \begin{align*}
           I_{n+1} = (0.2)^{2^{n+1}} &= (0.2)^{2^n \cdot 2}\\
                                     &= \left( (0.2)^{2^n} \right)^2\\
                                     &= \left( I_n \right)^2
       \end{align*}
       D'où $\lim_{n \to \infty} \frac{I_{n+1}}{\left( I_n \right)^2} = \lim_{n \to \infty} \frac{\left( I_n \right)^2}{\left( I_n \right)^2} = 1$
       D'où
       \begin{itemize}
           \item convergence \underline{d'ordre $2$}
           \item de constante $k_2 = 1$
       \end{itemize}
\end{enumerate}
\end{eg}
En pratique, on ne dispose pas de $K_q$
 \begin{definition}
    \[
        \substack{x_n \text{ converge vers } x^*\\\text{à l'ordre } q} \implies \exists N \in \N, \exists A,B \in \R \text{ tq } \forall n \ge N, 0 < A \le \frac{\|x_{n+1} - x^*\|}{\|x_n - x^*\|^q} \le B < +\infty
    \] 
    La convergence est au moins d'ordre $q$ si et seulement si on a (deuxieme partie d'équation)
\end{definition}

\subsection{Valeur ajoutée par l'itération}
Il est question de comparer $2$ suites qui ont la même vitesse de convergence.
\begin{remark}
    Si $|x_n - x^*| = 4 \cdot 10^{-8} = 0.\underbrace{0000000}_{7 \text{ chiffres}}4$. On dira que $x_n$ et  $x^*$ ont 7 chiffres exactes apres la virgule.
    \begin{align*}
    &\log_{10} |x_n - x^*| = \log_{10}4 - 8\log_{10}(10)\\
    &\frac{\log|x_n - x^*}{\log 10} = \frac{\log 4}{\log 10} - 8
    \end{align*}
    i.e $d_n = - \log_{10} |x_n - x^*|$ mesure de nombre de chiffres décimales entre $x_n$ et  $x^*$ qui coincident.
\end{remark}
\begin{remark}
   \[
       \lim_{n \to \infty} \frac{\|x_{n+1} - x^*\|}{\|x_n - x^*\|^q} = K_q \implies K_q \approx \frac{\|x_{n+1} - x^*\|}{\|x_n - x^*\|^q}
   \]  
   D'où $d_{n+1} - q d_n \approx -\log_{10} K_q$, i.e
   \[
   d_{n+1} + \frac{\log_{10} K_q}{1 - q} \approx q(d_n + \frac{\log_{10} K_q}{1 - q})
   \] 
   Donc, le nombre de chiffres significatives est multiplié par $q$.
\end{remark}
\begin{prop}
   Si $x_n$ converge à l'ordre  $1$ vers  $x^*$ de constante asymptotique  $K_1$, alors le nombre d'itérations nécessaires pour gagner un chiffre exacte est la partié enitère de  $-\frac{1}{\log_{10}K_1}$ 
\end{prop}
\begin{preuve}
Soit $m$ le nombre d'itérations pour gegner un chiffre. Comme  $d_{n+1} - d_n = -\log_{10} K_1$, en partant de $d_n$, après  $m$ itérations on aura
 \[
     d_{n+m} - d_{n} = -m \log_{10}K_1
\] 
D'où on aura gagné $1$ chiffre si  $d_{n+m} - d_n = 1$, i.e
\[
1 = -m \log_{10}K_1 \implies m = \left( -\frac{1}{\log_{10}K_1} \right) 
\] 
\end{preuve}

\subsection{Obtenir numériquement la vitesse de convergence}
On cherche $q$ tq:  $\lim_{n \to \infty} \frac{\|x_{n+1} - x^*\|}{\|x_n - x^*\|^q} = K_q \in  \R^*$
\begin{remark}
    \begin{align*}
        &\frac{\|x_{n+1} - x^*\|}{\|x_n - x^*\|^q} \approx K_q
        \implies& \underbrace{ \log \|x_{n+1} - x^*\|  }_{Y} - \underbrace{q \log\|x_n - x^*\|}_{X} = \log K_q
    \end{align*}
    i.e $Y = aX + b$. 
    \par
    Conclusion: pour détérminer q:
     \begin{itemize}
        \item Traiter la courbe $\log\|x_n - x^*\| \mapsto \log\|x_{n+1} - x^*\|$
        \item Détérminer $q$ comme la parte de la droite passant par le maximum de points.
    \end{itemize}
\end{remark}
\begin{align*}
    &x_n = x_0, x_1, \ldots, x_N\\
    &x_n - x^* = x_0 - x^*, x_1 - x^*, \ldots, x_N - x^*\\
    &x_{n+1} - x^* = x_1 - x^*, x_2 - x^*, \ldots, x_{N+1} - x^*
\end{align*}
En python:

\begin{lstlisting}
xn = np.array([x0, ..., xN])
e = np.log(np.abs(xn - x^*))
ex = e[0:-1] #de premier a avant dernier
ey = e[1:] #de deuxieme au dernier
plt.scatter(ex, ey, label="miage")
a,b = np.polyfit(ex, ey, 1)
plt.plot(ex, b + a * ex, label=f"$x \mapsto {b:32f} + {a:32f}x$")
\end{lstlisting}
