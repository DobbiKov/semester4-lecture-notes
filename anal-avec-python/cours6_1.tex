\chapter{Intégration numérique}
\underline{But:} On souhaite calculer au mieux
\[
I(f) = \int_{{a}}^{{b}} {f(x)} \: d{x} {} 
\] 
où $f: [a, b] \to \R$ donné
\par
\underline{Contraintes}
\begin{itemize}
    \item $f$ n'a pas de primitive connue (ou évidente)
    \item  $f$ n'est connue ou ne peut être évaluée qu'en un certain nombre fini de points 
         \[
             (x_i, 0\le i \le n \text{ sur } [a, b])
        \] 
\end{itemize}

\begin{figure}[H]
    \centering
    \incfig{integration-exemple}
    \caption{integration-exemple}
    \label{fig:integration-exemple}
\end{figure}

\[
I(f) = \int_{{a}}^{{b}} {f(x)} \: d{x} {}
\] 
\[
    S(f, \sum_{N} = \sum_{i=0}^{N} f(\xi_i)\underbrace{(x_{i+1} - x_{i})}_{\omega_i}
\] 
somme de Rieman associée à $\sum_{N}$. Théorème: $\lim_{N \to \infty} S(f, \sum_{N}) = \int_{{a}}^{{b}} {f(x)} \: d{x} {}$

\section{Formule de quadratique}
\begin{definition}
    Étant donnée $N$ points  $x_1, \ldots, x_N$ de l'intervalle $[a, b]$  et $N$ poids  $\omega_1, \ldots, \omega_N \in \R$ associées à chaque points.
    \par
    On appelle \underline{formule de quadratique} associé aux $(x_i),(\omega_i)$ l'application linéaire sur  $\mathcal{C}^0([a, b])$
     \[
         \tilde{I}(f) = \sum_{i=1}^{N} \omega_i f(x_i)
    \] 
    On dit que la formule de quadratique est \underline{d'ordre} $p$ si elle est \underline{exacte} pour les polynôme de degré  $p_1$. i.e
     \[
         \tilde{I}(Q) = \int_{{a}}^{{b}} {Q} \: d{x} {} \forall Q \in \R_{p-1}[X]
    \] 
    et s'il existe $Q \in \R_{p-1}[X]$ tq $I(Q) \neq \int_{{a}}^{{b}} {Q} \: d{x}$, autrement dit si elle exacte pour le polyôme de degré \underline{au plus} $p-1$.
\end{definition}
\begin{remark}
   On note:
   \[
   \int_{{a}}^{{b}} {f(x)} \: d{x} \approx \sum_{i=1}^{n} \omega_i f(x_i)
   \] 
   \begin{center}
    \begin{tabular}{| c | c | c | c |}
        \hline
        Points & $x_1$ &  $x_2$ &  \ldots \\  
        \hline
        Poids & $\omega_1$ &  $\omega_2$ &  \ldots \\
        \hline
    \end{tabular}
\end{center}
\end{remark}
\begin{eg}
   Soit la formulle de quadratique 
   \[
   \int_{{a}}^{{b}} {f(x)} \: d{x} \approx (b-a)f(a)
   \] 
   \begin{itemize}
       \item si $f = 1$, on a  
           \[
               \int_{{a}}^{{b}} {f(x)} \: d{x} {} = b - a = (b-a)f(a)
           \] 
           elle exacte pour les polynômes de degré 0.
        \item si $f(x) = x$ on a
             \[
                 \int_{{a}}^{{b}} {f(x)} \: d{x} = \left[ \frac{x^2}{2} \right]_{a}^{b}  = \frac{(b-a)(a+b)}{2} \neq (b-a)a
            \] 
            elle n'est pas exacte pour les polynômes de degré $1$.
   \end{itemize}
   Conclusion: elle est exacte pour les polynômes de degré au plus 0. Elle est donné d'ordre $1$.
\end{eg}

\subsection{Construction de formule de quadratique à points donnés}
\begin{prop}
    Soit $x_1, \ldots, x_N$, $N$ points 2 à 2 distincts de  $[a, b]$. 
    \begin{enumerate}
        \item Il existe un unique $(\omega_1, \ldots, \omega_N) \in \R^N$ tels que 
            \[
                \tilde{I}(Q) \overset{def}{\equiv} \sum_{i=1}^{N} \omega_iQ(x_i) = \int_{{a}}^{{b}} {Q(x)} \: d{x} \quad \forall Q \in \R_{N-1}[X]
            \] 
        \item Pour toute fonction $f: [a, b] \to \R$ de calsse $\mathcal{C}^N$
             \[
                 \left| \int_{{a}}^{{b}} {f(x)} \: d{x} - \tilde{I}(f) \right| \le \frac{(b-a)^{N+1}}{N!}\|f^{(N)}\|_{\infty}
            \] 
    \end{enumerate}
\end{prop}
\begin{preuve}
   Soit $l_i, i = 1, \ldots, N$  la base de Lagrange associé aux $x_i$, i.e
    \[
        l_i(x) = \prod_{\substack{j=1}\\ \substack{j\neq i}}^{N}  \frac{X - x_j}{x_i - x_j} 
   \] 
   on a $l_i \in \R_{N-1}[X]$.
   \par
   Soit $Q \in \R_{N-1}[X]$, $Q$ coïncide avec le polynôme d'interpolation de lagrange aux points  $x_1, \ldots, x_N$
   \[
   Q(X) = \sum_{i=1}^{N} Q(x_i)l_i(X)
   \] 
   d'où
   \begin{align*}
       \int_{{a}}^{{b}} {Q(x)} \: d{x} &= \sum_{i=1}^{N} Q(x_i)\int_{a}^{b} l_i(x) \, d{x}\\ 
                                       &= \sum_{i=1}^{n} Q(x_i)\omega_i
   \end{align*}
   où $\omega_i = \int_{a}^{b} l_i(x) \, d{x} $. D'où l'existence.\par
   \underline{Unicité:}
   Soit $\tilde{\omega_i} \, i = 1, \ldots, N$, 
   \[
       k: \int_{a}^{b} Q(x) \, d{x} = \sum_{i=1}^{N} \tilde{\omega_i}Q(x_i) \quad \forall Q \in \R_{N-1}[X] 
   \] 
   alors, puisque $l_i \in \R_{N-1}[X]$, on a
   \[
       \int_{a}^{b} l_i(x) \, d{x} = \tilde{\omega_i} \quad i = 1, \ldots, N 
   \] 
   D'où $(\tilde{\omega}_i = \omega_i)$ et on a l'unicité

   \par
   \underline{Estimation d'erreur}:
   \par
   Soit $f$ de classe  $\mathcal{C}^N$ sur  $[a, b]$ et  $R_f$ un poly d'interpolation aux points  $x_i$  $\quad i = 1, \ldots, N$. On a:
   \begin{align*}
       \tilde{I}(f) = \sum_{i=1}^{N} f(x_i)\omega_i &= \sum_{i=1}^{N} P_f(x_i)\omega_i\\
                                                    &= \int_{a}^{b} P_f(x) \, d{x} 
   \end{align*}
   D'où 
   \begin{align*}
       \left| \int_{a}^{b} f(x) \, d{x} - \tilde{I}(f) \right| &= \left| \int_{a}^{b} f(x) \, d{x} - \int_{a}^{b} P_f(x) \, d{x}   \right| \\
                                                               &\le \int_{a}^{b} |f(x) - P_f(x)| \, d{x} \\
                                                               &\le \frac{\|f^{(N)}\|_{\infty}(b-a)^N}{N!}(b-a)
   \end{align*}
   \[
       \left| \int_{a}^{b} f(x) \, d{x} - \tilde{I}(f)  \right| \le \frac{\|f^{(N)}\|_{\infty}}{N!}(b-a)^N(b-a)
   \] 
\end{preuve}

python:
\begin{lstlisting}
   from scipy.Integrate  import quad
   quad(f, a, b) = 
\end{lstlisting}

