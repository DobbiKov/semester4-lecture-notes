\chapter{Résolution approchée d'équations différentielles ordinaires (EDOs)}
\section{Motivations}
\subsection{Définitions}
\begin{definition}
    Soit \begin{align*}
        f: [a, b] \times \R^d&\longrightarrow \R^d \\
        (t, x) &\longmapsto f((t, x))
    \end{align*} 
    avec $a,b \in \R$ et $d \in \N^*$ donnée par des d composantes
    \begin{align*}
        f_i: [a, b] \times \R^d&\longrightarrow \R^d \\
        (t, x) &\longmapsto f_i(t, x)
    \end{align*}
\end{definition}
On note $g^{(p)}$ la dérivée d'ordre  $p$ d'une fonction  $g: \R \to \R$ et $g'$ sa dérivée d'ordre  $1$. 

Si $g: [a, b] \to \R^d$ est continue ainsi que toutes ses dérivées jusqu'à l'ordre $p$, on notera  $g \in \mathcal{C}^p([a, b], \R^d)$ ou simplement $g \in \mathcal{C}^p([a, b])$ s'il n'ya pas ambiguité. On a:
 \[
     \left( g \in \mathcal{C}^{1}([a, b], \R), i = 1, \ldots, d \right) \iff \left( g \in \mathcal{C}^{1}([a,b], \R^d) \right) 
\] 

\begin{definition}
    On appelle équation différentielle d'ordre $1$ une équation de la forme 
     \[
         y'(t) = f(t, y(t)), \quad t \in ]t_0, t_0 + \tau[
    \] 
    On appelle EDO d'ordre $p$ une équation de la forme 
     \[
         y^{(p)}(t) = f(t, y(t), y'(t), \ldots, y^{(p-1)}(t))
    \] 
    où $f: [a, b] \times (\R^d)^p \to \R^d$ est continue.
\end{definition}
\begin{definition}-
    \begin{itemize}
        \item 
            Une fonction $y$ de classe  $\mathcal{C}^1$ vérifiant une EDO est dite solution de l'EDO.

        \item 
            Résoudre une EDO c'est détérminer toutes les solutions de cette EDO.     

        \item 
            Lorsque $d \neq 1$, on parle de système d'EDOs.
    \end{itemize}
\end{definition}
\begin{remark}
   Toute EDO d'ordre $p > 1$ peut ce ramener à un système d'EDOs d'ordre  $1$. 
\end{remark}
\begin{definition}
    On appelle condition de Cauchy de l'EDO, la donnée de la valeur de la solution en un point 
    \[
        t_0 \in [a, b]: \quad y(t_0) = y^0
    \] 
    Le couple $(t_0, y^0)$ est appelé \textbf{condition initiale} et le problème de Cauchy consiste à la recherche d'une fonction de classe $\mathcal{C}^1$ vérifiant:
     \[
    \begin{cases}
        y'(t) = f(t, y(t)) \quad t \in ]t_0, t_0 + \tau[\\
        y(t_0) = y^0, \quad t_0 \text{ donné des } \R^d  
    \end{cases}
    \] 
\end{definition}
\subsection{Exemple}
\begin{eg}
   Pendule. 
\begin{figure}[H]
    \centering
    \incfig{pendule-exemple-edo}
    \caption{pendule-exemple-edo}
    \label{fig:pendule-exemple-edo}
\end{figure}
\[
\begin{cases}
    m = 1\\
    \phi(t) = ?\\
    \phi' + \frac{g}{L}\sin(\phi) = 0
\end{cases}
\] 
c'est une EDO d'ordre $2$.

 \[
\begin{cases}
    x_1 = \phi \implies x_1' = \phi' = x_2\\
    x_2 = \phi' \implies x_2' = \phi'' = -\frac{g}{L}\sin(\phi) = -\frac{g}{L}\sin(x_1)
\end{cases}
\] 

D'où $X(t) = \begin{pmatrix} x_1 \\ x_2 \end{pmatrix} $ on a $X'(t) = \begin{pmatrix} x_2 \\ -\frac{g}{L}\sin(x_1) \end{pmatrix} = f(t, X(t)) $ 
où \begin{align*}
    f: \R \times \R^2 &\longrightarrow \R^2 \\
    (t, X\begin{pmatrix} x_1\\x_2 \end{pmatrix} ) &\longmapsto f((t, X\begin{pmatrix} x_1\\x_2 \end{pmatrix} )) = \begin{pmatrix} x_2 \\ -\frac{g}{L}\sin(x_1) \end{pmatrix} 
.\end{align*}

\end{eg}
\begin{eg} Objet en chasse libre.
\begin{figure}[H]
    \centering
    \incfig{asteroid-edo}
    \caption{asteroid-edo}
    \label{fig:asteroid-edo}
\end{figure}
\[
\begin{cases}
    \text{vitesse: } v\\
    \text{altituted: } z
\end{cases}
\] 
$k$ a coef. de frottement.
 \[
     z'' = -g + k(z')^{2}e^{-az} \text{ càd EDO d'ordre } 2 
\] 
ou encore
\[
\begin{cases}
    z' = v\\
    v' = -g + kv^2e^{-az}
\end{cases}
\text{ càd système d'EDOs d'ordre 1}
\] 
Pososn $Y = \begin{pmatrix} z \\ z' \end{pmatrix} = \begin{pmatrix} y_1 \\ y_2 \end{pmatrix}  $ $Y = f(t, Y)???$
 \[
     f(t, Y\begin{pmatrix} y_1 \\ y_2 \end{pmatrix} ) = \begin{pmatrix} y_2 \\ -g + ky_2^2e^{-ay_1} \end{pmatrix} 
\] 
\begin{lstlisting}
def f(t, Y):
    x, y = Y
    return np.array([y, -g + (k*y**2)*np.exp(-a*x)])
\end{lstlisting}
\end{eg}

\begin{eg}
   Taux d'inféction 
\begin{figure}[H]
    \centering
    \incfig{taux-d-infection-edo}
    \caption{taux-d-infection-edo}
    \label{fig:taux-d-infection-edo}
\end{figure}
$y$: inféctés,  $x$: soins,  $\alpha$: taux d'inféction

\[
\begin{cases}
    \dot{y} = \alpha x y\\
    x + y = n
\end{cases} \implies
y' = \alpha y (n - y)
\] 
\end{eg}
\subsection{Nécessite de la solution approchée}
On considère le problème de Cauchy
\begin{equation}\label{eq:probleme-de-cauchy}
    \begin{cases}
        x'(t) = f(t, x(t)) \quad t \in ]t_0, t_0 + \tau[\\
        x(t_0) = x^0 \in \R^d
    \end{cases}
\end{equation}
On ne sait résourdre \ref{eq:probleme-de-cauchy} dans des cas particulièrs.

$d=1$,  $f$ est à variables séparées.
 \begin{eg}
     \[
     \begin{cases}
         \dot{L} = \tau_L L \quad ]0, T[\\
         L(0) = L_0
     \end{cases} \implies L(t) = L_0 l^{t \tau_L}
     \] 
\end{eg}
\underline{Illustation graphique}
\begin{figure}[H]
    \centering
    \incfig{illustation-graphique-sol-approche}
    \caption{illustation-graphique-sol-approche}
    \label{fig:illustation-graphique-sol-approche}
\end{figure}
\begin{itemize}
    \item On se donne $t_n = n \Delta t$
         \[
        n = 0, \ldots, N, \text{ où } \Delta t = \frac{T}{N}, N \in \N^*
        \] 
    \item On calcule $L_n = L_0 e^{t_n \tau_L}$, $n = 0, \ldots, N$
    \item On place $(t_n, L_n)$ sur un figure et on les relie pour obtenir un graphe de  $t \mapsto L(t)$
\end{itemize}

\section{Problème d'évolution de population des lapins}
$L$: population des lapins,  $R$: population renards.

On a le problème de Cauchy
 \[
\begin{cases}
    \dot{L} = L(\tau_L - p R)\\
    \dot{R} = R\tau_R (\alpha L - 1)\\
    L(0) = L_0, \quad R(0) = R_0
\end{cases}
\] 
Ce système n'est pas résoulable analytiquement.

On peut cependant le résoudre numériquement à condition de s'assurer que le problème est bien posé.
\begin{itemize}
    \item Existence et l'unicité de la solution
    \item Régularité de la solution
    \item Dépendance continue de la solution vis à vis des données du problème (où Stabilité)
\end{itemize}
\[
\Phi: (t_0, f) \mapsto y
\] 
\[
    \|\Phi(y_1 - y_2)\|_{*} \le C_1\|L- - L_0^2\| + C_2\|f_1 - f_2\|_{**}
\] 

\begin{definition}
    On dit que $f: [a, b] \times \R^d \to \R^d$  est lipschizienne par rapport à sa seconde variable s'il existe une constante positive $L$ (appellée constante de lipschitz) telle que
    \[
        \|f(t, y_2) - f(t, y_1)\| \le L\|y_2 - y_1\| \quad \forall t \in [a, b] \forall y_1, y_2 \in \R^d
    \] 
\end{definition}
\begin{theorem}
    de Cauchy lipshitz.

    Soit 
    \begin{equation}\label{eq:theoreme-de-cauchy-lipshitz}
        \begin{cases}
            x'(t) = f(t, x(t)), \quad t \in ]t_0, t_0 + \tau[\\
            x(t_0) = x^0
        \end{cases}
    \end{equation}
    \begin{itemize}
        \item Si $f: [t_0, t_0 + \tau] \times \R^d \to \R^d$ vérifie 
            \begin{enumerate}
                \item $f$ continue
                \item  $\|f(t, y) - f(t, z)\| \ll \|y - z\|$
            \end{enumerate}
            Alors \ref{eq:theoreme-de-cauchy-lipshitz} admet une unique solution (globale) de classe $\mathcal{C}^1([t_0, t_0+\tau], \R^d)$
    \end{itemize}
\end{theorem}
\section{Exemple de Schémas numériques}
\subsection{Formulation intégrale}
\begin{prop}
    $x$ solution de \ref{eq:theoreme-de-cauchy-lipshitz} ssi $x(t) = x^0 + \int_{{t_0}}^{{t}} {f(s, x(s))} \: d{s} {\forall t \in [t_0, t_0 + \tau]}$
\end{prop}
\begin{preuve}
   - 
   \begin{itemize}
       \item[$\implies$)] 
           \[
           \begin{cases}
               x(t_0) = x^0\\
               x'(t) = f(t, x(t))
           \end{cases}
           \] 
       \item[$\impliedby)$]
           \[
           \int_{{t_0}}^{{t}} {x'(s)} \: d{s} {\int_{{t_0}}^{{t}} {f(s, x(s))} \: d{s} {}}
           \] 
   \end{itemize}
\end{preuve}
\subsection{Construction de schema d'Euler explicite}
\begin{itemize}
    \item[Étape $1$] maillage du domaine
        \[
        N \text{ donné}, \text{ pose } \Delta t = \frac{T}{N},  t_n = n \Delta t, n = 0, \ldots, N
        \] 
    \item[Étape $2$]: Formulation intégrale:

        Suite $[t_n, t_{n+1}]$ problème discrèt ("continue")
        \begin{equation}
            \begin{cases}
                x(t_{n+1}) = x(t_n) + \int_{{t_n}}^{{t_{n+1}}} {f(s, x(s))} \: d{s} {, n=0, \ldots, N-1}\\
                x(t_0) = x^0
            \end{cases}
        \end{equation}
    \item[Étape $3$] Approximation des intégrales (Formules de quadratures)
\begin{figure}[H]
    \centering
    \incfig{etape-3-rectangles-a-gauche}
    \caption{etape-3-rectangles-a-gauche}
    \label{fig:etape-3-rectangles-a-gauche}
\end{figure}
Récrangles à gauche
\[
    \int_{{t_n}}^{{t_{n+1}}} {g(s)} \: d{s} {\Delta t g(t_n) + o(\Delta t^2) \approx \Delta t g(t_n)}
\] 
On a 
\begin{equation}
    \begin{cases}
        x(t_{n+1}) = x(t_n) + \Delta t f(t_n, x(t_n)) + o(\Delta t^2)\\
        x(t_0) = x^0
    \end{cases}
\end{equation}
\begin{equation}
    \begin{cases}
        x(t_{n+1}) \approx x(t_n) + \Delta t f(t_n, x(t_n))\\
        x(t_0) = x^0
    \end{cases}
\end{equation}
On pose $x_n \approx x(t_n)$,  $n= 0, \ldots, N$ lorsque des dans (PDC) on se sépare des restes
\begin{equation}
    \begin{cases}
        x_{n+1} = x_n + \Delta t f(t_n, x_n),  \quad n = 0, \ldots, N-1\\
        x_0 = x^0
    \end{cases}
\end{equation}
On a le schéma d'Euler explicite
\end{itemize}
\begin{remark}
   Schéma du point-millieu
   \[
   \int_{{t_n}}^{{t_{n+1}}} {g(s)} \: d{s} {\Delta t g(\frac{t_{n+1} + t_n}{2} + o(\Delta t^3))}
   \] 
   On aurait \[
       x(t_{n+1}) = x(t_n) + \Delta 3 f(t_{n + \frac{1}{2}}, x(t_{n + \frac{1}{2}})) + o(\Delta t^3)
   \] 
   Soit 
   \[
   \begin{cases}
       x_{n+1} = x_n + \Delta t f(t_{n + \frac{1}{2}}, x_{n + \frac{1}{2}}),  \quad n = 0, \ldots, N-1\\
       x_0 = x^0
   \end{cases}
   \] 

   Comme $x_{n + \frac{1}{2}}$ est inconnu, on l'approche par le schema d'Euler explicite.
   \[
   \text{i.e } x_{n+\frac{1}{2}} = x_n + \frac{\Delta t}{2} f(t_n, x_n)
   \] 
   D'où
   \begin{equation}
       \begin{cases}
           x_{n + \frac{1}{2}} = x_n + \frac{\Delta t}{2}f(t_n, x_n)\\
           x_{n+1} = x_n + \Delta t f(t_n + \frac{\Delta t}{2}, x_{n+ \frac{1}{2}})
       \end{cases}
   \end{equation}
   \begin{equation}
       \begin{cases}
           x_{n+1} = x_n + \Delta t f(t_n + \frac{\Delta t}{2}, x_n + \frac{\Delta t}{2}f(t_n , x_n)), n = 0, \ldots, N-1\\
           x_0 = x^0
       \end{cases}
   \end{equation}
\end{remark}

\subsection{Autres schémas et forme géneral des schemas explicites à un pas}

\begin{remark}
   Si Rectangles à droits, on aurait eu
   \begin{equation}
       \begin{cases}
            x_{n+1} = x_n + \Delta t f(t_{n+1}, x_{n+1}), n = 0, \ldots, N-1\\
            x_0 = x^0
       \end{cases}
   \end{equation}
   Il est implicite c'est le schema d'Euler implicite
\end{remark}
\begin{remark}
   Formule des trapeze - on aurait 
   \begin{equation}
       \begin{cases}
            x_{n+1} = x_n + \frac{\Delta t}{2}(f(t_n, x_n) + f(t_{n+1}, x_{n+1})), n = 0, \ldots, N-1\\
            x_0 \text{ donné}
       \end{cases}
   \end{equation}
    C'est le schema de CLANK-NICOLAS il est implicite. On peut expliciter le schema de C-N.
    \[
    x_{n+1} = x_n + \Delta t f(t_n, x_n)
    \] 
    On aura 
    \begin{equation}
        \begin{cases}
            x_{n+1} = x_n + \frac{\Delta t}{2}\left[ f(t_n, x_n) + f(t_{n+1}, x_n + \Delta t f(t_n, x_n)) \right] \\
            x_0 \text{ donné}
        \end{cases}
    \end{equation}
\end{remark}
