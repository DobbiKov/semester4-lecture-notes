\section{Comportement asymptotique "lorsque $N \to \infty$"}
\subsection{Observation}
On n'a pas toujours une convergence uniforme de l'interpolation
\begin{eg}
    $f(x) = \sqrt{x}$ avec $[a, b] = [0, 1]$,  $x_1, \ldots, x_n$ équirépartis sur $[a, b]$ 
    \[
        \underset{a \le t \le b}{max}|f(t) - P(t)| \xrightarrow[n \to \infty]{} +\infty
    \] 
    ce phénomène est appelé \underline{phénoème de Runge}.
\end{eg}

Il en reste une solution: 
\begin{itemize}
    \item si $f$ est lipschitziènne sur  $[a, b]$ ou Hölderienne 
         \[
             \exists a \in ]0, 1[, |f(x) - f(y)| \le C|x - y|
        \] 
    \item Si $x_1, \ldots, x_n$ sont les racines du n-ème polynôme de Tchebychev.
        \[
            |f(x) - P(x)| \le \frac{|f^{(n)}(x)}{n!}\prod_{i=1}^{n} (x - x_i)  
        \] 
\end{itemize}

\subsection{Polynôme de Tchebychev}
\begin{definition}
    Les polynômes de Tchebychev sont définis par la recurrence:
    \[
    \begin{cases}
        T_0 = 1\\
        T_1 = x\\
        T_n = 2xT_{n-1} - T_{n-2} \quad n\ge 2
    \end{cases}
    \] 
\end{definition}
\begin{prop}
   Le n-ième polynôme de Tchebychev vérifie:
   \begin{enumerate}
       \item $T_n$ est de degré exactement  $n$ et son coefficient de plus haut degré est  $2^{n-1}$,  $n \ge 1$
       \item $T_n$ a  $n$ racines distinctes simples
            \[
                T_n(x) = 0 \iff x \in \{x_1, \ldots, x_n\}, x_j = \cos( \frac{2j - 1}{2n} \pi)  \quad (1 \le j \le  n)
           \] 
       \item $|T_n(x)| \le 1, \quad \forall x \in [-1, 1], |T_n(x) = 1 \iff x \in \{x_0, \ldots, x_n\} x_k = \cos(k\frac{\pi}{n})$
           \[
               |T_n(x)| = 1 \text{ si } x \in \{x_k\} \quad x_k = \cos(k\frac{\pi}{n}) \quad (0 \le k \le n)
           \] 
   \end{enumerate}
\end{prop}

\begin{preuve}
    \begin{enumerate}
        \item 
   Par récurrence:
   \par
   Soit $(P_n)$ la propriété "$T_n$ est de degré  $n$ et son coef. de plus haut degré est  $2^{n-1}$", $n\ge 1$. $P_0$ et $P_1$ vraies ($k \le n$). \par
   Supposons $P_k$ vrai et montrons que  $P_{n+1}$ vrai. \par
   En effet, nous avons $T_{n+1} = 2xT_n - T_{n-1}$, on en déduit que $P_{n+1}$ est vraie.\par
   Maintenant,
   \[
       \forall x \in [-1, 1], T_n(x) = \cos(n \cdot \arccos(x))
   \] 
   En effet, pour $\begin{cases}
       n = 0, T_0(x) = 1 = \cos(0)\\
       n = 1, T_n(x) = \cos(\arccos(x))
   \end{cases}$ et  $n > 1$
    \begin{align*}
       \cos((n+1)\arccos(x)) = \cos(n\arccos(x))\cos(\arccos(x)) - \sin(n\arccos(x))\sin(\arccos(x))\\
       \cos((n-1)\arccos(x)) =  \cos(n\arccos(x))\cos(\arccos(x)) + \sin(n\arccos(x))\sin(\arccos(x))\\
   \end{align*}
   On a:
   \[
   \cos((n+1)\arccos(x)) = 2x\cos(n\arccos(x)) - \cos((n-1)\arccos(x))
   \] 
   D'où $x \mapsto \cos(n\arccos(x))$ vérifie la même récurrence sur $[-1, 1]$ que  $T_n$. Par conséquent les 2 coïncident sur  $[-1, 1]$. On en déduit $\forall x \in [-1, 1]$
   \item
   \begin{align*}
       T_n(x) = 0 &\iff \cos(n\arccos(x)) = 0\\
                  &\iff n\arccos(x) = \frac{\pi}{2} \text{ mod } \pi \\
                  &\iff \arccos(x) = \frac{\pi}{2n} \text{ mod } \frac{\pi}{n}
                  &\iff x = \cos(\frac{\pi}{2n} + k\frac{\pi}{n}) \quad 0 \le k \le n-1
   \end{align*}
\item $|\cos(x)| \le 1$ D'où $|T_n(x)| \le 1, \forall x \in [-1, 1]$
    \begin{align*}
        |T_n(x)| = 1 \iff n&\arccos(x) = 0 \text{ mod } \pi\\
                           &\arccos(x) = 0 \text{ mod } \frac{\pi}{n}
    \end{align*}
    \[
        \in x \in \{\cos(k\frac{\pi}{n}), k \in [0, n] \}
    \] 
\end{enumerate}
\end{preuve}

\begin{prop}
   Si $Q_n$ est un polyôme de degré  $n$ de même coeff. de plus haut degré que  $T_n$, alors:
   \[
       \underset{x \in [-1, 1]}{max} |Q_n(x)| \ge \underset{x \in [-1, 1]}{max} |T_n(x)| = 1
   \] 
\end{prop}

\begin{corollary}
    Si $\xi_1, \ldots, \xi_n$  sont $n$ points  $2$ à  $2$ distincts de  $[-1, 1]$, on a:
     \[
         \underset{x \in [-1, 1]}{max} \left| \prod_{j=1}^{n} (x - \xi_{j})  \right| \ge \underset{x \in [-1, 1]}{max} \left| \prod_{j=1}^{n} (x - x_j)  \right| = \underset{x \in [-1, 1]}{max} \frac{1}{2^{n-1}} |T_n(x)| = \frac{1}{2^{n-1}}
    \] 
    où $x_j$ sont les racines de  $T_n$
\end{corollary}

\subsection{Application}
Soit $\xi_1, \ldots, \xi_n$, 2 à 2 distincts, $P$ le polynôme de lagrange de  $f$ (suffisament régulière), alors:
 \begin{align*}
     |f(x) - P(x)| &\le \frac{\|f^{(n)}\|_{\infty}}{n!}|\omega_n(x)|\\
                   &\le \frac{\|f^{(n)}\|_{\infty}}{n!}\|\omega_n(x)\|_{\infty}
\end{align*}
où $\omega_i = \prod_{j=1}^{n} (x - \xi_j) $ et $\| . \|_{\infty}$ et loi norme inf sur $[-1, 1]$. Ainsi, le choix de $\xi_i$ qui possede la plus petite valeur de  $\| \omega_n\|_{\infty}$ est celui des racines du n-ìeme polynôme de Tchebychev.

\begin{remark}
    On se ramène à un intervalle quelconque $[a, b]$ par 
    \[
    x_j = \frac{a + b}{2} + \frac{b - a}{2}\cos(\frac{2j - 1}{2n}\pi) \quad (1 \le j \le n)
    \] 
    sont les racines des polynômes de Tchebychev sur $[a, b]$
\end{remark}

