\chapter{Résolution approchée d'équations ordinaires $(EO): f(x) = 0$}
\section{Généralités et exemples}
Soit $f: \R^n \to \R^n$, $I \subset \R^n$
\[
    (EO) \begin{cases}
        \text{chercher } x^* \in I\\
        f(x^*) = 0
    \end{cases}
\] 

\subsection{Définitions}
\begin{itemize}
    \item $(EO)$ est appelé \textit{équation non linéaire} si  $f$ est affine, elle est dite \textit{linéaire}. 
        \[
        f(x) = Ax + b \quad \text{ où } \begin{cases}
            A \in \mathcal{M}_m(\R)\\
            b \in \R^n
        \end{cases}
        \] 
    \item Tout $x^* \in I$ solution de $(EO)$ est dite reciné au zéro de  $f$ dans  $I$.
    \item Si  $f$ est de classe  $\mathcal{C}^r$ avec  $r \ge 1$ et $x^*$ est racine de  $f$, alors
         \begin{itemize}
            \item $x^*$ est dite racine simple  $f'(x^*) \neq 0$
            \item $x^*$ est dite racine de multiplicité $p < r$ si  $f^{(k)}(x^*) = 0, k = 0, \ldots, p-1$ et $f^{(p)}(x^*) \neq 0$
            \item lorsque $m = 1$, c'est cad  $f: I \subset \R \to  \R$, l'équation est dite \underline{scalaire}.
        \end{itemize}
\end{itemize}

\subsection{Exemples}
Schéma d'Euler implicite par EDO
\[
    \begin{cases}
        x' = f(t, x)\\
        x(t_0) = x^0
    \end{cases}
\] 

\[
    (EI) \begin{cases}
        x_{n+1} = x_n + \Delta t f(t_{n+1}, x_{n+1}), n=0, \ldots, N-1\\
        x_0 = x^0
    \end{cases}
\] 
À l'itération $n$, pour déterminer  $x_{n+1}$, il faut résoudre l'équation $g_n(z) = 0$ où 
 \[
    g_n(z) = z - x_n + \Delta t f(t_{n+1}, z)
\] 
\begin{itemize}
    \item 
        $x^0$ donné
    \item 
        Pour $n = 0, \ldots, N-1$,
        \begin{itemize}
            \item résoudre $\begin{cases}
                \text{chercher } z \in \R\\
                z = x_n - \Delta t f(t_n + \Delta t, z) = 0
            \end{cases}$
            \item pour $x_{n+1} = z$
        \end{itemize}
\end{itemize}

\[
    (CN) \begin{cases}
        x_{n+1} = x_n + \frac{\Delta t}{2}\left[ f(t_n, x_n) + f(t_{n+1}, x_{n+1}) \right], n= 0, \ldots, N-1 \\
        x_0 = x^0
    \end{cases}
\] 
\[
    (HEUN)\begin{cases}
       \overline{x}_{n+1} = x_n + \Delta t f(t_n, x_n)\\ 
       x_{n+1} = x_n + \frac{\Delta t}{2} \left[ f(t_n, x_n) + f(t_{n+1}, \overline{x}_{n+1}) \right] \\
       x_0 = x^0
    \end{cases}
\] 

\section{Position correcte du problème $(EO)$}
On se propose de vérifier si
\begin{itemize}
    \item $(EO)$ admet une solution
    \item Si oui, cette solution est-elle unique?
    \item La solution dépend continûment des données du problèmes (stabilité)
    \item La solution est suffisamment régulière.
\end{itemize}

Pour répondre à cela, on va se placer dans le cadre scalaire: $f: \R \to \R$
\begin{prop}
   Cas où $f(x) = 0$
    \begin{itemize}
        \item Si $I = [a, b]$ et  $f$ est telle que 
             \begin{itemize}
                \item $f(a)f(b) < 0$
                \item  $f$ continue sur  $I$
            \end{itemize}
            alors il existe $x^* \in I$ tq: $f(x^*) = 0$
        \item Si de plus  $f$ est strictement monotone, alors  $x^*$ est unique
   \end{itemize}
\end{prop}
\begin{preuve}
   \begin{itemize}
       \item Théorème des valeurs intérmidiaires
       \item unicité $f$ est injective (où $f$ bijective de  $I \to f(I)$, $x^* = f^{-1}(0)$ est unique)
   \end{itemize} 
\end{preuve}
\begin{prop}
   Cas $x = g(x)$ \\
    Si $g$ est telle que 
   \begin{itemize}
       \item  $g(I) \subset I$ 
       \item $g$ continue
   \end{itemize}
   Alors , il existe $x^* \in I$ tq $g(x^*) = x^*$
    \begin{itemize}
       \item Si $g$ contractante  ($|g(x) - g(y)| < k|x - y|$ avec  $0 \le k < 1$) alors $x^*$ est unique.
   \end{itemize}
\end{prop}
\begin{remark}
   Si $g$ est dérivable avec  $|g'(x)| < j \ll 1 \forall x \in I$, alors $g$ est contractante. 
\end{remark}
\begin{remark}
    Lorsque la racine cherchée sera de multiplicité $\ge 1$. Il faudra \underline{faire attention!}. 
    \begin{itemize}
        \item Autrement dit, le problème sera difficile à resoudre numériquement si le zero est une racine multiple.
    \end{itemize}
\end{remark}
\section{Construction de schéma pour $(EO)$}
\subsection{Méthode de dichotomie}
S'applique au cas $f(x) = 0$
\subsubsection*{Principe}
$I = [a, b]$ tq  $f(a)\cdot f(b) < 0$
\begin{figure}[H]
    \centering
    \incfig{principe-de-methode-de-dichotomie}
    \caption{principe-de-methode-de-dichotomie}
    \label{fig:principe-de-methode-de-dichotomie}
\end{figure}
$[a_0, b_0] = [a, b]$, $c = \frac{a_0 + b_0}{2}$ 
\[
    [a_1, b_1] = \begin{cases}
        [a_0, c] \text{ si } f(a_0)f(c) < 0\\
        [c, b_0] \text{ si } f(c)f(b_0) < 0
    \end{cases}
\] 
Si $[a_n, b_n]$ est construit, on construit  $[a_{n+1}, b_{n+1}]$, en prennant: $c = \frac{a_n + b_n}{2}$ 
\[
    [a_{n+1}, b_{n+1}] = \begin{cases}
        [a_n, c] \text{ si } f(a_n)f(c) < 0\\
        [c, b_n] \text{ sinon }
    \end{cases}
\] 
\subsubsection*{Algorithme}
\begin{itemize}
    \item \underline{Initialisation}
        \begin{itemize}
            \item Soit $a, b$ tq  $f(a)f(b) < 0$ et  $\varepsilon$ tolérance donné
            \item Calculer  $c = \frac{a + b}{2}$ 
            \item $k = 0$
        \end{itemize}
    \item \underline{Itération}: tant que $|f(c)| > \varepsilon$ et  $k$ pas trop grand, Faire
         \begin{itemize}
            \item Si $f(a)f(b) < 0$,  poser  $b = c$
            \item Sinon, poser  $a = c$
        \end{itemize}
        Calculer $c = \frac{a + b}{2}$, $k = k+1$. Fin Faire
\end{itemize}
\subsubsection*{Coût}:
\begin{itemize}
    \item $1$ dimension
    \item  $1$ évaluation de la fonction
\end{itemize}
\subsubsection*{Convergence}
$f$ est continue. Soit  $(a_n), (b_n)$ générées
 \begin{itemize}
     \item $x^* \in [a_n, b_n]$
     \item $[a_{n+1}, b_{n+1}] \subset [a_n, b_n], \forall n$
     \item $b_n - a_n = \frac{b - a}{2^n}$ 
     \item $|a_n - x^*| \le \frac{b - a}{2^n}$, $|b_n - x^*| \le \frac{b - a}{2^n}$ 
     \item $\lim_{n \to \infty} a_n = x^*$, $\lim_{n \to \infty} b_n = x^*$
\end{itemize}

\subsection{Méthode de fausse position}
\subsubsection*{Principe}
\begin{itemize}
    \item remplacer dans dichotomie $c = \frac{a + b}{2}$ par $c$ l'abscisse du point d'intersection avec l'axe des abscisses de la droite passante par  $(a, f(a)), (b, f(b))$
         \[
        c = b - \frac{b - a}{f(b) - f(a)}f(b)
        \] 
\end{itemize}
\subsubsection*{Algorithme}
Remplacer $c = \frac{a + b}{2}$ par $c = b - \frac{b - a}{f(b) - f(a)}f(b)$ dans dichotomie
\begin{itemize}
    \item \underline{Initialisation}
        \begin{itemize}
            \item Soit $a, b$ tq  $f(a)f(b) < 0$ et  $\varepsilon$ tolérance donné
            \item Calculer  $c = b - \frac{b - a}{f(b) - f(a)}f(b)$ 
            \item $k = 0$
        \end{itemize}
    \item \underline{Itération}: tant que $|f(c)| > \varepsilon$ et  $k$ pas trop grand, Faire
         \begin{itemize}
            \item Si $f(a)f(b) < 0$,  poser  $b = c$
            \item Sinon, poser  $a = c$
        \end{itemize}
        Calculer $c = b - \frac{b - a}{f(b) - f(a)}f(b)$, $k = k+1$. Fin Faire
\end{itemize}
\subsubsection*{Coût}:
\begin{itemize}
    \item $1$ dimension, 1 produit
    \item  $1$ évaluation de la fonction
\end{itemize}

\subsubsection*{Convergence} (Fausse position)
Soit $f \in  \mathcal{C}^2([a, b])$, $f(a)f(b) < 0$. Si $f''$ n'a aucune racine dans  $I = [a, b]$, alors une des suites  $(a_n)$ ou $(b_n)$ demeure constante.

 \begin{prop}
     Si $f$ est  $\mathcal{C}^2([a, b])$ , $f(a)f(b) < 0$ et  $f''$ n'a aucune racine sur  $[a, b]$. Soient  $(a_n), (b_n)$ générées.
      \begin{itemize}
         \item Si $(b_n)$ est constante, alors  $(a_n)$ converge linéairement vers  $x^*$ racine de  $f$ et on a 
              \[
             K_1 = \lim_{n \to \infty} \frac{x^* - x_{n+1}}{x^* - x_n} = 1 - f'\frac{x^* - b}{f(b)}
             \] 
            \item Si $(a_n)$ est constante  $(b_n)$ converge linéairement vers  $x^*$ et 
                 \[
             K_1 = \lim_{n \to \infty} \frac{x^* - b_{n+1}}{x^* - b_n} = 1 - f'\frac{x^* - a}{f(a)}
                \] 
     \end{itemize}
\end{prop}
