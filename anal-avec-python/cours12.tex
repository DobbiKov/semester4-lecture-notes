\section{Méthode d'itération (ou du point fixe)}
\subsection{Principe}
Elle est adaptée aux problèmes
\[
    (EO) \, \begin{cases}
    \text{Chercher } x^* \in \R\\
    f(x^*) = x^*
\end{cases}
\] 
\begin{remark}
   Une équation $f(x^*) = 0$ devra être transformée pour rentrer dans ce cadre $(g = ?)$ \par
   On cherche $x^*$ comme  $\lim_{n \to \infty} x_n$ où 
   \[
       (S) \, \begin{cases}
           x_{n+1} = y(x_n), \, n = 0, \ldots \\
           x_0 \text{ donné }
       \end{cases}
   \] 
\end{remark}

\subsection{Interpolation géomètrique}
Comment construire $x_{n+1}$ à partir de $x_n$? 

\begin{figure}[H]
    \centering
    \incfig{interpolation-geometrique}
    \caption{Interpolation géomètrique}
    \label{fig:interpolation-geometrique}
\end{figure}

\subsection{Algorithme}
\subsubsection*{Initialisation}
\begin{itemize}
    \item Choisir $x_0$ et $x$ donné  $\varepsilon > 0$
    \item Pour  
        \begin{itemize}
            \item $k = 0$
            \item  $r_k = \varepsilon + 1$
        \end{itemize}
\end{itemize}
\subsubsection*{Itérations}
\begin{itemize}
    \item Tant que $r_k > \varepsilon$ et  $k$ pas trop grand faire
         \begin{itemize}
            \item $x_{k+1} = g(x_k)$
            \item $r_{k+1} = |x_{k+1} - x_k|$
            \item $k = k + 1$
        \end{itemize}
    \item Fin faire
\end{itemize}
\subsubsection*{Coût:}
1 évaluation de la fonction

\subsubsection*{Inconvenient:}
Choix de $x^*$

\subsection{Convergence}
\begin{prop}
   Soit $I \subset \R$ tq $g(I) \subset I$, alors $g$ admet un point fixe  $x^*$ dans $\R$
   \begin{itemize}
       \item Si $\max_{x \in I} |g'(x)| = k < 1$, alors le point fixe est unique
       \item Si $\max_{x \in I} |g'(x)| = k < 1$, et de plus, $x_0 \in I$ et $(x_n)_n$ générée par  $(S)$, alors
            \begin{itemize}
               \item $|x^* - x_{n+1}| < k|x^* - x_n|$
               \item $(x_n)_n$ converge vers  $x^*$
               \item  $k_1 = \lim_{n \to \infty} \frac{x^* - x_{n+1}}{x^* - x_n} = g'(x^*)$ et $k_1 \le k$
           \end{itemize}
   \end{itemize}
\end{prop}
\begin{preuve}
   \begin{itemize}
       \item \textbf{Existence:} $I = [a, b]$. Posons  $f(x) = f(x) - x$, on a: Comme  $g(I) \subset I$, comme $g(a) \in [a, b]$ et $g(b) \in [a, b]$ on a: $g(a) \ge a$, $g(b) \le b$, d'où $f(a) = g(a) - a \ge 0$, $f(b) = g(b) - b \le 0$. D'après le théorème de VF, $\exists x^* \in [a, b]$ tq $f(x^*) = 0$ i.e  $g(x^*) = x^*$ 
       \item \textbf{Unicité:} Supposons $\exists x^*, x^{**}$ tq $g(x^*) = x^*$ et  $g(x^{* *}) = x^{* *}$ et  $x^{* *} \neq x^*$, alors 
           \begin{align*}
               |x^{* *} - x^*| &= |g(x^{* *}) - g(x^*)| \\
                               &\le |g'(x^* + o(x^{* *} - x^*))| |x^{* *} - x^*|\\
                               &\le \max_{n \in I} |g'(n)| |x^{* *} - x^*|
           \end{align*}
           \begin{align*}
               |x^{* *} - x^*| < K |x^{* *} - x^{*}| \text{ avec } 0 < K < 1
           \end{align*}
           i.e $1 < K < 1$ - absurde! 
        \item \textbf{Convergence:}
            \begin{itemize}
                \item $x_{n+1} = g(x_n)$
                \item $x^* = g(x^*)$
            \end{itemize}
            D'où $|x_{n+1} - x^*| \le |g(x_n) - g(x^*)| \le K |x_n - x^*|$. D'où $|x_{n+1} - x^*| \le K|x_n - x^*|$ donc
            \[
            |x_n - x^*| \le K^n |x_0 - x^*|
            \] 
            Comme $0 < K < 1$ on a  $\lim_{n \to \infty} x_n = x^*$ 
        \item \textbf{Ordre de convergence:} 
            \begin{align*}
                x_{n+1} - x^* &= g(x_n) - g(x^*)\\
                              &= g'(x^* + \theta(x_n - x^*))(x_n - x^*) \quad 0 < \theta < 1
            \end{align*}
            D'où  
            \[
            \frac{x_{n+1} - x^*}{x_n - x^*} = g'(x^* + \theta(x_n - x^*))
            \] 
            Comme $g$ est de classe  $C^1$, on a  $\lim_{n \to \infty} g'(x^* + \theta(x_n - x^*)) = g'(x^*)$
            Donc  
            \[
            \lim_{n \to \infty} \frac{x_{n+1} - x^*}{x_n - x^*} = g'(x^*)
            \] 
   \end{itemize} 
\end{preuve}
\begin{remark}
   Si $g'(x^*) \neq 0$, on a une convergence linéaire (d'ordre 1) avec pour constante asymtotique $K_1 = |g'(x^*)|$ \par 
\end{remark}
Que se passe-t-il si $g'(x^*) = 0$? \par
Soit  $p \ge 1$, le plus petit entrer tel que $g^{(p)}(x^*) \neq 0$

\begin{eg}
    
\[
\begin{cases}
    x^{n+1} = g(x^n) 
\end{cases}
\] 
avec 
\begin{enumerate}
    \item $g(x) = e^x - 1 - x$
    \item  $g(x) = e^x - 1 - x - \frac{x^2}{2}$
\end{enumerate}
Rq: $x^* = 0$
 \begin{enumerate}
    \item $g'(0) = 0$,  $g''(0) = 1$, alors  $p=2$
    \item  $g'(0) = 0$,  $g''(0) = 0$,  $g'''(0) = 1 \neq 0$, alors $p = 3$
\end{enumerate}
\end{eg}

Si $g \in C^p$, alors:
\[
    g(x_n) = g(x^*) + (x_n - x^*)g'(x^*) + \sum_{k=2}^{p-1} \frac{g^{(k)}(x^*)}{k!}(x_n - x^*)^k + \frac{(x_n - x^*)^p}{p!}g^{(p)}(x^* + \theta(x_n - x^*))
\] 
Comme $g'(x^*) = \ldots g^{(p-1)}(x^*) = 0$, on a: $g(x^*) = x^*$,  $g(x_n) = x_{n+1}$
 \[
     x_{n+1} - x^* = (x_n - x^*)^p \frac{g^{(p)}(x^* + \theta(x_n - x^*))}{p!}
 \] 
 D'où
 \[
     \lim_{n \to \infty} \frac{x_{n+1} - x^*}{(x_n - x^*)^p} = \frac{1}{p!}g^{(p)}(x^*)
 \] 
 Donc $x_n \to x^*$ à l'ordre $p$.

 \section{Méthode de Newton}
 \subsubsection*{Principe}
 Adaptée pour le problème  
 \[
     (EO) \, \begin{cases}
         \text{Cherche } x^* \in I\\
         f(x^*) = 0
     \end{cases}
 \] 
 Comment construire $x_{n+1}$ à partir de $x_n$?
 \par
 On remplace  $f$ par son polynôme de Taylor d'ordre 1 au voisinage de  $x_n$
  \[
 f(x) \approx f(x_n) + f'(x_n)(x - x_n) \equiv P_f(x)
 \] 
 On trouve $f(x_n) + f'(x_n)(x_{n+1} - x_n) = 0$. D'où $x_{n+1} = x_n - \frac{f(x_n)}{f'(x_n)}$, càd $x_{n+1} = g(x_n)$ avec $g(x) = x - \frac{f(x)}{f'(x)}$ 

 \subsubsection*{Comment calculer an ayant PointFixe donné}
 \begin{lstlisting}
def PointFixe(g, x_0, eps, IterMax):
    ...
 \end{lstlisting}
 \begin{lstlisting}
def Newton(f, f_prime, x0, eps, IterMax):
    g = lambda x: x - f(x)/f_prime(x)
    return PointFixe(g, x0, eps, IterMax)
 \end{lstlisting}

 \subsubsection*{Convergence}
 On adapte les résultats de la méthode du point fixe, ici:
 \par
 Comme $x_{n+1} = x_n - \frac{f(x_n)}{f'(x_n)}$, n'est rien d'autre que
 \[
 x_{n+1} = g(x_n) \text{ avec } g(x) = x - \frac{f(x)}{f'(x)}
 \] 
 \begin{remark}
    \[
    g'(x) = x - \frac{f(x)}{f'(x)}
    \]  
    On a:
    \[
    g'(x) = 1 - \frac{f'(x)^2 - f(x)f''(x)}{(f'(x))^2}
    \] 
    D'où
    \[
    g'(x^*) = 1 - \frac{f'(x^*)^2 - 0}{f'(x^*)^2} = 0
    \] 
    D'où la méthode de Newton. Pour un bon choix de $x_0$ coverge. Et est au moins d'ordre  $2$ pour une racine simple.
 \end{remark}

 \begin{remark}
     Si $f(x^*) = 0, f'(x^*) =0, \ldots, f^{(m)}(x^*) \neq 0$ 
     \begin{eg}
         $f(x) = (x - 1)^3$,  $x^* = 1$, on a: $f(1) = 0, f'(1) = 0, f''(1) = 0, f^{(3)}(1) \neq 0$, i.e $m=3$
     \end{eg}
     \[
     \begin{cases}
         x_{n+1} = x_n - \frac{f(x_n)}{f'(x_n)}\\
         x_0 \text{ donnée }
     \end{cases}
     \] 
     On aura convergence d'ordre $1$ i.e  $\lim_{n \to \infty} \frac{x_{n+1} - x^*}{x_n - x^*} = K_1 \neq 0$ avec $K_1 = 1 - \frac{1}{m} = \frac{2}{3}$ 
     \par
     I.e $x^*$ est racine d'ordre $m$ de  $f$ d'où  $f(x) = (x - x^*)^m h(x)$ avec  $h(x^*) \neq 0$
 \end{remark}
