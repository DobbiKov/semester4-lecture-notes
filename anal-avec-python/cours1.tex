\chapter{Équations Différentielles}
\section{Modèles discrètes}
On diésigne par $N(t)$ la population d'individus à l'instant  $t$.\\
Équation du modèle discret:
 \[
     \underbrace{N(t + \Delta t) - N(t)}_{\text{variation de la population}} = \underbrace{n}_{\text{nombre de naissances}} - \underbrace{m}_{\text{nombre de décès}} + \underbrace{ \underbrace{i}_{\text{immigration}} - \underbrace{e}_{\text{émigration}} }_{\text{sol de migration}}
\] 
\subsection{Modèle de croissance géomètrique}
\begin{itemize}
    \item \underline{hypothèse}:
        \begin{itemize}
            \item solde migration nul: i.e $i - e = 0$
            \item nombre de croissance proportionnel à la taille de la population  $\underbrace{n = \lambda \Delta t N(t)}_{\text{taux de natalité}}$
            \item Idem pour le mobre de décès: $\underline{m = \mu \Delta t N(t)}_{\text{taux de mortalité}}$
        \end{itemize}
    \item \underline{Modèle}: On pose $N_n = N(t_n)$ la taille de la population à l'instant  $t_n$.
         \[
             N_{n+1} - N_{n} = \lambda \Delta t N_n - \mu \delta t N_n
        \] 
        on pose $r = \lambda - \mu$
         \begin{align}
             N_{n+1} = ( 1 + r\Delta t )N_n, \qquad n = 0 
        \end{align}
    \item \underline{Solution}: $N_n = (1 + r \Delta t)^{n}N_0, \quad n \in \N$
    \item \underline{Visualisation}: $\Delta t$ fixé
\begin{figure}[H]
   \centering 
   \begin{subfigure}{0.3\textwidth}
       \centering
       \incfig{natalite-superieure}
       \caption{Natalité supérieure\\ à la mortalité}
       \label{fig:natalite-superieure}
   \end{subfigure}
   \begin{subfigure}{0.3\textwidth}
       \centering
       \incfig{natalite-egale}
       \caption{Natalité égale\\ à la mortalité}
       \label{fig:natalite-egale}
   \end{subfigure}
   \begin{subfigure}{0.3\textwidth}
       \centering
       \incfig{natalite-inferieur}
       \caption{Natalité inférieure\\ à la mortalité}
       \label{fig:natalite-inferieur}
   \end{subfigure}
\end{figure}
\end{itemize}
\begin{property}.
    \begin{itemize}
        \item Lorsque $t \to 0$, la population semble tendre vers une courbe $N(t) = N_0 e^{rt}$, solution de $\begin{cases}
                N'(t) = rN(t)\\
                N(0) = N_0
            \end{cases}$ 
        \item Si $r > 0$, la population croît indéfiniment
        \item Si  $r < 0$, il y a extinction de l'éspèce.
    \end{itemize} 
\end{property}
\underline{Inconvenients:}
\begin{enumerate}
    \item Une croissance infinie n'est pas réaliste
    \item Pour être rigoureux, on devrait écrire $E(rN_n)$ i.e partie entière.
\end{enumerate}

\section{Modèles continues}
\underline{Motivation:} L'observation qui prend $\Delta t$ proche de  $0$ aura beaucoup plus d'information. 
 \begin{remark}
   Le modèle de croissance géomètrique 
   \begin{align*}
       &N(t + \Delta t) - N(t) = \lambda \Delta t N(t) - \mu \Delta t N(t)\\
       \implies&\frac{N(t + \Delta t) - N(t)}{\Delta t} = \lambda N(t) - \mu N(t)
   \end{align*}
   en faisant $\Delta t \to 0$
    \[
        N'(t) = \lambda N(t) - \mu N(t)
    \] 
    D'où l'équation des modèles continues:
    \[
        \underbrace{N'(t)}_{\text{vitesse de variation}} = \underbrace{n(t)}_{\text{vitesse de naissance}} - \underbrace{m(t)}_{\text{vitesse de décès}} + \underbrace{i(t)}_{\text{vitesse d'immigration}} - \underbrace{e(t)}_{\text{vitesse d'émigration}}
    \] 
\end{remark}
\subsection{Modèle de Malthus}
\begin{itemize}
    \item \underline{hypothèse}: 
        \begin{itemize}
            \item solde migration nul: $i(t) - e(t) = 0$
            \item vitesse de naissance proportionnel à la population à l'instant  $t$:  $n(t) = \lambda N(t)$
            \item vitesse de décès: $m(t) = \mu N(t)$
        \end{itemize}
    \item \underline{Modèle}: $\begin{cases}
        N'(t) = (\lambda - \mu)N(t)\\
        N(0) = N_0
    \end{cases}$
\item \underline{Solution}: $N(t) = N_0e^{(\lambda - \mu)t}$
\item 
    \begin{property}
       \begin{itemize}
           \item Il peut être si comme limite du modèle de croissance géomètrique.
           \item Lorsque $r = \lambda - \mu > 0$ croissance est proportionnel.
           \item Lorsque  $r = \lambda - \mu = 0$ la population n'évolue pas.
           \item Lorsque  $r = \lambda - \mu < 0$ la population tend vers 0.
       \end{itemize} 
    \end{property}
\item \underline{Inconvenients}:
    \begin{itemize}
        \item croissance exponentielle pas réaliste. Il faut prendre en compte:
            \begin{itemize}
                \item la limitation des ressources
                \item l'interaction avec l'environnement
            \end{itemize}
    \end{itemize}
\end{itemize}
\subsection{Modèle Verhulst}
Corrige le modèle de Malthus en prennant en compte la limitation de ressources.
\begin{itemize}
    \item \underline{Idée}: limiter la croissance à un seuil $K$ appelé capacité biotique
\begin{figure}[H]
    \centering
    \incfig{malthus-prendre-en-compte}
    \caption{Modèle de Malthus}
    \label{fig:malthus-prendre-en-compte}
\end{figure}
\begin{figure}[H]
    \centering
    \incfig{verhulst}
    \caption{Modèle de Verhulst}
    \label{fig:verhulst}
\end{figure}
\item \underline{hypothèse}: Sole de migration nul
    \begin{itemize}
        \item taux de natalité fonction afiine décroissante de la population $\lambda \approx \lambda (1 - \frac{N(t)}{K})$ 
        \item taux de mortalité fonction affine croissante de la population $\mu \approx -\mu (1 - \frac{N(t)}{K})$
    \end{itemize}
\item \underline{Modèle}: $\begin{cases}
    N'(t) = rN(t)(1 - \frac{N(t)}{K})\\
    N(0) = N_0
\end{cases}$
\item \underline{Solutions}: $N(t) = \frac{K}{1 + (\frac{K}{N_0} - 1) e^{-rt}}$ \quad $t > 0$ 
\item \underline{Visualisation}:
    \begin{figure}[H]
    \centering
    \incfig{verhulst-visualisation}
    \caption{Verhulst solution}
    \label{fig:verhulst-visualisation}
\end{figure}
\item 
    \begin{property}
    Si $r>0$, on a:
    \begin{itemize}
        \item si $N_0 = 0$ $N_0 = K$ on a: $N(t) = N_0 \, \forall t > 0$
        \item si $0 < N_0 < K$, $N$ croissante
        \item si $N_0 > K$, $N$ décroissante 
        \item $N$ possède une limite si  $N_0 > 0$
            \[
            \lim_{t \to \infty} N(t) = K
            \] 
    \end{itemize}
    \end{property}
\end{itemize}
\section{Modèle de croissance logistique}
C'est un modèle discrét
\begin{itemize}
    \item \underline{hypothèse}: i.e $= 0$\\
         $n - m$ est une fonction affine de la population,  i.e $n - m = r \Delta t N(t)( 1 - \frac{N(t)}{K} )$ 
    \item \underline{Modèle}: On suppose $\Delta t = 1$: On pose  $N_n = N(t_n)$
         \[
        \text{On a:} \begin{cases}
            N_{n+1} - N_n = r N_n (1 - \frac{N_n}{K})\\
            N_0 \text{ donné}
        \end{cases}
        \] 
    \item 
        \begin{property} (À vérifier numeriquement) 
            \begin{itemize}
                \item si $r < 2$, la suite converge vers  $K$ 
                \item si  $2 < r < 2.449$, la suite converge vers un cycle
                \item si  $2.449 < r < 2.57$, la suite est encore un cycle mais plus complèxe
                \item si  $r > 2.57$, la suite devient chaotique
            \end{itemize}
        \end{property}
\end{itemize}
