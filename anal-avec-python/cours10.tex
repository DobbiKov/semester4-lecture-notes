\section{Étude de schémas pokes EDOs}
On aimerait savoir si la suite $(x_n)_{n}$ générée par $S$ converge. Si oui,  "vers" la solution de $P$.
\subsection{Définition}
\begin{definition}
    le schéma $S$ est dit convergent ssi pour toute solution x de  $P$ et toute suite de  $(x_n)_n$ construite par  $S$ courbe  $x^0 = x(t_0)$ on a
    \[
    \lim_{N \to +\infty} \left( \max_{0 \le n \le N} \|x(t_n) - x_n \| \right)  = 0
    \] 
    Où $x(t_n) - x_n =: e_n$ erreur globale à l'itération  $n$
    \begin{itemize}
        \item S'il existe $C > 0$, ne dépendant que de  $f, t_0, T$ tq  
            \[
                \max_{0 \le n \le N} \|e_n\| \le  C \Delta t^p \quad \forall t \in [0, \Delta t_0]
            \] 
    \end{itemize}
    Le schéma est dit convergent d'ordre $p$.
\end{definition}
\begin{remark}
   Pour étudier un schéma pour EDO, on procède en 2 étaps:
   \begin{itemize}
       \item On cherche l'ordre de consistance
       \item Puis, on regarde sa stabilité
   \end{itemize}
\end{remark}
\subsection{Ordre d'un shéma à un pas explicite}
\begin{definition}
    (erreur de consistance)\\
    Soit $x$ la solution de  $P$. On appelle erreur locale de troncature du schéma  $S$ à l'instant  $t_n$ la quantité 
    \[
        \xi_n = x(t_{n+1}) - x(t_n) - \Delta t \Phi (t_n, x(t_n), \Delta t)
    \] 
\end{definition}
\[
P \begin{cases}
    x'(t) = f(t, x(t)) \quad t \in ]t_0, t_0 + T[\\
    x^0 = x(t_0)
\end{cases}
\] 

\begin{definition}
    $S$ et dit consistant d'ordre au moins  $q$ si pour toute solution exacte  $x$ de  $P$, il existe une constante  $C > 0$ telle que  
    \[
        \xi(t, \Delta t) = x(t + \Delta t) - x(t) - \Delta t + \Phi(t, x(t), \Delta t)
    \] 
    vérifie
    \[
        \|\xi(t, \Delta t)\| \le C \Delta t^{q+1}
    \] 
    \begin{itemize}
        \item Si $S$ est d'ordre au moins  $q$ mais pas d'ordre au moins  $q+1$, alors il est dit consistant d'ordre exactement  $q$.
    \end{itemize}
\end{definition}
\begin{eg}
   Étude de consistance du Schéma d'Euler explicite
   \[
   S \begin{cases}
       x_{n+1} = x_n + \Delta f(t_n, x_n), \quad n = 0, \ldots, N\\
       x_0 = x^0
   \end{cases}
   \] 
   \begin{itemize}
       \item 
   L'erreur de consistance à l'instant $t$ est 
    \[
        \xi(t, \Delta t) = x(t + \Delta t) - x(t) - \Delta t f(t, x(t)) 
   \] 
   où $x$ est solution suffisamment régulière de  $P$ (i.e $x'(t) = f(t, x(t))$)
\item On cherche le petit $o$ de  $\Delta t$ dans  $\xi(t, \Delta t)$.
\end{itemize}
Comme $x$ est solution exacte, on a:
 \[
\xi(t, \Delta t) = x(t + \Delta t) - x(t) - \Delta t x'(t)
\] 
Effectons les $\Delta t$
 \[
x(t + \Delta t) = x(t) + \Delta x'(t) + \frac{\Delta t^2}{2}x''(t) + o(\Delta t^2)
\] 
D'où
\[
\xi(t, \Delta t) = \Delta t^2 \frac{x''(t)}{2} + o(\Delta t^2)
\] 
càd $\xi(t, \Delta t) = o(\Delta t^2)$ d'où le schema est consistant d'ordre au moins  $1$. 
Or pour le problème $\begin{cases}
    x^0(t) = f(t, x(t))
\end{cases}$ avec $f(t, y) = 2t$., on a la solution exacte  $x(t) = x_0 + \frac{t^2}{2}$. Et on a:
\[
\xi(t, \Delta t) = \Delta t^2
\] 
D'où le schema ne peut pas être consistant d'ordre au moins $2$. \underline{Il est donc consistant d'ordre exactement $1$.}
\end{eg}
\begin{prop}
    Soit $\Phi: I \times \R^d \times [0, \Delta t_0] \to \R^d$ telle que $\frac{\partial \Phi}{\partial \Delta t}, \frac{\partial^2 \Phi}{\partial \Delta t^2}, \ldots, \frac{\partial^q \Phi}{\partial \Delta t^q}$ existant et sont continues sur $I \times \R^d \times [0, \Delta t_0]$, $f$ étant de classe  $C^q$.
    Alors le schema  
    \[
    S \begin{cases}
        x_{n+1} = x_n + \Delta t \Phi(t_n, x_n, \Delta t) \quad n = 0, \ldots, N-1\\
        x_0 = x^0
    \end{cases}
    \] 
    est consistant d'ordre au moins $q$ ssi,
    \begin{align*}
        \Phi(t, y, 0) = f(t, y)\\
        \frac{\partial \Phi}{\partial \Delta t}(t, y, 0) = \frac{1}{2}f^{[1]}(t, y)\\
        \vdots\\
        \frac{\partial^{q-1} \Phi}{\partial \Delta t^{q-1}}(t, y, 0) = \frac{1}{q}f^{[q-1]}(t, y)
    \end{align*}
    où $f^{[0]}(t, y) = f(t, y)$,  $f^{[j]}(t, y) = \frac{\partial f^{[j-1]}}{\partial t}(t, y) + f(t, y) \frac{\partial f^{[j-1]}}{\partial y}(t, y)$ et $j = 1, \ldots, q-1$
\end{prop}
\begin{eg}
   Étude du schema du PM
   \[
   \begin{cases}
       x_{n + \frac{1}{2}} = x_n + \frac{\Delta t}{2}f(t_n, x_n)\\
       x_{n+1} = x_n + \Delta t f(t_{n + \frac{1}{2}}, x_{n + \frac{1}{2}})\\
       x_0 = x^0
   \end{cases}
   \] 
   ce schéma est explicite à un pas avec
   \[
   \Phi(t, y, \Delta t) = f(t + \frac{\Delta t}{2}, y + \frac{\Delta t}{2}f(t, y))
   \] 
   Comme 
   \begin{itemize}
       \item $\Phi(t, y, 0) = f(t, y)$ ce schema est consistant d'ordre au moins  $1$.
       \item  $\frac{\partial \Phi}{\partial \Delta t}(t, y, \Delta t) = \frac{1}{2}\frac{\partial f}{\partial t}(t + \frac{\Delta t}{2}, y + \frac{\Delta t}{2}f(t, y)) + \frac{1}{2}f(t, y) \frac{\partial f}{\partial y}(y + \frac{\Delta t}{2}, y + \frac{\Delta t}{2}f(t, y))$
           \[
               \frac{\partial \Phi}{\partial \Delta t}(t, y, 0) = \frac{1}{2} \left[ \frac{\partial f}{\partial t}(t, y) + f(t, y) \frac{\partial f}{\partial y}(t, y) \right] = \frac{1}{2}f^{[1]}(t, y)
           \] 
   \end{itemize}
   D'où le schema des $(PM)$ est consistant d'ordre au moins  $2$.
\end{eg}
\begin{ex}
   Mq qu'il ne peut pas être consistant d'ordre. Choisissez $f(t, y) = g(t)$ et revenez à la définition de $\xi(t, \Delta t)$
\end{ex}

\subsection{Stabilité des schémas à un pas}
\begin{definition}
    $(S)$ est di stable pour une classe de fonction  $f$, s'il existe une constante  $S \ge 0$ indépendante de $\Delta t$ tq:\\
    Pour toute suite  $(x_n)_{0 \le n \le N}$, $(\tilde{x_n})_{0 \le n \le N}$, $(r_n)_{0 \le n \le N-1}$ :
    \begin{align*}
        x_{n+1} = x_n + \Delta t \Phi(t_n, x_n, \Delta t)\\
        \tilde{x}_{n+1} = \tilde{x}_n + \Delta t \Phi(t_n, \tilde{x}_{n}, \Delta t) + r_n
    \end{align*}
    on a la majoration 
    \[
        \underbrace{\max_{0 \le n \le N} \{ \|\tilde{x}_n - x_n\|  \}}_{\substack{\text{erreur maximale}\\ \text{pour toutes les itérations}}} \le S \{ \underbrace{\|\tilde{x}_0 - x_0\|}_{\text{erreur initiale}} + \sum_{j=0}^{N-1} \underbrace{\|\eta_n\|}_{\substack{\text{erreur ajoutée}\\ \text{à chaque itération}}} \}
    \] 
\end{definition}
\begin{prop}
    (Stabilité du schéma d'Euler explicite) \\ 
    Si $f$ est lipschitzienne en espace, alors le schema d'Euler explicite est stable.
\end{prop}
\begin{preuve}
    \[
        \|x_{n+1} - \tilde{x}_{n+1}\| \le (1 + L\Delta t) \|x_n - \tilde{x}_n\| + \|\eta_n\|
    \] 
    Et on conclut pas le lemme de Gramwall discret.
\end{preuve}
\begin{lemma}
   de Graomwall discret:\\  
   $(a_n)_{0 \le n \le N}$, $(b_n)_{0 \le n \le N-1}$ termes positifs et $h > 0, L > 0$ tq
    \[
   a_{n+1} \le (1 + Lh)a_n + b_n \quad n = 0, \ldots, N-1
   \] 
   alors
   \[
       a_n \le e^{Lnh} a_0 + \sum_{j=0}^{n-1} e^{L(n - j - 1)h}b_j
   \] 
   D'où
   \[
       a_n \le e^{nhL}(a_0 + n\max_{0 \le j \le n-1}|b_j|)
   \] 
\end{lemma}
\begin{prop}
   Si la fonction $\Phi$  est telle que: il existe $\Lambda \ge 0$ tq:
   \[
       \forall \Delta t \in [0, \Delta t_0], \forall t \in [t_0, t_0 + T - \Delta t] \forall y_1, y_2 \in \R^d
   \] 
   \[
   \left\| \Phi(t, y_1, \Delta t) - \Phi(t, y_2, \Delta t) \right\| \le \Lambda \|y_1 - y_2\|
   \] 
   alors le schéma $(S)$ est stable de constante de stabilité  $S = e^{\Lambda T}$
\end{prop}

\subsection{Convergence des schémas à un pas explicite}
\begin{theorem}
    \begin{enumerate}
        \item Si la méthode $(S)$ est stable et consistant, alors elle est convergente
        \item Si la méthode  $(S)$ est stable et consistant d'ordre  $q$, alors elle est convergente d'ordre  $q$ et pour toute suite $(x_n)_{0 \leq n \leq N}$ donnée par $S$, initialisée par $x_0 \in \R^d$, on a 
            \[
            \max_{0 \le n \le N} \|x(t_n) - x_n\| \le \left( \|x(t_n) - x_0\| + CT\Delta t^q \right) 
            \] 
            Où $S$ est une constante de stabilité et  $C$ - constante de consistance.
    \end{enumerate}
\end{theorem}
