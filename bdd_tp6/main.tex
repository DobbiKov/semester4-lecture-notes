\documentclass[a4paper]{article}

\usepackage[utf8]{inputenc}
\usepackage[T1]{fontenc}
\usepackage{textcomp}
\usepackage[english]{babel}
\usepackage{amsmath, amssymb, amsthm}
\usepackage{listings}


% figure support
\usepackage{import}
\usepackage{xifthen}
\pdfminorversion=7
\usepackage{pdfpages}
\usepackage{transparent}
\usepackage{hyperref}
\usepackage[margin=0.8in]{geometry}

\usepackage{setspace}
\setlength{\parindent}{0in}

\newcommand{\incfig}[1]{%
    \def\svgwidth{\columnwidth}
    \import{./figures/}{#1.pdf_tex}
}

\pdfsuppresswarningpagegroup=1

\newcommand{\N}{\mathbb{N}}
\newcommand{\R}{\mathbb{R}}
\newcommand{\Z}{\mathbb{Z}}
\newcommand{\Q}{\mathbb{Q}}

\newtheorem{theoreme}{Théorème}[section]
\newtheorem{definition}{Définition}[section]
\newtheorem{exemple}{Exemple}[section]
\newtheorem{proposition}{Proposition}[section]
\newtheorem{propriete}{Propriété(s)}[section]
\newtheorem*{notation}{Notation}
\newtheorem*{remarque}{Remarque}

\title{TP 6 de BDD}

\begin{document}
\maketitle
\subsection*{Question 18}
\begin{lstlisting}[language=SQL]
SELECT personne.nom, personne.prenom FROM personne WHERE personne.pid 
in 
( SELECT sub2.pid FROM 
( SELECT sub.pid, sub.an FROM
(
SELECT role1.pid, film.fid, film.an FROM role1 JOIN film ON film.fid = role1.fid
) as sub 
GROUP BY sub.pid, sub.an HAVING COUNT(DISTINCT sub.fid) > 1 )
as sub2); 
\end{lstlisting}
\subsection*{Question 19}
\begin{lstlisting}[language=SQL]
SELECT personne.nom, personne.prenom FROM personne WHERE personne.pid in 
(
  SELECT sub.pid FROM 
(SELECT role1.pid, film.rang FROM role1 INNER JOIN film ON role1.fid = film.fid) as sub
  WHERE sub.rang = (SELECT MAX(rang) FROM film)
);
\end{lstlisting}

\subsection*{Question 20}
\begin{lstlisting}[language=SQL]
SELECT personne.nom, personne.prenom FROM personne WHERE NOT EXISTS
(
  SELECT 1 FROM role1 JOIN film ON role1.fid = film.fid 
      JOIN mes ON mes.fid = film.fid
      JOIN personne p2 ON p2.pid = mes.pid and p2.prenom = 'Woody' and p2.nom = 'Allen'
  WHERE role1.pid = personne.pid
); 
\end{lstlisting}
\subsection*{Question 21}
Ici, on a changé l'énoncé à rendre les personnes qui ont joué seulement 'Nola Rice' car il n'y avait pas de personnes d'id 798
\begin{lstlisting}[language=SQL]
SELECT p1.nom, p1.prenom FROM personne p1 JOIN role1 r1 ON r1.pid = p1.pid 
GROUP BY p1.pid, p1.nom, p1.prenom 
HAVING COUNT(DISTINCT r1.nom) = 1 AND MIN(r1.nom) = 'Nola Rice'; 
\end{lstlisting}

\subsection*{Question 22}
\begin{lstlisting}[language=SQL]
SELECT p1.nom, p1.prenom FROM personne p1
WHERE
(SELECT COUNT(f.fid) FROM film f 
  JOIN mes ON mes.fid = f.fid 
  JOIN role1 r2 ON r2.fid = f.fid AND r2.pid = p1.pid
  JOIN personne p2 ON p2.pid = mes.pid and p2.prenom = 'Woody' AND p2.nom = 'Allen'
) >= 
(SELECT COUNT(f2.fid) FROM film f2 
  JOIN mes ON mes.fid = f2.fid 
  JOIN personne p2 ON p2.pid = mes.pid and p2.prenom = 'Woody' and p2.nom = 'Allen'
)
;   
\end{lstlisting}


\subsection*{Question 23}
\begin{lstlisting}[language=SQL]
SELECT p1.nom, p1.prenom FROM personne p1
JOIN role1 r1 ON r1.pid = p1.pid
JOIN film f ON f.fid = r1.fid
GROUP by f.fid, p1.pid HAVING COUNT(DISTINCT r1.nom) >= 
(SELECT COUNT(DISTINCT r2.nom) FROM role1 r2 JOIN film f2 ON f2.fid = r2.fid WHERE f2.fid = f.fid);
\end{lstlisting}

\end{document}
