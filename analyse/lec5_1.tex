
\chapter{Fonctions de plusieurs variables}
\underline{Cadre:} $\R^n, \R^p$ $D \subset \R^n$
\[
F:D \to \R^p
\] 
sur $\R^n$, $\R^p$ distances usuelles, sur $D$ la distance héritée de  $\R^n$.
% \begin{figure}[ht]
%     \centering
%     \incfig{d-distance-exemple}
%     \caption{d-distance-exemple}
%     \label{fig:d-distance-exemple}
% \end{figure}

avec des coordonnées cartésiennes
\[
F(x_1, \ldots, x_n) = (F_1(x_1, \ldots, x_n), F_2(x_1, \ldots, x_n), \ldots, F_p(x_1, \ldots, x_n))
\] 
où $F_i: D \to \R$

\[
F: D \to \R^p \text{ continue}
\] 

on connaît:
\begin{lemma}
   \[
   F: D \to  \R^p \text{ continue ssi:}
   \]  
   chaque $F_i: D \to \R$ est continue
\end{lemma}
\begin{preuve}
    $Y_n = (Y_{1,n}, \ldots, Y_{p, n})$  suite des $\R^p$. $Y_n \to Y$ ssi $Y_{i,n} \to Y_i$ ($1 \le i \le p$)
\end{preuve}
