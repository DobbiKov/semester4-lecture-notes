\begin{eg}
   \begin{enumerate}
       \item $E = \R$, \quad  $d(x, y) = |x - y|$
            \[
                B(x_0, r) = ]x_0 -r, x_0 + r[
           \] 
       \item $E = \R^d$, \quad $d = 2,3$, \quad $X = (x_1, \ldots, x_d)$ 
           \begin{align*}
               &\|X\|_2 = \left( \sum_{i=1}^{d} x_i^2 \right)^{\frac{1}{2}} \\
               &\|X\|_1 = \sum_{i=1}^{d} x_i\\
               &\|X\|_{\infty} = \underset{1 \le i \le d}{\text{max}}|x_i|
           \end{align*}
           \begin{align*}
              &d_2(X, Y) = \|Y - X\|_2 = \|\vec{XY}\|_2\\ 
              &d_1(X, Y), d_{\infty}(X, Y)
           \end{align*}
   \end{enumerate} 
\end{eg}
\begin{property} Dans $\R^n$
   \begin{itemize}
       \item $d_{\infty}(X, Y) \le d_1(X, Y) \le n d_{\infty}(X, Y)$
       \item $d_{\infty}(X, Y) \le d_2(X, Y) \le \sqrt{n} d_{\infty}(X, Y)$
   \end{itemize} 
\end{property}
\section{Parties bornées de $(E, d)$}
\begin{definition}
    Soit $A \subset E$. $A$ est bornée si  $\exists R > 0$ et $\exists x_0 \in E$ tel que 
    \[
    A \subset B(x_0, R)
    \] 
\end{definition}
\begin{figure}[H]
    \centering
    \incfig{exemple-bornee}
    \caption{Exemple d'un enesemble borné}
    \label{fig:exemple-bornee}
\end{figure}
\begin{lemma}
   Les propriétés suivantes sont équivalentes:
   \begin{enumerate}
       \item $A$ est bornée
       \item  $\forall x_0 \in E, \exists r > 0$ tel que $A \subset B(x_0, r)$
       \item $\exists r > 0$ tel que $\forall x, y \in A$ on a $d(x, y) < r$
   \end{enumerate}
\end{lemma}
\begin{explanation} de lemme
   \begin{itemize}
       \item $(1) \implies (2)$ :\\
           \underline{Hyp}: $\exists x_1 \in E, \exists r_1 \in E$ tq $A \subset B(x_1, r_1)$\\
           Soit $x_0 \in E$. But: trouver $r$ tel que  $A \subset B(x_0, r)$ si $x \in A$, on a:  $d(x_1, x) < r_1$\\
           \underline{Je veux}: $d(x_0, x) <r$\\
          \begin{align*}
              d(x_0, x) \le d(x_0, x_1) + d(x_1, x) \le d(x_0, x_1) + r_1 < r \quad \text{ si } r > d(x_0, x_1) + r_1
          \end{align*} 
   \end{itemize} 
\end{explanation}
\begin{property}
   \begin{enumerate}
       \item Toute partie finie est bornée
       \item Si $A$ botnée et  $B \subset A$ alors $B$ bornée
       \item L'union d'un nombre \underline{fini} de bornés est borné
   \end{enumerate} 
\end{property}
\begin{preuve}{de (3).}\\
    $A_1, \ldots, A_n$ sont bornés. \underline{Je fixe $x_0 \in E$}, $A_i$ borné ($1 \le i \le n$), donc $\exists r_i > 0$ tel que $A_i \subset B(x_0, r_i)$ si $r = \underset{1 \le  i \le n}{max} r_i$ 
    \[
        A_i \subset B(x_0, r), \, \forall i \implies \bigcup\limits_{i=1}^{n} A_i \subset B(x_0, r)
    \] 
\end{preuve}
\section{Fonctions bornées}
\begin{definition}
    Soit $B$ un ensemble. Une fonction  $F: B \to E$ est bornée si $F(B) = \{ F(b): b \in B\} \subset E$ est borné.
\end{definition}
\section{Distance entre ensembles}
\begin{definition}
    Soit $A, B$ deux parties de  $E$. On pose  $d(A, B) = \underset{ x \in A, \, y \in B}{inf} d(x, y)$
\end{definition}
\begin{property}
   $\forall x \in A, \, y \in B, \, d(x, y) \ge d(A, B)$ et $\forall \epsilon > 0, \exists x \in A, \, y \in B$ tq $d(x, y) \le d(A, B) + \epsilon$
\end{property}
\begin{figure}[ht]
    \centering
    \incfig{distance-entre-ensembles}
    \caption{Distance entre ensembles}
    \label{fig:distance-entre-ensembles}
\end{figure}
\begin{definition}
    Distance entre un point et un ensemble:
    \[
        d(x, A) = d(\{x\}, A) = \underset{y \in A}{inf}\: d(x, y)
    \] 
    $x \neq \{x\}$
\end{definition}
\section{Topologie des espaces métriques}
distance $d(x, y)$ $\longrightarrow$ boules  $B(x_0, r)$ $\longrightarrow$ ensembles ouverts
\begin{definition}
    Soit (E, d) espace métrique.
    \begin{enumerate}
        \item $U \subset E$ est ouvert si $\forall x_0 \in U, \, \exists r > 0 \: r(x_0)$ tel que $B(x_0, r) \subset U$
        \item $F \subset E$ est fermé si $E \setminus F$ est ouvert
    \end{enumerate}
    $\O$ est ouvert et $E$ est ouvert.  $\O$ est fermé et $E$ est fermé.
\end{definition}
\begin{remark}
   dans $\R$ les intervalles ouverts sont des ouverts (pareil pour fermés) 
\end{remark}
\begin{remark}
   Une distance entre deux ensembles ouverts toujours existe et elle est infimum (qui n'est jamais atteint) 
\end{remark}
\begin{lemma}
   \begin{enumerate}
       \item $B(x_0, r_0)$ est ouvert.
       \item $B_f(x_0, r_0)$ est fermé.
   \end{enumerate} 
\end{lemma}
\begin{preuve}
   \begin{enumerate}
       \item Soit $x_1 \in B(x_0, r_0)$ ($d(x_0, x_1) < r_0$).\\
           But: touver $r_1 > 0$ tel que $B(x_1, r_1) \subset B(x_0, r_0)$?\\
           \begin{align*}
               &x \in B(x_1, r_1): \: d(x_1, x) < r_1\\
               &x \in B(x_0, r_0) \text{ si } d(x_0, x) < r_0
           \end{align*}
           facile:
           \begin{align*}
               d(x_0, x) &\le d(x_0, x_1) + d(x_1, x)\\
                         &\le d(x_0, x_1) + r_1\\
                         &< r_0 \text{ si}
           \end{align*}
           $r_1 < r_0 - d(x_0, x_1) > 0$
   \end{enumerate} 
\end{preuve}
\begin{eg} bizzare.\\
    Soit $E = \R$, $d(x, y) = |y - x|$,  $A = ]0, 1[$ ouvert, pas fermé dans  $\R$.\\
    \begin{center}
       \begin{tikzpicture}
          \draw (-2, 0) -- (2, 0); 
          \node (x) at (0, 0){]};
          \node (y) at (1, 0){[};
          \node[below] (x) at (0, -0.2){$0$};
          \node[below] (y) at (1, -0.2){$1$};
          \draw[color=red] (-2, 0) -- (0, 0);
          \draw[color=red] (1, 0) -- (2, 0);
       \end{tikzpicture} 
    \end{center}
    Je regarde $A$ comme partie de  $(A, d)$. Comme  $A \setminus A = \O$ qui est ouvert, donc $A$ est fermé dans $A$. Par contre, les bornes ne sont jamais atteints, alors $A$ est ouvert dans  $(A, d)$.
\end{eg}
\begin{theorem}.
    \begin{enumerate}
        \item Soit $U_i$,  $i \in I$ une collection d'ouverts. Alors,  $\cup_{i \in I} \,U_i$ est ouvert.\\
            Translate: Une union des ensembles ouverts est ouvert.
        \item Si $U_1, \ldots, U_n$ sont ouverts
            \[
                \bigcap\limits_{i=1}^{n} \, U_i \text{ est ouvert.}
            \] 
            Translate: intersection des ensembles ouverts est ouvert.
    \end{enumerate}
    \begin{enumerate}
        \item Soit $U_i$,  $i \in I$ une collection de fermés. Alors,  $\cup_{i \in I} \,U_i$ est fermé.\\
            Translate: Une union des ensembles fermés est fermé.
        \item Si $U_1, \ldots, U_n$ sont fermés 
            \[
                \bigcap\limits_{i=1}^{n} \, U_i \text{ est fermé.}
            \] 
            Translate: intersection des ensembles fermés est fermé.
    \end{enumerate}
\end{theorem}
\begin{preuve}.
    \begin{enumerate}
        \item Soit $x \in U := \bigcup\limits_{i \in I} U_i$. Il existe un $i$ noté  $i_0$ tel que $x \in U_{i_0}$, $U_{i_0}$ est ouvert, donc $\exists r > 0$ tel que $B(x, r) \subset U_{i_0} \subset U := \bigcup\limits_{i \in I} U_i$.
        \item Soit $x \in U := \bigcap\limits_{1 \le i \le n} U_i$.\\
            On fixe $i$.  $x \in U_i$,  $U_i$ ouvert, donc  $\exists r_i > 0$ tel que $B(x, r) \subset U_i$, $1 \le i \le n$, donc $B(x, r) \subset U := \bigcap\limits_{1 \le i \le n} U_i$
    \end{enumerate}
\end{preuve}
