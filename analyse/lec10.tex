\begin{prop}
   Soit $A \in B(E, F)$ et $\|A\| = \sup_{\|x\|_E \le 1} \|A\|_F$  une norme uniforme. $\|A\| = $ plus petit  $c$ tel que 
    \[
   \|Ax\|_F \le c \|x\|_E \quad \forall x \in E
   \] 
\end{prop}
\begin{preuve}
    $E = \mathcal{C}([a, b], \R)$ et $\|f\|_1 = \int_{{a}}^{{b}} {|f(x)|} \: d{x}$ norme sur $\mathcal{C}([a, b], \R)$. Je fixe $m \in \mathcal{C}([a, b], \R)$ et $A: f \to mf$. $Af(x) = m(x)f(x)$. 
    \begin{itemize}
        \item $A \in L(E)$ évident
        \item $A \in B(E)$?
    \end{itemize}
    Trouver $c \ge 0$ telle que 
    \[
    \|Af\|_{1} \le c \|f\|_1 \quad \forall f \in E
    \] 
    \begin{align*}
        \|Af\|_1 = \int_{{a}}^{{b}} {|m(x)f(x)|} \: d{x}
    \end{align*}
    \begin{align*}
        |m(x)f(x)| \le |m(x)| |f(x)| \le \|m\|_{\infty} |f(x)|
    \end{align*}
    \[
        \|m\|_{\infty} = \sup_{x \in [a, b]}|m(x)|
    \] 
    \begin{align*}
        \int_{{a}}^{{b}} {|m(x)f(x)|} \: d{x} \le \|m\|_{\infty}\int_{{a}}^{{b}} {|f(x)|} \: d{x} = \|m\|_{\infty}\|f\|_1
    \end{align*}
    \[
    c = \|m\|_{\infty}
    \] 
    On a: $A \in B(E)$ et $\|A\| \le \|m\|_{\infty}$. Montrons que $\|A\| = \|m\|_{\infty}$
    \[
        \|A\| = \sup_{\|f\|_1 \le 1} \|Af\|_{1} \overset{?}{=} \|m\|_{\infty} = \sup I \text{ avec } I = \{ \|Af\|_{1} : \|f\|_{1} \le 1 \}
    \] 
    Notons $\alpha = \sup I$
     \begin{enumerate}
        \item $\alpha$ majorant de  $I$
        \item  $\exists (a_n) \, a_n \in I$ avec $a_n \xrightarrow[n \to \infty]{} \alpha$
    \end{enumerate}
    Dans notre cas:
    \begin{itemize}
        \item But: trouver une suite $f_n \in E$ $\|f_n\|_1 \le 1$ et $\|Af_n\|_{1} \to \|m\|_{\infty}$
    \end{itemize}
    $a_n = \|Af_n\|_{1}$ $\|m\|_{\infty} = \sup$ de la fonction $|m|$ sur  $[a, b]$.
     \begin{itemize}
         \item $|m|$ continue:  $\exists x_0 \in [a, b]$ tel que $\|m\|_{\infty} = |m|(x_0)$
    \end{itemize}
    \[
    |m|(x) = |m(x)|
    \] 
\begin{figure}[H]
    \centering
    \incfig{preuve-prop-line-cont-borne}
    \caption{$f_n$}
    \label{fig:preuve-prop-line-cont-borne}
\end{figure}
\[
|m(x)f_n(x)| = |Af(x)| \text{ proche de } |m(x_0)| |f_n(x)|
\] 
$\|f_n\|_{1} = 1$ si $c_n \le 2n$
\[
f_n(x) = \begin{cases}
    0 \text{ si } a \le x \le x_0 - \frac{1}{2n}\\
    2n(1 - n|x - x_0|) \text{ si } |x - x_0| \le \frac{1}{2n}\\
    0 \text{ si } x_0 + \frac{1}{2n} \le x \le b
\end{cases}
\] 
\begin{figure}[H]
    \centering
    \incfig{preuve-lin-cont-bornee-2}
    \caption{$f_n$}
    \label{fig:preuve-lin-cont-bornee-2}
\end{figure}
\[
|m(x)f_n(x) - m(x_0)f_n(x)| \le |m(x) - m(x_0)| |f_n(x)| \le \varepsilon_n|f_n(x)|
\] 
Là où $f_n(x) \neq 0$ $|x - x_0| \le \frac{1}{n}$ donc 
\[
|m(x) - m(x_0)| \le \varepsilon_n \quad \varepsilon_n \xrightarrow[n \to \infty]{} 0
\] 
alors $m$ continue en  $x_0$.
 \begin{align*}
    \|Af_n\|_{1} = \int_{{a}}^{{b}} {|m(x)f_n(x)|} \: d{x} \le \int_{{a}}^{{b}} {|m(x) - m(x_0)| |f_n(x)|} \: d{x} + \int_{{a}}^{{b}} {|m(x_0)| |f_n(x)|} \: d{x} 
\end{align*}
\begin{itemize}
    \item $1^{er}$ terme: $\le \varepsilon_n\|f_n\|_{1} = \varepsilon_n$
    \item $2^{eme}$ terme: $:= \|m\|_{\infty}\|f_n\|_{1} = \|m\|_{\infty}$
\end{itemize}
Alors:
\begin{align*}
    &\|f_n\|_{1} = 1\\
    &\|Af_n\|_{1} \to \|m\|_{\infty}\\
    &\text{donc } \|A\| = \|m\|_{\infty}
\end{align*}
\end{preuve}
\begin{prop}
   Le cas de $B(E)$:
   \par
   Si $A, B \in B(E)$, $A \circ B$ (noté $AB$) $\in B(E)$ et 
   \[
   \|AB\| \le \|A\|\|B\|
   \] 
   (très utile)
\end{prop}
\begin{preuve}
   \begin{align*}
       \|A\underbrace{Bx}_{Y}\|_{E} \le \|A\|\|Bx\|_{E} \le \overbrace{\|A\|\|B\|}^{c} \cdot \|x\|_{E}
   \end{align*} 
    donc $\|AB\| \le \|A\| \|B\|$
\end{preuve}
\begin{theorem}
    Si $N_1, N_2$ sont deux normes sur $E$.  $N_1$ et $N_2$ sont topologiquement équivalentes $\iff$ $N_1$ et $N_2$ sont équivalentes.
\end{theorem}
\begin{preuve}
   $E_1$ c'est $(E, N_1)$, $E_2 = (E, N_2)$.
   \par
   $N_1$ et $N_2$ topologiquement équivalentes veut exactement dire:
   \begin{enumerate}
       \item $Id: E_1 \to E_2$ sont continue 
       \item et $Id: E_2 \to E_1$
   \end{enumerate}
   Donc:
   \begin{enumerate}
       \item $\Omega$ ouvert  pour $N_2$ $\implies$ $\Omega$ ouvert  pour  $N_1$
       \item $\Omega$ ouvert  pour $N_1$ $\implies$ $\Omega$ ouvert pour  $N_2$
   \end{enumerate}

   \begin{enumerate}
       \item $\iff$ $N_2(Id \, u) (= N_2(u)) \le c_1N_1(u)$
       \item $\iff$ $N_1(Id \, u) (= N_1(u)) \le c_2N_2(u)$
   \end{enumerate}
   car $Id$ continue et linéaire, donc bornée  $\exists c$ tq $\underbrace{Id \, u}_{\in E_2} \le c \underbrace{u}_{E_1}$ donc $N_2(Id \, u) \le c N_1(u)$
   \par
   $(1)$ et  $(2)$  $\iff$ $N_1$ et $N_2$ équivalentes.
\end{preuve}

\section{La norme des matrices}
$A \in \mathcal{M}_n(\mathbb{C})$ identifié à $A \in L(\mathbb{C}^n)$
\[
    \left( (Ax)_{i} = \sum_{j=1}^{n} a_{i, j}x_j \right) \quad x = (x_1, \ldots, x_n) \in \mathbb{C}^n
\] 
\begin{itemize}
    \item $(x|y) = \sum_{i=1}^{n} \overline{x_i}y_i$
    \item $\|x\| = (x|x)^{\frac{1}{2}} = \left( \sum_{i=1}^{n} |x_i|^2 \right)^{\frac{1}{2}}$
\end{itemize}
Matrice adjointe $A^*$  $(A^*)_{i,j} = \overline{(A)_{j,i}}$
 \[
     (x|Ay) = (A^*x|y) \quad \forall x,y
 \] 
 \subsection{\underline{Bonne norme} sur $L(\mathbb{C}^n)$ (ou sur $\mathcal{M}_n(\mathbb{C})$ )}
 $\|A\|$ norme uniforme sur  $L(\mathbb{C}^n)$ ($= B(\mathbb{C}^n)$) obtenue à partir de $\| \cdot \|_2$
  \begin{lemma}
    \begin{align*}
        \|A\| = \|A^*\| = \|A^*A\|^{\frac{1}{2}}
    \end{align*} 
 \end{lemma}
 \begin{preuve}
    $\|x\|_2 = \sup_{\|y\|_2 \le 1} |(y|x)|$. Donc: 
    \[
        \|A\| = \sup_{\|x\|_2 \le 1} \|Ax\|_2 = \sup_{\|x\|_{2} \le 1, \|y\|_{2} \le 1} |(y|Ax)| 
    \] 
    \[
        (y|Ax) = (A^*y|x) = \overline{(x|A^*y)}
    \] 
    donc $|(y|Ax)| = |(x|A^*y)|$
     \[
        \|A\| = \sup_{\|x\|\le 1, \|y\| \le 1} |(x|A^*y)| = \|A^*\|
    \] 
    \[
        \|A^*A\| \le \|A^*\| \|A\| = \|A\|^2 = \sup_{\|x\| \le 1} \|Ax\|^2
    \] 
    \begin{align*}
        \|Ax\|^2 = (Ax|Ax) = (x|A^*Ax) &\le \|x\| \|A^*Ax\| \text{ (Cauchy-Schwarz)}\\
                                       &\le  \|x\| \|A^*A\| \|x\| = \|A^*A\| \|x\|^2
    \end{align*}
    \[
        \|Ax\|^2 \le \|A^*A\| \|x\|^2
    \] 
    \[
        \|Ax\|_2 \le \|A^*A\|^{\frac{1}{2}}\|x\|_{2} \implies \|A\|^2 \le \|A^*A\|^{\frac{1}{2}}
    \] 
    \[
        \|A\| = \|A^*A\|^{\frac{1}{2}}
    \] 
 \end{preuve}

 \subsection{Comment "calculer" $\|A\|$?}
 \begin{theorem}
     $\|A\| = \max_{1 \le i \le n} \mu_i$ avec $\mu_i = \lambda_i^{\frac{1}{2}}$ où $\lambda_1, \ldots, \lambda_n \in \R^+$ valeurs propres de $A^*A$.
 \end{theorem}
 \begin{preuve}
    \[
        \|A\| = \|A^*A\|^{\frac{1}{2}}
    \]  
    À montrer: $\|A^*A\| = \max_{1 \le i \le n} \lambda_i$ ($\lambda_i \ge 0$)
    \[
        (AB)^* = B^*A^*
    \] 
    \[
        (A^*A)^* = A^*A^{* *} = A^*A
    \] 
    Soit $B = A^*A$,  $B = B^*$ et  $(x|Bx) = (x|A^*Ax) = (Ax|Ax) = \|Ax\|^2 \ge 0$. Donc:
    \[
    \forall x, \quad (x|Bx) \ge 0
    \] 
    Il existe une b.o.n $(u_1, \ldots, u_n)$ de $\mathbb{C}^n$ et  $\lambda_1, \ldots, \lambda_n \in \R$ tels que 
    \[
    Bu_i = \lambda_iu_i \quad 1\le i \le n
    \] 
    \[
        \lambda_i = (u_i|\lambda_iu_i) = (u_i|Bu_i) \ge 0 
    \] 
    Si $u = \sum_{i=1}^{n} x_iu_i$ $\|u\|^2 = \sum_{i=1}^{n} |x_i|^2$ 
    \[
    Bu = \sum_{i=1}^{n} x_iBu_i = \sum_{i=1}^{n} \lambda_ix_iu_i
    \] 
    \[
    \|Bu\|^2 = \sum_{i=1}^{n} \lambda_i^2|x_i|^2 \le \max \lambda_i^2 \cdot \sum_{i=1}^{n} |x_i|^2 = \max \lambda_i^2 \|u\|^2
    \] 
    \[
    \|B\| \le \max_{1 \le i \le n} \lambda_i
    \] 
    Si $\lambda_1 = \max_{1 \le i \le n} \lambda_i$
    \[
    \|Be_1\| = \|\lambda_1e_1\| = \lambda_1\|e_1\| \le \|B\|\|e_1\|
    \] 
    donc $\|B\| \ge \lambda_1$
 \end{preuve}
 \subsection{Comment majorer $\|A\|$}
 \begin{prop}
    On a: $\|A\| \le \|A\|_{HS}$ où
    \[
    \|A\|_{HS}^2 = \sum_{1\le i,j\le n}^{} |a_{i,j}|^2
    \] 
 \end{prop}
 \begin{preuve}
     
    \[
        \mathcal{M}_n(\mathbb{C}) \sim \mathbb{C}^{n \times n}
    \] 
    $\| \cdot \|_{HS}$ norme canonique sur $\mathbb{C}^{n \times n}$ !
    \[
        (Ax)_{i} = \sum_{i=1}^{n} a_{i, j}x_j
    \] 
    \[
        (y|Ax) = \sum_{i=1}^{n} \overline{y_i}\sum_{j=1}^{n} a_{i,j}x_j = \sum_{1 \le i,j \le n}^{} a_{i, j}\overline{y_i}x_{j}
    \] 
    Soit $b_{i, j} = y_i \overline{x_j}$ 
    \[
        (y|Ax) = \sum_{i,j}^{} \overline{b_{i,j}}a_{i,j}
    \] 
    \[
        |(y|Ax)| \le \left( \sum_{i,j}^{} |a_{i,j}|^2 \right)^{\frac{1}{2}} \times \left( \sum_{i,j}^{} |b_{i,j}|^2 \right)^{\frac{1}{2}}
    \] 
    \[
        \left( \sum_{i,j}^{} |b_{i,j}|^2 \right)^{\frac{1}{2}} = \left( \sum_{1\le i,j \le n}^{} |y_i|^2 |x_i|^2 \right)^{\frac{1}{2}} = \left( \sum_{1\le i \le n}^{} |y_i|^2 \right)^{\frac{1}{2}} \times \left( \sum_{1 \le j \le n}^{} |x_j|^2 \right)^{\frac{1}{2}} = \|y\| \|x\|
    \] 
    \[
    \left| (y|Ax) \right| \le \|A\|_{HS} \|x\| \|y\| \implies \|A\| \le \|A\|_{HS}
    \] 
 \end{preuve}
