\begin{definition}
    Une suite $(x_n)_{x \in \N}$ est de Cauchy si  $\forall \epsilon > 0, \exists N \in \N$ tel que:
    \[
    \forall n, p \ge N, \, d(x_n, x_p) \le \epsilon
    \] 
\end{definition}
\begin{intuition}
   Une suite de Cauchy c'est comme on mesure un point et on le localise, i.e:
   \begin{enumerate}
       \item On dit qu'il est entre $0$ et  $1$.
       \item Ensuite, on precise plus et on dit qu'il est entre  $0.5$ et  $0.6$.
       \item Puis, entre  $0.55$ et  $0.56$
   \end{enumerate}
   On peut infiniment augmenter le niveau de précision. C'est ça l'idée d'une suite de Cauchy.
\end{intuition}
\begin{definition}
    Un éspace métrique $(E, d)$ est \textbf{complet} si toute suite  $(x_n)_{n \in \N}$ d'éléments de  $E$ converge vers une limite  $x$ qui appartient aussi à  $E$.
\end{definition}
\begin{eg}
    Un éspace métrique $(]0, 1], d)$ avec $d$ une distance euclidienne n'est pas complet, car  soit une suite: $x_n = \frac{1}{n}$ dont la limite est $0$. Par contre,  $0 \not\in ]0, 1]$. Donc cet éspace n'est pas complet. 
\end{eg}
\begin{figure}[h]
   \centering 
   \begin{tikzpicture}
       \draw[->] (-1, 0) -- (2, 0); 
       \node[below] (_) at (2,0){$x$};

       \node (_) at (0,0){]};
       \node[below] (_) at (0,-0.3){$0$};
       \node (_) at (1,0){]};
       \node[below] (_) at (1,-0.3){$1$};
       \draw[color=red] (0,0)--(1,0);
   \end{tikzpicture}
   \caption{$(]0, 1], d)$ n'est pas complet}
\end{figure}
\begin{eg}
   Un éspace $(\Q, d)$ n'est pas complet. Car on peut prendre une suite  $x_n$ tendant vers  $\sqrt{2} \not\in \Q$.
\end{eg}

\begin{figure}[H]
    \centering
    \incfig{q_not_complete}
    \caption{$\Q$ pas complet}
    \label{fig:q_not_complete}
\end{figure}

\begin{definition}
    Soit une suite $(x_n)_{n \in \N}$ et une application  $\phi:\N \to \N$ \underline{strictement croissante}. Une suite $(x_n)_{\phi(n)}$ est appellée une sous-suite.
\end{definition}
\begin{eg}
    Soit une application $\phi: \N \to \N$ telle que $\phi(n) = 2n$. Donc  $(x_n)_{\phi(n)}$ est une sous-suite de  $(x_n)_{n \in \N}$ et:
    \[
        (x_n)_{\phi(n)} = \{x_0, x_2, x_4, \ldots\}
    \] 
\end{eg}

\section{Compacité}
\begin{definition}
    Soit $F \subset E$. Un \textbf{recouvrement ouvert} de $F$, est une union des enesembles ouverts:  $\bigcup_{i \in I} U_i$ tel que $F \subset \bigcup_{i \in I} U_i$
\end{definition}
\begin{eg}
    Soit $F = ]0, 1[$. Soit $A = \left\{]\frac{1}{n}, 1 + \frac{1}{n}[, n \in N\right\}$. $F \subset \bigcup_{n \in N^{*}} A_n$ i.e union infinie des $A_i$ couvre $F$.
\end{eg}
\begin{definition}
    Un ensemble $F \subset E$ est \textbf{compact} si \underline{pour tout} recouvrement ouvert, i.e \underline{pour tout} union des ensembles ouvert $\bigcup_{i \in I} U_i$ qui couvre $F$, on peut prendre un nombre \underline{fini} des  $U_i$ et couvrir $F$.
\end{definition}
\begin{theorem}
    Un ensemble $K \subset E$ est compact, si toute suite $(x_n)_{n \in \N}$ des éléments de $K$, possede une sous-suite qui converge  vers un éléments $x \in K$.
\end{theorem}
\begin{intuition}
    S'il existe tel suite $(x_n)_{n \in \N}$ sans sous-suite convergente vers un éléments de  $K$, donc les valeurs sont en-dehors de  $K$ et donc il existe un ensemble qui couvre $K$  seulement avec un nombre infini des ensembles. 
\end{intuition}
Pourquoi a-t-on besoin de compacité? Car cela nous donne une
\begin{prop}
    Si $K \subset E$ est compact, alors $K$ est fermé et borné.\\
    Si $K$ est compact et  $F$ est borné, donc  $K \cap F$ est compact\\
    Si $K$ est compact, donc  $K$ est complet
\end{prop}
\begin{property}
    La différence entre \textit{compacité} et {complecité}:
    \begin{itemize}
        \item complecité nous assure qu'il n'y a pas de trou dans un espace
        \item compacité nous assure qu'un ensemble est fermé et borné
    \end{itemize}
\end{property}
\section{Limites et applications continues}

