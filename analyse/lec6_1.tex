
\chapter{Dérivation des fonctions de plusieurs variables}
\section{Introduction}
$n = 1$: comment définir  $f'(x_0)$?
\begin{enumerate}
    \item $f'(x_0) = \lim_{x \to x_0} \frac{f(x) - f(x_0)}{x - x_0}$
    \item DL: $f(x) = f(x_0) + a_1(x - x_0) + (x - x_0)\epsilon(x)$ où $a_1 = f'(x_0)$
\end{enumerate}

\[
f: D \to R \quad D \text{ ouvert } \quad X_0 \in D \quad D \subset \R^n
\] 

\begin{definition}
    $f$ est dérivable en  $X_0$ dans la direction $\vec{u}$ $(\neq \vec{0})$ si la fonction
    \begin{align*}
        g: \R &\longrightarrow \R \\
        t &\longmapsto g(t) = f(X_0 + t\vec{u})
    .\end{align*}
    est dérivable en $t = 0$
\end{definition}
Autrement dire, la dérivée directionnelle (dans la direction de vecteur $\vec{u}$) est donnée par:
\begin{equation}\label{eq:derivee-directionnelle}
    D_{u}f(X_0) = \lim_{t \to 0} \frac{f(X_0 + t\vec{u}) - f(X_0)}{t}
\end{equation}
Dans le cas $\R$ on a eu la définition de la dérivée:
\[
f'(x_0) = \lim_{t \to 0} \frac{f(x_0 + t) - f(x_0)}{t} 
\] 
La diréction était toujours la même (l'axe $x$), on peut voir ça comme prendre un vecteur $u = (1)$ et utiliser comme la direction seulement l'axe $x$ et on obtient  l'eq. (~\ref{eq:derivee-directionnelle})
\begin{figure}[H]
    \centering
    \incfig{continuite-multidimensionelle}
    \caption{Dérivée directionnelle}
    \label{fig:continuite-multidimensionelle}
\end{figure}
$\vec{e_1}, \ldots, \vec{e_n}$ base canonique de $\R^n$, $f$ admet des dérivées partielles en  $X_0$ si $f$ dérivable en  $X_0$ dans les directions $\vec{e_1}, \ldots, \vec{e_n}$.
\[
    \frac{d}{dt} f(X_0 + t \vec{e_i}) \mid_{t = 0}
\] 
noté
\[
\frac{\partial f}{\partial x_i}(X_0)
\] 
Par contre, une fonction peut être dérivable dans \underline{toutes les diréctions} en un point mais \underline{ne pas être} continue en ce point, voici 

\begin{eg}
   \[
   f(x_1, x_2) = \begin{cases}
       1  \text{ si } x_2 = x_1^2 \text{ et } (x_1, x_2) \neq (0, 0)\\
       0 \text{ sinon}
   \end{cases}
   \]  
\begin{figure}[H]
    \centering
    \incfig{exemple-dervie-partielle}
    \caption{Exemple dérivable mais pas continue}
    \label{fig:exemple-dervie-partielle}
\end{figure}
\[
    f((0, 0) + t \vec{u}) = f(t \vec{u}) = 0
\] 
si $t\neq 0$ et $t$ petit, on a $f$ dérivable dans toutes les directions.
\par
Mais, $f$ n'est pas continue en  $(0, 0)$:
 \[
X_n = (\frac{1}{n}, \frac{1}{n^2}) \quad X_n \to (0, 0)
\] 
\[
\forall n, f(X_n) = 1 \quad f(X_n) \not\xrightarrow[n \to \infty]{} f(0, 0)
\] 
\end{eg}

\begin{definition}\label{def:fonction-differentiable}
    Soient $D \subset \R^n$ ouvert et $X_0 \in D$, la fonction
    $f: D \to \R$
    est \textbf{différentiable} en $X_0$ s'il existe un vecteur $\vec{u} \in \R^n$ tel que 
    \[
        f(X_0 + \vec{X}) = f(X_0) + \vec{u} \cdot \vec{X} + \|\vec{X}\|\epsilon(\vec{X})
    \] 
    où $\lim_{\vec{X} \to \vec{0}} \epsilon(\vec{X}) = 0$
\end{definition}
\begin{intuition}
   Je propose de réflechir sur ce que cette définition signifie. 
   Rappelons ce que signifie intuitivement la dérivée au cas $\R^n = \R$ ($n = 1$). 
   Intuitivement, si on zoom la fonction qu'on dérive elle se comporte et a l'air d'être une ligne. 
   Dans le cas $R^n = \R^2$, si on zoom la fonction elle a l'air d'être un plan. 
   En effet, c'est ça l'idée de la dérivée, que si on fait un petit petit pas d'un fourmit, le deplacement et aussi petit et uniforme. 
   En augmentant $n$, la dérivée donne des scalaire pour contruire un sous-éspace de dimension $n-1$ de l'espace $\R^n$. 
\end{intuition}

\section{DL à l'ordre 1}
Cette représentation de la dérivée comme un sous-éspace lorsqu'on zoom est représenté par le DL à l'ordre $1$.
De la définition ~\ref{def:fonction-differentiable}, ce vecteur $\vec{u}$ se note $\vec{\nabla}f(X_0)$ (gradient de $f$ en  $X_0$)

\begin{prop}
   $f$ différentiable en  $X_0$ $\implies$ $f$ dérivable dans toutes les directions en  $X_0$, et alors: 
   \[
       \vec{\nabla}f(X_0) = \begin{pmatrix} \frac{\partial f}{\partial x_1}f(X_0)\\ \ldots \\  \frac{\partial f}{\partial x_n}f(X_0)\end{pmatrix} 
   \] 
   dans la base $\vec{e_1}, \ldots, \vec{e_n}$
\end{prop}

\begin{preuve}
$f$ est continue en  $X_0$ $|\vec{u} \cdot X| \le |\vec{u}| |X|$
\begin{enumerate}
    \item continuité
        \begin{align*}
            |f(X_0 + X) - f(X_0)| &\le |\vec{u} \cdot X| + \|X\| |\epsilon(X)|\\
                                  &\le \|X\|\left( \|\vec{u}\| + |\epsilon(x)|  \right) \le c\|X\|
        \end{align*}
        donc: $f(X_0 + X) \xrightarrow[X \to \vec{0}]{} f(X_0)$
    \item .
        \begin{align*}
            g(t) = f(X_0 + t \vec{v}) &= f(X_0) + \vec{\nabla}f(X_0) \cdot t \vec{v} + \|t \vec{v}\| \cdot \epsilon(t \vec{v})\\
                                      &= f(X_0) + t\vec{\nabla}f(X_0) \cdot \vec{v} + |t|\|\vec{v}\|\epsilon_1(t)\\
                                      &= f(X_0) + t\vec{\nabla }f(X_0)\cdot \vec{v}
        \end{align*}
        donc:
        \[
            \frac{d}{dt} f(X_0 + t\vec{v})\mid_{t = 0} = \vec{\nabla}f(X_0)\cdot \vec{v}
        \] 
        (prendre $\vec{v} = \vec{e_1}, \ldots, \vec{e_n}$ pour les coordonnées de $\vec{\nabla}f(X_0)$)
\end{enumerate}
\end{preuve}
\begin{definition}
    \[
        D \subset \R^n \quad D \text{ ouvert } \quad f: D \to \R \text{ est } \mathcal{C}^1 \text{ sur } D
    \] 
    Soit $D \subset \R^n$ ouvert, alors la fonction $f: D \to \R$ est de classe $\mathcal{C}^1$ sur  $D$ si 
    $f$ est différentiable en tout  $X \in D$ et la fonction
    \begin{align*}
        : D &\longrightarrow \R^n \\
        X &\longmapsto \vec{\nabla}f(X)
    \end{align*}
    est continue.
\end{definition}
\begin{theorem}
    $f$ de classe  $\mathcal{C}^1$ sur  $D$ ssi  $f$ admet des dérivées partielles continues en tout point de  $D$. 
\end{theorem}

\begin{eg}
   \[
       f(X) = f(X_0) + \vec{\nabla}f(X_0) \cdot (X - X_0) + \|X - X_0\| \epsilon(X - X_0)
   \]  
   linéaire
   \par
   Dans $\R^3$: $f(x, y, z)$
    \[
        S = \{(x, y, z): f(x, y, z) = 0 \}
   \] 
   $S$: surface dans  $\R^3$, $X_0 \in S$ plan tangent à $S$ en  $X_0$, plan d'équation:
   \[
       f(X_0) + \vec{\nabla}f(X_0)\cdot X = 0
   \] 
\begin{figure}[H]
    \centering
    \incfig{surface-s-differentiable}
    \caption{Exemple d'une surface differentiable}
    \label{fig:surface-s-differentiable}
\end{figure}
\end{eg}
