\section{Compléments sur les espaces vectoriels normés}
\subsection{Suites de fonctions}
$X$ ensemble ($X \subset \R$), $f_n: X \to \R(\mathbb{C})$ et $(f_n)_{n \in \N}$.
\subsection{Convergence simple: }
\begin{definition}
    $(f_n)_{n \in \N}$ converge simplement vers $f$ si  $\forall x_0 \in X$, $f_n(x_0) \xrightarrow[n \to \infty]{} f(x_0)$ (ne provient pas d'une norme).
\end{definition}
\subsection{Convergence uniforme: }
\begin{definition}
    $f \in B(X, \R)$ si $\sup_{x \in X} |f(x)| = \|f\|_{\infty} < \infty$ ($f$ bornée sur  $X$). \\
    Convergence uniforme: $\forall \varepsilon > 0, \exists N \in \N$ tq $\forall n \ge N \forall x \in X |f_n(x) - f(x)| < \varepsilon$ équivalent à
    \[
    \forall \varepsilon > 0 \exists N \in \N \text{ tq } \forall n \ge N, \|f_n - f\|_{\infty} < \varepsilon
    \] 
    $f_n \to f$ dans $(B(X, \R), \| \cdot \|_{\infty})$
\end{definition}
\begin{definition}
    Limite uniforme de fonctions continues: $X = [a, b]$,  $\mathcal{C}([a, b], \R) \subset B([a, b], \R)$(sous espaces vectoriels). $\mathcal{C}([a, b], \R)$ est fermé dans $(B([a, b], \R), \| \cdot \|_{\infty})$
\end{definition}
\subsection{Séries à valeurs dans un espace vectoriel normé.}
\begin{definition}
     Soient $(E, \| \cdot \|_{\infty})$ e.v.n\footnote{espace vectoriel normé}, $(u_n)_{n \in N}$ suite dans $E$. La série  $\sum u_n$ converge dans  $(E, \| \cdot \|)$ si la suite $S_N = \sum_{n=0}^{N} u_n$ converge dans $(E, \| \cdot \|)$. $\lim_{N \to \infty} S_N$ notée $\sum_{n=0}^{\infty} u_n (\in E)$
\end{definition}
\begin{remark}
   Si $\sum u_n$ et  $\sum v_n$ convergent, alors  
   \begin{itemize}
       \item $\sum u_n + v_n$ converge et  $\sum \lambda u_n$ converge
       \item 
           $\sum_{n=0}^{\infty} u_n u_n + v_n = \sum_{n=0}^{\infty} u_n + \sum_{n=0}^{\infty} v_n$ 
       \item $\sum_{n=0}^{\infty} \lambda u_n = \lambda \sum_{n=0}^{\infty} u_n$
   \end{itemize}
\end{remark}

\subsection{Convergence normale}
\begin{definition}
   $\sum u_n$ converge normalement dans  $(E, \| \cdot \|)$ si  $\sum \|u_n\|$ converge dans  $\R$. 
\end{definition}
\begin{eg}
   $E = \R$, $\|x\| = |x|$. cv. normale = cv. absolue ($\sum u_n$ converge) 
\end{eg}
\begin{eg}
   $\sum u_n$ peut converger sans converger normalement, comme:  $u_n = \frac{(-1)^n}{n}$ 
\end{eg}

\begin{theorem}
    Si $(E, \| \cdot \|)$ est complet, toute série normalement convergente est convergente et 
     \[
    \|\sum_{n=0}^{\infty} u_n\| \le \sum_{n=0}^{\infty} \|u_n\|
    \] 
\end{theorem}
\begin{preuve}
   $S_n = \sum_{k=0}^{n} u_k$ et $T_n = \sum_{k=0}^{n} \|u_k\|$ 
   \begin{align*}
       n > p \quad \|S_n - S_p\| = \|\sum_{k = p+1}^{n} u_k\| \le \sum_{k=p+1}^{n} \|u_k\| = T_n - T_p = |T_n - T_p|
   \end{align*}
    $(T_n)$ converge dans  $\R$, donc $(T_n)$ de Cauchy:
     \[
    \forall \varepsilon > 0, \exists N \text{ tq } \forall n > p \ge N |T_n - T_p| \le \varepsilon
    \] 
    donc $(S_n)$ de Cauchy dans  $(E, \| \cdot \|)$.  \underline{$E$ complet:}  $(S_n)$ converge vers  $S \in E$.
\end{preuve}
\section{Applications linéaires continues}
Pour toute section $B_E$ désigne une boule \underline{fermé}!
\par
Soient $E, F$ 2 espaces vectoriels normés avec  $\| \cdot \|_{E}$ et $\| \cdot \|_{F}$ les normes associés, 
\begin{itemize}
    \item $A \in \mathcal{L}(E, F)$
    \item $\lambda A \in \mathcal{L}(E, F)$ et $\lambda Ax = \lambda(Ax)$
    \item  $A + B \in \mathcal{L}(E, F)$ et $(A + B)x = Ax + Bx$
    \item  $0x = 0_F$  $\forall x \in E$
\end{itemize}
\[
    \mathcal{L}(E) = \mathcal{L}(E, E)
\] 
\begin{itemize}
    \item $(AB)x = A(Bx)$ où  $AB = A \circ B$
    \item  $(\lambda A)B = \lambda (AB)$
    \item  $A(B + C) = AB + AC$
    \item  $(A + B)C = AC + BC$
    \item  $0A = 0$ 
    \item $AB \neq BA$ (en général)
    \item $A(BC) = (AB)C$
\end{itemize}

\begin{theorem}
    Soit $A \in \mathcal{L}(E, F)$. Les propriétés suivantes sont équivalentes:
    \begin{enumerate}
        \item $A: E \to F$ est continue
        \item $A$ est continue en  $0_E$
        \item  $\exists C \ge 0$ telle que 
            \[
            \|Ax\|_F \le C\|x\|_E \quad \forall x \in E
            \] 
            cela s'appelle que $A$ est bornée
        \item $A$ est bornée sur  $B_E(0, R)$ $\forall R > 0$
    \end{enumerate}
    On dit que $A$ est bornée (si $A$ est continue et linéaire)
\end{theorem}
\begin{preuve}
    \begin{itemize}
        \item $1) \implies 2)$ : évident
        \item $2) \implies 3)$ : 
            \begin{itemize}
                \item 
                    \underline{Hyp:} $\forall \varepsilon >0, \exists \delta > 0$ tq $\|x - 0_E\|_E \le \delta \implies \|Ax - A0_E\|_F \le \varepsilon$ $\|x\|_E \le \delta \implies \|Ax\|_F \le \varepsilon$
                \item $\varepsilon = 1 \exists \delta > 0$ tq $\|x\|_E \le \delta \implies \|Ax\|_F \le 1$
                \item Soit $ x \in E$ et $x \neq 0_E$
                \item $y = \frac{\delta}{\|x\|_E}x$ donc $\|y\|_E = \delta$  $\implies\|Ay\|_F \le 1$
                \item $Ay = \frac{\delta}{\|x\|_{E}}Ax$ et $A$ linéaire
                \item  $\|Ay\|_{F} = \frac{\delta}{\|x\|_E}\|Ax\|F \le 1 \implies \|Ax\|_F \le \frac{1}{\delta}\|x\|_E$
            \end{itemize}
        \item $3) \implies 1)$ 
            \begin{itemize}
                \item Je fixe $x_0 \in E$. à voir: $A$ continue en  $x_0$?
                \item $\|Ax - Ax_0\|_F = \|A(x - x_0)\|_F \le C\|x - x_0\|_E$
                \item Donc si $\|x - x_0\|_E \le \frac{\varepsilon}{c} = \delta(\varepsilon)$, $\|Ax - Ax_0\|_F \le \varepsilon$
            \end{itemize}
    \end{itemize}
\end{preuve}

\begin{notation}
   \[
       B(E, F) = \{ A \in \mathcal{L}(E, F): A \text{ continue } \}
   \]  
   \[
   B(E, E) = B(E)
   \] 
\end{notation}
\begin{lemma}
   Si $E$ est de dimension finie, alors
   \[
       \mathcal{L}(E, F) = B(E, F)
   \] 
   C'est faux si $\dim E = \infty$
\end{lemma}
\begin{preuve}
    $(e_1, \ldots, e_n)$ base de $E$. Sur  $E$ toutes les normes sont équivalentes.  
    \begin{itemize}
        \item $\|x\|_E$ norme donnée.        
        \item $\|x\|_{\infty} = \max_{1 \le i \le n} |x_i|$ 
    \end{itemize}
    où $x = \sum_{i=1}^{n} x_ie_i$ 
    \[
    \|Ax\|_F = \|\sum_{i=1}^{n} x_iAe_i\| = \sum_{i=1}^{n} |x_i|\|Ae_i\|_F
    \] 
    \[
    \|Ax\|_F \le \|x\|_{\infty} \times \sum_{i=1}^{n} \|Ae_i\|_F = C\|x\|_{\infty}
    \] 
    ($\|x\|_{\infty}\| \le C'\|x\|_{E}$). Donc: $\|Ax\|_{F} \le CC'\|x\|_E$. Alors: $A \in B(E, F)$
\end{preuve}

\subsection{Norme sur $B(E, F)$}
\begin{theorem}
    Soit $A \in B(E, F)$, on pose $\displaystyle \|A\| = \sup_{x \in E, \|x\|_E \le 1} \|Ax\|_F = \sup_{x \in B_E(0, 1)} \|Ax\|_{F}$

    \begin{enumerate}
        \item $\| \cdot \|$ est une norme sur  $B(E, F)$ appelée norme uniforme.
        \item On a:  $\|Ax\|_F \le \|A\|\|x\|_{E} \quad \forall x \in E$
        \item $\|A\| = $ la plus petite constante  $C$ telle que  $\|Ax\|_{F} \le C\|x\|_{E} \quad \forall x \in E$
    \end{enumerate}
\end{theorem}
\begin{remark}
    \begin{enumerate}
        \item On peut écrire $\|A\|_{B(E, F)}$ au lieu de $\|A\|$ 
        \item Parfois on trouve $|||A|||$ pour $\|A\|$ 
        \item Soit $I^+ = $ ensemble des  $C \ge 0$ telle que $\|Ax\|_{F} \le C\|x\|_{E} \quad \forall x \in E$. \\
            $I^+ \neq \O$ (car $A \in B(E, F)$) et $I^+ \subset [0, +\infty[$. $(2)$ et  $(3)$ disent que  $\|A\|$ est le plus petit élément de  $I^+$
             \[
            \inf I^+ = \min I^+ = \|A\|
            \] 
    \end{enumerate}
\end{remark}
\begin{preuve}
   \begin{enumerate}
       \item $A \in B(E, F) \iff \sup_{x \in B_{E}(0, 1)} \|Ax\|_{F} < \infty \iff \|A\|$ bien définie.  
           \begin{align*}
               \|(A + B)x\|_F = \|Ax + Bx\|_F \le \|Ax\|_F + \|Bx\|_F\\ \implies \sup_{x \in B_E(0, 1)} \|(A + B)x\|_F \le \sup_{x \in B_E(0, 1)} \|Ax\|_F + \sup_{x \in B_E(0, 1)} \|Bx\|_F
           \end{align*}
           \[
               \|A + B\| \le \|A\| + \|B\| \text{ et } A, B \in B(E, F) \implies A + B \text{ aussi }
           \] 
           \[
           \|\lambda A\| = |\lambda|\|A\| \text{ et } A \in B(E, F) \implies \lambda A \text{ aussi }
           \] 
           Si $\|A\| = 0$, alors  $\|Ax\|_F = 0 \forall x \in B_E(0, 1) \implies Ax = 0_F \forall x \in B_E(0, 1)$
           \[
           Ax = \|x\|_E A \frac{x}{\|x\|_E}
           \] 
           \[
           Ax = 0_F \forall x \in E \implies A = 0_{L(E, F)}
           \] 
           \[
           C \in I^+ \text{ si } \|Ax\|_F \le C\|x\|_E \quad \forall x \in E
           \] 
           \[
           \|A\| \in I^+ \implies \|Ax\|_F \le \|A\|\|x\|_E \forall x
           \] 
           \begin{itemize}
               \item Clair si $x = 0_E$. 
               \item Si  $x \neq 0_E$, $y = \frac{x}{\|x\|_E} \in B_E(0, 1)$ donc 
                   \[
                       \|Ay\|_F = \frac{1}{\|x\|_E}\|Ax\|_F \le \|A\| \implies \|Ax\|_F \le \|A\|\|x\|_E
                   \] 
                   Soit $C \in I^+$ donc  
                   \[
                   \|Ax\|_F \le C\|x\|_E
                   \] 
                   donc $\|Ax\|_F \le C \quad \forall x \in B_E(0,1)$, donc $\|A\| \le C$, alors 
                   \[
                       \|A\| = \min I^+ = \text{ "meilleure constante $C$"}
                   \] 
           \end{itemize}
   \end{enumerate} 
\end{preuve}
\begin{eg}
    $E = \mathcal{C}([a, b], \R)$, $\|f\|_{\infty} = \sup_{x \in [a, b]} |f(x)|$, $F = \R$, $u \in \mathcal{C}([a, b], \R)$
    \begin{align*}
        A: E &\longrightarrow F \\
        f &\longmapsto A(f) = \int_{{a}}^{{b}} {f(x)u(x)} \: d{x} {}
    .\end{align*}
    \underline{$A$ est bornée}: à voir: $\exists C \ge 0$ telle que  
    \[
        \left| \int_{{a}}^{{b}} {f(x)u(x)} \: d{x} {} \right| \le C \sup_{x \in [a, b]} |f(x)|
    \] 
    ?
    \begin{align*}
        \left| \int_{{a}}^{{b}} {f(x)u(x)} \: d{x} {} \right| \le \int_{{a}}^{{b}} {|f(x)| |u(x)|} \: d{x} {} \le \int_{{a}}^{{b}} {\|f\|_{\infty}|u(x)|} \: d{x} {= \|f\|_{\infty} \int_{{a}}^{{b}} {|u(x)|} \: d{x} {}}
    \end{align*}
    \[
    C = \int_{{a}}^{{b}} {|u(x)|} \: d{x} \text{ convient }
    \] 
    (En fait $\|A\| = \int_{{a}}^{{b}} {|u(x)|} \: d{x} {}$). $E = \mathcal{C}^1([0,1], \R)$ muni de $\|f\|_{\infty}$, $F = \R$, $Af = f'(0)$ linéaire mais pas continue. On construit une suite  $(f_n)$ dans  $E$ telle que  $\|f_n\|_E \xrightarrow[n \to  \infty]{} 0$ mais $\|Af_n\|_F \not\to 0$
    \[
    f_n(x) = \frac{1}{n}\sin(nx)
    \] 
\end{eg}
