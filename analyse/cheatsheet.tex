\documentclass[a4paper]{article}

\usepackage[utf8]{inputenc}
\usepackage[T1]{fontenc}
\usepackage{textcomp}
\usepackage[english]{babel}
\usepackage{amsmath, amssymb, amsthm}


% figure support
\usepackage{import}
\usepackage{xifthen}
\pdfminorversion=7
\usepackage{pdfpages}
\usepackage{transparent}
\usepackage{hyperref}
\usepackage[margin=0.8in]{geometry}

\usepackage{setspace}
\setlength{\parindent}{0in}

\newcommand{\incfig}[1]{%
    \def\svgwidth{\columnwidth}
    \import{./figures/}{#1.pdf_tex}
}

\pdfsuppresswarningpagegroup=1

\newcommand{\N}{\mathbb{N}}
\newcommand{\R}{\mathbb{R}}
\newcommand{\Z}{\mathbb{Z}}
\newcommand{\Q}{\mathbb{Q}}

\newtheorem{theorem}{Théorème}[section]
\newtheorem{definition}{Définition}[section]
\newtheorem{exemple}{Exemple}[section]
\newtheorem{proposition}{Proposition}[section]
\newtheorem{propriete}{Propriété(s)}[section]
\newtheorem*{notation}{Notation}
\newtheorem*{remarque}{Remarque}

\title{CheatSheet d'Analyse et Géometrie}
\author{Yehor KOROTENKO}

\begin{document}
\maketitle
\section{Distances et normes}
   \begin{definition}
       Une \textbf{norme} sur $\R^d$ est une application $N: \R^d \to \R$ tel que:
       \begin{enumerate}
           \item $N(\lambda X) = |\lambda|N(X)$
           \item  $N(X + Y) \le N(X) + N(Y)$
           \item $N(X) \ge 0$ et $N(X) = 0 \iff X = 0_d$
       \end{enumerate}
   \end{definition} 
   \begin{definition}
       Une \textbf{distance} sur $\R^d$ est une application:  $d: \R^d \times \R^d \to \R$ tel que:
       \begin{enumerate}
           \item $d(X, Y) = d(Y, X)$
           \item  $d(X, Y) \le d(X, Z) + d(Z, Y)$
           \item $d(X, Y) \ge 0 \quad \forall X,Y$ et $d(X,Y) = 0 \iff X = Y$
       \end{enumerate}
   \end{definition}
   \begin{exemple} La distance et la norme canonique. Soit $X = (x_1, \ldots, x_n)$
      \begin{enumerate}
          \item La norme:
            \[
                \|X\| = \left( \sum_{i=1}^{n} x_i^2 \right)^{\frac{1}{2}}  = \sqrt{x_1^2 + \ldots + x_n^2} 
            \]               
        \item La distance: 
            \[
                d(X, Y) = \|X - Y\| = \sqrt{(x_1 - y_1)^2 + \ldots + (x_n - y_n)^2 } 
            \] 

      \end{enumerate} 
   \end{exemple}
\section{Éspaces métriques}
\begin{definition}
    Espace métrique est un ensemble muni de la distance.
\end{definition}
\begin{definition}
    $(E, d)$ espace métrique et $x_0 \in E$ et $r \ge 0$, alors:
    \begin{enumerate}
        \item $B(x_0, r) = \{ x \in E : d(x_0, x) < r \}$ boule ouverte de centre $x_0$ et de rayon $r$
        \item $B(x_0, r) = \{ x \in E : d(x_0, x) \le  r \}$ boule fermé de centre $x_0$ et de rayon $r$
    \end{enumerate}
\end{definition}
\begin{definition}
    Une partie $A \subset E$ est bornée si $\exists R > 0$ et $x_0 \in E$ tq $A \subset B(x_0, R)$. Autrement dit, s'il existe une boule dans laquelle $A$ est inclu.
\end{definition}
\section{Ouverts - fermés}
\begin{definition}
    Soit $(E, d)$ espace métrique.
     \begin{enumerate}
        \item $U \subset E$ est ouvert si $\forall x_0 \in U, \exists r > 0$ tel que $B(x_0, r) \subset U$ (si pour tout point de $U$ il existe une boule ouverte inclu dans  $U$)
        \item $F \subset E$ est fermé si $E \setminus F$ est ouvert (si le complémentaire de $F$ est ouvert).
    \end{enumerate}
\end{definition}

\begin{theorem} Très important!!
    \begin{enumerate}
        \item Soit $U_i$,  $i \in I$ une collection d'ouverts. Alors,  $\cup_{i \in I} \,U_i$ est ouvert.\\
            Translate: Une union quelconque des ensembles ouverts est ouvert.
        \item Si $U_1, \ldots, U_n$ sont ouverts
            \[
                \bigcap\limits_{i=1}^{n} \, U_i \text{ est ouvert.}
            \] 
            Translate: intersection \underline{finie} des ensembles ouverts est ouvert.
    \end{enumerate}
    \begin{enumerate}
        \item Soit $U_i$,  $i \in I$ une collection de fermés. Alors,  $\cup_{i \in I} \,U_i$ est fermé.\\
            Translate: Une union quelconque des ensembles fermés est fermé.
        \item Si $U_1, \ldots, U_n$ sont fermés 
            \[
                \bigcap\limits_{i=1}^{n} \, U_i \text{ est fermé.}
            \] 
            Translate: intersection \underline{finie} des ensembles fermés est fermé.
    \end{enumerate}
    Note: pour union quelconque, intersection finie!
\end{theorem}

\begin{proposition} Soient $N \in \N$ et $\forall 1 \le i \le N, U_i \subset E$
    \begin{enumerate}
        \item Si tout $U_i$ est ouvert, alors  $U_1 \times \ldots \times U_N$ est aussi ouvert.
        \item Si tout $U_i$ est fermé, alors  $U_1 \times \ldots \times U_N$ est aussi fermé.
    \end{enumerate}
\end{proposition}

\section{Intérieur, Adhérence, Frontière}
\begin{definition} 
    Un point $x_0 \in A$ est intérieur à $A$ s'il existe  $r > 0$ tq  $B(x_0, r) \subset A$. \textbf{Intérieur} de $A$ est l'ensemble de tous les points intérieurs à  $A$:
     \[
         \operatorname{Int}(A) = \{ x \in A : x \text{ est intérieur à } A \}
    \] 
    Note: le plus grand ouvert inclu dans $A$
\end{definition}
\begin{definition} 
    Un point $x_0 \in A$ est adhérent à $A$ s'il existe  $r > 0$ tq  $B(x_0, r)$ intersecte $A$. \textbf{Adhérence} de $A$ est l'ensemble de tous les points intérieurs à  $A$:
     \[
         \operatorname{Adh}(A) = \{ x \in E : x \text{ est adhérent à } A \}
    \] 
    Note: le plus petit fermé contenant $A$
\end{definition}
\begin{definition}
    La \textbf{frontière}: $\partial A = \operatorname{Adh}(A) \setminus \operatorname{Int}(A) = \operatorname{Adh}(A) \cap \operatorname{Adh}(E \setminus A)$
\end{definition}

\begin{definition}
    Soit $A \subset B$. On dit que $A$ est \textbf{dense} dans  $B$ si  $B \subset \operatorname{Adh}(A)$
\end{definition}
\begin{proposition} Caractérisations séquentielles:
   \begin{enumerate}
       \item \textbf{de férmé}: $A$ est fermé ssi pour toute suite  $(x_n)$ d'éléments de  $A$ qui covnverge (supposons vers $x$), sa limite $x$ est aussi dans  $A$ ($x \in A$)
       \item \textbf{d'adhérence}: $x \in \operatorname{Adh}(A)$ ssi il existe une suite  $(x_n)$ d'éléments de  $A$ tq  $\lim_{n \to \infty} x_n = x$
       \item \textbf{de compact}: $A \subset E$ est compact si pour toute suite $(x_n)$ d'éléments de  $A$, il existe une sous-suite  $(x_{\phi(n)})$ qui converge vers une limite  $x \in A$
   \end{enumerate} 
\end{proposition}
\begin{proposition}
   \begin{enumerate}
       \item Si une suite $(x_n)$ converge vers une limite  $x$, cette limite est \textbf{unique}.
       \item Si une suite  $(x_n)$ coverge vers une limite  $x$ et cette suite admet une sous-suite  $(x_{\phi(n)})$ qui converge vers une limite  $x'$. Alors on a toujours:  $x = x'$
   \end{enumerate} 
\end{proposition}
\section{Compact}
\begin{definition}
    Soient $U = \bigcup_{i \in I} U_i$ où $U_i$ est ouvert  $\forall i \in I$. Alors, si $A \subset U$, donc $U$ est un recouvrement ouvert de  $A$.
\end{definition}
\begin{definition}
    $K \subset E$ est \textbf{compact} si \underline{de TOUT recouvrement ouvert} de $K$ on peut extraire un sous-recouvrement ouvert fini: i.e, on peut trouver $i_1, \ldots, i_n \in I$ (un nombre fini de $i_i$) tels que:
    \[
        K \subset U_{i_1} \cup \ldots \cup U_{i_n}
    \] 
\end{definition}
\begin{proposition}
   \begin{enumerate}
       \item Un ensemble fini est compact.
       \item $K$ compact $\implies$ $K$ fermé et borné
       \item Si $K$ compact et $F$ fermé, alors  $K \cap F$ est compact.
       \item Si  $K$ compact, toute suite de Cauchy dans  $K$ coverge dans  $K$.
   \end{enumerate} 
\end{proposition}
\section{Utilisation des fonctions continues}
\begin{proposition}
   Soit $F: E_1 \to  E_2$ une fonction continue avec $E_1$, $E_2$ éspaces métriques, alors:
   \begin{enumerate}
       \item Pour tout $U_2 \subset E_2$ ouvert, $F^{-1}(U_2)$ est ouvert dans $E_1$.
       \item Pour tout $F_2 \subset E_2$ fermé, $F^{-1}(F_2)$ est fermé dans $E_1$. 
       \item Pour toute suite $(x_n)$ dans  $E_1$ ayant limite $x$ (i.e $\lim_{n \to \infty} x_n = x$) on a:
           \[
           \lim_{n \to \infty} F(x_n) = F(x)
           \] 
   \end{enumerate}
\end{proposition}

\section{Montrer qu'un ensemble et fermé ou ouvert}
\subsection*{Pour montrer qu'un ensemble \(\mathcal{U}\) est ouvert}

\begin{itemize}
    \item \textbf{Utiliser la définition :} 
    \[
    \forall x \in \mathcal{U}, \exists r > 0 \quad \text{tel que} \quad B(x, r) \subset \mathcal{U}
    \]
    
    \item Montrer que \( E \setminus \mathcal{U} \) est fermé.
    
    \item Montrer que \(\mathcal{U}\) est l'image réciproque d'un ouvert par une application continue.
    
    \item Exprimer \(\mathcal{U}\) comme une boule ouverte.
    
    \item Écrire \(\mathcal{U}\) comme :
    \begin{itemize}
        \item une réunion d'ouverts ;
        \item une intersection finie d'ouverts.
    \end{itemize}
    
    \item \(\mathcal{U} = \operatorname{Int}(U)\).
    
    \item Écrire \(\mathcal{U} = I_1 \times \dots \times I_n\) avec \( I_i \) ouvert.
\end{itemize}

\subsection*{Pour montrer qu'un ensemble \(V \subset E\) est fermé}

\begin{itemize}
    \item \textbf{Utiliser la définition :} \( E \setminus V \) est ouvert.
    
    \item \textbf{Caractérisation séquentielle :} Toute suite convergente dans \(V\), sa limite est aussi dans \(V\).
    
    \item Montrer que \(V\) est l'image réciproque d'un fermé par une application continue.
    
    \item Montrer que \(V\) est compact.
\end{itemize}
\end{document}
