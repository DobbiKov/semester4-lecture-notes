\chapter{Espaces vectoriels normés}
\section{Introduction}
\begin{definition}
    Soit $E$ un  $\mathbb{K}$-espace vectoriel et $\lambda \in \R$, la \textbf{norme} sur  $E$ est une application $N: E \to \R_{+}$ avec:
    \begin{enumerate}
        \item $N(\lambda u) = |\lambda|N(u) \quad u \in E$
        \item $N(u + v) \le N(u) + N(v)$
        \item $N(u) = 0 \iff u = 0_{E}$
    \end{enumerate}
    \textbf{semi-norme}: $1$ et  $2$ seulement.
\end{definition}

On peut interpreter $2$ comme:
 \[
\left| N(u) - N(v) \right| \le N(u - v)
\] 

\begin{prop}
    \textbf{Norme induite}: Si $F \subset E$ un sous-espace vectoriel, je restreins $N$ à  $F$, alors  $(F, N)$ est un espace vectoriel normé.
\end{prop}

\begin{eg}
    $E = \mathbb{K}^n$ avec  $x = (x_1, \ldots, x_n) \in E$ 
    \begin{itemize}
        \item $\|x\|_1 = \sum_{i=1}^{n} |x_i|$
        \item $\|x\|_2 = \left(\sum_{i=1}^{n} |x_i|^2\right)^{\frac{1}{2}}$
        \item $\|x\|_{\infty} = \max_{1 \le i \le n} |x_i|$
        \item $\|x\|_p = \left( \sum_{i=1}^{n} |x_i|^p \right)^{\frac{1}{p}}$ avec $1 \le p < \infty$
    \end{itemize}
\end{eg}

\begin{prop}
    L'inégalité triangulaire pour $p > 2$ s'appelle \textbf{l'inégalité de Minkowski}:
    \[
        \left( \sum_{i=1}^{n} |x_i + y_i|^p \right)^{\frac{1}{p}} \le  \left( \sum_{i=1}^{n} |x_i |^p \right)^{\frac{1}{p}} + \left( \sum_{i=1}^{n} |y_i|^p \right)^{\frac{1}{p}}
    \] 
\end{prop}

\begin{definition}
    Soit $U$ un ensemble et $E = \{f: U \to \mathbb{K} \text{ bornée} \}$ 
    \[
        \|f\|_{\infty} = \sup_{x \in U} |f(x)| \text{ norme sur E}
    \] 
\end{definition}
\begin{definition}
    $R([a, b], \mathbb{K})$ = $\{$ les  $f: [a, b] \to \mathbb{K}$ intégrables au sens de Riemann\footnote{La fonction est Riemann intégrable (pas forcément continue) si on peut calculer l'air en utilisant l'intégration par les sommes de Riemann. Alors, si $f$ discontinue, elle est Riemann intégrable si la discontinuité est négligable.} $\}$
\end{definition}

\begin{eg}
    \[
        \|f\|_{p} = \left( \int_{{a}}^{{b}} {|f(x)|^{p}} \: d{x} {} \right)^{\frac{1}{p}} \text{ avec } 1 \le p < \infty
    \] 
    $\| . \|_{p}$ est une semi-norme sur $R([a, b], \mathbb{K})$ (inégalité de Minkowski). $\|f\|_{p} = 0$ n'entraine pas que $f = 0$ (e.g: $[a, b] = [-1, 1]$,  $f(x) = x$, $p = 3$). 
    \[
        \|u + v\|_{p} \le \|u\|_p + \|v\|_p
    \] 
    Sur $E = \mathcal{C}([a, b], \mathbb{K})$,  $\| . \|_{p}$ est une norme:
    si $f: [a, b] \to \mathbb{K}$ continue et $\int_{{a}}^{{b}} {|f(x)|^p} \: d{x} = 0$ alors $f(x) = 0 \forall x \in [a, b]$
\end{eg}

\begin{eg}
    $E = \mathbb{K}^{\N}$ un ensemble des suites $u$ à valeurs dans  $\mathbb{K}$ 
    \[
    u = (u_1, u_2, \ldots, u_n, \ldots )
    \] 
    pour $1 \le p < \infty$
    \[
        l^{p}(\N, \mathbb{K}) = \{(u_n): \sum_{n \in \N}^{} |u_n|^{p} \text{ est convergente } \}
    \] 
    \[
        \|u\|_p = \left( \sum_{n=0}^{\infty} |u_n|^{p} \right)^{\frac{1}{p}}
    \] 
    est une norme sur $l^{p}(\N, \mathbb{K})$
    \[
        p = \infty \quad l^{\infty}(\N, \mathbb{K}) = \{u \text{ bornée } \}
    \] 
    \[
        \|u\|_{\infty} = \sup_{n \in \N} |u_n|
    \] 
\end{eg}
\section{Topologie des espaces vectoriels normés}
\begin{prop}
    Soit $(E, \| . \|)$ un espace vectoriel normé avec 
     \[
    d(u, v) = \|u - v\|
    \] 
    une distance sur $E$ (induite par $\| . \|$), alors  $(E, d)$ est un espace métrique.
\end{prop}
\begin{definition}
    Un espace vectoriel normé complet s'appelle \textbf{un espace de Banach}.
\end{definition}
Cas de dimension finie:
\begin{enumerate}
    \item Tout espace vectoriel normé de dimension finie est complet (rappel: proposition \ref{prop:rd-est-complet}) (voir plus bas)
    \item Si $E$ de dim finie:
         \[
        K \text{ compact} \iff K \text{ fermé et borné }
        \] 
\end{enumerate}
\begin{lemma}
   \[
       \left( \mathcal{C}([0, 1], \R), \|.\|_1 \right) 
   \]  
   n'est pas complet.
\end{lemma}
\begin{preuve}
    On construit une suite de fonctions continues $(f_n)_{n \in \mathbb{N}}$ sur $[0,1]$ qui converge en norme $\| \cdot \|_1$ vers une fonction $f$ discontinue. Cela montrera que la limite de cette suite dans la norme $\| \cdot \|_1$ n’appartient pas à $\mathcal{C}([0,1], \mathbb{R})$, donc que cet espace n’est pas complet.
    
\begin{figure}[H]
    \centering
    \incfig{lemma-n-est-pas-complet}
    \caption{Lemma avec un espace pas complet}
    \label{fig:lemma-n-est-pas-complet}
\end{figure}

\medskip

\textbf{Définition de la suite $(f_n)$ :} pour chaque $n \in \mathbb{N}$, on définit $f_n : [0,1] \to \mathbb{R}$ par
\[
f_n(x) = 
\begin{cases}
0 & \text{si } x \le \frac{1}{2} - \frac{1}{2n}, \\[4pt]
2n\left(x - \frac{1}{2} + \frac{1}{2n}\right) & \text{si } \frac{1}{2} - \frac{1}{2n} < x < \frac{1}{2} + \frac{1}{2n}, \\[4pt]
1 & \text{si } x \ge \frac{1}{2} + \frac{1}{2n}.
\end{cases}
\]
Chaque $f_n$ est continue sur $[0,1]$ car elle est affine par morceaux avec raccords continus.

\medskip

\textbf{Définition de la fonction limite :} posons
\[
f(x) = 
\begin{cases}
0 & \text{si } x < \frac{1}{2}, \\
1 & \text{si } x > \frac{1}{2}, \\
\text{valeur quelconque} & \text{si } x = \frac{1}{2}.
\end{cases}
\]
Alors $f$ est \textbf{discontinue} en $x = \frac{1}{2}$, donc $f \notin \mathcal{C}([0,1], \mathbb{R})$.

\medskip

\textbf{Convergence de $(f_n)$ vers $f$ dans $\| \cdot \|_1$ :}

On a
\[
\|f_n - f\|_1 = \int_0^1 |f_n(x) - f(x)|\, dx.
\]
Mais $f_n(x) = f(x)$ sauf sur l’intervalle $\left[\frac{1}{2} - \frac{1}{2n}, \frac{1}{2} + \frac{1}{2n}\right]$ de longueur $\frac{1}{n}$, et sur cet intervalle, $|f_n(x) - f(x)| \le 1$, donc :
\[
\|f_n - f\|_1 \le \int_{\frac{1}{2} - \frac{1}{2n}}^{\frac{1}{2} + \frac{1}{2n}} 1\, dx = \frac{1}{n} \xrightarrow[n \to \infty]{} 0.
\]

Ainsi, $f_n \to f$ dans la norme $\| \cdot \|_1$.

\medskip

\textbf{Conséquence :} la suite $(f_n)$ est de Cauchy dans $\left( \mathcal{C}([0,1], \mathbb{R}), \| \cdot \|_1 \right)$, car :
\[
\|f_n - f_p\|_1 \le \|f_n - f\|_1 + \|f - f_p\|_1 \le \frac{1}{n} + \frac{1}{p} \xrightarrow[n,p \to \infty]{} 0.
\]

Cependant, la limite $f$ n’est pas continue, donc $f \notin \mathcal{C}([0,1], \mathbb{R})$.

\medskip

\textbf{Conclusion :} Il existe une suite de Cauchy dans $\left( \mathcal{C}([0,1], \mathbb{R}), \| \cdot \|_1 \right)$ qui ne converge pas dans cet espace. Par conséquent, cet espace n’est pas complet.

\end{preuve}
\begin{lemma}
    Dans $E = l^{1}(\N, \R)$ muni de
    \[
    \|u\|_1 = \sum_{n=0}^{\infty} |u_n|
    \] 
    $B_f(0, 1)$ n'est pas compact.
\end{lemma}
\begin{preuve}
    On construit une suite d'éléments de $B_f(0, 1)$ \underline{sans sous-suite convergente}.
    \[
    u \in E \quad u: \N \to \R
    \] 
    Je note $u(p)$ au lieu de  $u_p$ suite dans  $E$ noté  $(u_n)$,  $u_n \in E$. $u_n(p)$ p-ième terme de  $u_n$. Je pose  
    \[
    u_n(p) = \delta_{n, p} = \begin{cases}
        1 \text{ si } n = p\\
        0 \text{ sinon}
    \end{cases}
    \] 
    \[
    \|u_n\|_1 = \sum_{p=0}^{\infty} |u_n(p)| = |u_n(n)| = 1
    \] 
    Donc $u_n \in B_f(0, 1) \forall n$.
    \par
    Si $v \in l^1(\N, \R)$
    \[
    |v(p)| \le \sum_{p=0}^{\infty} |v(p)| = \|v\|_1
    \] 
    si $\|v_n - v\|_1 \xrightarrow[n \to \infty]{} 0$ alors $\forall p, v_n(p) \to v(p)$. Supposons que $(v_n) = (u_{\phi(n)})$ est une sous-suite de $(u_n)$ qui converge vers  $v$ pour  $\| . \|$. Je fixe $p \in \N$, $v_n(p) = u_{\phi(n)}(p) \xrightarrow[n \to \infty]{} v(p)$, mais $v_n(p) \xrightarrow[n \to \infty]{} 0$, donc $v(p) = 0 \forall p$. $v$: suite nulle, aussi 
     \[
    \|v_n\|_1 = 1 \forall n \text{ et } \|v_n\|_1 \xrightarrow[n \to \infty]{} \|v\|_1
    \] 
    contradiction
\end{preuve}
\section{Normes équivalentes}
\begin{definition}
    Deux normes $N_1$ et  $N_2$ sur  $E$ sont équivalentes ($N_1 \sim N_2$) si $\exists c_1, c_2 > 0$ telles que 
    \begin{itemize}
        \item $N_1(u) \le c_1N_2(u) \quad \forall u \in E$
        \item $N_2(u) \le c_2N_1(u) \quad \forall u \in E$
    \end{itemize}
    $\exists c > 0$ telle que
    \[
    cN_1(u) \le N_2(u) \le cN_1(u)
    \] 
\end{definition}
\begin{remark}
   Si  $N_1 \sim N_2$ et $N_2 \sim N_3$, alors $N_1 \sim N_3$ 
\end{remark}
\begin{definition}
    Les normes $N_1$ et $N_2$ sont \textbf{topologiquement équivalentes} si elles définissent les mêmes ensembles ouverts.
\end{definition}
\begin{theorem}
    Soientt $N_1, N_2$ deux normes, alors:
    \[
    N_1, N_2 \text{ topologiquement équivalentes } \iff N_1, N_2 \text{ équivalentes }
    \] 
\end{theorem}
\begin{eg}
    \begin{enumerate}
        \item $E = \mathcal{C}([0, 1], \R)$
        \item $\|f\|_{\infty} = \sup_{x \in [0, 1]}|f(x)|$
        \item $\|f\|_1 = \int_{{0}}^{{1}} {|f(x)|} \: d{x}$
    \end{enumerate}
    On remarque que $\|f\|_1 \le \|f\|_{\infty}$. Est-ce que $\exists c > 0$ telle que 
    \[
    \|f\|_{\infty} \le c\|f_1\| \forall f \in E
    \] 
    ?
    Pour le voir, construire une suite $(f_n)$ dans  $E$ telle que  $\|f_n\|_1 \to 0$ mais $\|f_n\|_{\infty} \not\to 0$
\end{eg}
\begin{theorem}
    Soit $E$ un espace de dimension finie. Alors toutes normes sur  $E$ sont équivalentes.
\end{theorem}
\begin{preuve}

   Puisque \( E \) est de dimension finie, il existe une base de \( E \) et donc un isomorphisme linéaire entre \( E \) et \(\mathbb{R}^n\) (ou \(\mathbb{C}^n\)). En conséquence, on peut se ramener à l'étude de normes sur \(\mathbb{R}^n\).


   Considérons la norme \(\|\cdot\|_1\) sur \( E \) et définissons la sphère unité associée :
   \[
   S = \{ x \in E : \|x\|_1 = 1 \}.
   \]
   Dans un espace de dimension finie, la sphère unité \( S \) est compacte (cela repose sur le fait que dans \(\mathbb{R}^n\), les ensembles fermés et bornés sont compacts).


   La fonction
   \[
   f : S \to \mathbb{R}, \quad f(x) = \|x\|_2
   \]
   est continue car \(\|\cdot\|_2\) est une norme (et donc une fonction continue). Par le théorème de Weierstrass, \( f \) atteint ses bornes sur \( S \). Il existe donc :
   \begin{itemize}
       \item Un minimum \( m = \min_{x \in S} f(x) > 0 \) (la stricteté de \( m > 0 \) s’explique par le fait que \( x \neq 0 \) pour \( x \in S \)).
       \item Un maximum \( M = \max_{x \in S} f(x) \).
   \end{itemize}


   Soit \( x \in E \) quelconque, \( x \neq 0 \). On écrit \( x = \|x\|_1 \, y \) avec \( y = \frac{x}{\|x\|_1} \) qui appartient à \( S \). Alors,
   \[
   \|x\|_2 = \|x\|_1\,\|y\|_2.
   \]
   Or, puisque \( y \in S \), on a
   \[
   m \le \|y\|_2 \le M.
   \]
   Ainsi,
   \[
   m\,\|x\|_1 \le \|x\|_2 \le M\,\|x\|_1.
   \]
   En posant \( c = m \) et \( C = M \), nous obtenons exactement l'équivalence des normes.


   Pour \( x = 0 \), l'inégalité est triviale car \( \|0\|_1 = \|0\|_2 = 0 \).


\end{preuve}
